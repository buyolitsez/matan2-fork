% 2023.03.23 lecture 05
\documentclass[../complex-analysis.tex]{subfiles}
\begin{document}

\section{Логарифм и аргумент.}

\begin{df}[%
аналитическая ветвь логарифма]
 Пусть $ \Omega  \subset \CC$ --- область, $ f \colon\, \Omega \to \CC   $ --- аналитическая функция. Тогда функция $ g \colon\, \Omega \to \CC   $ называется \textit{аналитической ветвью} $ \log f $, если $ g $ аналитична и
 \begin{align*}
  e^{g(z)} = f(z)
 \end{align*} для всех $ z \in \Omega $.
\end{df}
\begin{remrk}
 Если существует точка $ z_0 $ такая, что $ f(z_0) = 0 $, то у функции $ f $ нет аналитической ветви логарифма. 
\end{remrk}
\begin{proof}[\normalfont\textsc{Доказательство}]
 $ e^{z} \neq 0 $ для $ z \in \CC $.
\end{proof}

\begin{exmpl}
 Пусть $ \Omega = \CC \setminus \left\{ 0 \right\} $, и функция $ f(z) = z $. Предположим, что у $ f $ есть аналитическая ветвь логарифма --- функция $ g $. Продифференцируем:
 \begin{align*}
  g' \cdot e^{g} = 1 \implies g' = \frac{1}{e^{g}} = \frac{1}{z}.
 \end{align*} Проинтегрируем
 \begin{align*}
  \int\limits_{\left| z \right|=1} g'\,dz = 0,
 \end{align*} так как $ g'\,dz = dg $ --- точная форма.

 С другой стороны,
 \begin{align*}
  \int\limits_{\left| z \right|=1}   \frac{dz}{z} = 2\pi i.
\end{align*} $\implies \not \exists$ аналитической ветви логарифма для $f$ в $\CC \setminus \left\{ 0 \right\} $ .   
\end{exmpl}

\begin{thm}[%
]
 Если $ \Omega \subset \CC $ --- односвязная область, и $ f(z) \neq 0 $ для любой $ z \in \Omega $, то существует аналитическая ветвь $ \log f $. Более того, любые две аналитические ветви $ \log f $ отличаются на $ 2\pi k i $, $ k \in \Z $.
\end{thm}
\begin{proof}[\normalfont\textsc{Доказательство}]
 Рассмотрим функцию $h = \frac{f'}{f}$. Функция $ h $ аналитична в $ \Omega $. Значит, форма  $ h\,dz $ замкнута в  $ \Omega $, по теореме Коши-Гурса-Морера \eqref{theorem:cauchy-gursa-morer}. Значит, $ h\,dz $ точна в  $ \Omega $, потому что  $ \Omega $  односвязна. Тогда существует аналитическая в $ \Omega $  функция $ g $  такая, что $ g' = h $ в  $ \Omega $. Наша цель: показать, что $ g $  почти-что и есть логарифм $ f $.

 \begin{align*}
  \left( \frac{e^{g}}{f} \right)' = \frac{g' e^{g} \cdot f - e^{g} \cdot f'}{f^{2}} = \frac{\frac{f'}{f} \cdot e^{g} \cdot f - f' \cdot e^{g}}{f^{2}} = 0.
\end{align*} Значит, $ \frac{e^{g}}{f} $  --- аналитическая функция на $ \Omega $, со всюду нулевой производной. Значит, по неравенству Лагранжа \eqref{theorem:Lagrange_inequality} $ \frac{e^{g}}{f} = c $ всюду в $ \Omega $. Выберем $ \tilde c $ так, что $ c \cdot e^{\tilde c} = 1 $, и тогда $ e^{g+\tilde c} = f $ всюду в $ \Omega $. Значит, $ g + \tilde c $ --- аналитическая ветвь логарифма $ f $.

В качестве $\tilde c$ можно взять: 
 \begin{align*}
  e^{\tilde c} = \frac{ f(z_0)}{e^{g(z_0)}}
 \end{align*} в какой-то точке $ z_0 \in \Omega $.

 Осталось доказать единственность. Пусть $ e^{g_1} = e^{g_2} = f $ в $ \Omega $. Тогда
 \begin{align*}
  e^{g_1 - g_2} = 1
\end{align*} в $ \Omega $. Тогда $ (g_1-g_2)(z) = 2\pi i k (z) $, по замечанию \eqref{remark:expeq1_implies_poweq2pk}, где $ k(z) \in \Z$. Так как функция $k$ непрерывна, то $ k $ --- это константа в $ \Omega $.

\end{proof}
\begin{remrk}
  \label{remark:expeq1_implies_poweq2pk}
	$e^{\lambda} = 1 \implies \lambda = 2 \pi i k, k \in \Z$ 
	\begin{proof}
		$\lambda = a + ib \ (a, b \in \R), \ |e^a \cdot e^{ib}| = 1 \implies e^a = 1 \implies a = 0, e^{ib} = 1 \implies \cos b + i \sin b = 1 $. Значит $\sin b = 0, \ \cos b = 1 \implies b = 2 \pi k$ 
	\end{proof}
\end{remrk}

\begin{df}
 Пусть $ f $ --- аналитическая в $ \Omega $. \textit{Непрерывной ветвью} аргумента $ f $ называется непрерывная функция $ \arg f \colon\, \Omega \to \R $ такая, что
 \begin{align*}
  \left| f(z) \right| \cdot e^{i \arg f(z)} = f(z).
 \end{align*}
\end{df}
\begin{thm}
 Если область $ \Omega \subset \CC  $ односвязна, $ f \neq 0 $ в $ \Omega $, то существует непрерывная ветвь аргумента $ f $, причём любые две такие ветви отличаются на $ 2\pi k $, $ k \in \Z $.
\end{thm}
\begin{proof}[\normalfont\textsc{Доказательство}]
 $ \arg f(z) = \Imaginary (\log f) $, где $ \log f $ --- аналитическая ветвь логарифма $ f $. Тогда
 \begin{align*}
 f = e^{\log f} = e^{\Real \log f} \cdot e^{i \Imaginary (\log f)} = \left| f \right| e^{i \arg f}.
 \end{align*} Если $ A_1, A_2 $ --- непрерывные ветви аргумента, то
 \begin{align*}
 e^{i(A_1 - A_2)} = \frac{\left| f \right|e^{iA_1}}{\left| f \right|e^{iA_2}} = \frac{f}{f} = 1.
\end{align*} Значит, $ A_1 - A_2 = 2\pi k(z) $ (По замечанию \eqref{remark:expeq1_implies_poweq2pk}), где $ k(z) \in \Z $. Так как $ k $ непрерывна, то $ k = k_0 $, $ k_0 \in \Z $.
\end{proof}
\begin{remrk}
 В ходе доказательства мы проверили, что $ e^{\Real f} = \left| f \right| = e^{\ln \left| f \right|} $, где $ \ln \left| f \right| $ --- вещественный логарифм в старом смысле. Иными словами,
 \begin{align*}
  \Real (\log f) &= \ln \left| f \right|, \\
  \Imaginary (\log f) &= \arg f.
 \end{align*}
\end{remrk}

\begin{df}
 Пусть $ \alpha \in\CC $, $ f \colon\, \Omega \to \CC   $ --- аналитическая функция, существует $ \log f $  --- аналитическая ветвь логарифма функции $ f $. Тогда
 \begin{align*}
  f^{\alpha} = e^{\alpha \log f}.
 \end{align*} В частности, в $ \CC \setminus \left\{ \left(-\infty, 0\right]   \right\} $  определена степень $ z^{\alpha} = e^{\alpha \log z} $, где $ \log z $ --- какая-то ветвь логарифма $ z $.
\end{df}

\begin{df}
 \textit{Главная ветвь} $ \log z $ --- это такая аналитическая ветвь логарифма $ z $, что $ \log 1 = 0 $.
\end{df}
\begin{exmpl}
 Вычислим $ i^{i} $, если степень понимается заданной с помощью главной ветви логарифма.

\begin{align*}
 i^{i} = e^{i \log i} = e^{i \cdot \left( \Real \log i + \Imaginary \log i \right)} = e^{i \cdot \left( \ln \left| i \right| + i \arg i \right)} = e^{-\arg i} = e^{-\pi / 2}.
\end{align*} Почему же $ \arg i = \pi / 2 $?  $ i = e^{i \frac{\pi}{2}} $ , и в области $ \CC \setminus \left(-\infty, 0\right]   $  полярные координаты задают непрерывную ветвь аргумента: $\varphi \in [-\pi, \pi], \ r > 0$ . 
\end{exmpl}

\newpage
\section{Принцип максимума.}

\begin{df}
 Пусть $ (X_1,\rho_1) $, $ (X_2,\rho_2) $ --- метрические пространства.
 $ f \colon\, X_1 \to X_2   $ --- \textit{открытое отображение}, если оно переводит открытые множества в открытые.
\end{df}
\begin{remrk}
 Композиция открытых отображений --- открытое отображение.
\end{remrk}
\begin{thm}
\label{theorem:analytic_implies_open}
 Пусть $ \Omega \subset \CC$ --- область, $ f \colon\, \Omega \to \CC  $  --- аналитическая функция, $ f \not\equiv \mathrm{const} $. Тогда $ f $ --- открытое отображение.
\end{thm}
\begin{thm}[принцип максимума]
\label{theorem:maximum_principle}
 Если $ f \colon\, \Omega \to \CC $ --- аналитическая и существует $ z_0 \in \Omega $ такая, что $ \left| f(z_0) \right| \geqslant \left| f(z) \right| $ для любой $ z \in \Omega $, то $ f \equiv \mathrm{const} $.
\end{thm}
\begin{prop}
Из теоремы \eqref{theorem:analytic_implies_open} следует теорема \eqref{theorem:maximum_principle} .
\end{prop}
\begin{proof}[\normalfont\textsc{Доказательство}]
 Так как $ f(B(z_0, \eps)) $ содержит точку $ f(z_0) $ в предположении теоремы \eqref{theorem:maximum_principle}, но не содержит точки $ w $ такие, что $ \left| w \right| > \left| f(z_0) \right| $. Значит $ f $ не открытое, и значит $ f \equiv \mathrm{const} $.
\end{proof}
\begin{lm} (в условиях теоремы \eqref{theorem:analytic_implies_open})
 Рассмотрим $ F(x, y) = (u(x,y), v(x,y)) $, где $ u(x,y) = \Real f(x + iy) $, $ v(x,y) = \Imaginary f(x +iy) $. Тогда
 \begin{align*}
  \det J_F (x_0, y_0) = \left| f'(z_0) \right|^{2}.
 \end{align*}, где $z_0 = x_0 + i y_0$.
\end{lm}
\begin{proof}[\normalfont\textsc{Доказательство}]
 \begin{align*}
  \det J_F = \det \begin{pmatrix}
   u'_x & u'_y \\
   v'_x & v'_y \\
\end{pmatrix} = [\text{усл. Коши-Римана \eqref{theorem:cauchy_riman}}] = \det \begin{pmatrix}
  u'_x & u'_y \\
  -u'_y & u'_x
  \end{pmatrix} = u^{'2}_x + u^{'2}_y.
 \end{align*} С другой стороны,
 \begin{align*}
  \left| f'(z_0) \right|^{2} = \left| \left. \frac{\partial}{\partial x}f(x + iy_0) \right|_{\substack{x = x_0}}  \right|^{2} = \left| u'_x(x_0 + iy_0) + iv'_x(x_0 + iy_0) \right| = \\
   = u^{'2}_x + v^{'2}_x = u^{'2}_x + u^{'2}_y,
 \end{align*} снова по условию Коши-Римана.
\end{proof}
\begin{crly}
	\label{crly:derivative_not_0_implies_open}
 Если $ z_0 $: $ f'(z_0) \neq 0 $, то в малой окрестности $ z_0 $ отображение $ f $ открыто.
\end{crly}
\begin{proof}[\normalfont\textsc{Доказательство}]
 Если $ f'(z_0) \neq 0 $, то по непрерывности $ \left| f'(z) \right|^{2} \neq 0 $ для $ z \in B(z_0, \eps) $, и значит $ \det J_F \neq 0 $ в шаре, и по теореме об открытости отображения с ненулевым якобианом: $ f $ открыто как отображение из $ B(z_0, \eps) \subset \R^{2} $ в $ \R^{2} $.
\end{proof}
\begin{lm}
	\label{lemma:z_to_n_is_open}
 $ z^{n} $ --- открытое отображение в $ \CC $ для любого $ n \in \N $.
\end{lm}
\begin{proof}[\normalfont\textsc{Доказательство}]
 Производная $(z^{n})' = nz^{n-1} $  --- не обнуляется в $ \CC \setminus \left\{ 0 \right\} $. Значит $ \forall z_0 \in \CC \setminus \left\{ 0 \right\} \colon\; \exists  \eps > 0 \colon\; f=z^{n}, f(B(z_0, \eps))  $ открыто по предыдущему следствию.
 \begin{align*}
  f(B(0,r)) = B(0, r^{n})
 \end{align*} --- открытое множество. Значит, $ f $ открыто.
\end{proof}
\begin{proof}[\normalfont\textsc{Доказательство теоремы \eqref{theorem:analytic_implies_open}}]
  Подберем так, чтобы $ f = h_1 \circ h_2 \circ h_3 $, где $ h_1 $, $ h_2 $, $ h_3 $ --- открытые в окрестности $ z_0 \in \Omega $.
 \begin{align}
  \label{equation:proof:theorem_open_representation}
  f(z) = f(z_0) + \left((z - z_0) \cdot g(z) \right)^{n},
\end{align} где $ g $ --- аналитическая в окр. $ z_0 $ и $ n \in \N $, $ g(z_0) \neq 0 $. Докажем \eqref{equation:proof:theorem_open_representation}. По лемме \eqref{lemma:zero_multiplicity} о кратности нуля
 \begin{align*}
  Q(z) = f(z) - f(z_0) = (z - z_0)^{n} \cdot P(z),
 \end{align*} где $ P $ --- аналитическая. Теперь
 \begin{align*}
  Q(z) = (z - z_0)^{n} \cdot e^{\log P}
 \end{align*} в односвязной окрестности $ z_0 $, и $ P \neq 0 $ в этой окрестности. Тогда
 \begin{align*}
  Q(z) = \left((z-z_0) \cdot e^{\frac{\log P}{n}}\right)^{n}.
 \end{align*} Итого
 \begin{align*}
  g = e^{\frac{\log P}{n}}.
 \end{align*} Формула \eqref{equation:proof:theorem_open_representation} доказана.

 \begin{align*}
  &h_3 \colon\, z \mapsto (z-z_0)(g(z)) \\
  &h_2 \colon z \mapsto z^{n} \\
  &h_1 \colon z \mapsto f(z_0) + z
 \end{align*} $ h_1 $ очевидное открытое, $ h_2 $ открытое (по лемме \eqref{lemma:z_to_n_is_open}). Почему $ h_3 $ открытое?
 \begin{align*}
	 h_3'(z_0) = \left. g + (z - z_0) g' \right|_{\substack{z = z_0}} = g(z_0) \neq 0
\end{align*} Тогда по следствию \eqref{crly:derivative_not_0_implies_open}  $ h_3 $ открыто в малой окрестности $ z_0 \in \Omega $.

\end{proof}


\end{document}
