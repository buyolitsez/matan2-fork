% 2023.03.23 lecture 05
\documentclass[../complex-analysis.tex]{subfiles}
\begin{document}

\newpage
\section{Экспонента, логарифм и аргумент.}

В этом параграфе мы изучим более конкретные, но фундаментальные функции в комплексном анализе: \emph{экспоненту}, \emph{логарифм} и \emph{аргумент}. С экспонентой все довольно просто: это очень хорошая целая функция с замечательными свойствами. Из неё также можно построить тригонометрические функции: косинус и синус. Логарифм и аргумент, напротив, невозможно сделать целыми функциями, поэтому к ним требуется более аккуратный подход.

\subsection{Экспонента.}

\begin{df}
 \emph{Экспонентой} называется целая функция
 \begin{align}
  \label{eq:def:exp}
  \exp z = \sum_{n=0}^{\infty} \frac{z^{n}}{n!}.
 \end{align}
\end{df}
\begin{remrk}
 Экспонента действительно является целой функцией: ряд~\eqref{eq:def:exp} по теореме~\ref{theorem:adamar} Адамара (пример~\ref{example:exp_function}) сходится всюду в $ \CC $, поэтому по условию~\ref{enum3:theorem:cauchy_gurs_morer} теоремы~\ref{theorem:cauchy-gursa-morer} Коши-Гурса-Морера функция аналитична в $ \CC $.
\end{remrk}

\begin{prop}[основное свойство экспоненты]
 \begin{align}
  \label{eq:exponent_common_property}
  \exp(z+w) = \exp z \cdot \exp w
 \end{align} для любых $ z,w\in\CC $.
\end{prop}
\begin{proof}[\normalfont\textsc{Доказательство}]
 Действительно,
 \begin{align*}
  \sum_{n=0}^{\infty} \frac{(z+w)^{n}}{n!} = \sum_{n=0}^{\infty} \sum_{k=0}^{n} \frac{z^{k}\cdot w^{n-k}}{k!(n-k)!} = \left[ \sum_{n=0}^{\infty} \frac{z^{n}}{n!} \right] \cdot \left[ \sum_{m=0}^{\infty}\frac{w^{m}}{m!} \right],
 \end{align*} переходы верны, так как все ряды сходятся абсолютно (это некоторая теорема из теории рядов).
\end{proof}

Можно проверить, что $ \exp x $ при $ x \in \R $ совпадает с <<обычной>> вещественной экспонентой $ e^{x} $. Во-первых,
\begin{align*}
 \exp 1 = \sum_{n=0}^{\infty} \frac{1}{n!} = \lim_{n \to \infty} \left( 1 + \frac{1}{n} \right)^{n} = e,
\end{align*} так как
\begin{align*}
 \left( 1 + \frac{1}{n} \right)^{n} &= \sum_{k=0}^{n} \frac{n(n-1)\ldots(n-k+1)}{n^{k} \cdot k!} = \\
 &= \sum_{k=0}^{n} 1 \cdot \left( 1 - \frac{1}{n} \right) \cdot \ldots \cdot \left( 1 - \frac{k}{n} \right) \cdot \frac{1}{k!} \to \sum_{k=0}^{\infty}\frac{1}{k!}.
\end{align*} Во-вторых, по свойству~\eqref{eq:exponent_common_property} экспонента в новом и старом смысле совпадают на $ \Z $, так как
\begin{align*}
 \exp n = \underbrace{\exp 1 \cdot \ldots \cdot \exp 1}_{n \text{ раз}} = \underbrace{e \cdot e \cdot \ldots e}_{n \text{ раз}} = e^{n},
\end{align*} и
\begin{align*}
 1 = \exp 0 = \exp n \cdot \exp (-n) \implies \exp (-n) = e^{-n}.
\end{align*} В третьих, по тому же свойству~\ref{eq:exponent_common_property} функции совпадают при показателях вида $ 1 / n $:
\begin{align*}
 \exp 1 = \left( \exp \frac{1}{n} \right)^{n} \implies \exp \frac{1}{n} = e^{1 / n},
\end{align*} а значит и на $ \Q $. Наконец, по непрерывности функции совпадают на $ \R $. По этой причине вместо $ \exp z $ мы будем просто писать $ e^{z} $.

\begin{prop}
 $ e^{z} \neq 0 $.
\end{prop}
\begin{proof}[\normalfont\textsc{Доказательство}]
 Предположим, что $ e^{z_0} = 0 $ в точке $ z_0 \in \CC $. Но тогда
 \begin{align*}
  1 = e^{0} = e^{z_0} \cdot e^{-z_0} = 0 \cdot e^{-z_0} = 0.
 \end{align*}
\end{proof}

\begin{prop}
 $ (e^{z})' = e^{z} $.
\end{prop}
\begin{proof}[\normalfont\textsc{Доказательство}]
 Действительно, это легко проверяется по следствию~\ref{corollary:complex_differential_of_power_series}:
 \begin{align*}
  (e^{z})' = \sum_{n=1}^{\infty} n \cdot \frac{z^{n-1}}{n!} = \sum_{k=0}^{\infty}\frac{z^{k}}{k!}.
 \end{align*}
\end{proof}

Более того, можно доказать, что экспонента --- это единственная с точностью до мультипликативной константы целая функция~$ f $, удовлетворяющая функциональному уравнению $ f' = f $.

\begin{prop}
 \label{prop:|e^it| is 1}
 $ \left| e^{it} \right| = 1 $ при $ t \in \R $.
\end{prop}
\begin{proof}[\normalfont\textsc{Доказательство}]
 Заметим, что для $ t \in \R $
 \begin{align*}
  \overline{e^{it}} = e^{-it},
 \end{align*} ведь $ \overline \imath = -i $. Тогда
 \begin{align*}
  1 = e^{0} = e^{it - it} = e^{it} \cdot e^{-it} = e^{it} \cdot \overline{e^{it}} = \left| e^{it} \right|^{2},
 \end{align*} что и требовалось доказать.
\end{proof}

Выделим в ряде $ e^{it} $ для $ t \in \R $ вещественную и мнимую часть:
\begin{align*}
 e^{it} = \sum_{n=0}^{\infty} \frac{(it)^{n}}{n!} = \sum_{k=0}^{\infty} \frac{(-1)^{k}t^{k}}{(2k)!} + i \cdot \left[ \sum_{k=0}^{\infty} \frac{(-1)^{k}t^{2k+1}}{(2k+1)!} \right].
\end{align*} Перестановку слагаемых делать можно, так как все ряды сходятся абсолютно. Это выражение наталкивает на следующее определение.

\begin{df}
 \label{def:cos_sin}
 \emph{Косинусом} называется целая функция
 \begin{align*}
  \cos z = \frac{e^{iz} + e^{-iz}}{2} = \sum_{k=0}^{\infty} \frac{(-1)^{k}z^{2k}}{(2k)!}.
 \end{align*} \emph{Синусом} называется целая функция
 \begin{align*}
  \sin z = \frac{e^{iz} - e^{-iz}}{2i} = \sum_{k=0}^{\infty} \frac{(-1)^{k}z^{2k+1}}{(2k+1)!}.
 \end{align*}
\end{df}
\begin{remrk*}
 Косинус и синус действительно целые функции, так как они выражаются через экспоненту.
\end{remrk*}

Из определения~\ref{def:cos_sin} непосредственно следует \emph{формула Эйлера}
\begin{align}
 \label{eq:exp:euler_formula}
 e^{iz} = \cos z + i \sin z.
\end{align} А из формулы~\eqref{eq:exp:euler_formula} и предложения~\ref{prop:|e^it| is 1} следует основное тригонометрическое тождество:
\begin{align*}
 \cos^{2}t+\sin^{2}t=1,\qquad t\in\R.
\end{align*} Формулу Эйлера можно также записать в виде
\begin{align*}
 e^{x+iy} = e^{x} \cdot (\cos y + i \sin y),
\end{align*} где экспонента, косинус и синус берутся от вещественного числа.

\subsection{Аналитическая ветвь логарифма.}

В классическом смысле логарифм --- это функция, обратная экспоненте. В комплексном анализе нам понадобится некоторое более замысловатое определение.

\begin{df}[%
 аналитическая ветвь логарифма]
 Пусть функция~$ f\colon\,\Omega\to\CC $ аналитична в области~$ \Omega \subset \CC $. Функция $ g\colon\,\Omega\to\CC $ называется \emph{аналитической ветвью логарифма $ f $} в области~$ \Omega $, если $ g $ аналитична в $ \Omega $, и выполнено
 \begin{align*}
  e^{g(z)} = f(z)
 \end{align*} всюду в $ \Omega $. Если из контекста ясно, о какой ветви логарифма идёт речь, то её обозначают $ \log f(z) = g(z) $.
\end{df}

Разумеется, не во всех ситуациях существует аналитическая ветвь логарифма функции. Приведём конкретные примеры.

\begin{remrk}
 Если существует точка $ z_0 \in \Omega $ такая, что $ f(z_0) = 0 $, то у функции $ f $ нет аналитической ветви логарифма. 
\end{remrk}
\begin{proof}[\normalfont\textsc{Доказательство}]
 В самом деле, $ e^{w} \neq 0 $ при всех $ w \in \CC $.
\end{proof}

\begin{exmpl}
 Рассмотрим тождественную функцию~$ f(z)=z $ в области~$ \Omega = \CC \setminus \left\{ 0 \right\} $. Предположим, что у неё есть аналитическая ветвь логарифма --- функция $ g  $. Продифференцируем функцию $ f $:
 \begin{align*}
  1 = (z)' = (e^{g(z)})' = g'(z) \cdot e^{g(z)} = g'(z) \cdot z.
 \end{align*} Следовательно,
 \begin{align*}
  g'(z) = \frac{1}{z}.
 \end{align*} Так как дифференциальная форма $ g'(z)\,dz $ точна в $ \Omega $ (лемма~\ref{lemma:derivative_of_analytic_f_is_exact_form}), то
 \begin{align*}
  \int_{\left| z \right|=1} g'(z)\,dz = 0.
 \end{align*} С другой стороны, по формуле~\eqref{eq:int_form:dz/z-a} (лемма \ref{lemma:form:dz/z-a}), этот интеграл равен
 \begin{align*}
  \int_{\left| z \right|=1}   \frac{dz}{z} = 2\pi i.
 \end{align*} Следовательно, у тождественной функции $ f(z) = z $ нет аналитической ветви логарифма в области $ \CC \setminus \left\{ 0 \right\} $.
\end{exmpl}

\begin{prop}
 Аналитическая ветвь логарифма аналитической функции $ f\colon\,\Omega\to\CC $ в области~$ \Omega \subset\CC $ единственна с точностью до прибавления $ 2\pi k i $, $ k \in \Z $, если, конечно, она существует.
\end{prop}
\begin{proof}[\normalfont\textsc{Доказательство}]
 Пусть $ g_1,g_2 $ --- две аналитические ветви логарифма функции $ f $. Тогда при всех $ z \in \Omega $ выполнено
 \begin{align*}
  e^{g_1(z)} = e^{g_2(z)} = f(z),
 \end{align*} и, значит
 \begin{align*}
  e^{g_1(z) - g_2(z)} = 1.
 \end{align*} Это означает (замечание~\ref{remark:expeq1_implies_poweq2pk}), что
 \begin{align*}
  g_1(z) - g_2(z) = 2\pi i k(z),
 \end{align*} где $ k(z) $ --- аналитическая в $ \Omega $ функция, принимающая лишь целые значения. Так как $ \Omega $ --- связное множество, то $ k(z) = k $ --- константа в $ \Omega $.
\end{proof}
\begin{remrk}
 \label{remark:expeq1_implies_poweq2pk}
 Пусть $ e^{z} = 1 $, $ z = x+yi $, $ x,y\in\R $. Тогда
 \begin{align*}
  1 = \left| e^{z} \right| = \left| e^{x} \cdot e^{iy} \right| = e^{x},
 \end{align*} и, следовательно, $ x = 0 $. В таком случае
 \begin{align*}
  \cos y + i \sin y = 1,
 \end{align*} следовательно, $ \cos y = 1 $, и $ y = 2\pi k $, $ k \in \Z $.
\end{remrk}

Теперь рассмотрим частую ситуацию, когда логарифм всё же существует.

\begin{thm}
 \label{theorem:exist_log_f}
 Пусть функция~$ f\colon\,\Omega\to\CC $ аналитична в односвязной области~$ \Omega \subset \CC $, и $ f(z) \neq 0 $ на $ \Omega $. Тогда существует аналитическая ветвь логарифма функции $ f $ в области~$ \Omega $.
\end{thm}
\begin{proof}[\normalfont\textsc{Доказательство}]
 Функция
 \begin{align*}
  h(z) = \frac{f'(z)}{f(z)}
 \end{align*} аналитична в области $ \Omega $ как частное аналитичных в $ \Omega $ функций, ведь $ f(z) \neq 0 $ в $ \Omega $. По теореме~\ref{theorem:cauchy-gursa-morer} Коши-Гурса-Морера дифференциальная форма $ h\,dz $ замкнута в $ \Omega $, а так как область односвязна, то по следствию~\ref{corollary:closed_form_is_exact_in_simply_connected} форма $ h\,dz $ точна в $ \Omega $. Тогда по лемме~\ref{lemma:o_pervoobraznoi} о первообразной существует такая аналитическая в $ \Omega $ функция $ g $, что
 \begin{align*}
  g'(z) = h(z)
 \end{align*} всюду в $ \Omega $. Покажем, что функция~$ g $ почти-что и есть искомый логарифм. Для всех $ z \in \Omega $ выполнено
 \begin{align*}
  \left[ \frac{e^{g(z)}}{f(z)} \right]' = \frac{g'(z) \cdot e^{g(z)} \cdot f(z) - e^{g(z)} \cdot f'(z)}{f(z)^{2}} = \frac{\frac{f'(z)}{f(z)} \cdot e^{g(z)} \cdot f(z) - e^{g(z)} \cdot f'(z)}{f(z)^{2}} = 0.
 \end{align*} Тогда по замечанию~\ref{remark:zero_derivative_analytical} данная функция постоянна на $ \Omega $:
 \begin{align*}
  \frac{e^{g(z)}}{f(z)} = \zeta, \qquad \zeta\in\CC,
 \end{align*} причём $ \zeta\neq 0 $, так как экспонента не имеет нулей. Возьмём такое число $ \theta \in \CC $, что $ e^{\theta} = \zeta^{-1} $. Тогда функция $ z \mapsto g(z) + \theta $ и есть искомая ветвь логарифма функции~$ f $:
 \begin{align*}
  \frac{e^{g(z) + \theta}}{f(z)} = \zeta \cdot \zeta^{-1} = 1 \implies e^{g(z)+\theta} = f(z).
 \end{align*}
\end{proof}

\subsection{Непрерывная ветвь аргумента.}

\begin{df}
 Пусть функция~$ f $ аналитична в области~$ \Omega $. \textit{Непрерывной ветвью аргумента $ f $ в области~$ \Omega $} называется непрерывная функция~$ g \colon\, \Omega \to \R $ такая, что
 \begin{align*}
  \left| f(z) \right| \cdot e^{i g(z)} = f(z)
 \end{align*} для всех $ z\in\Omega $. Если из контекста ясно, о какой ветви аргумента идёт речь, то её обозначают $ \arg f(z) = g(z) $.
\end{df}

Как и для ветвей логарифма, у ветвей аргумента есть некоторая единственность.

\begin{prop}
 Непрерывная ветвь аргумента аналитической функции~$ f $ в области~$ \Omega \subset \CC $ единственна с точностью до прибавления $ 2\pi k $, $ k\in\Z $, если, конечно, она существует.
\end{prop}
\begin{proof}[\normalfont\textsc{Доказательство}]
 Пусть $ g_1,g_2 $ --- непрерывные ветви аргумента $ f $ в области~$ \Omega $. Тогда в $ \Omega $ выполнено
 \begin{align*}
  e^{i(g_1(z) - g_2(z))} = \frac{\left| f(z) \right| \cdot e^{ig_1(z)}}{\left| f(z) \right| \cdot e^{ig_2(z)}} = \frac{f(z)}{f(z)} = 1.
 \end{align*} По замечанию~\ref{remark:expeq1_implies_poweq2pk}
 \begin{align*}
  g_1(z)-g_2(z) = 2\pi k(z),
 \end{align*} где $ k\colon\,\Omega\to\Z $ --- непрерывная функция. Так как $ \Omega $ --- связное множество, то $ k(z) = k $ --- константа на $ \Omega $.
\end{proof}

По аналитической ветви логарифма всегда можно построить непрерывную ветвь аргумента.

\begin{thm}
 \label{theorem:exist_arg_f}
 Пусть аналитическая в области~$ \Omega \subset \CC $ функция~$ f\colon\,\Omega\to\CC $ обладает аналитической ветвью логарифма $ \log f $. Тогда функция
 \begin{align*}
  \arg f(z) = \Imaginary \log f(z)
 \end{align*} является непрерывной ветвью аргумента $ f $ в области~$ \Omega $.
\end{thm}
\begin{proof}[\normalfont\textsc{Доказательство}]
 В самом деле,
 \begin{align*}
  f(z) = e^{\log f(z)} = e^{\Real \log f(z) + i \Imaginary \log f(z)} = e^{\Real \log f(z)} \cdot e^{i\arg f(z)} = \left| f(z) \right| \cdot e^{i\arg f(z)},
 \end{align*} где равенство
 \begin{align*}
  e^{\Real \log f(z)} = \left| f(z) \right|
 \end{align*} выполнено, так как $ \left| e^{it} \right|=1 $ при $ t \in \R $.
\end{proof}

Соединяя вместе с теоремой~\ref{theorem:exist_log_f}, получаем следующее следствие.

\begin{crly}
 Если функция~$ f\colon\,\Omega\to\CC $ аналитична в односвязной области $ \Omega \subset \CC  $, и $ f \neq 0 $ в $ \Omega $, то существует непрерывная ветвь аргумента $ f $.
\end{crly}

\begin{remrk}
 В ходе доказательства теоремы~\ref{theorem:exist_arg_f} мы проверили, что
 \begin{align*}
  e^{\Real \log f(z)} = \left| f(z) \right| = e^{\ln \left| f(z) \right|},
 \end{align*} где $ \ln x $, $ x > 0 $ --- вещественный логарифм в старом смысле. Тогда
 \begin{align*}
  \Real \log f(z) &= \ln \left| f(z) \right|, \\
  \Imaginary \log f(z) &= \arg f(z),
 \end{align*} или же
 \begin{align*}
  \log f(z) = \ln \left| f(z) \right| + i \cdot \arg f(z).
 \end{align*}
\end{remrk}

\subsection{Примеры применения.}

\begin{df}[степенная функция]
 Пусть функция~$ f $ аналитична в области~$ \Omega $ и обладает аналитической ветвью логарифма~$ \log f $. Пусть $ \alpha\in\CC $. Тогда \emph{степенной функцией с показателем $ \alpha $, отвечающей ветви логарифма $ \log f $} называется аналитичная в $ \Omega $ функция
 \begin{align*}
  f(z)^{\alpha} = e^{\alpha \log f(z)}.
 \end{align*}
\end{df}

В частности, в односвязной области~$ \CC \setminus \left(-\infty, 0\right]$ для любого показателя $ \alpha \in \CC $ определена степенная функция
\begin{align*}
 z^{\alpha} = e^{\alpha \log z},
\end{align*} где $ \log z $ --- любая аналитическая ветвь логарифма тождественной функции $ z \mapsto z $ (которая существует по теореме~\ref{theorem:exist_log_f}). Имеет смысл зафиксировать одну ветвь логарифма для данной ситуации.

\begin{df}
 \textit{Главной ветвью логарифма} $ \log z $ называется такая аналитическая ветвь логарифма тождественной функции~$ z \mapsto z $ в области~$ \CC \setminus (-\infty, 0] $, что $ \log 1 = 0 $.
\end{df}

По теореме~\ref{theorem:exist_log_f} главная ветвь логарифма существует: достаточно взять любую ветвь, и вычесть из неё число $ \log 1 = 2\pi i k $, $ k \in \Z $.

\begin{exmpl}
 Вычислим выражение $ i^{i} $, где степенная функция отвечает главной ветви логарифма:
 \begin{align*}
  i^{i} = e^{i \log i} = e^{i \cdot \left( \Real \log i + \Imaginary \log i \right)} = e^{i \cdot \left( \ln \left| i \right| + i \arg i \right)} = e^{-\arg i} = e^{-\pi / 2}.
 \end{align*}
\end{exmpl}

\newpage
\section{Принцип максимума.}

В этом параграфе мы изучим лишь одну теорему, называемую \emph{принципом максимума}, которая оказывается крайне полезной на практике при работе с аналитическими функциями. Сформулируем её.

\begin{thm}[принцип максимума]
 \label{theorem:maximum_principle}
 Пусть у аналитической в области~$ \Omega $ функции $f$ есть точка максимума $ z_0 \in \Omega $: для всех $ z\in\Omega $ выполнено
 \begin{align*}
  \left| f(z_0) \right| \geqslant \left| f(z) \right|.
 \end{align*} Тогда функция~$ f $ постоянна на $ \Omega $.
\end{thm}

Несмотря на кажущуюся простоту, доказательство принципа максимума достаточно хитрое.

\begin{df}
 Отображение~$ f\colon\,X_1 \to X_2 $ между топологическими пространствами~$ X_1 $ и $ X_2 $ называется \emph{открытым}, если оно переводит открытые подмножества $ X_1 $ в открытые подмножества $ X_2 $.
\end{df}

\begin{thm}
 \label{theorem:analytic_implies_open}
 Непостоянная аналитическая функция~$ f\colon\,\Omega\to\CC $ в области~$ \Omega \subset \CC $ является открытым отображением.
\end{thm}

Оказывается, из открытости аналитических отображений элементарно выводится принцип максимума.

\begin{proof}[\normalfont\textsc{Вывод теоремы~\ref{theorem:maximum_principle} из теоремы~\ref{theorem:analytic_implies_open}}]
 По теореме~\ref{theorem:analytic_implies_open} образ $ f(\Omega) $ --- открытое подмножество $ \CC $. Этот образ содержит точку $ f(z_0) $, но не содержит все такие точки $ w \in \CC $, что $ \left| w \right| > \left| f(z_0) \right| $. Поэтому $ f(z_0) $ не входит в $ f(\Omega) $ вместе со своей окрестностью, и $ f(\Omega) $ не может быть открыто.
\end{proof}

\begin{lm}
 Пусть функция~$ f\colon\,\Omega\to\CC $ аналитична в области~$ \Omega \subset \CC $ и записана в вещественном виде:
 \begin{align*}
  F(x,y) = (u(x,y), v(x,y)), && f(x + iy) = u(x,y) + iv(x,y).
 \end{align*} Тогда якобиан отображения $ F $ равен
 \begin{align*}
  \det J_F (x_0,y_0) = \left| f'(z_0) \right|^{2},
 \end{align*} где $ z_0 = x_0 + iy_0 $.
\end{lm}
\begin{proof}
 С одной стороны, по условию Коши-Римана (теорема~\ref{theorem:cauchy_riman})
 \begin{align*}
  \det J_F = \begin{vmatrix}
   u_x' & u_y' \\
   v_x' & v_y'
  \end{vmatrix} = u_x' \cdot v_y' - u_y' \cdot v_x' = u_x' \cdot u_x' - u_y' \cdot (-u_y') = \left( u_x' \right)^{2} + \left( u_y' \right)^{2}
 \end{align*} С другой стороны,
 \begin{align*}
  f'(z_0) = \lim_{h \to 0}  \frac{f(z_0+h)-f(z_0)}{h} = \lim_{\substack{h \to 0 \\ h \in \R}} \frac{f(z_0+h) - f(z_0)}{h} = \left.\frac{\partial}{\partial x}f(x + iy_0)\right|_{x = x_0},
  \end{align*} поэтому
  \begin{align*}
   \left| f'(z_0) \right|^{2} = \left| u'_x(x_0 + iy_0) + iv'_x(x_0 + iy_0) \right|^2 =  \left( u'_x \right)^{2} + \left( v_x' \right)^{2} = \left( u_x' \right)^{2} + \left( u_y' \right)^{2}
  \end{align*} также по условию Коши-Римана.
 \end{proof}

 \begin{crly}
  \label{corollary:open_map_derivative}
  Пусть функция~$ f $ аналитична в области~$ \Omega \subset \CC $. Если точка~$ z_0 \in \Omega $ такова, что $ f'(z_0) \neq 0 $, то в малой окрестности точки $ z_0 $ $ f $ --- открытое отображение.
 \end{crly}
 \begin{proof}[\normalfont\textsc{Доказательство}]
  Если $ f'(z_0) \neq 0 $, то по непрерывности производной верно
  \begin{align*}
   \det J_F =  \left| f'(z) \right|^{2} \neq 0
  \end{align*} в некоторой окрестности $ U \ni z_0 $. Тогда по теореме об открытости отображения с ненулевым якобианом, отображение $ \left.f\right|_U $ открыто.
  \end{proof}

  \begin{lm}
   \label{lemma:open_map_z_n}
   Для любого натурального $ n \geqslant 1 $ функция $ z^{n} $ --- открытое отображение из $ \CC $ в $ \CC $.
  \end{lm}
  \begin{proof}[\normalfont\textsc{Доказательство}]
   Производная $(z^{n})' = nz^{n-1} $  не обнуляется в области~$ \CC \setminus \left\{ 0 \right\} $. Значит, по следствию~\ref{corollary:open_map_derivative} отображение $ z^{n} $ открыто в области~$ \CC\setminus \left\{ 0 \right\} $. Кроме того,
   \begin{align*}
    f(B(0,r)) = B(0,r^{n})
   \end{align*} --- открытое подмножество $ \CC $, поэтому отображение открыто в $ \CC $.
  \end{proof}

  Наконец мы готовы доказать теорему об открытости непостоянной аналитической функции и завершить обоснование принципа максимума.

  \begin{proof}[\normalfont\textsc{Доказательство теоремы~\ref{theorem:analytic_implies_open}}]
   Пусть $ f \colon\,\Omega\to\CC $ --- непостоянная аналитическая функция. Пусть $ z_0 \in \Omega $ --- любая точка. Наша цель: выразить $ f $ как
   \begin{align}
    \label{eq:desired_composition:analytic_implies_open}
    f(z) = h_3(h_2(h_1(z))),
   \end{align} где функции~$ h_j $, $ j=1,2,3 $ аналитичны в $ \Omega $ и являются открытыми отображениями в окрестности точки~$ z_0 $; тогда и функция~$ f $ будет открытым отображением в окрестности $ z_0 $ как композиция открытых отображений. Так как мы это покажем для любой точки~$ z_0\in\Omega $, то $ f $ будет открытым отображением и в самом $ \Omega $.

   По лемме~\ref{lemma:zero_multiplicity} о кратности нуля существует $ n \geqslant 1 $ такое, что
   \begin{align*}
    f(z) - f(z_0) = (z - z_0)^{n} \cdot P(z),
   \end{align*} где функция~$ P $ аналитична в $ \Omega $, и $ P(z) \neq 0 $. Так как $ P(z_0) \neq 0 $ в малой односвязной окрестности точки $ z_0 $, то по теореме~\ref{theorem:exist_log_f} в этой окрестности существует аналитическая ветвь логарифма $ \log P $. Тогда
   \begin{align*}
    f(z) - f(z_0) = (z - z_0)^{n} \cdot e^{\log P(z)} = \left[ \left( z-z_0 \right) \cdot e^{\frac{\log P(z)}{n}} \right]^{n},
   \end{align*} и, следовательно,
   \begin{align}
    \label{eq:explicit_composition:analytic_implies_open}
    f(z) &= f(z_0) + ((z-z_0) \cdot g(z))^{n},
   \end{align} где функция
   \begin{align*}
    g(z) = e^{\frac{\log P(z)}{n}}
   \end{align*} аналитична в окрестности точки~$ z_0 $, $ n \geqslant 1 $ и $ g(z_0) \neq 0 $. Разложение~\eqref{eq:explicit_composition:analytic_implies_open} имеет в точности вид~\eqref{eq:desired_composition:analytic_implies_open}, где
   \begin{align*}
    h_1(z) &= (z - z_0) \cdot g(z), & h_2(z) &= z^{n}, & h_3(z) &= f(z_0) + z.
   \end{align*} Отображение $ h_3 $ открыто по очевидным причинам, а $ h_2 $ открыто по лемме~\ref{lemma:open_map_z_n}. Осталось понять открытость отображения $ h_1 $. Продифференцируем его:
   \begin{align*}
    h_1'(z) = g(z) + (z - z_0) \cdot g'(z).
   \end{align*} Тогда в точке $ z = z_0 $
   \begin{align*}
    h_1'(z_0) = g(z) \neq 0.
   \end{align*} Раз так, то по следствию~\ref{corollary:open_map_derivative} отображение~$ h_1 $ также открыто в окрестности точки~$ z_0 $.
  \end{proof}

  \end{document}
