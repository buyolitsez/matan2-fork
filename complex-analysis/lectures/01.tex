\documentclass[../complex-analysis.tex]{subfiles}
\begin{document}

В этом семестре мы будем изучать теорию функций комплексного переменного (комплексный анализ). Эта наука изучает не произвольные функции комплексного переменного, а, так называемые, \textit{аналитические} функции (в литературе они также называются \textit{голоморфными} функциями). 

\section{Точные и замкнутые дифференциальные формы.}

\begin{conventn*}
 Начиная с этого параграфа, плоскость $\R^{2}$ мы иногда будем отождествлять с областью комплексных чисел $\CC$ естественным образом. Так, вектор $ \begin{psmallmatrix}
  x \\ y
 \end{psmallmatrix} \in \R^{2}$ отождествляется с комплексным числом $z = x+iy \in \CC$. В каких-то случаях нам будет удобнее говорить о $\R^{2}$, а в других случаях --- о $\CC$. Буквы $ x $, $ y $ и $ z $, использованные выше, обычно будут именно в таких ролях.
\end{conventn*}


Несмотря на то, что предмет изучения комплексного анализа --- аналитические функции, чтобы эффективно и быстро построить основную теорию, неожиданно оказывается, что нужно прибегнуть к дифференциальным формам порядка $1$. Вспомним соответствующие определения из предыдущего семестра.

\subsection{Дифференциальные формы первого порядка.}

В определении дифференциальной формы, которое мы рассмотрим сейчас,  есть два важных отличия от предыдущего семестра. Во-первых, формам теперь разрешено принимать комплексные значения. Во-вторых, мы будем рассматривать формы, действующие только из $\R^{2}$ и имеющие порядок $1$ (но изредка будут появляться формы порядка $ 2 $).

\begin{df}[линейная форма]
 \label{definition:linear_form}
 \textit{Линейной формой} (порядка $1$) будем называть функцию $L \colon\, \R^{2} \to \CC$, линейную по $ \R $, то есть обладающую следующими свойствами.
 \begin{enumerate}
  \item $L(h_1 + h_2) = L(h_1) + L(h_2)$ для всех $h_1, h_2 \in \R^{2}$.
  \item \label{enum2:definition:linear_form} $L(\alpha h) = \alpha \cdot L(h)$ для всех $\alpha \in \R$ и $h \in \R^{2}$.

 \end{enumerate}
\end{df}
\begin{exmpl}
 \label{example:dx_dy_linear_forms}
 Линейные формы $dx, dy \colon\, \R^{2} \to \CC$ определены формулами
 \begin{align*}
  dx \begin{pmatrix}
   x \\ y
   \end{pmatrix} = x, & &dy \begin{pmatrix}
   x \\ y
  \end{pmatrix} = y.
 \end{align*} Легко проверить, что $dx$ и $dy$ удовлетворяют определению \ref{definition:linear_form}.
\end{exmpl}

\begin{remrk}
 Лучше считать, что линейные формы действуют из $\R^{2}$. Конечно, можно считать, что линейная форма действует из $\CC$, но в таком случае возникает следующий нюанс. Условие $\alpha \in \R$ в пункте \ref{enum2:definition:linear_form} становится критическим. Хоть линейная форма и может принимать комплексные значения, и должна быть линейной по $\R$, она совершенно не обязана быть линейной по $\CC$!

 Например, форма $dx$ не линейна по $\CC$: возьмём вектор $h = \begin{psmallmatrix}
  1 \\ 0
 \end{psmallmatrix}$, который отождествлён с комплексным числом $z = 1$ и комплексный коэффициент $\alpha = i$:
 \begin{align*}
  dx(\alpha z) = dx(i) = 0, & &\alpha \cdot dx(z) = i \cdot 1 = i.
 \end{align*} 
\end{remrk}

\begin{remrk}
 \label{remark:linear_forms_is_linear_space}
 Линейные формы порядка $ 1 $ образуют линейное пространство над полем $\CC$, которое мы будем обозначать $\LF$. Разумеется, это даже банахово пространство (то есть линейное нормированное пространство, полное относительно заданной на нём нормы).

 Формы $dx$ и $dy$, определённые в примере \ref{example:dx_dy_linear_forms}, образуют базис в этом пространстве: всякая линейная форма $ L \in \LF $ имеет вид $L = p\,dx + q\,dy$ для некоторых чисел $p,q \in \CC$.
\end{remrk}

\begin{exmpl}
 Ещё одна крайне важная линейная форма --- это \[
  dz = dx + i\,dy.
  \] Форма $ dz \colon\,\R^{2} \to \CC $ по сути задаёт отождествление вещественной плоскости $ \R^{2} $ и комплексной плоскости $ \CC $: вектору $ \begin{psmallmatrix}
  x \\ y
 \end{psmallmatrix} \in \R^{2} $ она сопоставляет комплексное число $ z = x + iy \in \CC $.
\end{exmpl}

\begin{df}[дифференциальная форма]
 \textit{Дифференциальной формой} порядка~$1$ (или \textit{$1$-формой}) на области $ \Omega \subset \CC $ называется отображение $\omega \colon\, \Omega \to \LF$, которое каждой точке $z \in \Omega$ сопоставляет линейную форму $\omega(z) \in \LF$.
\end{df}

\begin{remrk}
 В силу замечания \ref{remark:linear_forms_is_linear_space} всякая $1$-форма $\omega \colon\, \Omega \to \LF$ имеет вид
 \begin{align}
  \label{equation:1-form}
  \omega(z) = P(z)\,dx + Q(z)\,dy
 ,\end{align} где $P, Q \colon\, \Omega \to \CC$ --- функции комплексного переменного.
\end{remrk}

\begin{exmpl*}
 $\omega = x \, dy$ --- $1$-форма на области $ \Omega = \CC $. В точке $2 + i \in \CC$ на векторе $ \begin{psmallmatrix}
  1 \\ 3
 \end{psmallmatrix} \in \R^{2}$ эта форма принимает значение
 \begin{align*}
  \left(\omega (2 + i)\right) \begin{psmallmatrix}
   1 \\ 3
   \end{psmallmatrix}= (2 \, dy) \begin{psmallmatrix}
   1 \\ 3
  \end{psmallmatrix} = 2 \cdot 3 = 6
 .\end{align*} 
\end{exmpl*}

\begin{exmpl*}
 $ \omega=z\,dz $ --- $ 1 $-форма на $ \CC $. Её можно переписать так:
 \begin{align*}
  z\,dz = z \cdot (dx + i\,dy) = z\,dx + iz\,dy.
 \end{align*} В записи \eqref{equation:1-form} имеем  $ P(z)=z $ и $ Q(z)=iz $.
\end{exmpl*}

\begin{exmpl*}
 Обобщая предыдущий пример, если $ f \colon\,\Omega \to \CC $ --- функция, заданная на области $ \Omega \subset \CC $, то $ f\,dz $ --- дифференциальная форма порядка $ 1 $ на $ \Omega $. Раскрывая её, получаем
 \begin{align*}
  f(z)\,dz = f(z)\,dx + if(z)\,dy,
 \end{align*} то есть $ P(z) = f(z) $ и $ Q(z) = if(z) $.

 Очень много дифференциальных форм, которые мы будем встречать, будут иметь вид $ f\,dz $. Однако важно понимать, что на них дифференциальные формы не ограничиваются.
\end{exmpl*}

\begin{df*}
 $1$-форма $\omega \colon\, \Omega \to \LF$ называется \textit{непрерывной}, если функции $P,Q \colon\, \Omega \to \CC$ из представления \eqref{equation:1-form} являются непрерывными на $ \Omega $. Форма $\omega$ называется $C^{k}$-гладкой, если функции $P$ и $Q$ --- $C^{k}$-гладкие (где $k \in \N \cup \left\{ \infty \right\}$).
\end{df*}

$ 1 $-формы у нас всегда будут как минимум непрерывными. В других источниках непрерывность (или даже гладкость!) включается в само определение дифференциальной формы.

Для гладкой $ 1 $-формы есть понятие \emph{дифференциала}.

\begin{df}
 Пусть $ \omega\colon\,\Omega \to \LF $ --- гладкая дифференциальная форма в области~$ \Omega $, записанная в виде
 \begin{align*}
  \omega = P\,dx + Q\,dy,
 \end{align*} где $ P,Q \in C^{1}(\Omega,\CC) $ --- гладкие функции. Тогда \emph{внешним дифференциалом} (или просто \emph{дифференциалом}) формы~$ \omega $ называется $ 2 $-форма
 \begin{align*}
  d\omega = dP \land dx + dQ \land dy,
 \end{align*} где $ \land $ --- оператор внешнего дифференцирования.
\end{df}

Определение дифференциальных форм высших порядков и оператора внешнего дифференцирования можно найти в последнем параграфе конспекта по теории меры (третий семестр).

\subsection{Гладкие пути в \texorpdfstring{$\CC$}{комплексной плоскости}.}

Мы вспомнили определение дифференциальных форм порядка $1$. Теперь вспомним определение объектов, по которым мы умеем интегрировать такие формы. Эти объекты --- поверхности размерности $1$ в $\CC$, то есть \textit{пути}.

\begin{df}[путь]
 \emph{Путём} в $\CC$ называется непрерывная функция $\gamma \colon [a,b] \to \CC$. Путь $\gamma$ называется \textit{гладким}, если $\gamma \in C^{1}([a,b],\CC)$ --- гладкая функция. Путь $\gamma$ называется \textit{кусочно-гладким}, если $\gamma$ --- кусочно-гладкая функция, то есть $\gamma = \gamma_1 + \ldots + \gamma_n$, где $\gamma_i$ --- гладкие пути.
\end{df}

\begin{df}[сумма путей]
 Если $\gamma_1 \colon [a,b] \to \CC$ и $\gamma_2 \colon [b,c] \to \CC$ --- пути, то \textit{суммой путей $\gamma_1$ и $\gamma_2$} называется путь $(\gamma_1 + \gamma_2)\colon\, [a,c] \to \CC$, определённый по формуле
 \begin{align*}
  (\gamma_1 + \gamma_2)(t) = \begin{cases}
   \gamma_1(t), \text{ если } t \in [a,b],  \\
   \gamma_2(t), \text{ если } t \in [b,c].
  \end{cases} 
 \end{align*} Сумма путей определена при условии $\gamma_1(b) = \gamma_2(b)$.
\end{df}

\begin{notatn}
 Если $\gamma \colon\, [a,b] \to \CC$ --- путь, то путь $-\gamma \colon [a,b] \to \CC$ определяется по формуле:
 \begin{align*}
  -\gamma(t) = \gamma(b - t + a).
 \end{align*} Это тот же самый путь $ \gamma $, но проходимый в обратном направлении.
\end{notatn}

\begin{df}
 Путь $ \gamma \colon\, [a,b]\to\CC $ называется \emph{петлёй} (или \emph{замкнутым путём}), если его начало совпадает с его концом: $ \gamma(a)=\gamma(b) $.
\end{df}

\begin{prop}
 \label{prop:smooth_path_existance}
 Пусть в области $ \Omega \subset \CC $ существует путь между двумя точками $ p, q\in\Omega $. Тогда между точками $ p $ и $ q $ существует путь, представляющий из себя ломанную линию (в частности, этот путь кусочно-гладкий).
\end{prop}
\begin{proof}[\normalfont\textsc{Доказательство}]
 Пусть существует путь $ \gamma\colon[a,b]\to\Omega $, $ \gamma(a) = p $, $ \gamma(b)=q $. Так как $ \Omega $ --- открытое подмножество $ \CC $, то для каждой точки $ \gamma(t) \in \Omega $, $ t \in [a,b] $, существует открытый шар $ B_t $ с центром в точке $ t $. Так как образ $ \gamma([a,b]) $ --- компактное множество (как непрерывный образ компакта), то из открытого покрытия $ \left\{B_t\right\}_{t \in [a,b]} $  можно выделить конечное подпокрытие $ B_{t_1}, \ldots, B_{t_N} $. Пусть $ r $ --- наименьший из радиусов шаров этого конечного подпокрытия. Тогда можно построить ломаную линию, соединяющую точки $ a $ и $ b $, звенья которой будут иметь длину меньше $ r $.
\end{proof}

\subsection{Интегрирование 1-форм по путям.}

Теперь вспомним (и обобщим) понятие интеграла $1$-формы по пути в $\CC$.

\begin{df}[интеграл $1$-формы по пути]
 Пусть $\Omega \subset \CC$ --- область, $\gamma \colon [a,b] \to \Omega$ --- кусочно-гладкий путь, а $\omega \colon\, \Omega \to \LF$ --- непрерывная $1$-форма на $ \Omega $, записанная в виде
 \begin{align*}
  \omega(z) = P(z)\,dx + Q(z)\,dy.
 \end{align*} Тогда \textit{интегралом} от $1$-формы $\omega$ по пути  $\gamma$ называется число
 \begin{align}
  \label{eq:definition:integral_of_1_form_on_path}
  \int_{\gamma} \omega = \int_{a}^{b} \left( P(\gamma(t)) \cdot \gamma_1'(t) + Q(\gamma(t)) \cdot \gamma_2'(t) \right) dt,
 \end{align} где $\gamma(t) = \gamma_1(t) + i \gamma_2(t)$.
\end{df}
\begin{remrk*}
 Определение интеграла \eqref{eq:definition:integral_of_1_form_on_path} можно <<вывести>>, если записать $ 1 $-форму в виде \eqref{equation:1-form} и применить замену переменной $ z = \gamma(t) $:
 \begin{align*}
  \int_{\gamma} \omega &= \int_{\gamma} \left( P(z)\,dx + Q(z)\,dy \right) = \int_{a}^{b} (P(\gamma(t)) \, d (\gamma_1(t)) + Q(\gamma(t)) \, d(\gamma_2(t))) = \\
  &= \int_{a}^{b} \left( P(\gamma(t)) \cdot \gamma_1'(t) + Q(\gamma(t)) \cdot \gamma_2'(t)\right)dt.
 \end{align*}
\end{remrk*}

Для интеграла от $ 1 $-форм по гладкому пути верны всевозможные свойства, непосредственно следующие из соответствующих свойств интеграла Лебега: линейность, аддитивность, монотонность и прочие, а также верны более сложные результаты, например, теорема Фубини. Мы не будем сейчас проверять их все (они легко выводятся из соответствующих результатов для интеграла Лебега), ограничимся пока лишь полезной оценкой модуля интеграла.

\begin{lm}
 \begin{align}
  \label{eq:bound_on_absolute_value_of_int}
  \left| \int_{\gamma} \omega  \right| \leqslant l(\gamma) \cdot \max_{t \in [a,b]} \sqrt{\left| P(\gamma(t)) \right|^{2} + \left| Q(\gamma(t)) \right|^{2}},
 \end{align} где $l(\gamma)$ --- длина кусочно-гладкого пути $\gamma$.
\end{lm}
\begin{proof}
 Воспользуемся аналогичной оценкой для интеграла Лебега, а также неравенством Коши-Буняковского-Шварца для пространства $\CC^{2}$:
 \begin{align*}
  \left| \int_{\gamma} \omega  \right| &\leqslant \int_{a}^{b} \left| P(\gamma(t))\cdot\gamma_1'(t) + Q(\gamma(t))\cdot\gamma_2'(t) \right|dt \leqslant \\
  &\leqslant \int_{a}^{b} \sqrt{\left| P(\gamma(t)) \right|^{2} + \left| Q(\gamma(t)) \right|^{2}} \cdot \sqrt{\left| \gamma_1'(t) \right|^{2} + \left| \gamma_2'(t) \right|^{2}} dt \leqslant \\
  &\leqslant \max_{t \in [a,b]} \sqrt{\left| P(\gamma(t)) \right|^{2} + \left| Q(\gamma(t)) \right|^{2}} \cdot \int_{a}^{b} \left| \gamma'(t) \right|  dt
 .\end{align*} Интеграл $\int_{a}^{b} \left| \gamma'(t) \right|dt$ и есть длина гладкого пути $l(\gamma)$ (как мы выяснили во втором семестре матана). Неравенство КБШ было применено к векторам
 \begin{align*}
  \begin{pmatrix}
   P(\gamma(t)) \\ Q(\gamma(t))
   \end{pmatrix}, \begin{pmatrix}
   \gamma_1'(t) \\ \gamma_2'(t)
  \end{pmatrix} \in \CC^{2}.
 \end{align*}
\end{proof}

Докажем ещё одно простое, но часто используемое равенство.

\begin{prop}
 \begin{align*}
  \int_{-\gamma} \omega = -\int_{\gamma} \omega.  
 \end{align*}
\end{prop}
\begin{proof}[\normalfont\textsc{Доказательство}]
 \begin{align*}
  \int_{-\gamma} \omega &= \int_{a}^{b} \left( P(-\gamma(t)) \cdot (-\gamma)_1'(t)  + Q(-\gamma(t)) \cdot (-\gamma)_2'(t) \right) dt = \\
  &= \begin{bmatrix}
   u = b - t + a, & du = -dt
  \end{bmatrix} = \\
  &= \int_{a}^{b} \left( P(\gamma(t)) \cdot \gamma_1'(u) + Q(\gamma(u)) \cdot \gamma_2'(t) \right)(-du) = - \int_{\gamma} \omega. 
 \end{align*}
\end{proof}

\begin{conventn}
 Если $ \omega = f\,dz $, то интеграл
 \begin{align*}
  \int_{\gamma} f\,dz 
 \end{align*} попросту называют \emph{интегралом функции $ f $ по пути $ \gamma $}.
\end{conventn}

\subsection{Гомотопии путей.}

Дадим новое важное топологическое определение, которое часто применяется при работе с путями.

\begin{df}[гомотопные пути]
 \label{definition:path-homotopy}
 Два пути $ \gamma_0,\gamma_1\colon [a,b] \to \Omega$ \textit{гомотопны} в области $ \Omega \subset \CC $, если $ \gamma_0 $ можно перевести в $ \gamma_1 $ <<непрерывным образом>>: существует непрерывное отображение $ H \colon [0,1] \times [a,b] \to \Omega $, $ H = H(s,t) $, называемое \textit{гомотопией}, которое переводит $ \gamma_0 $ в $ \gamma_1 $:
 \begin{align*}
  H(0, t) &= \gamma_0(t), & H(1,t)&=\gamma_1(t)
 \end{align*} а также сохраняет концы путей:
 \begin{align*}
  H(s, a) &= \gamma_0(a) = \gamma_1(a), & H(s,b) &= \gamma_0(b) = \gamma_1(b).
 \end{align*} В частности, пути могут быть гомотопны, только если их концы совпадают.
\end{df}

\begin{figure}[ht]
 \centering
 \incfig[0.5]{path-homptopy}
 \caption{Гомотопные пути.}
 \label{fig:path-homptopy}
\end{figure}

\begin{remrk*}
 Легко видеть, что гомотопия путей --- это отношение эквивалентности. Путь гомотопен самому себе; если $ \gamma_0 $ гомотопен $ \gamma_1 $, то и $ \gamma_1 $ гомотопен $ \gamma_0 $; если $ \gamma_0 $ гомотопен $ \gamma_1 $, а $ \gamma_1 $ гомотопен $ \gamma_2 $, то и $ \gamma_0 $ гомотопен $ \gamma_2 $.
\end{remrk*}

\begin{exmpl}
 \label{example:non_homotopy_paths}
 Одна и та же пара путей может быть гомотопна в одной области, но не гомотопна в другой области --- область $\Omega$ в определении \ref{definition:path-homotopy} важна.

 Рассмотрим рисунок \ref{fig:paths-without-homptopy}. Путь $\gamma_0$ --- верхняя полуокружность, направленная по часовой стрелке, а путь $\gamma_1$ ---  нижняя полуокружность, направленная против часовой стрелки. Пути $\gamma_0$ и $\gamma_1$ гомотопны в области $\CC$, но не гомотопны в области $\CC \setminus \left\{ 0 \right\}$.

 Интуитивное объяснение негомотопности такое: мы не можем непрерывно перевести $\gamma_0$ в $\gamma_1$, сохраняя концы, миновав выколотую точку $0$. Формальное доказательство давать не будем (если интересно доказательство, ключевое слово --- фундаментальная группа единичной окружности).
\end{exmpl}

\begin{figure}[ht]
 \centering
 \incfig[0.7]{paths-without-homptopy}
 \caption{Пути гомотопны в $\CC$ (слева), но не гомотопны в $\CC \setminus \left\{ 0 \right\}$ (справа).}
 \label{fig:paths-without-homptopy}
\end{figure}

Неудобное обстоятельство состоит в том, что интеграл формы определён только для кусочно-гладких путей. В дальнейшем мы захотим интегрировать по промежуточным путям гомотопии, и даже если пути $ \gamma_0 $ и $ \gamma_1 $ являются кусочно-гладкими, то априори никто не гарантировал, что промежуточные пути $ H(s, \cdot) $ будут также кусочно-гладкими. Но оказывается, для гомотопных путей всегда можно подобрать <<хорошую>> гомотопию.

\begin{prop}
 \label{prop:piecewise_linear_homotopy}
 Пусть пути $ \gamma_0, \gamma_1 \colon\,[a,b] \to \Omega $ гомотопны в области~$ \Omega \subset \CC $. Тогда для них существует такая гомотопия~$ H \colon [0,1] \times [a,b] \to \Omega $, что все пути вида $ \gamma_s \colon\, [a,b] \to \Omega $, $ \gamma_s(t) = H(s,t) $ и вида $ \eta_t \colon\, [0,1] \to \Omega $, $ \eta_t(s) = H(s,t) $ являются ломаными линиями.
\end{prop}

Доказательство предложения~\ref{prop:piecewise_linear_homotopy} оставлена в качестве упражнения.

\subsection{Односвязные области.}

Мы уже знаем, что если пути гомотопны, то их начало и конец обязаны совпадать, но обратное, как показывает пример \ref{example:non_homotopy_paths}, в общем случае не верно. Тем не менее, во многих областях обратное следование всё же верно, и для этих областей есть специальное название.

\begin{df}[односвязная область]
 Область $ \Omega \subset\CC $ называется \emph{односвязной}, если любые два пути в $ \Omega $, имеющие одинаковые начало и конец, гомотопны.
\end{df}

\begin{claim}
 Область $ \Omega \subset \CC $ односвязна тогда и только тогда, когда любую петлю $ \gamma \colon\,[a,b] \to \Omega $ <<можно стянуть в точку>>, то есть путь $ \gamma $ гомотопен постоянному пути $ \gamma_0 \colon\,[a,b] \to \Omega $, $ \gamma_0(t) = \gamma(a) $ в области $ \Omega $.
\end{claim}

Это утверждение было задачей в листочке.

\begin{proof}
 Следование из односвязности очевидно: петля и точка --- это два пути, имеющие одинаковые начало и конец, поэтому они должны быть гомотопны. Поэтому, нам нужно лишь доказать следование в обратную сторону.

 Пусть в области $ \Omega $ можно стянуть всякую петлю. Пусть теперь $ \gamma_0, \gamma_1 \colon\, [0,1] \to \Omega $ --- два пути с совпадающими началом $ A = \gamma_0(0) = \gamma_1(0) $ и концом $ B = \gamma_0(1) = \gamma_1(1) $.

 Пусть $ \eta = \gamma_0 + (-\gamma_1)$, $ \eta \colon\, [0,1] \to \Omega $ --- петля, образованная этими двумя путями (сначала проходим по $ \gamma_0 $ в прямую сторону, а затем по $ \gamma_1 $ в обратную сторону). По предположению петлю $ \eta $ можно стянуть в точку $ A $ гомотопией $ H \colon\, [0,1] \times [0,1] \to \Omega $.

 Посмотрим на путь $ \xi $, который проходит точка $ B $ под действием гомотопии $ H $:
 \begin{align*}
  \xi(t) = H(t, 1 / 2),
 \end{align*} смотрите рисунок \ref{fig:central-path}.
 \begin{figure}[ht]
  \centering
  \incfig[0.7]{central-path}
  \caption{Путь $ \xi $, проходимой точкой $ B $ под действием гомотопии $ H $.}
  \label{fig:central-path}
 \end{figure}

 Утверждается, что в путь $ -\xi $ можно перевести оба пути $ \gamma_0 $ и $ \gamma_1 $ гомотопией. Построим гомотопию $ \hat H \colon [0,1] \times [0,1] \to \Omega $, которая переводит $ \gamma_0 $ в $ -\xi $ (гомотопия для $ \gamma_1 $ построится так же). Пусть $ t \in [0, 1] $. Путь $ \hat H(t) \colon\, [0,1] \to \Omega $ будет устроен следующим образом: сначала мы пройдём по петле $ H(t) $ до точки $ C_t = H(t, 1 / 2) $, которая лежит на пути $ -\xi $. Затем мы пройдём кусок пути $ -\xi $ от точки $ C_t $ до точки $ B $. Для понимания смотрите рисунок \ref{fig:homotopy_to_central_path}.

 \begin{figure}[ht]
  \centering
  \incfig[0.9]{homotopy_to_central_path}
  \caption{Гомотопия $ \hat H $ пути $ \gamma_0 $ и $ -\xi $: слой $ \hat H(t) $.}
  \label{fig:homotopy_to_central_path}
 \end{figure}

 Более формально, путь $ \hat H(t) $ при $0 \leqslant t \leqslant 1$ задаётся по формуле:
 \begin{align*}
  \hat H(t, u) = \begin{cases}
   H(t, u), \text{ если } 0 \leqslant u \leqslant 1 / 2, \\
   H(t(1 - (u - 1 / 2)), 1 / 2), \text{ если } 1 / 2 \leqslant u \leqslant 1.
  \end{cases} 
 \end{align*}

 Так как областей значений $ \hat H $ является подмножеством области значений $ H$, то $ \hat H $ --- действительно гомотопия пути $ \gamma_0 $ и $ -\xi $.

\end{proof}

Возникает естественный вопрос: а бывают ли вообще односвязные области? Например, область $ \CC $ односвязна? Оказывается, что да.

\begin{prop}
 Выпуклые области односвязны.
\end{prop}
\begin{proof}[\normalfont\textsc{Доказательство}]
 Пусть $ \Omega \subset \CC $ --- выпуклая область, и два пути $ \gamma_0,\gamma_1\colon\,[a,b] \to \Omega $ имеют одинаковые начало и конец. Построим их гомотопию $ H \colon[0,1] \times [a,b]\to\Omega $ в явном виде:
 \begin{align*}
  H(s,t) = (1 - s) \cdot \gamma_0(t) + s \cdot \gamma_1(t).
 \end{align*} Ясно, что по построению отображение $ H $ непрерывно, и оно переводит $ \gamma_0 $ в $ \gamma_1 $, сохраняя начало и конец. По выпуклости области $ \Omega $ образ $ H $ не выходит за пределы $ \Omega $, поэтому пути действительно гомотопны.
\end{proof}

\subsection{Точные формы.}

Вспомним определение \textit{точных} дифференциальных форм --- аналога неопределённого интеграла, что мы уже рассматривали в прошлом семестре.

\begin{df}[%
 точная $1$-форма]
 $1$-форма $\omega \colon\, \Omega \to \LF$ называется \textit{точной} в области $\Omega \subset \CC$, если существует гладкая функция $F \colon\, \Omega \to \CC$ ($0$-форма), дифференциал которой совпадает с формой $ \omega $:
 \begin{align*}
  \omega = dF = F_x'\,dx + F_y'\,dy
 \end{align*} при всех $ z \in \Omega $. Функция $ F $ называется \textit{первообразной} формы $ \omega $.
\end{df}

У точности дифференциальной формы есть следующий нетривиальный критерий.

\begin{thm}[критерий точности формы]
 \label{theorem:exact_1_form}
 Пусть $\omega$ --- непрерывная $1$-форма в области $\Omega \subset \CC$. Тогда следующие условия равносильны.
 \begin{enumerate}
  \item \label{enum1:theorem:exact_1_form} Форма $\omega$ точна в $\Omega$.
  \item \label{enum2:theorem:exact_1_form} Для любых кусочно-гладких путей $\gamma_0, \gamma_1 \colon [a,b] \to \Omega$ с совпадающими началом и концом \textup{(}$\gamma_0(a)=\gamma_1(a)$ и $\gamma_0(b)=\gamma_1(b)$\textup{)}, интеграл формы $ \omega $ по этим двум путям совпадает:
   \begin{align}
    \label{eq:exact_1_form:integral_equals}
    \int_{\gamma_0}  \omega = \int_{\gamma_1} \omega.
   \end{align} То есть, интеграл формы $ \omega $ по пути зависит лишь от начала и от конца пути.
 \end{enumerate}
\end{thm}
\begin{proof}\
 \begin{itemize}
  \item Из пункта \ref{enum1:theorem:exact_1_form} следует пункт \ref{enum2:theorem:exact_1_form}. Пусть существует гладкая функция $F \colon\,\Omega \to \CC$ такая, что
   \begin{align*}
    \omega = d F = F'_x \, dx + F'_y \, dy
   .\end{align*} Возьмём любой кусочно-гладкий путь $\gamma \colon [a,b] \to \Omega$ и проинтегрируем по нему форму $\omega$, используя формулу Ньютона-Лейбница:
   \begin{align*}
    \int_{\gamma} \omega &= \int_{\gamma} \left( F_x'\,dx + F_y'\,dy \right) = \int_{a}^{b} \left( F'_x(\gamma(t)) \cdot \gamma_1'(t)  + F'_y(\gamma(t)) \cdot\gamma_2'(t) \right) dt = \\
    &= \int_{a}^{b} \left[ F(\gamma(t)) \right]'_t dt = F(\gamma(b)) - F(\gamma(a))
   .\end{align*} Таким образом, интеграл зависит лишь от начала и конца пути.

  \item Из пункта \ref{enum2:theorem:exact_1_form} следует пункт \ref{enum1:theorem:exact_1_form}. Имея знание о выполнении условия из пункта~\ref{enum2:theorem:exact_1_form}, мы красивым образом сконструируем первообразную~$F$.

   Выберем произвольную точку~$p_0 \in \Omega$. По линейной связности области $ \Omega $ для любой точки~$ p \in \Omega $ существует путь $ \gamma(p) $, соединяющий точку~$ p_0 $ с точкой~$ p $, а по предложению~\ref{prop:smooth_path_existance} можно считать, что этот путь кусочно-гладкий. Тогда для любой точки~$ p \in \Omega $ определим
   \begin{align*}
    F(p) = \int_{\gamma(p)}  \omega.
   \end{align*} Это определение корректно, так как по пункту~\ref{enum2:theorem:exact_1_form} интеграл не зависит от выбора пути~$ \gamma(p) $, лишь бы только этот путь был.

   Проверим, что построенная функция~$F$ действительно является первообразной формы~$ \omega $. Возьмём произвольную точку~$p \in \Omega$, и рассмотрим малое приращение $h \in \CC$, $h \to 0$:
   \begin{align*}
    &F(p + h) - F(p) = \int_{\gamma(p + h)}  \omega - \int_{\gamma(p)}  \omega = \int_{\gamma(p) + [p,p+h]} \omega - \int_{\gamma(p)}   \omega = \int_{[p,p+h]} \omega.
   \end{align*} Здесь $[p,p+h] \subset \Omega$ --- отрезок, соединяющий точки $p$ и $p + h$, рассматриваемый как путь в $\Omega$. Этот отрезок параметризуется функцией
   \begin{align*}
    \gamma(t) = p + th, \qquad t \in [0,1].
   \end{align*} Производная этой функции равна
   \begin{align*}
    \gamma'(t) = (\gamma'_1(t), \gamma'_2(t)) = (h_1, h_2),
   \end{align*} где $h = h_1 + i h_2$, $ h_1,h_2\in\R $. Запишем форму в виде $\omega = P(z)\,dx + Q(z)\,dy$, где $P,Q$ --- непрерывные функции. Подставим всё это обратно:
   \begin{align*}
    F(p+h)-F(p) &= \int_{0}^{1} \left( P(p+th) \cdot h_1 + Q(p+th)\cdot h_2 \right)dt = \\
    &=  \int_{0}^{1} \left( (P(p) + o(1))\cdot h_1 + (Q(p) + o(1))\cdot h_2 \right)dt = \\
    &= \int_{0}^{1} \left( P(p) \cdot h_1 + Q(p)\cdot h_2 + o(h) \right)dt = \\
    &= P(p) \cdot h_1 + Q(p)\cdot h_2 + o(h).
   \end{align*} Получается, функция~$F$ по определению дифференцируема в точке~$p$, и её дифференциал на приращении~$h$ равен
   \begin{align*}
    d_p F \begin{pmatrix}
     h_1 \\ h_2
     \end{pmatrix} = P(p)\cdot  h_1 + Q(p)\cdot h_2 = \left( \omega(p) \right) \begin{pmatrix}
     h_1 \\ h_2
    \end{pmatrix}
   ,\end{align*} что и требовалось доказать.
 \end{itemize}
\end{proof}

\begin{remrk}
 \label{remark:integral_of_exact_form}
 В процессе доказательства теоремы~\ref{theorem:exact_1_form} мы вывели формулу интеграла точной формы $ \omega $ по гладкому пути $ \gamma \colon\,[a,b] \to \Omega $:
 \begin{align}
  \label{eq:integral_of_exact_form}
  \int_{\gamma} \omega =  F(\gamma(b)) - F(\gamma(a)),
 \end{align} где $ F $ --- первообразная формы~$ \omega $. В частности,
 \begin{align*}
  \int_{\gamma} \omega = 0, 
 \end{align*} если $ \gamma $ --- петля.
\end{remrk}

\subsection{Замкнутые формы.}

Теперь введём понятие \textit{замкнутой} дифференциальной формы, которое имеет неразрывную связь с точностью формы.

\begin{df}[замкнутая $1$-форма]
 $1$-форма $\omega \colon\,\Omega\to\LF$ называется \textit{замкнутой} в области $\Omega$, если она <<локально точна>>: для любой точки $p \in \Omega$ существует окрестность $U_p \ni p$, $U_p \subset \Omega$ такая, что форма~$\omega$ точна в $U_p$.
\end{df}

Как и для точных форм, есть следующие нетривиальные критерии замкнутости формы.

\begin{thm}[критерии замкнутости формы]
 \label{theorem:closed_1_form}
 Пусть $\omega$ --- непрерывная $1$-форма в области $\Omega \subset \CC$. Тогда следующие условия равносильны.
 \begin{enumerate}
  \item \label{enum1:theorem:closed_1_form} Форма~$\omega$ замкнута в области~$\Omega$.
  \item \label{enum2:theorem:closed_1_form} Если два кусочно-гладких пути~$\gamma_0, \gamma_1 \colon [0,1] \to \Omega$ гомотопны в области~$\Omega$, то интеграл формы~$ \omega $ по ним совпадает:
   \begin{align*}
    \int_{\gamma_0} \omega = \int_{\gamma_1} \omega.
   \end{align*}
  \item \label{enum3:theorem:closed_1_form} Пусть $\Pi$ --- замкнутый прямоугольник, полностью лежащий в области~$\Omega$, стороны которого параллельны осям координат $ x $ и $ y $. Тогда
   \begin{align*}
    \int_{\partial \Pi} \omega = 0,
   \end{align*} где $ \partial\Pi $ --- граница прямоугольника, рассматриваемая как кусочно-гладкий путь. Для понимания см. рисунок~\ref{fig:closed-rectangle-and-its-border}.
   \begin{figure}[ht]
    \centering
    \incfig[0.5]{closed-rectangle-and-its-border}
    \caption{Замкнутый прямоугольник в области и его граница.}
    \label{fig:closed-rectangle-and-its-border}
   \end{figure}
 \end{enumerate}
\end{thm}
\begin{remrk*}
 Условие~\ref{enum3:theorem:closed_1_form} из теоремы~\ref{theorem:closed_1_form} часто применяется как достаточно простой признак замкнутости дифференциальной формы. Мы иногда будем называть его <<тестом на прямоугольнике>>.
\end{remrk*}
\begin{proof}[\normalfont{\textsc{Доказательство теоремы \ref{theorem:closed_1_form}}}]\
 \begin{itemize}
  \item Из пункта~\ref{enum1:theorem:closed_1_form} следует пункт~\ref{enum2:theorem:closed_1_form}. Это самая сложная часть теоремы. Основная идея её доказательства: всё  локализовать и свести к теореме~\ref{theorem:exact_1_form} о точных формах.

   Пусть форма~$\omega$ замкнута в области~$\Omega$. Пусть кусочно-гладкие пути~$\gamma_0$, $\gamma_1$ гомотопны в $\Omega$, а $H \colon [0,1] \times [0,1] \to \Omega$ --- их гомотопия. По предложению~\ref{prop:piecewise_linear_homotopy} можно считать, что гомотопия~$ H $ <<сохраняет кусочно-гладкость>>, то есть все пути вида $ H(s, \cdot) $ и $ H(\cdot, t) $ являются кусочно-гладкими.

   Заметим, что образ
   \begin{align*}
    H([0,1] \times [0,1]) = K
   \end{align*}  является компактным как непрерывный образ компакта.

   Мы знаем, что форма~$ \omega $ локально точна на множестве~$ K $. Однако, пользуясь компактностью $ K $ это условие можно усилить, добавив <<равномерность>>.
   \begin{prop}
    \label{prop:eps_0:theorem:closed_1_form}
    Существует число $\eps_0 > 0$ такое, что для любой точки $p \in K$ форма $\omega$ точна в шаре $B(p,\eps_0) \subset \Omega$. 
   \end{prop}
   \begin{proof}[\normalfont\textsc{Доказательство}]
    Действительно, так как $\omega$ замкнута, то для любой точки $p \in K$ существует число $\eps(p) > 0$ такое, что $\omega$ точна в шаре $B(p,2\eps(p))\subset\Omega$. Семейство открытых шаров $\{B(p,\eps(p))\}_{p \in K}$ образует открытое покрытие компакта $K$:  выделим из него конечное подпокрытие $\left\{ B(p_k, \eps(p_k)) \right\}_{k=1}^{N}$, и возьмём число $\eps_0 = \min \left\{ \eps(p_1), \ldots, \eps(p_N) \right\}$.

    Проверим, что $\eps_0$ подходит: пусть $p \in K$ --- любая точка. Тогда $p \in B(p_k,\eps(p_k))$ для некоторого $k$. В таком случае по неравенству треугольника
    \begin{align*}
     B(p, \eps_0)\subset B(p_k,2\eps(p_k)).
    \end{align*} Тогда форма~$\omega$ уж тем более точна в $B(p,\eps_0)$.
   \end{proof}

   По теореме Кантора гомотопия~$H$ равномерно непрерывна на компакте~$[0,1]\times[0,1]$: для любого $\eps > 0$ существует число~$\delta(\eps) > 0$, такое, что для любых двух точек $w_1,w_2 \in [0,1] \times [0,1]$ условие $\left\| w_1 - w_2 \right\| < \delta(\eps)$ влечёт
   \begin{align*}
    \left| H(w_1)-H(w_2) \right| < \eps.
   \end{align*}

   Мы теперь готовы к локализации. Возьмём достаточно большое натуральное число $ N \geqslant 1 $, которое мы подберём позже. Для каждого $ k = 0, 1, \ldots, N $ рассмотрим путь
   \begin{align*}
    \gamma_{\frac{k}{N}}(t) = H\left(\frac{k}{N}, t\right), \quad t \in [0,1],
   \end{align*} который к тому же кусочно-гладкий по выбору гомотопии~$ H $. Покажем для всех $ k = 0,1,\ldots,N-1 $ равенство
   \begin{align}
    \label{equation:small_gamma_step:theorem:closed_1_form}
    \int_{\gamma_{\frac{k}{N}}}  \omega = \int_{\gamma_{\frac{k+1}{N}}}  \omega.
   \end{align} Ясно, что из \eqref{equation:small_gamma_step:theorem:closed_1_form} всё последует, ведь тогда можно будет построить цепочку равенств
   \begin{align*}
    \int_{\gamma_0}\omega = \int_{\gamma_{\frac{1}{N}}} \omega = \ldots = \int_{\gamma_{\frac{N-1}{N}}} \omega  = \int_{\gamma_{1}}\omega.
   \end{align*} Итак, зафиксируем число~$k$ и обозначим $\eta = \gamma_{\frac{k}{N}}$  и $\rho = \gamma_{\frac{k+1}{N}}$. Продолжим локализацию. Разобьём оба пути на маленькие кусочки:
   \begin{align*}
    \eta = \eta_0 + \eta_1 + \ldots + \eta_{N-1}, & &\rho = \rho_0 + \rho_1 + \ldots + \rho_{N-1},
   \end{align*} где
   \begin{align*}
    \eta_j = \eta \rvert_{\left[\frac{j}{N},\frac{j+1}{N}\right]  }, & &\rho_j = \rho \rvert_{\left[\frac{j}{N},\frac{j+1}{N}\right]  }.
   \end{align*} Кроме того, для каждого $ j = 0,1,\ldots,N $ рассмотрим путь
   \begin{align*}
    \xi_j(t) = H \left( \frac{k + t}{N}, \frac{j}{N} \right), \quad t \in [0,1],
   \end{align*} который также кусочно-гладкий из-за свойства гомотопии~$ H $. Наконец, для каждого $ j = 0,\ldots,N-1 $ рассмотрим замкнутый контур
   \begin{align*}
    C_j = \xi_j+\eta_j+(-\xi_{j+1})+(-\rho_j),
   \end{align*} изображённый на рисунке~\ref{fig:closed_1_form_small_contur}.

   \begin{figure}[ht]
    \centering
    \incfig{closed_1_form_small_contur}
    \caption{Малый контур $C_j = \xi_j+\eta_j + (-\xi_{j+1}) + (-\rho_j)$.}
    \label{fig:closed_1_form_small_contur}
   \end{figure}

   Докажем теперь для каждого контура $C_j$ равенство
   \begin{align}
    \label{eq:int_on_C_j_eq_0:theorem:closed_1_form}
    \int_{C_j} \omega = 0. 
   \end{align} Из равенств~\eqref{eq:int_on_C_j_eq_0:theorem:closed_1_form} последует равенство~\eqref{equation:small_gamma_step:theorem:closed_1_form} для данного $ k $. Действительно, просуммируем \eqref{eq:int_on_C_j_eq_0:theorem:closed_1_form} по всем $ j $ и воспользуемся аддитивностью интеграла:
   \begin{align*}
    \sum_{j=0}^{N-1} \int_{C_j} \omega = \int_{\eta + (-\rho)} \omega = 0 \implies \int_{\eta} \omega = \int_{\rho} \omega.  
   \end{align*} Здесь важно отметить, что интегралы по всем $ \xi_j $, $ j=1,\ldots,N-1 $ сокращаются, а интегралы по $ \xi_0 $ и $ \xi_N $ и так равны нулю, так как эти пути постоянные.

   Итак, докажем \eqref{eq:int_on_C_j_eq_0:theorem:closed_1_form}. За точку~$P$ обозначим начало пути~$\xi_j$:
   \begin{align*}
    P = H \left( \frac{k}{N}, \frac{j}{N} \right).
   \end{align*}Докажем, что число~$ N $ можно подобрать так, чтобы выполнилось $ C_j \subset B(P,\eps_0) $, где $ \eps_0 $ взято из предложения~\ref{prop:eps_0:theorem:closed_1_form}. Действительно, любая точка~$ Q $ из контура~$ C_j $ имеет прообраз --- точку $ (q_1, q_2) $, такую что  $ Q = H(q_1, q_2) $, причём по построению контура~$ C_j $ выполнено
   \begin{align*}
    \frac{k}{N} \leqslant q_1 \leqslant \frac{k+1}{N}, && \frac{j}{N} \leqslant q_2 \leqslant \frac{j+1}{N}.
   \end{align*} Тогда расстояние между прообразами $ P $ и $ Q $ не превосходит $ \frac{\sqrt 2}{N} $. Если выбрать $ N $ достаточно большим, чтобы было выполнено $ \frac{\sqrt 2}{N} < \delta(\eps_0) $, то по равномерной непрерывности гомотопии~$ H $ получаем
   \begin{align*}
    \left| P - Q \right| < \eps_0,
   \end{align*} что доказывает включение $C_j \subset B(P,\eps_0)$.

   Теперь, так как по предложению~\ref{prop:eps_0:theorem:closed_1_form} форма~$\omega$ точна в шаре~$B(P,\eps_0)$, содержащем в себе контур~$ C_j $, то
   \begin{align*}
    \int_{C_j} \omega = 0.
   \end{align*} Равенство~\eqref{eq:int_on_C_j_eq_0:theorem:closed_1_form} доказано.

  \item Из пункта~\ref{enum2:theorem:closed_1_form} следует пункт~\ref{enum3:theorem:closed_1_form}. Действительно, так как весь прямоугольник~$\Pi$ лежит в области~$ \Omega $, то его границу $\partial \Pi$ можно стянуть в точку~$z_0\in\Omega$ (процесс стягивания изображён на рисунке~\ref{fig:rectangle_into_point}). Таким образом, граница $\partial\Pi$ гомотопна постоянному пути $\gamma_{z_0}(t) = z_0$. Следовательно,
   \begin{align*}
    \int_{\partial\Pi} \omega = \int_{\gamma_{z_0}}   \omega = 0.
   \end{align*} 
   \begin{figure}[ht]
    \centering
    \incfig[0.5]{rectangle_into_point}
    \caption{Стягивание границы прямоугольника в точку.}
    \label{fig:rectangle_into_point}
   \end{figure}

  \item Из пункта~\ref{enum3:theorem:closed_1_form} следует пункт~\ref{enum1:theorem:closed_1_form}. Рассмотрим любую точку~$ z_0 \in \Omega $. Возьмём радиус~$ r > 0 $ такой, что открытый шар~$ B_0 = B(z_0, r) \subset \Omega $. Нам нужно найти $ C^{1} $-гладкую функцию~$ F \colon B_0 \to \CC $, такую, что $ dF = \omega $ всюду в $ B_0 $. Построим первообразную явно:
   \begin{align*}
    F(z) = \int_{\gamma(z_0, z)} \omega, \qquad z \in B_0,
   \end{align*} где $ \gamma(z_0, z) $ --- путь от точки $ z_0 $ до точки $ z $, состоящий из двух отрезков, параллельных осям $ x $ и $ y $; он изображен на рисунке \ref{fig:parrallel_path_in_oblast}. Заметим, что так как интеграл по границе прямоугольника со сторонами, параллельными осям координат, и с углами в точках $ z_0 $ и $ z $ равен нулю, то не важно, какой из двух возможных путей брать. Кроме того, любой из этих двух путей обязательно лежит в $ B_0 $, так как шар~$ B_0 $ выпуклый. Поэтому, функция~$ F $ корректно задана на $ B_0 $ --- нам осталось проверить, что она действительно является первообразной формы~$ \omega $.

   \begin{figure}[ht]
    \centering
    \incfig[0.7]{parrallel_path_in_oblast}
    \caption{Путь $ \gamma(z_0, z) $.}
    \label{fig:parrallel_path_in_oblast}
   \end{figure}

   Пусть $ p = p_1 + ip_2 \in B_0 $ --- произвольная точка в шаре. Проверим дифференцируемость функции~$ F $ в точке~$ p $. Пусть $ h = h_1 + ih_2 \to 0 $ --- малое приращение. Тогда
   \begin{align*}
    F(p + h) - F(p) &= \int_{\gamma(p, p + h)}   \omega,
   \end{align*} где $ \gamma(p, p + h) $ --- путь от  $ p $  до $ p + h $, состоящий из двух отрезков, параллельных осям координат. Это верно, так как интегралы формы $ \omega $ по пути $ \gamma(z_0,p+h) $ и по пути $ \gamma(z_0, p) + \gamma(p,p+h) $ совпадают, для понимания см. рисунок~\ref{fig:differential_of_parallel_path_f}.

   \begin{figure}[ht]
    \centering
    \incfig[0.7]{differential_of_parallel_path_f}
    \caption{Разность $ F(p+h) - F(p) $ наглядно.}
    \label{fig:differential_of_parallel_path_f}
   \end{figure}

   Запишем  $ \omega(z) = P(x,y)\,dx + Q(x,y)\,dy $, где  $ P,Q $ --- непрерывные функции двух вещественных переменных ($ z = x+iy $). Тогда
   \begin{align*}
    F(p+h)-F(p) &= \int_{0}^{h_1} P(p_1 + x, p_2)\,dx + \int_{0}^{h_2} Q(p_1 + h_1, p_2 + y)\,dy = \\
    &= \int_{0}^{h_1} \left( P(p) + o(1) \right)dx + \int_{0}^{h_2} \left( Q(p) + o(1) \right)dy = \\
    &= P(p) \cdot h_1 + Q(p) \cdot h_2 + o(h).
   \end{align*} Таким образом, функция $ F $ дифференцируема в каждой точке $ p \in B_0 $, и
   \begin{align*}
    d_p F = P(p)\,dx + Q(p)\,dy = \omega(p).
   \end{align*} Значит, форма $ \omega $ точна в шаре~$ B_0 $, то есть $ \omega $ замкнута в области~$ \Omega $.
 \end{itemize}
\end{proof}

\end{document}
