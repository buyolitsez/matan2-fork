% 2023.05.18 lecture 13
\documentclass[../complex-analysis.tex]{subfiles}
\begin{document}

\newpage
\section{Ряды Фурье по тригонометрической системе.}

В этом параграфе мы перейдём от абстрактных гильбертовых пространств к конкретному гильбертову пространству $ L^{2} $ с конкретным ортонормированным базисом (тригонометрическая система), которые изучать тоже интересно.

\subsection{Теорема Фейера.}

\begin{claim}
 \label{claim:e^int is Orhonorm System}
 Множество функций
 \begin{align*}
  \left\{ t \mapsto \frac{e^{int}}{\sqrt{2\pi}} \right\}_{n \in \Z}
 \end{align*} является ортонормированной системой в гильбертовом пространстве~$ L^{2}[-\pi,\pi] $.
\end{claim}
\begin{proof}[\normalfont\textsc{Доказательство}]
 \begin{align*}
  \frac{1}{2\pi} \int_{-\pi}^{\pi} e^{int} \cdot e^{-imt}\,dt = \begin{cases}
   1, \text{ если } n = m, \\
   0, \text{ если } n \neq m.
  \end{cases} 
 \end{align*}
\end{proof}

\begin{thm}[Фейера]
 \label{theorem:Feyer}
 Пусть непрерывная функция $ f \in C(\R) $ $ 2\pi $-периодична. Обозначим \emph{частичные суммы Фурье:}
 \begin{align}
  \label{eq:Partial Fourier Sums:Feyer Thm}
  S_{N} (f,t) = \sum_{-N}^{N} c_ne^{int}
 \end{align} где
 \begin{align*}
  c_n = \frac{1}{2\pi} \int_{-\pi}^{\pi} f(x) \cdot e^{-inx}\,dx.
 \end{align*} Тогда ряд сходится по Чезаро к функции $ f $:
 \begin{align}
  \label{eq:Cesaro Convergence:Thm Feyer}
  \frac{1}{M+1} \sum_{N=0}^{M} S_N(f,t) \rightrightarrows f(t), \qquad t \in \R
 \end{align} при $ M \to + \infty$.
\end{thm}
\begin{proof}[\normalfont\textsc{Доказательство}]\
 \begin{enumerate}
  \item \label{enum1:Feyer Thm} Докажем, что $ S_N(f,t) = S_N(g, 0) $ для функции
   \begin{align*}
    g(x) = f(x + t).
   \end{align*} Действительно,
   \begin{align*}
    S_N(f,t) &= \sum_{-N}^{N} \frac{1}{2\pi} \int_{-\pi}^{\pi} f(x)\cdot e^{-in(x-t)}\,dx =  \\
    &= \begin{bmatrix}
     y = x - t, & dy = dx
    \end{bmatrix} = \\
    &= \sum_{-N}^{N} \frac{1}{2\pi} \int_{-\pi - t}^{\pi - t} f(y + t) \cdot e^{-iny}\,dy = \\
    &= \sum_{-N}^{N}\frac{1}{2\pi} \int_{-\pi}^{\pi} f(y+t) \cdot e^{-iny}\,dy = \\
    &= \sum_{-N}^{N} \frac{1}{2\pi} \int_{-\pi}^{\pi} g(y) \cdot e^{-iny}\,dy,
   \end{align*} где последний переход справедлив, так как функция $f$ имеет период $2 \pi$.  
  \item \label{enum2:Feyer Thm} Рассмотрим функции $ p_N \colon\, [-\pi,\pi] \to \R $ такие, что
   \begin{itemize}
    \item $ p_N \geqslant 0 $.
    \item $ \int_{-\pi}^{\pi} p_N\,dx = 1  $.
    \item Для любого числа~$ \eps > 0 $:
     \begin{align}
      \label{eq:enum2 Property 3:Feyer Thm}
      \lim_{N \to \infty} \max_{ \eps \leqslant \left| x \right| \leqslant \pi } p_N(x) = 0.
     \end{align}
   \end{itemize} Тогда 
   \begin{align}
    \label{eq:Lim Dot Product with p_N:Feyer Thm}
    \lim_{N \to \infty} \int_{-\pi}^{\pi} g(s) \cdot p_N(s)\,ds = g(0).
   \end{align} Докажем:
   \begin{align*}
    &\left|\int_{-\pi}^{\pi} g(s) \cdot p_N(s)\,ds - g(0) \right| \leqslant \int_{-\pi}^{\pi} \left| g(s) - g(0) \right|p_N(s)\,ds = \\
    = &\int_{\left| x \right| < \eps} \left| g(s)-g(0) \right|p_N(s)\,ds  + \int_{\eps \leqslant \left| x \right| \leqslant \pi} \left| g(s)-g(0) \right|p_N(s)\,ds \leqslant \\
    \leqslant &\underbrace{\sup_{\left| x-y \right| < \eps} \left| g(x)-g(y) \right|}_{\to 0 \text{ при } \eps \to 0} \cdot \underbrace{\int_{\left| x \right| < \eps}   p_N(s)\,ds}_{\leqslant 1} + \underbrace{\sup_{\eps \leqslant \left| x \right| \leqslant \pi} p_N(x)}_{\to 0 \text{ при } N \to \infty} \cdot \underbrace{\int_{\eps \leqslant \left| x \right| \leqslant \pi}  \left| g(s)-g(0) \right|\,ds}_{\leqslant 2\pi \cdot 2 \max \left| g \right|}.
   \end{align*} Кроме того, оценка на скорость сходимости~\eqref{eq:Lim Dot Product with p_N:Feyer Thm} зависит лишь от функций~$ p_N $ и от
   \begin{align*}
    \delta(\eps) := \max_{\left| x-y \right| \leqslant \eps} \left| g(x)-g(y) \right|.
   \end{align*} В частности, оценка равномерная по $ t $ из шага~\ref{enum1:Feyer Thm}.

  \item Перепишем левую часть \eqref{eq:Cesaro Convergence:Thm Feyer}:
   \begin{align}
    \begin{aligned}
     \label{eq:LHS Rewritten in p_M:Feyer Thm}
     \frac{1}{M+1} \sum_{N=0}^{M} S_N(g,0) &= \frac{1}{M+1} \sum_{N=0}^{M} \sum_{n=-N}^{N} \frac{1}{2\pi} \int_{-\pi}^{\pi} g(s) \cdot e^{-ins}\,ds = \\
     &= \int_{-\pi}^{\pi} g(s) \cdot p_M(s)\,ds,
    \end{aligned}
   \end{align} где
   \begin{align}
    \label{eq:Defintion of p_M:Feyer Thm}
    p_M(s) = \frac{1}{2\pi(M+1)} \sum_{N=0}^{M} \sum_{-N}^{N} e^{-ins}.
   \end{align} Далее будем переписывать функцию~$ p_M $. Перепишем сначала двойную сумму: докажем индукцией по $ M $ равенство
   \begin{align}
    \label{eq:Double Sum of Exps:Feyer Thm}
    \sum_{N=0}^{M} \sum_{-N}^{N} e^{-ins} = \left| \sum_{k=0}^{M} e^{iks} \right|^{2}.
   \end{align} База $ M=0 $ очевидна, проверим переход $ M \to M+1 $. Раскроем правую часть \eqref{eq:Double Sum of Exps:Feyer Thm} по формуле
   \begin{align*}
    \left| a+b \right|^{2} = \left| a \right|^{2} +  a \overline b + b \overline a + \left| b \right|^{2}, \quad a,b\in\CC\colon
   \end{align*}
   \begin{align*}
    &\left| \sum_{k=0}^{M+1} e^{iks} \right|^{2} = \left| \sum_{k=0}^{M}e^{iks} + e^{i(M+1)s} \right|^{2} =  \\
    =\;&\left| \sum_{k=0}^{M} e^{iks} \right|^{2} + e^{-i(M+1)s} \sum_{k=0}^{M} e^{iks} + e^{i(M+1)s} \sum_{k=0}^{M} e^{-iks} + \left| e^{i(M+1)s} \right|^{2} = \\
    =\;& \sum_{N=0}^{M} \sum_{-N}^{N} e^{-ins} + \sum_{1}^{M+1} e^{-iks} + \sum_{-(M+1)}^{-1} e^{-iks} + 1 = \\
    =\;& \sum_{N=0}^{M} \sum_{-N}^{N} e^{-ins} + \sum_{-(M+1)}^{M+1} e^{-iks} = \sum_{N=0}^{M+1} \sum_{-N}^{N} e^{-ins}.
   \end{align*} Формула~\eqref{eq:Double Sum of Exps:Feyer Thm} проверена.

   Тогда при $ s \neq 2\pi k $ можно свернуть сумму геометрической прогрессии:
   \begin{align*}
    p_M(s) &= \frac{1}{2\pi(M+1)} \left| \sum_{k=0}^{M} e^{iks} \right| = \frac{1}{2 \pi (M+1)} \left| \frac{1-e^{i(M+1)s}}{1-e^{is}} \right|^{2} = \\
    &= \frac{1}{2 \pi (M+1)}  \left|\frac{(e^{-i(M+1)s / 2} - e^{i(M+1)s / 2})/(2i)}{(e^{-is / 2} - e^{is / 2}) / (2i)} \right|^{2} = \\
    &= \frac{1}{2 \pi (M+1)} \frac{ \sin^{2} \frac{(M+1)s}{2}}{\sin^{2} \frac{s}{2}}.
   \end{align*} Полученная функция называется \emph{ядром Фейера} --- это очень известная функция; её график изображён на рисунке \ref{fig:feyer_kernel_plot}.

   \begin{figure}[ht]
    \centering
    \incfig[0.5]{feyer_kernel_plot}
    \caption{График ядра Фейера.}
    \label{fig:feyer_kernel_plot}
   \end{figure}

   Проверим все три свойства функции~$ p_M $, требуемые на шаге~\ref{enum2:Feyer Thm}.
   \begin{itemize}
    \item $ p_M(s) \geqslant 0 $ --- очевидно.
    \item $ \int_{-\pi}^{\pi} p_M\,ds = 1 $ --- это легче проверить в форме \eqref{eq:Defintion of p_M:Feyer Thm} с двойной суммой:
     \begin{align*}
      \int_{-\pi}^{\pi} p_M(s)\,ds &= \frac{1}{2 \pi (M+1)} \sum_{0}^{M+1} \sum_{-N}^{N} \int_{-\pi}^{\pi} e^{-ins}\,ds = \frac{1}{2 \pi (M+1)} \sum_{0}^{M+1} 2\pi = 1.
     \end{align*}
    \item Наконец, проверим свойство~\eqref{eq:enum2 Property 3:Feyer Thm}:
     \begin{align*}
      \max_{\eps \leqslant \left| s \right| \leqslant \pi} p_M(s) &=  \frac{1}{2\pi(M+1)} \max_{\eps \leqslant \left| s \right| \leqslant \pi} \frac{\sin^{2} \frac{(M+1)s}{2}}{\sin^{2} \frac{s}{2}} \leqslant \\
      &\leqslant \frac{1}{2\pi(M+1)} \max_{\eps \leqslant \left| s \right| \leqslant \pi} \frac{1}{\sin^{2} \frac{s}{2}} \leqslant \frac{C(\eps)}{2\pi (M+1)} \to 0,
     \end{align*} где $ C(\eps) $ зависит лишь только от $ \eps $.
   \end{itemize}
   Теперь, по шагу~\ref{enum2:Feyer Thm} верно \eqref{eq:Lim Dot Product with p_N:Feyer Thm}, что в терминах \eqref{eq:LHS Rewritten in p_M:Feyer Thm} означает
   \begin{align*}
    \frac{1}{M+1} \sum_{0}^{M} S_N(g,0) \to g(0)
   \end{align*} при $ M \to \infty $, причём сходимость равномерная по букве~$ t $.
 \end{enumerate}
\end{proof}

\subsection{Ортонормированные базисы в \texorpdfstring{$L^{2}[-\pi, \pi]$}{пространстве Лебега}.}

\begin{crly}
 \label{corollary:e^int System Density in Periodic Continuous Funcs}
 Линейная оболочка
 \begin{align}
  \label{eq:Span of e^int Orthogonal System}
  \mathrm{span} \left\{ t \mapsto e^{int} \right\}_{n \in \Z}
 \end{align} всюду плотна в пространстве непрерывных периодических функций
 \begin{align*}
  C_p[-\pi,\pi] = \left\{ f \in C[-\pi,\pi] : f(-\pi) = f(\pi) \right\}.
 \end{align*}
\end{crly}
\begin{proof}[\normalfont\textsc{Доказательство}]
 Ясно, что любая функция~$ f \in C_p [-\pi,\pi] $ совпадает с сужением некоторой $ 2\pi $-периодической функции~$ \tilde f \in C(\R) $ на отрезок $ [-\pi,\pi] $. По теореме~\ref{theorem:Feyer} Фейера:
 \begin{align*}
  \frac{1}{M+1} \sum_{n=0}^{M} S_N(\tilde f, t) \rightrightarrows \tilde f(t) = f(t), \qquad t \in [-\pi,\pi]
 \end{align*} Но, как можно видеть из \eqref{eq:Partial Fourier Sums:Feyer Thm}, левая часть лежит в линейной оболочке~\eqref{eq:Span of e^int Orthogonal System}, поэтому функцию~$ f $ можно сколь угодно хорошо приближается элементами линейной оболочки (равномерная сходимость является сходимостью в пространстве $ C_p [-\pi,\pi] $).
\end{proof}

\begin{crly}
 \label{corollary:e^int is ONB in L^2 space}
 Система
 \begin{align*}
  \left\{t \mapsto \frac{e^{int}}{\sqrt{2\pi}} \right\}_{n \in \Z}
 \end{align*} является ортонормированном базисом в пространстве $ L^{2}[-\pi,\pi] $.
\end{crly}
\begin{proof}[\normalfont\textsc{Доказательство}]
 В утверждении~\ref{claim:e^int is Orhonorm System} мы уже показали, что это система ортонормированная. Предположим, что она не всюду плотна. Это означает, что замыкание линейной оболочки системы
 \begin{align*}
  V = \mathop{\mathrm{Cl}} \left[ \mathop{\mathrm{span}} \left\{ t \mapsto e^{int} \right\}_{n \in \Z} \right]
 \end{align*}
 не совпадает со всем пространством, а в этом случае найдётся функция~$ f \in L^{2}[-\pi,\pi] $, ортогональная к подпространству~$ V $:
 \begin{align*}
  \int_{-\pi}^{\pi} f p \,dx = 0, \qquad p \in V
 \end{align*} Так как по следствию~\ref{corollary:e^int System Density in Periodic Continuous Funcs} подпространство~$ V $ содержит в себе пространство $ C_p [-\pi,\pi] $, то
 \begin{align*}
  \int_{-\pi}^{\pi} f \cdot g_\eps \, dx = 0,
 \end{align*} где функция~$ g_\eps \in C_p [-\pi,\pi] $ построена как на картинке~\ref{fig:eint_full_in_l2_g_eps}.

 \begin{figure}[ht]
  \centering
  \incfig[0.8]{eint_full_in_l2_g_eps}
  \caption{График функции $g_{\eps}$. }
  \label{fig:eint_full_in_l2_g_eps}
 \end{figure}

 Тогда, устремляя $ \varepsilon \to 0 $, по теореме Лебега о мажорируемой сходимости с мажорантой $ \left| f \right| $ можно совершить предельный переход под знаком интеграла:
 \begin{align*}
  0 = \lim_{ \varepsilon \to 0 } \int_{-\pi}^{\pi} f(x) \cdot g_{\varepsilon}(x) \,dx = \int_{-\pi}^{\pi} f(x) \cdot \chi_{[a,b]}(x) \,dx
 \end{align*} Таким образом, для любых $ a,b \in (-\pi,\pi) $ верно
 \begin{align*}
  \frac{1}{b-a} \int_a^{b} f(x)\,dx = 0.
 \end{align*} Следовательно, $ f(x) = 0 $ в любой точке Лебега $ x \in [-\pi,\pi] $. Поскольку почти все точки есть точки Лебега, то $ f = 0 $ как элемент $ L^{2}[-\pi,\pi] $.
\end{proof}

\begin{crly}
 \label{corollary:sin cos Orthogonal Basis}
 Система
 \begin{align}
  \label{eq:sin cos System}
  T = \left\{t \mapsto \cos nt\right\}_{n \geqslant 0} \cup \left\{t \mapsto \sin nt\right\}_{n \geqslant 1}
 \end{align} является ортогональным базисом в пространстве~$ L^{2}[-\pi,\pi] $.
\end{crly}
\begin{proof}[\normalfont\textsc{Доказательство}]
 Так как $ e^{int} = \cos nt + i \sin nt$, то
 \begin{align*}
  \mathop{\mathrm{span}}T \supset \mathop{\mathrm{span}} \left\{ t \mapsto e^{int} \right\}_{n \in \Z},
 \end{align*} а система в правой части полна по следствию~\ref{corollary:e^int is ONB in L^2 space}. Осталось проверить ортогональность системы. Во-первых,
 \begin{align*}
  \int_{-\pi}^{\pi} \cos nt \cdot \sin kt\,dt &= 0,
 \end{align*}  так как функция $ \cos nt \cdot \sin kt $ нечётная. Во-вторых,
 \begin{align}
  \label{eq:cos cos int:trig basis}\int_{-\pi}^{\pi} \cos nt \cdot \cos mt \,dt &= \begin{cases}
   0, &\text{если } n \neq m, \\
   \pi, &\text{если } n = m \geqslant 1,\\
   2\pi &\text{если } n = m = 0;
  \end{cases}\\
  \label{eq:sin sin int:trig basis} \int_{-\pi}^{\pi} \sin nt \cdot \sin mt\,dt &= \begin{cases}
   0, &\text{если } n \neq m,  \\
   \pi, &\text{если } n = m \geqslant 1,
  \end{cases}
 \end{align} Упражнение: проверить \eqref{eq:cos cos int:trig basis} и \eqref{eq:sin sin int:trig basis}.
\end{proof}

\begin{crly}
 \label{corollary:Fourier Series in L^2}
 Для любой функции $ f \in L^{2}[-\pi,\pi] $ выполнено разложение в ряд Фурье
 \begin{align}
  \label{eq:Fourier Series e^int}
  f(t) &= \sum_{n \in \Z} c_n e^{int},
 \end{align} где
 \begin{align*}
  c_n &= \frac{1}{2\pi} \int_{-\pi}^{\pi} f(t) \cdot e^{-int}\,dt;
 \end{align*} а также
 \begin{align}
  \label{eq:Fourier Series sin cos}
  f(t) &= \frac{a_0}{2}+ \sum_{k=1}^{\infty} a_k \cos kt + \sum_{k=1}^{\infty} b_k\sin kt, \\
  \intertext{где}
  \notag a_k &= \frac{1}{\pi} \int_{-\pi}^{\pi} f(t) \cdot \cos kt\,dt, \qquad k \geqslant 0, \\
  \notag b_k &= \frac{1}{\pi} \int_{-\pi}^{\pi} f(t) \cdot \sin kt\,dt, \qquad k \geqslant 1,
 \end{align} где ряды~\eqref{eq:Fourier Series e^int} и \eqref{eq:Fourier Series sin cos} сходятся в пространстве~$ L^{2}[-\pi,\pi] $.
\end{crly}
\begin{proof}[\normalfont\textsc{Доказательство}]\
 \begin{itemize}
  \item По следствию~\ref{corollary:e^int is ONB in L^2 space} система $ \left\{ t \mapsto e^{int} / \sqrt{2\pi} \right\}_{n \in \Z} $ является ортонормированным базисом в пространстве~$ L^{2}[-\pi,\pi] $, поэтому для любой функции~$ f \in L^{2}[-\pi,\pi] $:
   \begin{align*}
    f(t) = \sum_{k \in \Z} (f, e^{int}) \cdot \frac{e^{int}}{\sqrt{2\pi}},
   \end{align*} где
   \begin{align*}
    (f, e^{int}) = \frac{1}{\sqrt{2\pi}}\int_{-\pi}^{\pi} f(t) \cdot e^{-int}\,dt.
   \end{align*} Перенося множитель $ 1 / \sqrt{2\pi} $ из ряда в коэффициент, получаем \eqref{eq:Fourier Series e^int}.

  \item По следствию~\ref{corollary:sin cos Orthogonal Basis} система~$ T $ из~\eqref{eq:sin cos System} является ортогональным базисом. Из \eqref{eq:cos cos int:trig basis} и \eqref{eq:sin sin int:trig basis} следует
   \begin{align*}
    \left\| 1 \right\|_{L^{2}[-\pi,\pi]} &= 2\pi, \\
    \left\| \cos kt \right\|_{L^{2}[-\pi,\pi]} &= \pi, \quad k \geqslant 1,\\
    \left\| \sin kt \right\|_{L^{2}[-\pi,\pi]} &= \pi, \quad k \geqslant 1.
   \end{align*} Поэтому, нормировочные коэффициенты такие, какие нужно.
 \end{itemize}
\end{proof}

\subsection{Коэффициенты Фурье гладких функций.}

\begin{df}
 \emph{$ n $-м коэффициентом Фурье} ($ n \in \Z $) функции~$ f \in L^{2}[-\pi,\pi] $ называется число
 \begin{align*}
  \hat f(n) = c_n = \frac{1}{\sqrt{2\pi}} \int_{-\pi}^{\pi} f(t) \cdot e^{-int}\,dt.
 \end{align*}
\end{df}

\begin{thm}
 \label{theorem:Fourier Coeffiecents of Smooth Funs}
 Пусть функция~$ f \in C^{r}(\R) $, $ r \geqslant 1 $ $ 2\pi $-периодична. Тогда
 \begin{align*}
  \hat f(n) = \OO\left( \frac{1}{\left| n \right|^{r}}\right)
 \end{align*} при $ n \to \pm\infty $.
\end{thm}
\begin{proof}[\normalfont\textsc{Доказательство}]
 Проинтегрируем по частям коэффициент Фурье, пользуясь периодичностью $ f $:
 \begin{align*}
  \hat f(n) &= \int_{-\pi}^{\pi} f(t) \cdot e^{-int}\,dt = \\
  &= \underbrace{f(t) \cdot \left.\frac{e^{-int}}{-in}\right|_{-\pi}^{\pi}}_{0} + \frac{1}{in} \int_{-\pi}^{\pi} f'(t) \cdot e^{-int}\,dt = \\
   &= \underbrace{\left.f'(t) \cdot \frac{e^{-int}}{-(in)^{2}}\right|_{-\pi}^{\pi}}_{0} + \frac{1}{(in)^{2}} \int_{-\pi}^{\pi} f^{(2)}(t) \cdot e ^{-int}\,dt = \\
    &= \ldots = \\
    &= \frac{1}{(in)^{r}} \int_{-\pi}^{\pi} f^{(r)}(t) \cdot e^{-int}\,dt.
 \end{align*} Тогда
  \begin{align*}
   \left| \hat f(n) \right| \leqslant \max_{t \in [-\pi,\pi]} \left| f^{(r)}(t) \right| \cdot 2\pi \cdot \frac{1}{n^{r}}.
  \end{align*}
 \end{proof}

 \subsection{Теорема Харди.}

 В результате следствия~\ref{corollary:Fourier Series in L^2} мы доказали, что всякая функция из $ L^{2}[-\pi,\pi] $ раскладывается в ряд Фурье. Однако этот ряд пока-что сходится лишь в пространстве Лебега, либо же сходится по Чезаро по теореме~\ref{theorem:Feyer} Фейера, а нам хотелось бы, чтобы сходимость была поточечной. Как раз эту проблему решает теорема Харди.

 \begin{thm}[Харди]
  Пусть гладкая функция $ f \in C^{1}(\R) $ $ 2\pi $-периодична. Тогда ряд Фурье функции~$ f $ 
  \begin{align}
   \label{eq:Fourier Series:Hardy Thm}
   \sum_{n \in \Z} \hat f(n) \cdot  e^{int}
  \end{align} сходится к ней самой всюду в $ \R $.
 \end{thm}
 \begin{proof}[\normalfont\textsc{Доказательство}]
  По теореме~\ref{theorem:Feyer} Фейера ряд Фурье~\eqref{eq:Fourier Series:Hardy Thm} сходится по Чезаро к функции~$ f $ всюду в $ \R $. Кроме того, по теореме~\ref{theorem:Fourier Coeffiecents of Smooth Funs}
  \begin{align*}
   \left| \hat f(n) \cdot e^{int} \right| = \OO(1 / n).
  \end{align*} Тогда по тауберовой теореме~\ref{theorem:tauber} ряд~\eqref{eq:Fourier Series:Hardy Thm} сходится в обычном смысле всюду в~$ \R $.
 \end{proof}

 \begin{thm}[тауберова теорема]
  \label{theorem:tauber}
  Пусть числовой ряд
  \begin{align}
   \label{eq:Series:Tauber Thm}
   \sum_{n=1}^{\infty} a_n, \quad a_n \in \CC
  \end{align} сходится по Чезаро:
  \begin{align*}
   \lim_{N \to \infty} \frac{1}{N} \sum_{n=1}^{N} s_n = S \in \R,
  \end{align*} где
  \begin{align*}
   s_n = \sum_{k=1}^{n} a_k.
  \end{align*} Пусть также выполнено \emph{тауберово условие:} $ a_n = \OO(1 / n) $. Тогда ряд~\eqref{eq:Series:Tauber Thm} сходится в обычном смысле.
 \end{thm}

 \begin{proof}[\normalfont\textsc{Доказательство тауберовой теоремы}]
  Обозначим
  \begin{align*}
   \sigma_n &= \frac{s_1 + \ldots + s_n}{n}, & \sigma_{n,l} &= \frac{s_{n+1} + \ldots + s_{n+l}}{l}.
  \end{align*}

  Пусть $ l = \left\lfloor \eps n \right\rfloor  $, где $ \eps \in (0,1) $ --- число, которое мы будем подбирать позже. Тогда при $ n \to \infty $:
  \begin{align*}
   \sigma_{n,l} = \frac{\sigma_{n+l} \cdot (n+l) - \sigma_n \cdot n}{l} = \underbrace{(\sigma_{n+l}-\sigma_n)}_{\to 0 \text{ по Коши}} \cdot \underbrace{\frac{n+l}{l}}_{\OO(1)} + \sigma_n \cdot \frac{l}{l} = \sigma_n + o(1).
  \end{align*} и, следовательно,
  \begin{align*}
   \lim_{n \to \infty} \left| \sigma_n - \sigma_{n,l} \right| = 0.
  \end{align*}

  Далее,
  \begin{align*}
   \left|\sigma_{n,l} - s_n \right| &\leqslant \frac{\left|s_{n+1} - s_n \right| + \left| s_{n+2} - s_n \right| + \ldots + \left| s_{n+l} - s_n \right|}{l} \leqslant \\
   &\leqslant \frac{\left| a_{n+1} \right| + (\left| a_{n+1} \right| + \left| a_{n+2} \right|) + \ldots + \left( \left| a_{n+1} \right| + \ldots + \left| a_{n+l} \right| \right)} {l} \leqslant \\
   &\leqslant \frac{l |a_{n+1}| + (l - 1) |a_{n + 2}| + \ldots + |a_{n + l}|}{l} \leqslant  \\
   &\leqslant |a_{n + 1}| + |a_{n + 2}| + \ldots + |a_{n + l}| \leqslant \\
   &\leqslant \frac{Cl}{n} \leqslant C\eps.
  \end{align*} Следовательно,
  \begin{align*}
   \limsup_{n \to \infty} \left| s_n - \sigma_n \right| \leqslant C \eps,
  \end{align*} где $ C $ не зависит от $ \eps $. Устремив $ \varepsilon \to 0 $, получаем требуемую сходимость.
 \end{proof}

 \end{document}

