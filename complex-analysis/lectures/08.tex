% 2023.04.13 lecture 08
\documentclass[../complex-analysis.tex]{subfiles}
\begin{document}

\newpage
\section{Принцип аргумента.}

В параграфе~\ref{section:Singularities of Analytic Functions}, изучая изолированные особые точки аналитических функций, мы могли заметить некоторую схожесть между нулями и полюсами аналитической функции. Например, порядок полюса равен номеру наименьшего ненулевого коэффициента ряда Лорана (со знаком минус), при условии что хотя бы один ненулевой коэффициент с отрицательным номером всё-таки есть. В то же время, кратность нуля равна номеру наименьшего ненулевого коэффициента в ряде Тейлора (который совпадает с рядом Лорана, если функция аналитична в окрестности изучаемой точки). Таким образом, можно считать, что полюс порядка~$ n \geqslant 1 $ --- это <<нуль кратности $ -n $>>, или же что нуль кратности~$ k \geqslant 1 $ --- это <<полюс порядка $ -k $>>. При этом точку, в которой аналитическая функция определена и не равна нулю, можно считать как нуль кратности~$ 0 $, так и полюс порядка~$ 0 $.

В этом параграфе мы изучим \emph{принцип аргумента} --- утверждение, которое во многом раскрывает это сходство между нулями и полюсами. Упрощая, принцип аргумента говорит о том, что изменение аргумента функции~$ f $ при обходе вдоль границы области~$ \Omega $ равно $ 2\pi(N - P) $, где $ N $ --- число нулей функции~$ f $ в $ \Omega $ с учётом кратности, а $ P $ --- число полюсов функции~$ f $ в $ \Omega $ с учётом порядка. Или же, повторяя предыдущую мысль, изменение аргумента равно $ 2\pi S $, где $ S $ --- сумма кратностей всех нулей функции~$ f $ в $ \Omega $, где полюс порядка $ n \geqslant 1 $ считается как нуль кратности~$ -n $.

Приведём пример <<на пальцах>>. У функции $ f(z) = z^{n} $ в области $ \mathbb D $ аргумент вдоль границы $ \partial\mathbb D $ равен $ \arg e^{int} = nt $, где $ t \in [0,2\pi] $. При обходе точки~$ 0 $ по окружности $ \partial\mathbb D $ против часовой стрелки, аргумент изменяется на
\begin{align*}
 \Delta_{\partial\Omega} \arg z^{n} = 2\pi n = \arg z^{n}(2\pi) - \arg z^{n}(0)
\end{align*} так как $ 0 $ --- нуль функции $ z^{n} $ кратности $ n $ (а других нулей и полюсов у функции нет).

\subsection{Аргумент вдоль пути.}

Для понятия <<аргумент>> ранее у нас было определение~\ref{def:Continious Branch of Argument} непрерывной ветви аргумента аналитической функции в заданной области. Здесь же нам понадобится <<аргумент вдоль пути>>, поэтому дадим новое определение.

\begin{df}
 \label{definition:continuous_branch_argument_path}
 Пусть $ \gamma\colon\,[0,1]\to\CC $ --- путь, а функция~$ f \colon\,\gamma([0,1]) \to \CC \setminus \left\{ 0 \right\}$ определена и не равна нулю на носителе пути. Непрерывная функция~$ \psi\colon\,[0,1] \to \R $ называется \emph{непрерывной ветвью аргумента функции~$ f $ вдоль пути~$ \gamma $}, если для любого $ t\in[0,1] $ выполнено
 \begin{align*}
  \left| f(\gamma(t)) \right| \cdot e^{i\psi(t)} = f(\gamma(t)).
 \end{align*}
\end{df}

\begin{claim}
 \label{claim:Argument Along Path Uniqueness}
 В условиях определения~\ref{definition:continuous_branch_argument_path} непрерывная ветвь аргумента функции~$ f $ вдоль пути~$ \gamma $ единственна (если, конечно, она существует): любые две такие ветви отличаются на $ 2\pi k $, $ k\in\Z $.
\end{claim}
\begin{proof}[\normalfont\textsc{Доказательство}]
 Пусть $ \psi_1 $, $ \psi_2 $ --- непрерывные ветви аргумента функции $ f $ вдоль пути $ \gamma $. Тогда для любого $ t\in[0,1] $
 \begin{align*}
  \left| f(\gamma(t)) \right| \cdot e^{i\psi_1(t)} = \left| f(\gamma(t)) \right|\cdot e^{i\psi_2(t)}.
 \end{align*} Так как $ \left| f(\gamma(t)) \right| \neq 0 $, то $ e^{i(\psi_1(t)-\psi_2(t))} = 1 $ для всех $ t\in[0,1] $. Следовательно,
 \begin{align*}
  \psi_1(t) - \psi_2(t) = 2\pi k(t),
 \end{align*} где $ k \colon\,[0,1] \to \Z $ --- непрерывная функция со значениями в $ \Z $, то есть $ k(z) = k $ --- целое число.
\end{proof}

\begin{thm}
 \label{theorem:continious_branch_argument_path_exists}
 Пусть в условиях определения~\ref{definition:continuous_branch_argument_path}  функция~$ f $ вдобавок является $ C^{1} $-гладкой на носителе пути $ \gamma([0,1]) $, а путь $ \gamma $ является кусочно-гладким. Тогда непрерывная ветвь аргумента функции~$ f $ вдоль пути~$ \gamma $ существует.
\end{thm}
На самом деле, теорему~\ref{theorem:continious_branch_argument_path_exists} можно доказать и не предполагая кусочно-гладкость пути $ \gamma $, но мы это опустим.

План доказательства будет следующий (схожий с доказательством теоремы~\ref{theorem:exist_arg_f}): мы построим непрерывный логарифм вдоль пути, и тогда аргумент вдоль пути просто будет равен его мнимой части.
\begin{proof}[\normalfont\textsc{Доказательство теоремы~\ref{theorem:continious_branch_argument_path_exists}}]
 Рассмотрим функцию~$ g \colon\,[0, 1] \to \CC $:
 \begin{align}
  \label{eq:Explicit Def of Log Along Path}
  g(t) = \int_{0}^{t} \frac{f'(\gamma(u))}{f(\gamma(u))} \cdot  \gamma'(u)\,du.
 \end{align} Функция~$ g $ по построению непрерывная и кусочно-гладкая. Проверим, что она является логарифмом функции~$ f $ вдоль пути~$ \gamma $:
 \begin{align*}
  \left[ \frac{e^{g(t)}}{f(\gamma(t))} \right]' &= \frac{\left(e^{g(t)}\right)' \cdot f(\gamma(t)) - e^{g(t)} \cdot (f(\gamma(t)))'}{f(\gamma(t))^{2}} = \\
  &= \frac{e^{g(t)} \cdot \left( g'(t) \cdot f(\gamma(t)) - f'(\gamma(t)) \cdot \gamma'(t) \right)}{f(\gamma(t))^{2}},
 \end{align*} но по построению~$ g $
 \begin{align*}
  g'(t) \cdot f(\gamma(t)) - f'(\gamma(t)) \cdot \gamma'(t) = \frac{f'(\gamma(t))}{f(\gamma(t))} \cdot \gamma'(t) \cdot f(\gamma(t)) - f'(\gamma(t)) \cdot \gamma'(t) = 0,
 \end{align*} поэтому
 \begin{align*}
  \left[ \frac{e^{g(t)}}{f(\gamma(t))} \right]' = 0.
 \end{align*} Значит, функция $ t \mapsto e^{g(t)} / f(\gamma(t)) $ кусочно-постоянна, а раз она непрерывна на $ [0,1] $, то она постоянна на $ [0,1] $:
 \begin{align*}
  \frac{e^{g(t)}}{f(\gamma(t))} = e^{-c} 
 \end{align*} для некоторой константы $ c \in \CC $. Обозначим
 \begin{align}
  \label{eq:Explicit Def Of True Log Along Path}
  \hat g (t) = g(t) + c.
 \end{align} Тогда
 \begin{align}
  \label{eq:Logarithm Along Path:Exists Argument Along Path}
  e^{\hat g(t)} = f(\gamma(t)), \quad t \in [0,1].
 \end{align} Логарифм построен.

 Теперь рассмотрим функцию
 \begin{align}
  \label{eq:Explicit Def of Argument Along Path}
  \psi(t) = \Imaginary \hat g(t), \quad t \in [0,1].
 \end{align} Функция $ \psi $ непрерывна на $ [0,1] $, действует в $ \R $, и по \eqref{eq:Logarithm Along Path:Exists Argument Along Path}
 \begin{align*}
  f(\gamma(t)) = e^{\hat g(t)} = e^{\Real \hat g(t) + i \psi(t)} = \left| f(\gamma(t)) \right| \cdot e^{i\psi(t)}.
 \end{align*} Аргумент вдоль пути построен.
\end{proof}

\begin{remrk}
 \label{remark:Piecewise Smooth Argument Along Path}
 При выполнении условий теоремы~\ref{theorem:continious_branch_argument_path_exists} все непрерывные ветви аргумента функции~$ f $ вдоль пути~$ \gamma $ являются кусочно-гладкими.
\end{remrk}
\begin{proof}
 Покажем, что построенная в доказательстве теоремы~\ref{theorem:continious_branch_argument_path_exists} ветвь аргумента на самом деле является кусочно-гладкой. Действительно, построенная функция $ \hat g \colon\,[0,1] \to \CC $  является кусочно-гладкой. Её можно воспринимать как кусочно-гладкое отображение $ \hat g \colon\,[0,1] \to \R^{2} $. Так отображение $ \Imaginary\colon\,\R^{2}\to\R $ также является гладким (оно даже линейное), то по формуле~\eqref{eq:Explicit Def of Argument Along Path} и ветвь аргумента $ \psi $ кусочно-гладкая.

 По утверждению~\ref{claim:Argument Along Path Uniqueness} о единственности все остальные непрерывные ветви аргумента функции~$ f $ вдоль пути~$ \gamma $ также являются кусочно-гладкими.
\end{proof}

\begin{df}
 В условиях теоремы~\ref{theorem:continious_branch_argument_path_exists} \textit{изменением аргумента функции~$ f $ вдоль пути $ \gamma $} называется число
 \begin{align*}
  \Delta_\gamma \arg f := \psi(1) - \psi(0).
 \end{align*}
\end{df}

\begin{remrk}
 По существованию аргумента функции вдоль пути (теорема~\ref{theorem:continious_branch_argument_path_exists}) изменение аргумента $ \Delta_\gamma \arg f $ определено, а по единственности (утверждение~\ref{claim:Argument Along Path Uniqueness}) оно не зависит от выбора непрерывной ветви аргумента.
\end{remrk}

\begin{remrk}
 \label{remark:Argument Along Paths Adds When Functions Multiply}
 Пусть $ \gamma\colon\,[0,1] \to \CC $ --- кусочно-гладкий путь, $ f,g\in C^{1}(\gamma([0,1]), \CC \setminus \left\{ 0 \right\}) $, и $ \psi_f, \psi_g, \psi_{fg} $ --- непрерывные ветви аргумента функций~$ f, g, f \cdot g $ (соответственно) вдоль пути~$ \gamma $. Тогда
 \begin{align}
  \label{eq:Argument Along Paths Adds When Functions Multiply}
  \psi_{fg}(t) = \psi_f(t) + \psi_g(t) + 2\pi k, \quad k \in \Z.
 \end{align}
\end{remrk}
\begin{proof}[\normalfont\textsc{Доказательство}]
 По замечанию~\ref{remark:Piecewise Smooth Argument Along Path} функции~$ \psi_f, \psi_g, \psi_{fg} $ кусочно-гладкие. Тогда равенство~\eqref{eq:Argument Along Paths Adds When Functions Multiply} равносильно
 \begin{align}
  \label{eq:Derivatives:Arg Along Paths Adds When Fs Mul}
  \psi_{fg}'(t) = \psi_f'(t) + \psi_g'(t),
 \end{align} а~\eqref{eq:Derivatives:Arg Along Paths Adds When Fs Mul}, в свою очередь, по формулам~\eqref{eq:Explicit Def of Argument Along Path}, \eqref{eq:Explicit Def Of True Log Along Path} и \eqref{eq:Explicit Def of Log Along Path}, последует из
 \begin{align*}
  &\frac{(f \cdot g)'(\gamma(t))}{(f \cdot g)(\gamma(t))} \cdot \gamma'(t) = \frac{f'(\gamma(t))}{f(\gamma(t))} \cdot \gamma'(t) + \frac{g'(\gamma(t))}{g(\gamma(t))} \cdot \gamma'(t) \impliedby \\
  \impliedby\;& \frac{f'(\gamma(t)) \cdot g(\gamma(t)) + f(\gamma(t)) \cdot g'(\gamma(t))}{f(\gamma(t)) \cdot g(\gamma(t))} = \frac{f'(\gamma(t))}{f(\gamma(t))} + \frac{g'(\gamma(t))}{g(\gamma(t))}.
 \end{align*} Последнее равенство, как можно видеть, верно.
\end{proof}

\subsection{Принцип аргумента.}

\begin{notatn}
 Пусть $ \Omega \subset \CC $ --- стандартная область, $ \partial \Omega  = \bigsqcup_{k=1}^{N} \gamma_k $  --- её граница, ориентированная каноничным способом (в смысле соглашения~\ref{convention:Standard Region Canonical Parameterization}). Пусть функция $ f \in C^{1}(\partial\Omega, \CC \setminus \left\{ 0 \right\}) $. Тогда \emph{изменением аргумента функции~$ f $ вдоль границы~$ \partial\Omega $} называется число
 \begin{align*}
  \Delta_{\partial \Omega} \arg f = \sum_{k=1}^{N}\Delta_{\gamma_k} \arg f.
 \end{align*}
\end{notatn}

\begin{thm}[%
 принцип аргумента]
 \label{theorem:Argument Principle}
 Пусть $ \Omega \subset\CC$ --- стандартная область, $ E \subset \Omega $ --- конечное число точек. Пусть функция~$ f $ аналитична в области~$ \Omega \setminus E $, $ C^{1} $-гладко продолжается на границу: $ f \in C^{1}(\overline\Omega \setminus E) $, не обнуляется на границе: $ f(z) \neq 0 $ при $ z \in \partial\Omega $, а также не имеет существенных особых точек в области~$ \Omega $. Тогда
 \begin{align*}
  \Delta_{\partial \Omega} \arg f = 2\pi(N - P),
 \end{align*} где $ N $ --- суммарное количество нулей $ f $ в $ \Omega $ (с учётом кратности), а $ P $ --- суммарное количество  полюсов $ f $ в $ \Omega $ (с учётом порядка).
\end{thm}

В условии теоремы~\ref{theorem:Argument Principle} важно, что в величине~$ N $ также учитываются нули, попавшие во множество~$ E $ как устранимые особые точки (от того, что мы искусственно выкинем нуль, изменение аргумента не изменится).

\begin{remrk}
 В условиях теоремы~\ref{theorem:Argument Principle} число нулей и число полюсов у функции~$ f $ в области~$ \Omega $ конечно.
\end{remrk}
\begin{proof}[\normalfont\textsc{Доказательство}]
 Число полюсов конечно по условию, так как $ E $ --- конечное множество. Предположим, что у функции~$ f $ бесконечное число нулей в $ \Omega $. Функция~$ f $ не равна тождественно нулю в $ \Omega $, поскольку она непрерывно продолжается на границу и не принимает значение нуль на границе. Так как область~$ \Omega $ ограничена, то можно выбрать последовательность нулей, сходящуюся к некоторой точке~$ z_0 \in \overline\Omega $, причём по непрерывности $ f $ на $ \overline\Omega $ имеем $ f(z_0) = 0 $. Так как $ f $ не обнуляется на границе, то $ z_0 \in \Omega $, но тогда по теореме~\ref{theorem:uniqueness} единственности $ f \equiv 0 $ в $ \Omega $, чего не может быть.
\end{proof}

\begin{proof}[\normalfont\textsc{Доказательство теоремы~\ref{theorem:Argument Principle}}]
 Рассмотрим интеграл логарифмической производной функции~$ f $ по границе:
 \begin{align*}
  \varointctrclockwise_{\partial\Omega}  \frac{f'(z)\,dz}{f(z)}.
 \end{align*} Переходя к параметризации границы, он равен
 \begin{align*}
  \varointctrclockwise_{\partial\Omega} \frac{f'(z)\,dz}{f(z)} = \sum_{k=1}^{n} \int_{0}^{1} \frac{f'(\gamma_k(t))}{f(\gamma_k(t))} \cdot \gamma_k'(t)\,dt,
 \end{align*} где $ \partial\Omega = \bigsqcup_{k=1}^{n} \gamma_k $  --- представление границы стандартной области~$ \Omega $ в виде объединения $ n $ непересекающихся простых кусочно-гладких замкнутых путей. Тогда по формулам~\eqref{eq:Explicit Def of Log Along Path} и \eqref{eq:Explicit Def Of True Log Along Path}:
 \begin{align*}
  \varointctrclockwise_{\partial\Omega} \frac{f'(z)\,dz}{f(z)} = \sum_{k=1}^{n} \left( \hat g_k(1) - \hat g_k(0) \right),
 \end{align*} где $ \hat g_k \colon\,[0,1] \to \CC $ --- непрерывная ветвь логарифма функции~$ f $ вдоль пути~$ \gamma_k $, построенная в доказательстве теоремы~\ref{theorem:continious_branch_argument_path_exists}. Но так как путь $ \gamma_k $ замкнутый, то $ \log \left| \gamma_k(0) \right| = \log \left| \gamma_k(1) \right| $ , и, значит, $ \Real \hat g_k(0) = \Real \hat g_k(1) $. Следовательно, по формуле~\eqref{eq:Explicit Def of Argument Along Path}
 \begin{align}
  \begin{aligned}
   \label{eq:Logarithmic Derivative Is Delta Arg:Argument Principle}
   \varointctrclockwise_{\partial\Omega} \frac{f'(z)\,dz}{f(z)} &= i\sum_{k=1}^{n} \left( \Imaginary \hat g_k(1) - \Imaginary \hat g_k(0) \right) = i \sum_{k=1}^{n} \left( \psi_k(1) - \psi_k(0) \right) = \\
   &= i \sum_{k=1}^{n} \Delta_{\gamma_k} \arg f = i \cdot \Delta_{\partial\Omega} \arg f,
  \end{aligned}
 \end{align} где $ \psi_k $ --- непрерывная ветвь аргумента функции~$ f $ вдоль пути~$ \gamma_k $.

 С другой стороны, по теореме~\ref{theorem:cauchy_residue} Коши о вычетах
 \begin{align*}
  \varointctrclockwise_{\partial\Omega} \frac{f'(z)\,dz}{f(z)} = 2\pi i \sum_{a \in S} \residue_a \frac{f'(z)}{f(z)},
 \end{align*} где $ S \subset \Omega$  --- множество всех нулей и полюсов функции~$ f $ в области~$ \Omega $ (учитывая нули, попавшие во множество~$ E $ как устранимые особенности). В примерах~\ref{example:logarithmic_derivative1} и \ref{example:logarithmic_derivative2} мы вычислили вычеты логарифмической производной:
 \begin{align*}
  &\residue_a \frac{f'(z)}{f(z)} = m, &&\text{если $ a $ --- нуль~$ f $ кратности~$ m \geqslant 1 $};\\
  &\residue_a \frac{f'(z)}{f(z)} = -m, &&\text{если $ a $ --- полюс~$ f $ порядка~$ m \geqslant 1 $}.
 \end{align*} Поэтому
 \begin{align*}
  \varointctrclockwise_{\partial\Omega} \frac{f'(z)\,dz}{f(z)} = 2\pi i (N-P),
 \end{align*} где $ N $ --- суммарное число нулей, а $ P $ --- суммарное число полюсов функции~$ f $ в области~$ \Omega $ с учётом кратности/порядка. Совмещая полученное с~\eqref{eq:Logarithmic Derivative Is Delta Arg:Argument Principle}, заключаем
 \begin{align*}
  \Delta_{\partial\Omega} \arg f = 2\pi(N-P).
 \end{align*}
\end{proof}

Продемонстрируем применение принципа аргумента на конкретном примере.

\begin{exmpl}
 Докажите, что у функции~$ z \mapsto z^{2}-1 $ существует аналитический квадратный корень в области~$ \Omega = \CC \setminus [-1,1] $, то есть такая аналитическая в $ \Omega $ функция~$ f $, что $ f(z)^{2} = z^{2} - 1 $ всюду в $ \Omega $.
\end{exmpl}
\begin{proof}
 Сначала упростим задачу: найдём аналитический квадратный корень функции~$ h(z) = z^{2}-1 $ в односвязной области~$ \tilde\Omega = \CC \setminus [-1, +\infty) $. Так как функция~$ h $ не обнуляется в $ \tilde\Omega $, то по теореме~\ref{theorem:exist_log_f} у неё есть аналитическая ветвь логарифма $ \log h $ в $ \tilde\Omega $. Тогда функция
 \begin{align*}
  f(z) = e^{\frac{1}{2}\log h(z)}
 \end{align*} является аналитическим квадратным корнем функции~$h$ в области~$ \tilde\Omega $, что проверяется топорно:
 \begin{align*}
  f(z)^{2} = e^{\log h(z)} = h(z).
 \end{align*}

 Теперь покажем, что функцию~$ f $ можно аналитически продолжить в область~$ \Omega $. Для этого сначала непрерывно продолжим её на луч $ (1,+\infty) $.

 Зафиксируем точку $ x \in (1,+\infty) $. Так как функция~$ h $ непрерывна в окрестности точки~$ x $, и $ h(x) \neq 0 $, то у непрерывной ветви аргумента $ \arg h(z) $, порождённой ветвью логарифма $ \log h(z) $, есть предел при $ z \to x $ <<по верхнему берегу>>:
 \begin{align*}
  \lim_{\substack{z \to x \\ z \in \CC_+}} \arg h(z) = 2\pi k_0, \quad k_0\in\Z.
 \end{align*} При этом, не умаляя общности, можно считать
 \begin{align}
  \label{eq:Lim Upper Bank:Example Argument Principle}
  \lim_{\substack{z \to x \\ z \in \CC_+}} \arg h(z) = 0.
 \end{align} Аналогично, существует предел аргумента <<по нижнему берегу>>:
 \begin{align}
  \label{eq:Lim Lower Bank:Example Argument Principle}
  \lim_{\substack{z \to x \\ z \in \CC_-}} \arg h(z) = 2\pi k, \quad k \in \Z.
 \end{align}

 Рассмотрим кусочно-гладкую петлю~$ \gamma \colon\,[0,1] \to \tilde \Omega $, которая начинается в точке~$ x $, обходит отрезок~$ [-1,1] $ против часовой стрелки и завершается также в точке~$ x $ (риуснок~\ref{fig:example_analytic_square_path}).

 \begin{figure}[ht]
  \centering
  \incfig[0.6]{example_analytic_square_path}
  \caption{Путь из верхнего предела точки в нижний.}
  \label{fig:example_analytic_square_path}
 \end{figure}

 Непрерывная ветвь аргумента $ \arg h $ функции~$ h $ в области $ \tilde\Omega $ порождает непрерывную ветвь аргумента функции~$ h $ вдоль пути $ \gamma $ (учитывая тот факт, что $ \arg h $ имеет пределы~\eqref{eq:Lim Upper Bank:Example Argument Principle} и \eqref{eq:Lim Lower Bank:Example Argument Principle} по верхнему и нижнему берегу). В частности,
 \begin{align*}
  \lim_{\substack{z \to x \\ z \in \CC_-}}  \arg h(z) = \Delta_\gamma \arg h(z).
 \end{align*} Так как функция~$ h $ имеет два корня $ z=\pm 1 $ кратности~$ 1 $ в области, ограниченной путём $ \gamma $, то по принципу аргумента (теорема~\ref{theorem:Argument Principle}):
 \begin{align*}
  \Delta_\gamma \arg h(z) = 4\pi.
 \end{align*}

 Тогда, пользуясь \eqref{eq:Lim Upper Bank:Example Argument Principle} и \eqref{eq:Lim Lower Bank:Example Argument Principle}, запишем пределы функции~$ f $ по верхнему и нижнему берегу:
 \begin{align*}
  \lim_{\substack{z \to x \\ z \in \CC_+}}  f(z) &= \lim_{\substack{z \to x \\ z \in \CC_+}} e^{\frac{1}{2} \log h(z)} = \lim_{\substack{z \to x \\ z \in \CC_+}} e^{\frac{1}{2} \log \left| z^{2}-1 \right| + \frac{i}{2} \arg h(z)} = e^{\frac{1}{2} \log \left| x^{2}-1 \right| + \frac{i}{2} \cdot 0} = \sqrt{x^{2}-1},\\
  \lim_{\substack{z \to x \\ z \in \CC_-}} f(z) &= \lim_{\substack{z \to x \\ z \in \CC_-}} e^{\frac{1}{2}\log h(z)} = \lim_{\substack{z \to x \\ z \in \CC_-}} e^{\frac{1}{2} \log \left| z^{2}-1 \right| + \frac{i}{2} \arg h(z)} = e^{\frac{1}{2} \log\left| x^{2}-1\right| + \frac{i}{2} \cdot 4\pi} = \sqrt{x^{2}-1}.
 \end{align*} Так как пределы совпали, то функция $ f $ действительно непрерывно продолжается на луч $ (1,+\infty) $ по формуле $ f(x) = \sqrt{x^{2}-1} $.

 Осталось понять, что это продолжение аналитично в $ \Omega $. Действительно, по условию~\ref{enum2:theorem:cauchy_gurs_morer} теоремы~\ref{theorem:cauchy-gursa-morer} Коши-Гурса-Морера нам достаточно проверить замкнутость дифференциальной формы~$ f\,dz $. А замкнутость мы будем проверять <<тестом на прямоугольнике>> (условие~\ref{enum3:theorem:closed_1_form} теоремы~\ref{theorem:closed_1_form}).

 Интеграл формы~$ f\,dz $ по прямоугольникам, не касающихся луча $ (1,+\infty) $ (в том числе и строго пересекающих луч), и так равен нулю, потому что функция~$ f $ аналитична в $ \tilde\Omega $. А прямоугольники, касающиеся луча, мы будем приближать прямоугольником сверху (или снизу), интеграл по которому равен нулю (рисунок~\ref{fig:rectangle_test_example_analytic_square}). Пользуясь непрерывностью функции~$ f $, совершим предельный переход под знаком интеграла (по теореме Лебега о мажорируемой сходимости) и получим, что интеграл по касающемуся прямоугольнику также равен нулю.

 \begin{figure}[ht]
  \centering
  \incfig[0.6]{rectangle_test_example_analytic_square}
  \caption{Тест на прямоугольнике.}
  \label{fig:rectangle_test_example_analytic_square}
 \end{figure}
\end{proof}

\newpage
\section{Теорема Руше.}

Рассмотрим следующее приложение принципа аргумента.

\begin{thm}[Руше]
 \label{theorem:Rouche}
 Пусть $ \Omega \subset \CC $ --- стандартная область; функции $ f,g $ аналитичны в $ \Omega $, $ C^{1} $-гладко продолжаются на границу: $ f,g \in C^{1}(\overline\Omega) $, и выполнено строгое неравенство
 \begin{align}
  \label{eq:Bound on Border:Rouche Thm}
  \left| g(z) \right| < \left| f(z) \right|
 \end{align}
 всюду на $ \partial \Omega $. Тогда функции~$ f + g $ и $ f $ имеют одинаковое количество нулей в области~$ \Omega $ (с учётом кратности).
\end{thm}
\begin{proof}[\normalfont\textsc{Доказательство}]
 Заметим, что из~\eqref{eq:Bound on Border:Rouche Thm} в частности следует, что $ f(z) \neq 0 $ и $ f(z) + g(z) \neq 0 $ всюду на $ \partial\Omega $. Тогда по принципу аргумента (теорема~\ref{theorem:Argument Principle}):
 \begin{align*} 2\pi N_f = \Delta_{\partial\Omega} \arg f, && 2\pi N_{f+g} = \Delta_{\partial\Omega}\arg(f+g).
 \end{align*} где $ N_f $, $ N_{f+g} $ --- число нулей с учётом кратности функций $ f $ и $ f+g $ (соответственно) в области~$ \Omega $. Поэтому, нам достаточно доказать
 \begin{align*}
  \Delta_{\gamma_k} \arg f = \Delta_{\gamma_k} \arg (f+g)
 \end{align*} для всех путей $ \gamma_k $, из которых состоит граница $ \partial\Omega $. Распишем правую часть по формуле~\eqref{eq:Argument Along Paths Adds When Functions Multiply} (замечание~\ref{remark:Argument Along Paths Adds When Functions Multiply}):
 \begin{align*}
  \Delta_{\gamma_k} \arg (f+g)  = \Delta_{\gamma_k} \arg f + \Delta_{\gamma_k} \arg \left( 1 + \frac{g}{f}\right).
 \end{align*} Значит, достаточно показать
 \begin{align*}
  \Delta_{\gamma_k} \arg \left( 1 + \frac{g}{f} \right) = 0
 \end{align*} для всех путей~$ \gamma_k $. Заметим, что по условию
 \begin{align*}
  \Real \left(1 + \frac{g(\gamma_k(t))}{f(\gamma_k(t))}\right) \geqslant 1 - \left| \frac{g(\gamma_k(t))}{f(\gamma_k(t))} \right| > 0.
 \end{align*} Отсюда для всех $ t \in [0,1] $ имеем
 \begin{align*}
  \cos(\psi_k(t)) > 0,
 \end{align*} где $ \psi_k\colon\,[0,1]\to\R $ --- непрерывная ветвь аргумента функции~$ 1 + g(z) / f(z) $ вдоль пути~$ \gamma_k $. Следовательно, для каждого $ t \in [0,1] $
 \begin{align*}
  \psi_k(t) \in (-\pi / 2, \pi / 2) + 2\pi k(t),
 \end{align*} где $ k(t) \in \Z $ --- некоторая функция с целыми значениями. Иными словами,
 \begin{align*}
  \psi_k([0,1]) \subset \bigsqcup_{k\in\Z} (-\pi / 2, \pi / 2) + 2\pi k.
 \end{align*} Так как образ~$ \psi_k([0,1]) $ --- связное множество (как непрерывный образ связного), то оно лежит только в одном интервале-компоненте связности: существует целое $ k_0 \in \Z $:
 \begin{align*}
  \psi_k([0,1]) \subset (-\pi / 2, \pi / 2) + 2\pi k_0.
 \end{align*} В частности, получаем
 \begin{align*}
  \left| \psi_k(1) - \psi_k(0) \right| < \pi.
 \end{align*} С другой стороны,
 \begin{align*}
  \psi_k(1) - \psi_k(0) = 2\pi m, \quad m\in\Z
 \end{align*} поскольку путь $ \gamma_k $ замкнутый. Следовательно,
 \begin{align*}
  \Delta_{\gamma_k} \arg \left( 1 + \frac{g}{f} \right) = 0,
 \end{align*} что завершает доказательство теоремы Руше.
\end{proof}

\end{document}
