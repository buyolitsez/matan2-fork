% 2023.04.27 lecture 10
\documentclass[../complex-analysis.tex]{subfiles}
\begin{document}

\newpage
\section{Теорема Монтеля.}

Вернёмся к комплексному анализу. Последний шаг перед доказательством теоремы Римана --- это теорема Монтеля.

\begin{thm}[Монтеля]
 \label{theorem:montel}
 Пусть $ \mathbf S = \left\{f_\alpha\right\}_{\alpha \in A} $  --- семейство аналитических в области~$ \Omega \subset \CC $ функций, удовлетворяющее условию \emph{локальной равномерной ограниченности:} для любой точки~$ z \in \Omega $ существует такой замкнутый диск~$ \overline B(z,\eps(z)) \subset \Omega$, что
 \begin{align}
  \label{eq:Local Uniform Bound:Montel Theorem}
  \sup_{\substack{\zeta \in \overline B(z, \eps(z)) \\ \alpha \in A}} \left| f_\alpha(\zeta) \right| < +\infty.
 \end{align} Тогда данное семейство функций <<предкомпактно>>: из всякой последовательности функций $ \{f_{n}\}_{n=1}^{\infty} \subset \mathbf S  $ можно выделить подпоследовательность~$ \{f_{i_n}\}_{n=1}^{\infty}  $, сходящуюся равномерно на компактах в $ \Omega $.
\end{thm}
\begin{proof}[\normalfont\textsc{Доказательство}]
 Первым шагом локализуем теорему на окрестность некоторой точки. Во-первых, существует возрастающая последовательность компактов $ K_1 \subset K_2 \subset \ldots \subset \Omega $, приближающая снизу область~$ \Omega $:
 \begin{align*}
  \Omega = \bigcup_{n=1}^{\infty}K_n.
 \end{align*} Например, можно взять
 \begin{align*}
  K_n = \left\{ z \in \Omega : \mathop{\mathrm{dist}}(z, \CC \setminus \Omega) \geqslant 1 / n,\; \left| z \right| \leqslant n \right\}.
 \end{align*} Для каждого компакта $ K_n $ из его открытого покрытия $ \left\{B(z, \eps(z) / 2)\right\}_{z \in K_n} $, где $ \eps(z) $ взято из условия теоремы, можно выделить конечное подпокрытие
 \begin{align*}
  K_n \subset \bigcup_{j=1}^{N_n} B(z_j, \eps(z_j) / 2).
 \end{align*} Следовательно, существует такой счётный набор точек $ \{a_{n}\}_{n=1}^{\infty} \subset \Omega $, что
 \begin{align*}
  \Omega = \bigcup_{n=1}^{\infty} B(a_n, \eps(a_n) / 2).
 \end{align*}

 Достаточно показать, что для любой точки~$ a \in \Omega $ из последовательности $ \{f_{n}\}_{n=1}^{\infty}  $ можно выделить подпоследовательность, сходящуюся равномерно на $ B(a, \eps(a)/2) $. Тогда в силу диагонального аргумента Кантора существует подпоследовательность, которая равномерно сходится одновременно на всех дисках $ B(a_j, \eps(a_j) / 2) $. Эта подпоследовательность будет сходится равномерно на любом компакте $ K \subset \Omega $, так как любой компакт можно покрыть конечным числом дисков вида $ B(a, \eps(a) / 2) $.

 В таком случае зафиксируем точку~$ a \in \Omega $ и рассмотрим семейство функций~$\mathbf S$, суженное на компакт $ X = \overline{B}(a, \eps(a) / 2) $. Это семейство функций равномерно ограничено по условию~\eqref{eq:Local Uniform Bound:Montel Theorem}. Проверим, что оно равностепенно непрерывно. Для этого достаточно доказать, что производные равномерно ограничены:
 \begin{align}
  \label{eq:Bounded Derivative:Montel Theorem}
  \sup_{\substack{z_0 \in X,\, f \in \mathbf S}}\left| f'(z_0) \right| \leqslant C
 \end{align} для некоторого~$ C > 0 $; тогда по неравенству Лагранжа~\eqref{eq:lagrange_inequality} (теорема~\ref{theorem:Lagrange_inequality}) для всех $ f \in \mathbf S $, $ z_1,z_2 \in X $ будет верно
 \begin{align*}
  \left| f(z_1) - f(z_2) \right| \leqslant C \left| z_1 - z_2 \right|
 \end{align*} (можно даже сказать, что верна равностепенная липшецевость). Проверим \eqref{eq:Bounded Derivative:Montel Theorem} с помощью формулы~\eqref{eq:Derivative Cauchy Integral Formula} (лемма~\ref{lemma:Derivative Cauchy Integral Formula}), где интегрировать мы будем по окружности~$ C(a,\eps(a)) $ с центром в $ a $ и радиусом~$ \eps(a) $:
 \begin{align}
  \label{eq:Derivative Cauchy Integral Formula:Montel Theorem}
  \left| f'(z_0) \right| = \left| \frac{1}{2\pi i} \varointctrclockwise_{C(a,\eps(a))} \frac{f(z)\,dz}{(z-z_0)^{2}}   \right|.
 \end{align} Оценим~\eqref{eq:Derivative Cauchy Integral Formula:Montel Theorem} пока что только для точек $ z_0 \in C(a, \eps(a) / 2) $, пользуясь неравенством~\eqref{eq:bound_on_absolute_value_of_int}:
 \begin{align*}
  \left| f'(z_0) \right| \leqslant \frac{2\pi \cdot \eps(a)}{2\pi \cdot (\eps(a) / 2)^{2}} \cdot \max_{\substack{z \in \overline B(a,\eps(a)) \\ f \in \mathbf S}} \left| f(z) \right| \leqslant C < +\infty.
 \end{align*} Но тогда по принципу максимума (следствие~\ref{corollary:maximum_principle_border}) неравенство~\eqref{eq:Bounded Derivative:Montel Theorem} верно и для всех $ z_0 \in \overline B(a, \eps(a) / 2) $. Таким образом, семейство~$ \mathbf S $ равностепенно непрерывно. Тогда по теореме~\ref{theorem:Arzela-Ascoli} Арцела-Асколи из любой последовательности функций~$ \{f_{n}\}_{n=1}^{\infty} \subset \mathbf S $  можно выделить подпоследовательность, сходящуюся равномерно на компакте~$ X = \overline B(a,\eps(a) / 2) $.
\end{proof}

\begin{remrk}
 В условиях теоремы~\ref{theorem:montel} Монтеля предельная функция~$ f $ аналитична в области~$ \Omega $.
\end{remrk}
\begin{proof}[\normalfont\textsc{Доказательство}]
 По лемме~\ref{lemma:Lim of Analytic Functions is Analytic} предел последовательности аналитических в области~$ \Omega $ функций, сходящихся равномерно на компактах в $ \Omega $, является аналитической функцией.
\end{proof}

\newpage
\section{Теорема Римана.}

\subsection{Конформные отображения.}

\begin{df}
 Пусть $ \Omega_1, \Omega_2 \subset \CC $ --- области. Функция $ f \colon\; \Omega_1 \to \Omega_2 $ называется \emph{конформным отображением}, если функция~$ f $ аналитична и является биекцией между $ \Omega_1 $  и $ \Omega_2 $.
\end{df}

\begin{prop}\
 \label{proposition:conformal_map_group_properties}
 \begin{enumerate}
  \item Если $ f\colon\,\Omega_1 \to \Omega_2 $ и $ g\colon\,\Omega_2\to\Omega_3 $ --- конформные отображения, то их композиция $ (g \circ f) \colon\,\Omega_1 \to \Omega_3 $ --- конформное отображение.
  \item Если $ f \colon\,\Omega_1 \to \Omega_2 $ --- конформное отображение, то обратное отображение $ f^{-1}\colon\,\Omega_2\to\Omega_1 $ также является конформным.
 \end{enumerate}
\end{prop}
\begin{proof}[\normalfont\textsc{Доказательство}]\
 \begin{enumerate}
  \item Очевидно: композиция аналитичных функций аналитична (следствие~\ref{corollary:Simple Operations with Analytics Funs}).
  \item Теорема об обратном отображении из вещественного анализа.
 \end{enumerate}
\end{proof}

Физики очень любят конформные отображения, так как они сохраняют углы между кривыми. Конформная инвариантность очень важна в приложениях.

\begin{df}
 Области $ \Omega_1, \Omega_2 \subset \Omega $ называются \textit{конформно эквивалентными}, если существует конформное отображение $ f \colon\, \Omega_1 \to \Omega_2 $.
\end{df}
\begin{prop}
 Конформная эквивалентность является отношением эквивалентности, то есть
 \begin{itemize}
  \item область~$ \Omega $ конформно эквивалентна сама себе;
  \item если $ \Omega_1 $ конформно эквивалентна $ \Omega_2 $, то и $ \Omega_2 $ конформно эквивалентна $ \Omega_1 $;
  \item если $ \Omega_1 $ конформно эквивалентна $ \Omega_2 $, а $ \Omega_2 $ конформно эквивалентна $ \Omega_3 $, то и $ \Omega_1 $ конформно эквивалентна $ \Omega_3 $.
 \end{itemize}
\end{prop}
\begin{proof}[\normalfont\textsc{Доказательство}]
 Очевидно из предложения~\ref{proposition:conformal_map_group_properties}.
\end{proof}

В связи с данными определениями возникает естественный вопрос: какие области в $ \CC $ конформно эквивалентны, а какие нет? Приведём сначала отрицательные примеры, показывающие, что не все области конформно эквивалентны.

\begin{exmpl}
 Области $ \CC $ и $ \mathbb D $ не являются конформно эквивалентными.
\end{exmpl}
\begin{proof}[\normalfont\textsc{Доказательство}]
  Если бы существовало конформное отображение~$ f\colon\,\CC \to \mathbb D $, то по теореме \ref{theorem:liuvill} Лиувилля оно было бы постоянным, чего не может быть.
\end{proof}
\begin{exmpl}
 Области $ \mathbb D $ и $ \mathbb D \setminus \left\{ 0 \right\} $ не являются конформно эквивалентными, так как односвязность сохраняется при конформных отображениях.
\end{exmpl}
\begin{proof}[\normalfont\textsc{Доказательство}]
 Пусть область $ \Omega_1 \subset \Omega $ односвязна, и $ f \colon\, \Omega_1 \to \Omega_2 $ --- конформное отображение на область~$ \Omega_2 \subset \CC $. Пусть $ \gamma_0, \gamma_1 \colon\,[a,b] \to \Omega_2 $ --- два пути в $ \Omega_2 $, имеющие одинаковые начало и конец. Тогда пути $ \eta_0,\eta_1 \colon [a,b] \to \Omega_2 $, $ \eta_0 = f^{-1} \circ \gamma_0 $, $ \eta_1 = f^{-1} \circ \gamma_1 $ гомотопны в односвязной области~$ \Omega_1 $; пусть $ H \colon\,[0,1] \times [a,b] \to \Omega_1 $ --- их гомотопия. Тогда отображение $ f \circ H $ является гомотопией путей $ \gamma_0 $ и $ \gamma_1 $.
\end{proof}

Приведём теперь положительный пример.
\begin{exmpl}
 Области $ \left\{ z \in \CC : \Real z < 0 \right\} $ и $ \mathbb D $ конформно эквивалентны.
\end{exmpl}
\begin{proof}[\normalfont\textsc{Доказательство}]
 Для краткости обозначим $ \Omega = \left\{ z \in \CC : \Real z < 0 \right\} $. Возьмём любую точку~$ a \in \Omega $ и рассмотрим отображение
 \begin{align*}
  f(z) = \frac{z-a}{z+\overline a}.
 \end{align*} Проверим, что $ f $ и есть искомое конформное отображение $ \Omega \to \mathbb D $. Проверим сначала что оно действует в $ \mathbb D $:
 \begin{align*}
  1 - \left| f(z) \right|^{2} &= \frac{\left| z+\overline a \right|^{2} - \left| z-a \right|^{2}}{\left| z+\overline a \right|^{2}} = \frac{\left| z \right|^{2} + 2 \Real(za) + \left| a \right|^{2} -\left| z \right|^{2} + 2 \Real(z \overline a) - \left| a \right|^{2}}{\left| z+\overline a \right|^{2}} = \\
  &= \frac{2 \Real(z(a+\overline a))}{\left| z+\overline a \right|^{2}} = \frac{4\Real a \cdot \Real z}{\left| z+\overline a \right|^{2}} > 0, \quad \text{так как } \Real a < 0,\; \Real z < 0.
 \end{align*}
То есть $ \left| f(z) \right|< 1 $ для любого $ z \in \Omega$. Докажем теперь, что $ f $ обратима. Пусть $ w \in \mathbb D $. Найдём точку $ z \in \Omega $ такую, что
\begin{align*}
 &\frac{z-a}{z+\overline a} = w \iff z-a = wz + \overline a w \iff \\
 \iff\; &z(1-w) = a + \overline a w \iff z = \frac{a + \overline a w}{1 - w} = g(z).
\end{align*} Утверждается, что при $ w \in \mathbb D $ $ \Real z < 0 $:
\begin{align*}
 &\Real z < 0 \iff \Real \frac{a + \overline a w}{1 - w} < 0 \iff \frac{\Real ((1-\overline w)(a + \overline a w))}{\left| 1-w \right|^{2}} < 0 \iff \\
 \iff\;& \Real(a + \overline a w - a \overline w - \overline a \left| w \right|^{2}) < 0 \iff \Real(a - \overline a \left| w \right|^{2}) < 0 \iff \\
 \iff\;& \Real  a < \Real (\overline a \left| w \right|^{2}) \iff \Real a < \left| w \right|^{2} \Real a,
\end{align*} а последнее верно, так как $\Real a < 0$ и $ \left| w \right| < 1 $.

Таким образом, аналитическая функция~$ f $ действует в $ \mathbb D $, и обратима, значит это конформное отображение между $ \Omega $ и $ \mathbb D $.

\end{proof} 

\end{document}

