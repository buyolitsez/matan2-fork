% 2023.04.20 lecture 09
\documentclass[../complex-analysis.tex]{subfiles}
\begin{document}

Теорему Руше удобно применять в практических задачах.

\begin{exmpl}
 Найдём число корней многочлена $ p(z) = 0.1+2z^{3} + z^{10} = 0 $ в кольце~$ 1 < \left| z \right|  < 2 $ (с учётом кратности). Рассмотрим открытый диск~$ D_1 = B(0,1) $. Рассмотрим функции
 \begin{align*}
  f(z) = 2z^{3}, &&g(z) = 0.1 + z^{10}.
 \end{align*} Тогда на границе $ \left| z \right|=1 $ выполнено
 \begin{align*}
  \left| f(z) \right| = 2 > 1.1 \geqslant \left| g(z) \right|.
 \end{align*} Значит, по теореме~\ref{theorem:Rouche} Руше функция $f(z) + g(z) = p(z)$
 имеет столько же нулей в $ D_1 $, сколько и у $ f(z) $: то есть три нуля (с учётом кратности). Кроме того, на $ \left| z \right| = 1 $ нулей нет.

 А теперь посчитаем сколько нулей в диске $ D_2 =B(0,2) $. Но здесь выберем другие буквы $ f $ и $ g $:
 \begin{align*}
  f(z) = z^{10}, && g(z) = 0.1  + 2z^{3}.
 \end{align*} Тогда на границе $ \left| z \right|=2 $ выполнено $ \left| f \right| = 1024 $ и $ \left| g \right| < 0.1 + 16 $. Снова по теореме Руше у функции $ f(z)+g(z) = p(z) $ столько же нулей в $ D_2 $, сколько и у $ f(z) $: то есть $ 10 $ нулей (с учётом кратности).

 Итого, в кольце $ 1 < \left| z \right| < 2 $ будет $ 10-3 = 7 $ нулей.
\end{exmpl}

Теорема Руше часто используется в задачах теории возмущений. Другое приложение: оператор Шрёдингера --- про частицу в потенциальном поле.

\begin{crly}[теорема Гурвица]
 \label{corollary:Gurovitz}
 Пусть последовательность $ \{f_{n}\}_{n=1}^{\infty}  $ функций, аналитических в области~$ \Omega \subset \CC $, сходится равномерно на компактах в $ \Omega $ к функции~$ f $, не равной тождественно нулю в области~$ \Omega $. Пусть $ z_0 \in \Omega $ --- корень функции~$ f $ кратности $ k_0 \geqslant 1 $. Тогда существует такой открытый диск $ B(z_0,\eps) \subset \Omega $ с центром в $ z_0 $, что при достаточно больших~$ n $ функция $ f_n $ имеет ровно $ k_0 $ нулей с учётом кратности в диске~$ B(z_0,\eps) $.
\end{crly}

\begin{lm}
 \label{lemma:Lim of Analytic Functions is Analytic}
 Пусть последовательность $ \{f_{n}\}_{n=1}^{\infty}  $ функций, аналитических в области~$ \Omega \subset \CC $, сходится равномерно на компактах в $ \Omega $ к функции~$ f $. Тогда функция~$ f $ аналитична в области~$ \Omega $.
\end{lm}
\begin{proof}[\normalfont\textsc{Доказательство}]
 Проверим с помощью <<теста на прямоугольнике>>. Для любого замкнутого прямоугольника~$ \Pi \subset \Omega $ имеем
 \begin{align*}
  \int_{\partial\Pi} f\,dz = \lim_{n \to \infty} \int_{\partial\Pi} f_n\,dz = \lim_{n \to \infty} 0 = 0,
 \end{align*} по теореме Лебега о мажорируемой сходимости, так как функции~$ f_n $ сходятся равномерно к $ f $ на компакте $ \partial\Pi \subset \Omega $.
\end{proof}

\begin{proof}[\normalfont\textsc{Доказательство следствия~\ref{corollary:Gurovitz}}]
 По лемме~\ref{lemma:Lim of Analytic Functions is Analytic} функция~$ f $ аналитична. По лемме~\ref{lemma:zero_multiplicity} о кратности нуля $ f(z) = (z - z_0)^{k_0}g(z) $, где функция~$ g $ аналитична в~$ \Omega $, и при этом $ g(z_0) \neq 0 $. Тогда существует такой замкнутый диск $ \overline B(z_0, \eps) \subset \Omega $ на котором выполнено $\left| g(z) \right| > \delta > 0$ для некоторого числа~$ \delta>0 $. В частности, на окружности $ C_\eps = \left\{ z \in \CC : \left| z-z_0 \right|=\eps \right\} $ выполнено
 \begin{align*}
  \left|f(z) \right| > \delta \eps^{k_0}.
 \end{align*} Обозначим $ r_n(z) = f_n(z) - f(z) $. Так как $ f_n \rightrightarrows f $ равномерно на компакте~$ C_\eps \subset \Omega $, то для числа $ \delta \eps^{k_0} / 2 $ существует число $ N \geqslant 1 $ такое, что для всех $ n > N $ и для всех $ z \in C_\eps $ верно
 \begin{align*}
  \left|r_k(z) \right| < \frac{\delta \cdot \eps^{k_0}}{2} < \left| f(z) \right|.
 \end{align*} По теореме~\ref{theorem:Rouche} Руше функции $ f(z) $ и $ f(z) - r_n(z) = f_n(z) $ имеет одинаковое число нулей в открытом диске $ B(z_0,\eps) $ (то есть ровно $ k_0 $ нулей) с учётом кратности при всех $ n > N $.
\end{proof}

На самом деле мы доказали чуть больше: для каждого $ \eps > 0 $, меньшего некоторого заведомо известного $ \eps_0 > 0 $, существует $ N $, при котором функции~$ f_n $ имеет ровно $ k_0 $ нулей в $ B(z_0,\eps) $ с учётом кратности при всех $ n > N $.

\begin{thm}
 \label{theorem:analytic_injective}
Пусть последовательность $ \{f_{n}\}_{n=1}^{\infty}  $ функций, аналитических и инъективных в области~$ \Omega \subset \CC $, сходится равномерно на компактах в $ \Omega $ к функции~$ f $. Тогда либо $ f $ постоянна, либо инъективна на $ \Omega $.
\end{thm}
\begin{proof}[\normalfont\textsc{Доказательство}]
 По лемме~\ref{lemma:Lim of Analytic Functions is Analytic} функция~$ f $ аналитична. Предположим, что $ f $ не постоянна и не инъективна: существуют различные точки $ z_0,w_0 \in \Omega $ такие, что $ f(z_0) - f(w_0) = 0 $. Рассмотрим функции
 \begin{align*}
  F_n(z) &= f_n(z) - f_n(w_0),\\
  F(z) &= f(z) - f(w_0).
 \end{align*} От прибавления константы ничего не поменялось: функции~$ F_n $ также аналитичны в $ \Omega $, и сходятся равномерно на компактах к функции~$ F $, аналитичной в $ \Omega $. Но при этом $ F(z_0) = 0 $.

 Возьмём такое число $ \eps > 0 $, чтобы $ \eps < \left| z_0 - w_0 \right| / 2 $, и достаточно маленькое для того, чтобы выполнялась теорема Гурвица (следствие~\ref{corollary:Gurovitz}) для функций $ F_n $ и $ F $. Тогда при достаточно больших $ n $ функция~$ F_n $ имеет нуль в открытом диске~$ B(z_0,\eps) $: существует точка $ z_n \in B(z_0,\eps) $, для которой $ f_n(z_n) = f_n(w_0) $. Но так как функции~$ f_n $ инъективные, то $z_n = w_0$, а это противоречие!
\end{proof}

\begin{thm}
 \label{theorem:Derivative of Injective Analytic Fun is not 0}
 Пусть $ f $ --- инъективная аналитическая функция в области~$ \Omega \subset \CC $. Тогда $ f'(z) \neq 0 $ для всех $ z \in \Omega $.
\end{thm}

Теорему~\ref{theorem:Derivative of Injective Analytic Fun is not 0} можно доказать и вещественными методами, рассматривая $ f $ как функцию $ f\colon\,\Omega \subset \R^{2} \to \R^{2} $.

\begin{proof}[\normalfont\textsc{Доказательство}]
 Пусть $ z_0 \in \Omega $. Рассмотрим функции
 \begin{align*}
  f_n(z) = \frac{f\left(z + 1/n\right) - f(z)}{1 / n}, \quad n \geqslant 1,
 \end{align*} они заданы при достаточно больших $ n $ в замкнутом диске~$ \overline B(z_0, 2\eps) \subset \Omega $, и аналитичны в открытом диске~$ B(z_0,2\eps) $. Покажем, что $ f_n \rightrightarrows f' $ равномерно на компакте~$ \overline B(z_0,\eps) $. Перепишем
 \begin{align*}
  f_n(z) = g_z(1 / n),
 \end{align*} где
 \begin{align*}
  g_z(\zeta) = \frac{f(z + \zeta) - f(z)}{\zeta}, &&g_z(0) = f'(z).
 \end{align*} Для каждого $ z \in \overline B(z_0,\eps) $ функция~$ g_z $ определена и аналитична в открытом диске~$ B(0,\eps) $ (аналитичность в нуле получается с помощью леммы~\ref{lemma:ob_ustranenii_osobennosti} об устранении особенности, см. доказательство теоремы~\ref{theorem:cauchy-gursa-morer} Коши-Гурса-Морера). Тогда по неравенству Лагранжа~\eqref{eq:lagrange_inequality} (теорема~\ref{theorem:Lagrange_inequality}):
 \begin{align}
  \label{eq:Bound 1:Derivative of Injective Analytic Fun is not 0}
  \left|f_n(z) - f'(z) \right| = \left|g_z(1 / n) - g_z(0) \right| \leqslant \frac{1}{n} \cdot \max_{\left| \zeta \right| \leqslant \eps} \left| g'_z(\zeta) \right| = \frac{1}{n} \cdot \max_{\left| \zeta \right|=\eps} \left| g_z'(\zeta) \right|,
 \end{align} где последний переход справедлив по принципу максимума (следствие~\ref{corollary:maximum_principle_border}). Продифференцируем функцию~$ g_z $:
  \begin{align*}
  g'_z(\zeta) = \frac{f'(z + \zeta) \cdot \zeta - f(z+\zeta)+f(z)}{\zeta^{2}}.
 \end{align*} Продолжим оценку~\eqref{eq:Bound 1:Derivative of Injective Analytic Fun is not 0}:
 \begin{align*}
  \left| f_n(z) - f'(z) \right| \leqslant \frac{1}{n} \cdot \max_{\left| \zeta \right|=\eps} \left| g'_z(\zeta) \right| \leqslant \frac{1}{n\eps^{2}} \cdot \max_{\left| \lambda - z_0 \right| \leqslant 2\eps} \left( \left| f'(\lambda) \right| \cdot \eps + 2 \left| f(\lambda) \right| \right) = \OO(1 / n).
 \end{align*} Равномерная сходимость на компакте $ \overline B(z_0,\eps) $ доказана. 
 
 Теперь предположим, что $ f'(z_0) = 0 $. Тогда по теореме Гурвица (следствие~\ref{corollary:Gurovitz}) при при достаточно больших $ n $ функция $ f_n $ имеет корень, то есть $ f(z + 1 / n)  = f(z)$, а этого не может быть, так как функция~$ f $ инъективна!
\end{proof}

\newpage
\section{Теорема Арцела-Асколи.}

Нашей целью в курсе комплексного анализа будет доказательство теоремы Римана --- по сути самой красивой теоремы в данном курсе.

Для этого нам придётся немного отвлечься от теории аналитических функций. В этом параграфе мы займёмся компактностью, и докажем классическую теорему из функционального анализа.

\begin{df}
 Пусть $ (X,\rho) $ --- метрическое пространство. Говорят, что семейство непрерывных функций $ \left\{f_\alpha\colon X \to \CC\right\}_{\alpha \in A}  $ \emph{равностепенно непрерывно}, если для любого $ \eps > 0 $ существует $ \delta > 0 $ такое, что из $ \rho(x_1, x_2) < \delta $ следует
 \begin{align}
  \left|f_\alpha(x_1) - f_\alpha(x_2) \right| < \eps \label{eq:equicontinuity}
 \end{align} для всех $ \alpha \in A $.
\end{df}
\begin{df}
 Говорят, что семейство функций $ \left\{f_\alpha\colon X \to \CC \right\}_{\alpha \in A}  $ \textit{равномерно ограничено}, если
 \begin{align*}
  \sup_{x \in X, \alpha \in A} \left| f_\alpha(x) \right| < \infty.
 \end{align*}
\end{df}

\begin{lm}
 \label{lemma:metric_compact_is_seperable_space}
 Любой метрический компакт сепарабельный.
\end{lm}
\begin{proof}[\normalfont\textsc{Доказательство}]
 Возьмем покрытие метрического компакта $X$ открытыми шарами радиуса $1/2$, извлечем конечное подпокрытие по компактности. Затем возьмём покрытие $X$ открытыми шарами радиуса $1/3$ и опять извлечем конечное подпокрытие. Проделаем эту процедуру для всех $ n \in \N $.

 Тогда центры этих шаров образуют счётное всюду плотное множество. Действительно, для каждой точки $ x \in X $ и для любого $\eps > 0$ найдется шар радиуса меньше~$ \eps$ в котором точка~$ x $ лежит; следовательно найдется центр на расстоянии меньше~$\eps$. 
\end{proof}

\begin{df}
 Подмножество~$ Y \subset X $ метрического пространства~$ X $ называется \emph{предкомпактным}, если его замыкание $ \overline Y $ --- компактное пространство.

 В терминах секвенциальной компактности, предкомпактность означает, что из любой последовательности $ y_1, y_2, \ldots \in Y $ можно выделить подпоследовательность $ y_{j_1}, y_{j_2}, \ldots $, сходящуюся к $ x \in X $. В отличие от компактности, мы не требуем, чтобы $ x \in Y $.
\end{df}

\begin{thm}[Арцела-Асколи]
 Пусть $ (X,\rho) $ --- метрический компакт, и $ C(X,\CC) $ --- банахово пространство непрерывных функций $ f \colon\,X \to \CC $ с $ \sup $-нормой:
 \begin{align*}
  \left\| f \right\| = \max_{x \in X} \left| f(x) \right|, \quad f \in C(X,\CC).
 \end{align*} Пусть $S \subset C(X,\CC)$ --- равномерно ограниченное и равностепенно непрерывное семейство непрерывных функций. Тогда семейство~$ S $ предкомпактно, то есть из любой последовательности функций $\{f_{n}\}_{n=1}^{\infty} \subset S$  можно выделить равномерно сходящуюся подпоследовательность $\{f_{i_n}\}_{n=1}^{\infty} $:
 \begin{align*}
  f_{i_n} \rightrightarrows f,
 \end{align*} где $ f \in C(X,\CC) $.
\end{thm}
\begin{proof}[\normalfont\textsc{Доказательство}]
 Пусть $\{f_{n}\}_{n=1}^{\infty} \subset S $ --- произвольная последовательность функций из семейства~$ S $.

 \begin{enumerate}
  \item Так как пространство~$ X $ сепарабельно по лемме~\ref{lemma:metric_compact_is_seperable_space}, то существует счётное всюду плотное подмножество $ \{x_{k}\}_{k=1}^{\infty} \subset X  $. Наша задача: найти подпоследовательность, которая сходится во всех точках $ x_j $ из этого счётного всюду плотного подмножества.

   Так как семейство~$ S $  равномерно ограничено, то из последовательности функций $ \{f_{n}\}_{n=1}^{\infty}  $ можно выделить подпоследовательность $ \{f_{n}^{(1)}\}_{n=1}^{\infty}  $, сходящуюся в точке~$ x_1 $. Из неё так же можно выделить подпоследовательность $ \{f_{n}^{(2)}\}_{n=1}^{\infty}  $, сходящуюся в точке~$ x_2 $, причём подпоследовательность всё ещё будет сходится в точке~$ x_1 $. Продолжая эту процедуру, для каждого  $ m \geqslant 1 $  мы найдём подпоследовательность $ \{f_{n}^{(m)}\}_{n=1}^{\infty}  $ исходной последовательности, которая сходится в точках $ x_1, x_2, \ldots, x_m $.

   Теперь мы проведём рассуждение, аналогичное диагональному методу Кантора: рассмотрим последовательность функций~$ \{f_{n}^{(n)}\}_{n=1}^{\infty}  $ (рисунок~\ref{fig:cantor_diagonalization_theorem_arcel_askol}). Эта последовательность по построению является подпоследовательностью исходной последовательности $ \{f_{n}\}_{n=1}^{\infty}  $, а также она сходится в любой точке $ x_j $; в самом деле, начиная с элемента $ f_j^{(j)} $ хвост будет являться подпоследовательностью последовательности $ \{f_{n}^{j}\}_{n=1}^{\infty}  $, поэтому хвост будет сходится в точке~$ x_j $.

   \begin{figure}[ht]
    \centering
    \incfig[0.5]{cantor_diagonalization_theorem_arcel_askol}
    \caption{Диагональный метод Кантора.}
    \label{fig:cantor_diagonalization_theorem_arcel_askol}
   \end{figure}

   Итак, мы нашли подпоследовательность $\{\hat f_{n}\}_{n=1}^{\infty} $, которая сходится во всех точках $ x_j $.

  \item Докажем, что та же самая подпоследовательность $ \{\hat f_{n}\}_{n=1}^{\infty}  $ сходится во всех точках $ x \in X $.

   Зафиксируем точку~$ x \in X $. Так как пространство~$ \CC $  полное, нам достаточно доказать, что последовательность чисел $\{\hat f_{n}(x)\}_{n=1}^{\infty} \subset \CC$  фундаментальна. Возьмём любое число~$ \eps > 0 $, и докажем, что при достаточно больших $ m,n $ выполнено
   \begin{align*}
    \left| \hat f_m(x) - \hat f_n(x) \right| < \eps.
   \end{align*} Воспользуемся $ \eps / 3 $-приёмом:
   \begin{align*}
    \left| \hat f_m(x) - \hat f_n(x) \right| \leqslant \left| \hat f_m(x) - \hat f_m(x_j) \right| + \left| \hat f_m(x_j) - \hat f_n(x_j) \right| + \left| \hat f_n(x_j) - \hat f_n(x) \right|.
   \end{align*} Так как семейство~$ S $ равностепенно непрерывно, то для числа~$ \eps / 3 $ существует такое число~$ \delta>0 $, что для любого $ n \geqslant 1 $ и для любых точек~$ p,q \in X $ условие~$ \rho(p,q) < \delta $ влечёт $ \left| \hat f_n(p) - \hat f_n(q) \right| < \eps / 3$. Поэтому, найдём точку $ x_j $ такую, что $ \rho(x,x_j) < \delta $. Тогда
   \begin{align*}
    \left| \hat f_m(x) - \hat f_n(x) \right| \leqslant 2\eps / 3 + \left| \hat f_m(x_j) - \hat f_n(x_j) \right| < \eps
   \end{align*} при достаточно больших $ n $ и $ m $, так как последовательность $\{\hat f_{n}(x_j)\}_{n=1}^{\infty} $  сходится. Таким образом, последовательность $ \{\hat f_{n}\}_{n=1}^{\infty}  $ сходится всюду в $ X $ к некоторой функции $ f \colon X \to \CC $.

  \item Осталось воспользоваться тем фактом, что сходящаяся последовательность непрерывных на компакте функций сходится равномерно, и её предел --- тоже непрерывная функция. Этот факт был доказан в первом семестре анализа, но доказательство также было дано и на лекциях: см. лемму~\ref{lemma:Convergence of Functions on Compact}.
 \end{enumerate}
\end{proof}

\end{document}

