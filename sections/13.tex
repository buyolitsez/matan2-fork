% 2022.11.25 Lecture 13

\section{Мера Лебега на поверхностях}

До этого параграфа в пространстве $\R^{n}$ мы умели измерять только $n$-мерные объекты. Мы не могли посчитать, например, длину кривой на плоскости, или площадь поверхности в пространстве.

В этом параграфе мы построим меру Лебега на поверхностях. С помощью неё мы сможем, например, измерить площадь сферы.

\begin{df}
 \label{definition:surface}
 $(\Omega, \Phi, S)$ --- \textit{поверхность}, если $\Omega \subset \R^{k}$ --- область, $S \subset \R^{n}$, $\Phi \in C^{1}(\Omega, S)$ --- гладкая биекция, а также $\rk d_{x} \Phi = k$ для любой точки $x \in \Omega$ (дифференциал всюду инъективен).
\end{df}
\begin{remrk*}
 Часто необходимо интегрировать (вычислять площадь) по многообразиям (например, по сфере), но их всегда можно разбить на несколько поверхностей. Поэтому в этом параграфе изучаем только поверхности.
\end{remrk*}

Прежде чем определять меру Лебега на поверхностях, определим её естественным образом на линейных подпространствах $\R^{n}$.

\begin{df}
 Пусть $V \subset \R^{n}$ --- линейное подпространство размерности $k$. \textit{Мера Лебега} $\lambda_k^{V} \colon\, \B(V) \to [0, \infty]$ в подпространстве $V$ задаётся по формуле
 \begin{align*}
  \lambda_k^{V}(E) = \lambda_k(UE)
 ,\end{align*} где $U \colon\, \R^{n} \to \R^{n}$ --- ортогональный линейный оператор такой, что $UV = \R^{k}$. Здесь $\R^{k}$ рассматривается как линейное подпространство $\R^{n}$, состоящее из векторов вида $(x_1, \ldots, x_k, 0, \ldots, 0)$.
\end{df}

Меру Лебега на подпространстве $V$ можно выразить в терминах обратимого линейного отображения $L \colon\, \R^{k} \to V$.

\begin{lm}
 \label{lemma:lebesgue_measure_on_subspace}
 Пусть $L \colon\, \R^{k} \to \R^{n}$ --- линейное отображение, $\rk L = k$, а $V = \Imaginary L$ --- его образ. Тогда для любого $E \in \B(\R^{k})$ верно
 \begin{align}
  \label{equation:lebesgue_measure_on_subspace}
  \lambda_k^{V}(LE) = \sqrt{\det L^{\top}L} \cdot \lambda_k(E)
 .\end{align} 
\end{lm}
\begin{proof}[\normalfont\textsc{Доказательство}]
 Пусть $U \colon\, \R^{n} \to \R^{n}$ --- ортогональный оператор такой, что $UV = \R^{k}$. Тогда
 \begin{align*}
  \lambda_k^{V}(LE) &= \lambda_k(ULE) = \left| \det (UL) \right| \cdot \lambda_k(E) = \\
  &= \sqrt{\det(UL)\det(UL)} \cdot \lambda_k(E) = \\
  &= \sqrt{\det(L^{\top}U^{\top})\det(UL)} \cdot \lambda_k(E) = \\
  &= \sqrt{\det(L^{\top}U^{\top}UL)} \cdot \lambda_k(E) = \\
  &= \sqrt{\det L^{\top}L} \cdot\lambda_k(E)
 .\end{align*} 
\end{proof}

Теперь определим меру Лебега на поверхности аксиоматически.

\begin{df}
 \label{definition:lebesgue_measure_on_surface}
 Пусть $(\Omega, \Phi, S)$ --- поверхность, $\B(S)$ --- борелевская $\sigma$-алгебра на поверхности $S$ (как на топологическом пространстве). Пусть $\mu$ --- мера на $\B(S)$, {\color{red} конечная на компактах (зачем?)}.

 Будем говорить, что $\mu$ --- это \textit{поверхностная мера Лебега} на $S$, если:
 \begin{itemize}
  \item $\mu(\Phi(e)) = 0$ для любого $e \subset \Omega$ такого, что $\lambda_k(e) = 0$.
  \item Мера образа шара $B(x,r) \subset \Omega$ равна 
   \begin{align}
    \label{equation:surface_lebesgue_measure_ball}
    \mu(\Phi(B(x,r))) = \lambda_k^{T_{\Phi(x)}}\Bigl((d_x \Phi)(B(0, r))\Bigr) + o(r^{k})
   \end{align}
   при $r \to 0$, где $T_{\Phi(x)}$ --- касательное пространство к поверхности $S$ в точке $\Phi(x)$ (оно же есть образ дифференциала $d_x \Phi$). См. рисунок \ref{fig:ball_image_on_surface}.
 \end{itemize}
\end{df}

\begin{figure}[ht]
 \centering
 \incfig{ball_image_on_surface}
 \caption{Образ шара на поверхности.}
 \label{fig:ball_image_on_surface}
\end{figure}

\begin{thm}
 \label{theorem:lebesgue_measure_on_surface_formula}
 Пусть $(\Omega, \Phi, S)$  --- поверхность.

 Тогда мера Лебега на ней существует и задаётся формулой
 \begin{align}
  \label{equation:lebesgue_measure_on_surface_formula}
  \mu(\Phi(E)) = \int\limits_{E} \sqrt{ \det G_{\Phi}(x) } \, d\lambda_k(x)
 \end{align} для всех $E \in \B(\Omega)$, где
 \begin{align*}
  G_{\Phi}(x) = J_{\Phi}^{\top}(x) J_{\Phi}(x)
 .\end{align*}  $G_{\Phi}(x)$ называется \textit{матрицей Грама} отображения $\Phi$ в точке $x$.
\end{thm}
Формула \eqref{equation:lebesgue_measure_on_surface_formula} есть в точности формула \eqref{equation:lebesgue_measure_on_subspace}, проинтегрированная по множеству $E$, где в каждой точке в качестве линейного отображения $L$ мы берём дифференциал $\Phi$ в этой точке.
\begin{proof}\
 \begin{enumerate}
  \item Сначала покажем, что $\mu$ действительно задана на $\B(S)$. Для этого достаточно показать, что $\Phi(\B(\Omega)) \supset \B(S)$. Действительно, $\B(S)$ --- наименьшая $\sigma$-алгебра, содержащая открытые множества пространства $S$, а $\Phi(\B(\Omega))$ --- какая-то $\sigma$-алгебра, содержащая открытые множества $S$. $\Phi(\B(\Omega))$ --- $\sigma$-алгебра, так как $\Phi$ --- биекция ($\Phi(\varnothing) = \varnothing$; $\Phi(\bigcup A_k) = \bigcup \Phi(A_k)$; $ \Phi(A^c) = \Phi(A)^c$). $\Phi(\B(\Omega))$ содержит открытые множества пространства $S$, так как прообраз открытого открыт при непрерывном отображении.

  \item $\mu(\Phi(e)) = 0$, если $\lambda_k(e) = 0$. Действительно,
   \begin{align*}
    \mu(\Phi(e)) = \int\limits_{e} \sqrt{\det G_{\Phi}(x)} \, d\lambda_k = 0
   ,\end{align*} так как $e$ --- множество нулевой меры.

  \item Проверим равенство \eqref{equation:surface_lebesgue_measure_ball} для шара $B(x_0, r) \subset \Omega$:
   \begin{align*}
    \mu(\Phi(B(x_0,r))) &= \int\limits_{B(x_0, r)} \sqrt{\det G_{\Phi}(x)} \, d\lambda_k(x) = \\
    &= [G_{\Phi} \text{ --- непрерывное отображение на } \Omega] = \\
    &= \int\limits_{B(x_0,r)} \left( \sqrt{ \det G_{\Phi}(x_0) } + o(1) \right) \, d\lambda_{k}(x) = \\
    &= \sqrt{\det G_{\Phi}(x_0)} \lambda_k(B(x_0, r)) + o(\lambda_k(B(x_0, r))) = \\
    &= \sqrt{ \det G_{\Phi}(x_0)  } \lambda_k(B(0, r)) + o(r^{k})
   .\end{align*} Теперь применим формулу \eqref{equation:lebesgue_measure_on_subspace} для $L = J_{\Phi}(x_0)$ и $E = B(0,r)$:
   \begin{align*}
    \mu(\Phi(B(x_0,r))) = \lambda_k^{T_{\Phi(x_0)}} \Bigl( (J_{\Phi}(x_0))(B(0,r)) \Bigr) + o(r^{k})
   .\end{align*} 

  \item Покажем, что мера Лебега на поверхности определяется своими свойствами: то есть если она удовлетворяет определению \ref{definition:lebesgue_measure_on_surface}, то выполняется формула \eqref{equation:lebesgue_measure_on_surface_formula}.

   Рассмотрим меру
   \begin{align*}
    \nu(E) = \mu(\Phi(E)), \quad E \in \B(\Omega)
   .\end{align*} По первому свойству для множества $e \in \B(\Omega)$ условие $\lambda_k(e) = 0$ влечёт  $\nu(e) = 0$. Иными словами, мера $\nu$ непрерывна относительно $\lambda_k$: $\nu \prec \prec \lambda_k$. Тогда по теореме \ref{theorem:radon_nikodim} Радона-Никодима
   \begin{align*}
    \nu(E) = \int\limits_{E} f(x) \, d\lambda_k(x)  
    \end{align*} для некоторой измеримой функции $f \geqslant 0$. По теореме \ref{theorem:almost_all_points_are_lebesgue_points} для $\lambda_k$-почти всех точек $x \in \Omega$ верно \begin{align*}
    f(x) = \lim_{r \to 0} \frac{\nu(B(x,r))}{\lambda_k(B(x,r))}
   .\end{align*} {\color{red}Тут та же проблема, что может оказаться $f \notin L^{1}(\Omega, \lambda_k)$.} Но тогда
   \begin{align*}
    f(x) &= \lim_{r \to 0}  \frac{\mu(\Phi(B(x,r)))}{\lambda_k(B(x,r))} = \\
    &= \lim_{r \to 0}  \frac{\lambda_k^{T_{\Phi(x)}}\Bigl((J_{\Phi}(x))(B(0,r))\Bigr) + o(r^{k})}{\lambda_k(B(0,r))} = \\
    &= \lim_{r \to 0} \frac{\sqrt{\det J_{\Phi}^{\top}J_{\Phi}(x)} \lambda_k(B(0,r)) + o(r^{k})}{\lambda_k(B(0,r))} = \\
    &= \sqrt{\det G_{\Phi}(x)}
   .\end{align*}  Формула \eqref{equation:lebesgue_measure_on_surface_formula} проверена.
 \end{enumerate}
\end{proof}

\begin{remrk*}
 Мы не будем доказывать, что мера Лебега на поверхности не зависит от параметризации (хорошее упражнение).
\end{remrk*}

Теперь рассмотрим более конкретные примеры --- приложения теоремы \ref{theorem:lebesgue_measure_on_surface_formula}.

\begin{exmpl}[длина гладкой кривой]
 \label{example:length_of_smooth_curve}
 Пусть $\Phi \colon\, (a, b) \to \R^{n}$ --- гладкая кривая без самопересечений, $\mu$ --- мера Лебега на этой кривой. Тогда для любого $E \in \B((a,b))$:
 \begin{align*}
  \mu(\Phi(E)) &= \int\limits_{E} \left\| \Phi'(t) \right\| \, d\lambda_1(t) = \\
  &= \int\limits_{E} \sqrt{ \Phi_1'(t)^{2} + \ldots + \Phi_n'(t)^{2} } \, d\lambda_1(t)
 .\end{align*} Это в точности формула длины гладкой кривой из второго семестра.
\end{exmpl}
\begin{proof}
 Вычислим матрицу Якоби:
 \begin{align*}
  J_{\Phi}(t) = \begin{pmatrix}
   \Phi_1'(t) \\
   \vdots \\
   \Phi_n'(t) 
  \end{pmatrix}
 .\end{align*} Вычислим матрицу Грама:
 \begin{align*}
  G_{\Phi}(t) &= J^{\top}_{\Phi} J_{\Phi} = \begin{pmatrix}
   \Phi_1'(t) & \ldots & \Phi_n'(t)
   \end{pmatrix} \cdot \begin{pmatrix}
   \Phi_1'(t) \\
   \vdots \\
   \Phi_n'(t) 
  \end{pmatrix} = \left(\Phi'_1(t)^{2} + \ldots + \Phi'_n(t)^{2}\right).
 \end{align*} --- матрица $1 \times 1$.
\end{proof}

\begin{exmpl}[площадь двумерной поверхности в $\R^{3}$]
 Пусть $(\Omega, \Phi, S)$ --- двумерная поверхность в $\R^{3}$ ($\Omega \subset \R^{2}$ --- область, $\Phi \colon\, \Omega \to S$ --- гладкая биекция, $\rk d_x \Phi = 2$, $S \subset \R^{3}$), а $\mu$ --- мера Лебега на этой поверхности. Тогда для любого $A \in \B(\Omega)$:
 \begin{align*}
  \mu(\Phi(A)) = \int\limits_{A} \sqrt{ EG - F^{2} } \, d\lambda_2
  ,\end{align*}  где \begin{align*}
  E &= \left\langle \Phi'_u, \Phi'_u \right\rangle  = \Phi_{1u}^{'2} + \Phi_{2u}^{'2} + \Phi_{3u}^{'2}, \\
  F &= \left\langle \Phi'_u, \Phi'_v \right\rangle = \Phi_{1u}' \Phi_{1v}' + \Phi_{2u}' \Phi_{2v}' + \Phi_{3u}' \Phi_{3v}', \\
  G &= \left\langle \Phi'_v, \Phi'_v \right\rangle  = \Phi_{1v}^{'2} + \Phi_{2v}^{'2} + \Phi_{3v}^{'2}.
 \end{align*} 
\end{exmpl}
\begin{proof}
 Вычислим матрицу Якоби:
 \begin{align*}
  J_{\Phi} = \begin{pmatrix}
   \Phi_{1u}' & \Phi_{1v}' \\
   \Phi_{2u}' & \Phi_{2v}' \\
   \Phi_{3u}' & \Phi_{3v}'
  \end{pmatrix}
 .\end{align*}  Посчитаем матрицу Грама:
 \begin{align*}
  G_{\Phi} = J^{\top}_{\Phi} J_{\Phi} = \begin{pmatrix}
   E & F \\
   F & G
  \end{pmatrix}
 ,\end{align*}  где
 \begin{align*}
  E &= \left\langle \Phi'_u, \Phi'_u \right\rangle  = \Phi_{1u}^{'2} + \Phi_{2u}^{'2} + \Phi_{3u}^{'2}, \\
  F &= \left\langle \Phi'_u, \Phi'_v \right\rangle = \Phi_{1u}' \Phi_{1v}' + \Phi_{2u}' \Phi_{2v}' + \Phi_{3u}' \Phi_{3v}', \\
  G &= \left\langle \Phi'_v, \Phi'_v \right\rangle  = \Phi_{1v}^{'2} + \Phi_{2v}^{'2} + \Phi_{3v}^{'2}.
 \end{align*} Тогда
 \begin{align*}
  \sqrt{\det G_{\Phi}} = \sqrt{EG-F^{2}}
 .\end{align*} 
\end{proof}

\begin{exmpl}[график функции]
 Пусть $(\Omega,\Phi,S)$ --- поверхность размерности $n-1$ в $\R^{n}$, где параметризация $\Phi \colon\, \Omega \to \R^{n}$ имеет вид
 \begin{align*}
  \Phi(x_1,\ldots,x_{n-1}) = (x_1, \ldots, x_{n-1},\; g(x_1,\ldots,x_{n-1}))
 \end{align*} для некоторой функции $g \colon\, \Omega \to \R$. Такая поверхность называется \textit{графиком функции $g$}. Пусть $\mu$ --- мера Лебега на этой поверхности. Тогда для любого $E \in \B(\Omega)$:
 \begin{align*}
  \mu(\Phi(E)) = \int\limits_{E} \sqrt{1 + \left\| \nabla_x g \right\|^{2}} \,d\lambda_{n-1}(x)
 .\end{align*} 

 Типичный пример: поверхность функции $z = z(x, y)$ двух аргументов (рисунок \ref{fig:plot_z_x_y}). В этом случае
 \begin{align*}
  \mu(\Phi(E)) = \int\limits_{E} \sqrt{1 + g_x^{'2} + g_y^{'2}} \, d\lambda_2
 .\end{align*} 
\end{exmpl}

\begin{figure}[ht]
 \centering
 \incfig{plot_z_x_y}
 \caption{График функции $z = z(x,y)$ как поверхность.}
 \label{fig:plot_z_x_y}
\end{figure}

\begin{proof}
 Вычислим матрицу Якоби:
 \begin{align*}
  J_{\Phi} = \begin{pmatrix}
   1 & 0 & \ldots & 0 \\
   0 & 1 & \ldots & 0 \\
   \vdots & \vdots & \ddots & \vdots \\
   0 & 0 & \ldots & 1 \\
   g'_{x_1} & g'_{x_2} & \ldots & g'_{x_{n-1}}
  \end{pmatrix}
 .\end{align*} По формуле Бине-Коши:
 \begin{align*}
  \det(J^{\top}_{\Phi} J_{\Phi}) &= \sum_{i = 1}^{n} \left( \det A_i \right)^{2}
 ,\end{align*} где $A_i$ --- это матрица $J_{\Phi}$ с выкинутой $i$-й строчкой. Тогда
 \begin{align*}
  \det G_{\Phi} &= 1 + g^{'2}_{x_1} + g^{'2}_{x_2} + \ldots + g^{'2}_{x_{n-1}}  = \\ 
  &= 1 + \left\| \nabla_x g \right\|^{2} 
 .\end{align*} 
\end{proof}

Мы научились считать меру на поверхностях. Теперь научимся считать интеграл на поверхности.

\begin{thm}
 Пусть $(\Omega, \Phi, S)$ --- поверхность, $f \colon\, S \to \R $  --- измеримая функция, суммируемая относительно меры Лебега $\mu$  на $S$. Тогда для любого множества $F \in \B(S)$:
 \begin{align}
  \label{equation:formula_integral_on_surface}
  \int\limits_{F} f \, d\mu  = \int\limits_{\Phi^{-1}(F)} f(\Phi(x)) \sqrt{\det G_{\Phi}(x)} \, d\lambda_k(x)
 .\end{align}
\end{thm}
\begin{proof}
 Левую часть можно переписать так:
 \begin{align*}
  \int\limits_{F} f \, d\mu = \int\limits_{S} \chi_F f \, d\mu
 ,\end{align*}  а правую часть так:
 \begin{align*}
  \int\limits_{\Phi^{-1}(F)} f(\Phi(x)) \sqrt{\det G_{\Phi}(x)}\, d\lambda_k &= \int\limits_{\Omega} \chi_F(\Phi(x)) f(\Phi(x)) \sqrt{\det G_{\Phi}(x)} \, d\lambda_k(x)
 .\end{align*} Поэтому теорему достаточно доказывать для случая $F = S$  (иначе заменим $f$ на $\chi_F f$).

 Для характеристических функций $\chi_{\Phi(E)}$, $E \in \B(\Omega)$ формула \eqref{equation:formula_integral_on_surface} --- это в точности равенство \eqref{equation:lebesgue_measure_on_surface_formula}:
 \begin{align*}
  \int\limits_{S} \chi_{\Phi(E)} \, d\mu &= \mu(\Phi(E)) = \int\limits_{E}  \sqrt{\det G_{\Phi}(x)}\,d\lambda_k(x)=  \\ &= \int\limits_{\Omega} \chi_{\Phi(E)}(\Phi(x)) \sqrt{\det G_{\Phi}(x)}  \, d\lambda_k(x)
 .\end{align*} По линейности формула \eqref{equation:formula_integral_on_surface} верна для неотрицательных простых. По теореме \ref{theorem:levi} Леви и теореме \ref{theorem:approximation} об аппроксимации \eqref{equation:formula_integral_on_surface} верна для любой неотрицательной измеримой. Рассмотрением функций $f_{\pm} = \max(\pm f, 0)$ обобщаем формулу для суммируемых функций $f$.
\end{proof}

Наконец, перейдём к каноническому примеру: посчитаем площадь сферы.

\begin{exmpl}[площадь сферы]
 Посчитаем площадь поверхности $\Sigma$, заданной уравнением $x^{2} + y^{2} + z^{2} = R^{2}$, $x, y, z > 0$. Это $1 / 8$ часть сферы.

 Параметризуем поверхность $\Sigma$ сферическими координатами:
 \begin{align*}
  \Phi(\varphi, \psi) = (R \cos \varphi \cos \psi, R \sin\varphi \cos \psi, R \sin \psi)
 ,\end{align*} где  $\varphi \in (0, \frac{\pi}{2})$, $\psi \in (0, \frac{\pi}{2})$. Тогда $\Omega = (0, \frac{\pi}{2}) \times (0, \frac{\pi}{2})$. Посчитаем матрицу Якоби:
 \begin{align*}
  J_{\Phi} = \begin{pmatrix}
   \frac{\partial \Phi_1}{\partial \varphi} & \frac{\partial \Phi_1}{\partial \psi} \\
   \frac{\partial \Phi_2}{\partial \varphi} & \frac{\partial \Phi_2}{\partial \psi} \\
   \frac{\partial \Phi_3}{\partial \varphi} & \frac{\partial \Phi_3}{\partial \psi} \\
   \end{pmatrix} = \begin{pmatrix}
   -R \sin \varphi \cos \psi & -R  \cos \varphi \sin \psi \\
   R \cos \varphi \cos \psi & -R \sin \varphi \sin \psi \\
   0 & R \cos \psi
  \end{pmatrix}
 .\end{align*} Транспонируем:
 \begin{align*}
  J_{\Phi}^{\top} = \begin{pmatrix}
   -R \sin \varphi \cos \psi & R \cos \varphi \cos \psi & 0 \\
   -R  \cos \varphi \sin \psi & -R \sin \varphi \sin \psi & R \cos \psi 
  \end{pmatrix}
 .\end{align*} Вычислим матрицу Грама:
 \begin{align*}
  &G_{\Phi} = J_{\Phi}^{\top} J_{\Phi} = \\
  = &\begin{pmatrix}
   {\scriptstyle R^{2} \sin^{2} \varphi \cos^{2} \psi + R^{2} \cos^{2} \varphi \cos^{2} \psi} & {\scriptstyle R^{2}\cos\varphi\sin\varphi\cos\psi\sin\psi - R^{2}\cos\varphi\sin\varphi\cos\psi\sin\psi} \\
   {\scriptstyle R^{2}\cos\varphi\sin\varphi\cos\psi\sin\psi - R^{2}\cos\varphi\sin\varphi\cos\psi\sin\psi} & {\scriptstyle R^{2}\cos^{2}\varphi \sin^{2}\psi + R^{2} \sin^2 \varphi \sin^{2} \psi + R^{2} \cos^{2} \psi}
  \end{pmatrix} = \\
  &= \begin{pmatrix}
   R^{2} \cos^{2} \psi & 0 \\
   0 & R^{2}\sin^{2}\psi + R^{2}\cos^{2}\psi
   \end{pmatrix} = \begin{pmatrix}
   R^{2}\cos^{2}\psi & 0 \\
   0 & R^{2}
  \end{pmatrix}
 .\end{align*} Тогда
 \begin{align*}
  \sqrt{\det G_{\Phi}}  = R^{2} \cos \psi
 .\end{align*} Значит,
 \begin{align*}
  \mu(\Sigma) &= \int\limits_{(0, \frac{\pi}{2}) \times (0, \frac{\pi}{2})} R^{2} \cos\psi  \, d\lambda_2 = \int\limits_{0}^{\pi / 2} \int\limits_{0}^{\pi / 2} R^{2}\cos\psi \,d\psi\,d\varphi = \\
  &= R^{2} \cdot \frac{\pi}{2} \cdot \int\limits_{0}^{\pi / 2} \cos\psi \,d\psi = \frac{\pi R^{2}}{2}
 .\end{align*} Таким образом, площадь всей сферы равна $8 \cdot \frac{\pi R^{2}}{2} = 4 \pi R^{2} = \left( \frac{4}{3} \pi R^{3} \right)'$ .
\end{exmpl} 
