Далее мы покажем, что в достаточно частой ситуации странное условие $\mu(A) < \infty$ в пункте \ref{enum3:theorem:caratheodory} несущественно.

\begin{df}
 Функция $\mu_0 \colon\, \p \to [0, \infty]$ на полукольце $\p \subset 2^{X}$ называется \textit{$\sigma$-конечной}, если существует счётный набор множеств $\{P_k\}_{k=1}^{\infty} \subset \p $ такой, что
 \begin{align*}
  X = \bigsqcup_{k=1}^{\infty} P_k
 \end{align*} и $\mu_0(P_k) < \infty$ для всех $k$.
\end{df}
\begin{exmpl*}
 Функция $\mu_0 \colon\, \left[a, b\right) \mapsto b - a$ на полукольце ячеек $\p_1$  $\sigma$-конечна. Действительно,
 \begin{align*}
  \R = \bigsqcup_{k \in \Z}^{\infty} \left[k, k + 1\right), \quad \mu_0(\left[k, k + 1\right)) = 1
 .\end{align*} 
\end{exmpl*}
\begin{df}
 Мера $\mu$ на $\sigma$-алгебре $\A \subset 2^{X}$ называется \textit{$\sigma$-конечной}, если существует счётный набор множеств $\left\{ A_k \right\}_{k=1}^{\infty} \subset \A$  такой, что
 \begin{align*}
  X = \bigsqcup_{k=1}^{\infty} A_k
 \end{align*} и $\mu(A_k) < \infty$ для всех $k$.
\end{df}
\begin{crly}
 \label{corollary:sigma-finite-caratheodory-continuation-is-unique}
 Если в формулировке теоремы \ref{theorem:caratheodory} Каратеодори функция $\mu_0$ $\sigma$-конечна, то $\mu(A) = \nu(A)$ для любого $A \in \A^{\ast} \cap \A_{\nu}$ (даже если $\mu(A) = \infty$).
\end{crly}
\begin{proof}
 Разбиение
 \begin{align*}
  A = \bigsqcup_{k=1}^{\infty} (A \cap Q_k), \quad \mu(Q_k) < \infty
 ,\end{align*} необходимое для финальной части доказательства пункта \ref{enum3:theorem:caratheodory} теоремы, можно теперь получить для любого $A$, даже если $\mu(A) = \infty$: можно просто взять $\{Q_{k}\}_{k=1}^{\infty} $ из определения $\sigma$-конечности функции $\mu_0$.
\end{proof}

Следующее замечание бывает крайне полезно: для меры-стандартного продолжения по теореме Каратеодори есть явная формула. 

\begin{remrk}[формула для вычисления $\mu(A)$]
 \label{remark:measure_explicit_formula}
 \begin{align*}
  \mu(A) = \inf \left\{ \sum_{k=1}^{\infty} \mu_0(P_k) \mid P_k \in \p \colon\; A \subset \bigcup_{k=1}^{\infty} P_k \right\}
 \end{align*}
 для стандартного продолжения $\mu_{0}$ на $\A^{\ast}$.
\end{remrk}

Теперь вернёмся к важнейшему примеру --- длина на $\R$.

\begin{df}
 Пусть $\mu_0 \colon\, \left[a, b\right) \mapsto b - a$  --- функция длины на полукольце ячеек $\p_1$. Стандартное продолжение $\lambda_1$ функции $\mu_0$ на $\sigma$-алгебру Каратеодори называется \textit{мерой Лебега}. Сама $\sigma$-алгебра Каратеодори называется \textit{$\sigma$-алгеброй Лебега} и обозначается $\A_{\lambda_1}$. Множества из $\A_{\lambda_1}$ называются \textit{измеримыми по Лебегу}.
\end{df}

Отметим, что по теореме \ref{theorem:cell_length_has_sigma_additivity} длина ячейки счётно-аддитивна, поэтому теорема Каратеодори применима.

\begin{lm}
 \label{lemma:open_subset_of_R_is_disjoint_countable_union_of_intervals}
 Любое открытое множество $G \subset \R$ может быть разбито на счётное число интервалов:
 \begin{align*}
  G = \bigsqcup_{k=1}^{\infty} I_k, \quad I_k = (a_k, b_k),\; a_k \leqslant b_k
 .\end{align*}
\end{lm}
\begin{proof}
 Рассмотрим следующее отношение эквивалентности на $G$: для $x, y \in G$ положим $x \sim y$, если существует интервал $I \subset G$ такой, что $x, y \in I$. Легко видеть, что это действительно отношение эквивалентности, поэтому множество $G$ разбивается на классы эквивалентности.

 Каждый класс эквивалентности $C$ является связным (выпуклым): если $a, b \in C$, то и все точки между $a$ и $b$ тоже лежат в $C$. Кроме того, $C$ --- открытое множество, потому что для любой точки $x \in G$ существует окрестность $G_x \subset G$, которая по-совместительству обязана лежать в $C$. Тогда $C$ является интервалом. При этом, классов эквивалентности не более, чем счётно, потому что внутри каждого класса есть рациональное число.
\end{proof}

\begin{lm}
 $ \B_1 = \B( \p_1 ) \subset \A_{\lambda_1} $.
\end{lm}

Эта лемма показывает, что очень широкий класс подмножеств $\R$ является измеримым по Лебегу. В частности, все открытые и замкнутые подмножества $\R$ измеримы по Лебегу.

\begin{proof}
 Включение $\B(\p_1) \subset \A_{\lambda_1}$ верно, так как по теореме \ref{theorem:caratheodory} Каратеодори $\A_{\lambda_1}$ содержит $\p_1$, а $\B(\p_1)$ --- наименьшая $\sigma$-алгебра, содержащая $\p_1$.

 Теперь докажем равенство $\B_1 = \B(\p_1)$. Сначала докажем включение $\B_1 \subset \B(\p_1)$, для этого докажем, что любое открытое множество $G \subset \R$ лежит в $\B(\p_1)$. По лемме \ref{lemma:open_subset_of_R_is_disjoint_countable_union_of_intervals} $G$ представимо в виде счётного объединения интервалов. В свою очередь, каждый интервал представим в виде счётного объединения ячеек:
 \begin{align*}
  (a_k, b_k) = \bigcup_{n=1}^{\infty} \left[a_k + 1 / n, b_k\right)
 .\end{align*} По этой причине $G$ обязано лежать в $\B(\p_1)$ как счётное объединение ячеек.

 Включение в обратную сторону $\B(\p_1) \subset \B_1$ верно, так как для любой ячейки $[a, b)$ имеет место равенство
 \begin{align*}
  [a, b) = (a - 1, b) \setminus (-\infty, a) \in \B_1
 .\end{align*}
\end{proof}

\begin{df*}
 Множества из $\B_1$ называются \textit{измеримыми по Борелю}.
\end{df*}

\begin{exmpl}[мера Лебега некоторых множеств]\
 \begin{enumerate}
  \item Найдём длину точки $x \in \R$. Множество $\left\{ x \right\}$ измеримо по Лебегу как замкнутое подмножество $\R$. Для любого $\eps > 0$ верно
   \begin{align*}
    0 \leqslant \lambda_1(\left\{ x \right\}) \leqslant \lambda_1([x, x + \eps)) = \eps
   .\end{align*} Значит, $\lambda_1(\left\{ x \right\}) = 0$.

  \item Найдём длину отрезка $[a,b]$. Отрезок измерим по Лебегу как замкнутое подмножество $\R$. Его длина равна:
   \begin{align*}
    \lambda_1([a,b]) = \lambda_1([a,b) \sqcup \left\{ b \right\}) = b - a + 0 = b - a
   .\end{align*}
  \item Если $A$ открытое и непустое, то $\lambda_1(A) > 0$.
   \begin{proof}
    Возьмём любую точку $x \in A$. Тогда существует $\eps > 0$ такой, что $(x - \eps, x + \eps) \subset A \implies [x, x + \eps) \subset A$. Тогда
    \begin{align*}
     \lambda_1(A) \geqslant \lambda_1([x,x+\eps)) = \eps
    .\end{align*} 
   \end{proof}
  \item $\lambda_1(\Q) = 0$.
   \begin{proof}
    \begin{align*}
     \lambda_1(\Q) = \sum_{x \in \Q} \lambda_1(\left\{ x \right\}) = 0
    .\end{align*}
   \end{proof}
  \item Рассмотрим $E = \left\{ x \in [0,1] \mid x \notin \Q \right\}$. Тогда
   \begin{align*}
    &1 = \lambda_1([0,1]) = \lambda_1(E \sqcup ([0,1] \cap Q)) = \lambda_1(E) + 0 \\
    \implies &\lambda_1(E) = 1.
   \end{align*} 
  \item Если есть $\{E_{k}\}_{k=1}^{\infty} $ такие, что $\lambda_1(E_k) = 0$, то
   \begin{align*}
    \lambda_1 \left( \bigcup_{k=1}^{\infty} E_k \right) = 0
   .\end{align*}
   \begin{proof}
    \begin{align*}
     \lambda_1 \left( \bigcup_{k=1}^{\infty} E_k \right) \leqslant \sum_{k=1}^{\infty} \lambda_1(E_k) = 0
    .\end{align*}
   \end{proof}
 \end{enumerate}
\end{exmpl}
\begin{exmpl}[общее канторово множество]
 \label{example:general_cantor_set}
 Построим обобщённое канторово множество следующей процедурой:

 \begin{enumerate}[start=0,label={Шаг \arabic*:},leftmargin=2cm]
  \item  из отрезка $[0,1]$ выкинем серединный интервал длины $a_0$.
  \item из оставшихся двух отрезков выкинем $2$ серединных интервала длины $a_1$.
  \item из оставшихся четырёх отрезков выкинем $2^{2}$  серединных интервала длины $a_2$, и так далее...
 \end{enumerate}

 \begin{figure}[ht]
  \centering
  \incfig{пояснение-обобщенного-канторова-множества}
  \caption{Построение общего канторова множества}
  \label{fig:пояснение-обобщенного-канторова-множества}
 \end{figure}

 Длины $a_0, a_1, a_2, \ldots \geqslant 0$ должны быть достаточно маленькими чтобы можно их было выкидывать. Для этого необходимо и достаточно, чтобы
 \begin{align}
  \label{equation:example:cantor_set_length_restriction}
  \sum_{n=0}^{\infty} 2^{n} a_n \leqslant 1
 .\end{align} Для стандартного канторова множества $a_n = \frac{1}{3^{n+1}}$ --- каждый раз из отрезков выкидываем среднюю треть. Проверим \eqref{equation:example:cantor_set_length_restriction}:
 \begin{align*}
  \sum_{n=0}^{\infty} \frac{2^{n}}{3^{n+1}} = \frac{1}{3} \sum_{n=0}^{\infty} \left( \frac{2}{3} \right)^{n} = \frac{1}{3} \cdot \frac{1}{1 - \frac{2}{3}} = 1
 .\end{align*} Можно, например, взять $a_n = \frac{1}{4^{n+1}}$ (каждый раз выкидывать среднюю четверть из отрезков). Тогда
 \begin{align*}
  \sum_{n=0}^{\infty} \frac{2^{n}}{4^{n+1}} = \frac{1}{4} \sum_{n=0}^{\infty} \left(\frac{2}{4}\right)^{n} = \frac{1}{4} \cdot \frac{1}{1 - \frac{2}{4}} = \frac{1}{2}
 .\end{align*}

 Обобщённое канторово множество с последовательностью $\left\{ a_n \right\}_{n=0}^{\infty}$ обозначим $C_{\left\{ a_n \right\}}$. Оно \textit{измеримо по Борелю}: $C_{\left\{ a_n \right\}} \in \B_1$, так как оно замкнуто (или, например, мы счётное число раз выкидываем интервалы из отрезка).

 Вычислим длину обобщённого канторова множества $C_{\left\{ a_n \right\}}$:
 \begin{align*}
  \lambda_1 \left( C_{\left\{ a_n \right\}} \right)
  & = \lambda_{1} \left( [0, 1] \setminus \bigsqcup_{n=0}^{\infty} \bigsqcup_{k=1}^{2^{n}} I_{n,k} \right) = \\
  &=  1 - \sum_{n=0}^{\infty} \sum_{k=1}^{2^{n}} \lambda_1(I_{n,k}) = \\
  &= 1 - \sum_{n=0}^{\infty} 2^{n} a_n
 .\end{align*} Таким образом, стандартное Канторово множество несчётное, но его длина равна нулю!

 Длина канторова множества с $a_n = \frac{1}{4^{n+1}}$ равна
 \begin{align*}
  \lambda_1 \left( C_{\left\{ \frac{1}{4^{n+1}} \right\}} \right) = \frac{1}{2}
 .\end{align*} Канторовы множества с ненулевой длиной называются \textit{толстыми канторовыми множествами}. Интересный факт: толстое канторово множество не содержит интервалов, но имеет положительную длину, в отличие от $\Q$!
\end{exmpl}

Есть ещё одно важное свойство мер, которое, к счастью, выполняется для меры-стандартного продолжения.

\begin{df}
 Мера $\mu$ на $\sigma$-алгебре $\A$ называется \textit{полной}, если для любого множества $A \in \A$ такого, что $\mu(A) = 0$ и для любого  $B \subset A$ имеет место $B \in \A$.
\end{df}
\begin{remrk}
 \label{remark:standard_continuation_measure_is_complete}
 Мера $\mu$ в теореме Каратеодори полна.
\end{remrk}
\begin{proof}
 Пусть $A \in \A^{\ast}$ такое, что $\mu(A) = 0$. Пусть  $B \subset A$. Проверим принадлежность $B \in \A^{\ast}$ по определению. Рассмотрим произвольное $E \subset X$. Тогда:
 \begin{align*}
  \mua(E) & \leqslant \mua(E \cap B) + \mua(E \cap B^{c}) \leqslant \\
  & \leqslant \mua(E \cap A) + \mua(E) \leqslant \\
  & \leqslant \mua(A) + \mua(E) \leqslant \\
  & \leqslant 0 + \mua(E)
 .\end{align*} Значит, $\mua(E) = \mua(E \cap B) + \mua(E \cap B^{c})$ для любого $E \subset X$, и следовательно, $B \in \A^{\ast}$.
\end{proof}

Следующие два утверждения полезны: они позволяют совершить предельный переход по мере, если есть монотонная последовательность множеств.

\begin{claim}[непрерывность меры сверху]
 \label{claim:upward_continuity_of_measure}
 Пусть $\mu$ --- мера на $\sigma$-алгебре $\A \subset 2^{X}$. Пусть  $E_1 \subset E_2 \subset \ldots$  --- последовательность возрастающих измеримых множеств $E_k \in \A$, $E = \bigcup_{k=1}^{\infty} E_k$  --- их объединение. Тогда
 \begin{align*}
  \mu(E) = \lim_{n \to \infty} \mu(E_n) 
 .\end{align*} 
\end{claim}
\begin{proof}
 Рассмотрим разностную последовательность множеств:
 \begin{align*}
  F_1 &= E_1 \\
  F_2 &= E_2 \setminus E_1 \\
  F_3 &= E_3 \setminus E_2 \\
  &\;\;\vdots
 \end{align*} Все $F_k \in \A$. Кроме того,
 \begin{align*}
  E = \bigsqcup_{k=1}^{\infty} F_k
 .\end{align*} Тогда по счётной и конечной аддитивности меры имеем равенство:
 \begin{align*}
  \mu(E) = \sum_{k=1}^{\infty} \mu(F_k) = \lim_{n \to \infty} \sum_{k=1}^{n} \mu(F_k) = \lim_{n \to \infty} \mu(E_k)  
 .\end{align*}
\end{proof}
\begin{claim}[непрерывность меры снизу]
 \label{claim:downward_continuity_of_measure}
 Пусть $\mu$ --- мера на $\sigma$-алгебре $\A \subset 2^{X}$. Пусть $E_1 \supset E_2 \supset \ldots $ --- последовательность убывающих измеримых множеств $E_k \in \A$, причём $\mu(E_1) < \infty$. Пусть $E = \bigcap_{k=1}^{\infty} E_k$  --- их пересечение. Тогда
 \begin{align*}
  \mu(E) = \lim_{n \to \infty} \mu(E_k) 
 .\end{align*} 
\end{claim}
\begin{proof}
 Перейдём к дополнению и сведём к непрерывности меры сверху. Рассмотрим множества $F_k = E_1 \setminus E_k$. Тогда $F_1 \subset F_2 \subset \ldots$ и 
 \begin{align*}
  F = \bigcup_{k=1}^{\infty} F_k = E_1 \setminus E
 .\end{align*} В силу того, что $\mu(E_1) < \infty$, имеем $\mu(F_k) = \mu(E_1) - \mu(E_k)$, а также $\mu(F) = \mu(E_1) - \mu(E)$. Тогда по непрерывности меры сверху
 \begin{align*}
  \mu(E_1) - \mu(E) = \mu(F) = \lim_{n \to \infty} \mu(F_n)  = \mu(E_1) - \lim_{n \to \infty} \mu(E_n) 
 ,\end{align*} и, следовательно,
 \begin{align*}
  \mu(E) = \lim_{n \to \infty} \mu(E_n) 
 .\end{align*} 
\end{proof}
\begin{exmpl*}
 Непрерывность меры снизу может быть не верна, если $\mu(E_1) = \infty$. Например,
 \begin{align*}
  E_n = \bigcup_{k \in \Z} \left[k, k + 1 / n \right)
 .\end{align*} Тогда $E_1 \supset E_2 \supset \ldots$ и $\mu(E_n) = \infty$  для любого $n$, но пересечение
 \begin{align*}
  \bigcap_{n=1}^{\infty} E_n = \Z
 \end{align*} имеет меру нуль.
\end{exmpl*}

\section{Измеримые функции. Теорема об аппроксимации.}

Начиная с этого параграфа мы будем изучать функции, которые могут принимать отрицательные значения. В связи с этим введём ещё одно обозначение.

\begin{df}
 \label{definition:extended_reals}
 За $[-\infty, +\infty]$ обозначим множество $\R$ с двумя дополнительными элементами: $+\infty$ и $-\infty$. На это множество можно продолжить отношение порядка с $\R$: для каждого $x \in \R$ положим
 \begin{align*}
  -\infty < x < +\infty
 ,\end{align*} а также $-\infty < +\infty$.

 Отметим, что операции сложения и умножения в общем случае продолжить нельзя: например, нет разумного способа определить $-\infty + (+\infty)$. Однако, их можно продолжить, если запретить складывать бесконечности разных знаков. Для остальных случаев правила такие:
 \begin{itemize}
  \item $a + (\pm \infty) = \pm \infty$, $a \in \R$.
  \item $+\infty + (+\infty) = +\infty$, $-\infty + (-\infty) = -\infty$.
  \item 
   \begin{align*}
    a \cdot (+\infty) = \begin{cases}
     +\infty, \text{ если } a > 0  \\
     0, \text{ если } a = 0 \\
     -\infty, \text{ если } a < 0
    \end{cases} \quad
    a \cdot (-\infty) = \begin{cases}
     -\infty, \text{ если } a > 0  \\
     0, \text{ если } a = 0 \\
     +\infty, \text{ если } a < 0
    \end{cases} 
   \end{align*} 
 \end{itemize}
\end{df}

\begin{df}
 \textit{Измеримым пространством} $(X, \A)$ называется множество $X$ с заданной на нём $\sigma$-алгеброй $\A \subset 2^{X}$.
\end{df}
В этом параграфе все функции будут заданы на некотором измеримом пространстве.
\begin{df}
 \label{definition:measurable_function}
 Функция $f \colon\, X \to [-\infty, \infty] $ называется \textit{измеримой} (относительно $\A$), если для любого $a \in \R$ прообраз луча $(a, +\infty]$ измерим:
 \begin{align*}
  f^{-1}((a, +\infty]) \in \A
 .\end{align*}
\end{df}
\begin{remrk}
 В определении \ref{definition:measurable_function} можно заменить луч $(a, +\infty]$, например, на  $[a, +\infty]$:
 \begin{align*}
  f^{-1}([a, +\infty]) &= \bigcap_{n=1}^{\infty} f^{-1}\left(\left(a - \frac{1}{n},\; +\infty \right]\right), \\
  f^{-1}((a, +\infty]) &= \bigcup_{n=1}^{\infty}  f^{-1} \left( \left[a + \frac{1}{n},\; +\infty\right] \right)
 .\end{align*} Можно также заменить на $[-\infty, a)$ или $[-\infty, a]$(тут нужно перейти к дополнению).
\end{remrk}
\begin{crly}
 Если функция $f$ измерима, то прообраз любой точки $x \in [-\infty, +\infty]$ измерим: $f^{-1}(\left\{ x \right\}) \in \A$.
\end{crly}
\begin{proof} Для $x \in \R$:
 \begin{align*}
  f^{-1}( \left\{ x \right\} ) = f^{-1}( [x, \infty] ) \setminus f^{-1} \left( (x, +\infty] \right)
 .\end{align*}  Для $x = +\infty$:
 \begin{align*}
  f^{-1} \left( \left\{ +\infty \right\} \right) = \bigcap_{n=1}^{\infty} f^{-1} \left( [n, +\infty] \right)
 .\end{align*} Для $x = -\infty$:
 \begin{align*}
  f^{-1}(\left\{ -\infty \right\}) = \bigcap_{n=1}^{\infty} f^{-1}([-\infty, -n])
 .\end{align*} 
\end{proof}

\begin{crly}
 \label{corollary:preimage_of_cell_in_measurable_function_is_measurable}
 Если функция $f$ измерима, то прообраз любой ячейки $[a, b)$ измерим:
 $ f^{-1}([a, b)) \in \A $.
\end{crly}
\begin{proof}
 \begin{align*}
  f^{-1}( [a, b) ) = f^{-1} \left( [a, +\infty] \right) \setminus f^{-1} \left( [b, +\infty] \right)
 .\end{align*}
\end{proof}
\begin{thm}
 Пусть функция $f \colon\, X \to [-\infty, +\infty] $ измерима. Тогда для любого $E \in \B_1$ верно $ f^{-1}(E) \in \A $.
\end{thm}
\begin{proof}
 Заведём специально обученную $\sigma$-алгебру:
 \begin{align*}
  \B = \left\{ A \subset \R \mid f^{-1}(A) \in \A \right\}
 .\end{align*} Проверим аксиомы $\sigma$-алгебры:
 \begin{enumerate}
  \item $\varnothing \in \B$, так как $f^{-1}(\varnothing) = \varnothing$.
  \item Пусть $\{A_{k}\}_{k=1}^{\infty} \subset \B $, тогда
   \begin{align*}
    f^{-1}\left(\bigcap_{k=1}^{\infty} A_k \right) = \bigcap_{k=1}^{\infty} f^{-1}(A_k) \in \A \implies \bigcap_{k=1}^{\infty} A_k \in \B
   .\end{align*}
  \item Пусть $A \in \B$. Тогда
   \begin{align*}
    f^{-1}(A^{c}) = f^{-1}(\R \setminus A) = f^{-1}(\R) \setminus f^{-1}(A) \in \B \implies A^{c} \in \B
   .\end{align*} 
 \end{enumerate}
 Но по следствию \ref{corollary:preimage_of_cell_in_measurable_function_is_measurable} $\p_1 \subset \B$. Тогда $\B_1 = \B(\p_1) \subset \B$. Значит, для любого  $A \in \B_1$ $f^{-1}(A) \in \A$.
\end{proof}
Иными словами, измеримая функция --- это та, у которой прообраз измеримого по Борелю множества измерим.

Пока-что довольно трудно что-то сказать про измеримые функции: например, мы не знаем, измерима ли сумма двух измеримых функций. Однако, мы можем показать измеримость предела последовательности измеримых функций.

\begin{lm}
 \label{lemma:limit_of_measurable_functions_is_measurable}
 Пусть $\{f_{n}\}_{n=1}^{\infty} $ --- последовательность измеримых функций $f_n \colon\; X \to [-\infty, +\infty]$. Тогда функции
 \begin{align*}
  \sup_{n \geqslant 1} f_n(x), \quad \inf_{n \geqslant 1} f_n(x), \quad \limsup_{n \to \infty} f_{n}(x), \quad \liminf_{n \to \infty} f_n(x)
 \end{align*} измеримы. Также, если существует предел  
 \begin{align*}
  \lim_{n \to \infty} f_n(x)
 \end{align*} для любой точки $x \in X$, то $\lim f_n(x)$ измерим.
\end{lm}
\begin{proof}\
 \begin{itemize}
  \item $f(x) = \sup f_n(x)$:
   \begin{align*}
    f^{-1}(\left(a, +\infty\right]  ) = \bigcup_{n=1}^{\infty} f^{-1}_n(\left(a, +\infty\right]  )
   .\end{align*} 
  \item $f(x) = \inf f_n(x)$:
   \begin{align*}
    f^{-1}(\left[-\infty, a\right)) = \bigcup_{n=1}^{\infty} f^{-1}_n(\left[-\infty, a\right))
   .\end{align*} 
  \item $f(x) = \limsup\limits_{n\to \infty} f_n(x)$:
   \begin{align*}
    f(x) = \limsup f_n(x) = \displaystyle\lim_{N \to \infty} \underbrace{\sup_{n \geqslant N} f_n(x)}_{\text{ убывает } \forall x \in X} = \inf_{N \geqslant 1} \left( \sup_{n \geqslant N} f_n(x) \right)
   .\end{align*} 
  \item $f(x) = \liminf\limits_{n \to \infty} f_n(x)$ аналогично.
  \item Если $f(x) = \lim\limits_{n \to \infty} f_n(x)$ существует, то $f = \limsup f_n = \liminf f_n$, а они измеримы.
 \end{itemize}
\end{proof}

Чтобы продолжить дальше, мы определим класс \textit{простых} функций. Многие теоремы и конструкции в теории меры будут сначала доказаны/построены для простых функций, а затем продолжены до класса измеримых функций.

\begin{df}
 Функция $f \colon\; X \to \R$ на измеримом пространстве $(X, \A)$ называется \textit{простой}, если она измерима и принимает конечное число значений.
\end{df}

Важное замечание: мы считаем, что простая функция не может принимать значения $\pm \infty$.

\begin{remrk}
 Каждая простая функция $f$ имеет вид
 \begin{align*}
  f = \sum_{k=1}^{N} c_k \chi_{A_k}
 ,\end{align*} где $X = \bigsqcup_{k=1}^{N} A_k $ --- разбиение измеримого пространства на измеримые множества $A_k \in \A$, а $c_k \in \R$ --- коэффициенты. $\chi_A$ здесь обозначает характеристическую функцию множества $A$:
 \begin{align*}
  \chi_A(x) = \begin{cases}
   1, \text{ если } x \in A, \\
   0, \text{ иначе. }
  \end{cases}
 \end{align*}
\end{remrk}
\begin{proof}
 По измеримости $f$ принимает конечное число значений:
 \begin{align*}
  f(X) = \{c_{k}\}_{k=1}^{N}
 .\end{align*} Тогда положим
 \begin{align*}
  A_k = f^{-1} \left( \left\{ c_k \right\} \right) \in \A
 .\end{align*}
\end{proof}

\begin{lm}
 \label{lemma:common_small_partition_of_simple_functions}
 Пусть функции $f$, $g$ простые на измеримом пространстве $(X, \A)$. Тогда существует общее мелкое разбиение $X = \bigsqcup_{k=1}^{K} E_k$, $E_k \in \A$ такое, что обе функции имеют вид
 \begin{align*}
  f = \sum_{k=1}^{K} u_k \chi_{E_k}, \quad g = \sum_{k=1}^{K} v_k \chi_{E_k}
 .\end{align*} 
\end{lm}
\begin{proof}
 Пусть
 \begin{align*}
  f &= \sum_{i=1}^{N} c_i \chi_{A_i}, \quad X = \bigsqcup_{i=1}^{N} A_i,\; A_i \in \A, \\
  g &= \sum_{i=1}^{M} d_i \chi_{B_i}, \quad X = \bigsqcup_{i=1}^{M} B_i,\; B_i \in \A
 .\end{align*} Общее мелкое разбиение имеет вид
 \begin{align*}
  X = \bigsqcup_{i=1}^{N} \bigsqcup_{j=1}^{M} (A_i \cap B_j)
 .\end{align*} Каждая функция постоянна на всяком $A_i \cap B_j$.
\end{proof}
\begin{crly}
 \label{corollary:simple_operations_with_simple_functions_result_in_simple_function}
 Пусть функции $f$, $g$ простые на измеримом пространстве $(X, \A)$. Тогда
 \begin{itemize}
  \item их линейная комбинация $\lambda_1 f + \lambda_2 g$;
  \item $f \cdot g$;
  \item $\max \left\{ f_1, \ldots, f_n \right\}$;
  \item $\min \left\{ f_1, \ldots, f_n \right\}$
 \end{itemize} являются простыми функциями.
\end{crly}
\begin{proof}
 Рассмотреть общее мелкое разбиение.
\end{proof}

Следующая теорема показывает главную идею Лебега: какую попало измеримую функцию можно приблизить простыми.

