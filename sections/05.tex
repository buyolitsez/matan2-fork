% 2022.10.07 lecture 5

\begin{claim}[интеграл как мера]
 \label{claim:integral_is_measure}
 Пусть $f \colon\, X \to [0,\infty] $ --- неотрицательная измеримая функция. Для каждого измеримого множества $E \in \A$ сопоставим число \begin{align*}
  \nu(E) = \int\limits_{E} f \, d\mu 
 .\end{align*} Тогда $\nu$ --- это мера на $\A$.
\end{claim}
\begin{proof}
 $0 \leqslant \nu \leqslant \infty$ очевидно. Проверим счётную-аддитивность: пусть $E = \bigsqcup_{k=1}^{\infty} E_k$, $E_k \in \A$. Тогда по формуле \eqref{equation:remark:integral_on_subset}:
 \begin{align*}
  \nu(E) = \int\limits_{E} f\, d\mu = \int\limits_{X} \chi_E f \, d\mu = \int\limits_{X} \lim_{k \to \infty} f_k \, d\mu 
 ,\end{align*} где 
 \begin{align*}
  f_k = \chi_{E_1 \sqcup \ldots \sqcup E_k} f = \sum_{i=1}^{k} \chi_{E_i} f
 .\end{align*} Так как неотрицательные функции $\{f_{k}\}_{k=1}^{\infty} $ возрастают к $\chi_E f$, то по теореме \ref{theorem:levi} Леви:
 \begin{align*}
  \nu(E) &= \lim_{k \to \infty} \int\limits_{X} f_k d\mu   = \lim_{k \to \infty} \int\limits_{X} \sum_{i=1}^{k} \chi_{E_i} f \, d\mu = \sum_{i=1}^{\infty} \int\limits_{X} \chi_{E_i} f \, d\mu = \sum_{i=1}^{\infty} \nu(E_i)
 .\end{align*} 
\end{proof}

\begin{df}
 \label{definition:measure_continious_relative_to_other_measure}
 Пусть $\mu, \nu$ --- меры на $\sigma$-алгебре $\A \subset 2^X$. Мера $\nu$ называется \textit{непрерывной} относительно меры $\mu$, если из того, что $\mu(E) = 0$ следует $\nu(E) = 0$. Обозначение: $\nu \prec \prec \mu$.
\end{df}
\begin{remrk}
 В утверждении \ref{claim:integral_is_measure} мера $\nu(E) = \int_E f \, d\mu  $ непрерывна относительно $\mu$, так как если $\mu(E) = 0$, то $\int_{E} f \, d\mu = 0$. 

 Более того, верно обратное (теорема \ref{theorem:radon_nikodim} Радона-Никодина): если $\nu$ непрерывна относительно $\mu$, то она обязательно имеет вид $\nu = \int f \, d\mu  $.
\end{remrk}

\begin{claim}[абсолютная непрерывность интеграла]
 \label{claim:absolute_continuity_of_integral}
 Пусть $f \geqslant 0$ и $\int_{X} f \, d\mu < \infty$. Тогда для любого $\eps > 0$ существует $\delta(\eps) > 0$ такое, что если $E \in \A$ и $\mu(E) < \delta(\eps)$, то \begin{align*}
  \int\limits_{E} f \, d\mu < \eps
 .\end{align*} 

 Менее формально: $\int_{E} f \, d\mu \to 0$ при $\mu(E) \to 0$.
\end{claim}
\begin{proof}
 Пусть $\eps > 0$. По определению \ref{definition:integral_supremum_simple} интеграла существует простая неотрицательная функция $g$ такая, что $0 \leqslant g \leqslant f$ на $X$ и \begin{align*}
  \int\limits_{X} f \, d\mu \leqslant \int\limits_{X} g \, d\mu + \frac{\eps}{2}
  .\end{align*} Пусть теперь $E \in \A$. Тогда \begin{align*}
  \int\limits_{E} f \, d\mu  &= \int\limits_{E} (f-g) \, d\mu + \int\limits_{E} g \, d\mu \leqslant \\ &\leqslant \int\limits_{X} (f-g) \, d\mu + \int\limits_{E} \max_E g \, d\mu \leqslant \\ &\leqslant \frac{\eps }{2} + \max_E g \cdot \mu(E)
 .\end{align*} Таким образом, можно взять $\delta(\eps) = \frac{\eps}{2 \cdot \max_E g}$ (если $\max_E g = 0$, можно взять любое $\delta(\eps)$). Отметим, что $\max_E g$ существует, так как $g$ --- простая функция.
\end{proof}

\begin{lm}[об исчерпывании интеграла]
 \label{lemma:about_exhaustion_of_integral}
 Пусть $f \geqslant 0$ и $\int_{X} f \, d\mu < \infty$. Тогда для любого $\eps > 0$ существует $A \in \A$ такое, что $\mu(A) < \infty$ и \begin{align*}
  \int\limits_{X \setminus A} f \, d\mu < \eps
 .\end{align*} 
\end{lm}
\begin{proof}
 Рассмотрим множества $E_n = \left\{ x \in X \mid f(x) \geqslant \frac{1}{n} \right\}$. $E_n \in \A$ (так как это прообраз луча $\left[1 / n, \infty\right]$). Кроме того, $E_1 \subset E_2 \subset \ldots$ и \begin{align*}
  \bigcup_{n=1}^{\infty} E_n = \left\{ x \mid f(x) > 0 \right\}
 .\end{align*} Рассмотрим меру 
 \begin{align*}
  \nu(E) = \int\limits_{E} f \, d\mu  
  .\end{align*} По непрерывности меры сверху (утверждение \ref{claim:upward_continuity_of_measure}) \begin{align*}
  \nu \left( \bigcup_{n=1}^{\infty} E_n \right) = \lim_{n \to \infty} \nu(E_n) 
  .\end{align*} Левая часть равна \begin{align*}
  \nu \left( \bigcup_{n=1}^{\infty} E_n \right) = \int\limits_{\left\{ x \mid f(x) > 0 \right\}} f \, d\mu  = \int\limits_{X} f \, d\mu  
  ,\end{align*} а правая часть равна \begin{align*}
  \lim_{n \to \infty} \nu(E_n)  = \lim_{n \to \infty} \int\limits_{E_n} f \, d\mu  
  .\end{align*} Тогда в качестве $A$ можно взять $E_N$ для достаточно большого $N$. Осталось проверить, что $\mu(E_n) < \infty$ для любого $n$. Для этого воспользуемся неравенством Чебышева: \begin{align*}
  \mu(E_n) = \int\limits_{X} \chi_{E_n} \, d\mu  \leqslant \int\limits_{X} \chi_{E_n} \underbrace{n \cdot f(x)}_{\geqslant 1 \text{ на $E_n$}} \, d\mu  \leqslant n \int\limits_{X} f \, d\mu  < \infty
 .\end{align*}
\end{proof}

\section{Предельные теоремы}
Как и в предыдущем параграфе, зафиксируем пространство с мерой $(X, \A, \mu)$.

Ранее для предельного перехода под знаком интеграла у нас была только теорема \ref{theorem:levi} Леви, которая рассматривает частный (но тем не менее, очень важный) случай: функции в последовательности неотрицательные, измеримые, и возрастают. В этом параграфе мы получим результаты с менее жестким ограничением на последовательность функций. 

Сначала формализуем очень удобный жаргон в теории меры.

\begin{df}
 Говорят, что свойство $P(x)$, зависящее от точки $x \in X$, выполнено \textit{$\mu$-почти всюду} (или \textit{$\mu$-почти наверное}, или \textit{$\mu$-почти везде}), если множество $E = \left\{ x \in X \mid P(x) \text{ не выполнено} \right\}$ измеримо и $\mu(E) = 0$.

 Также, будем говорить, что множество $E \in \A$ имеет \textit{полную меру}, если $\mu(X \setminus E) = 0$. Таким образом, свойство $P(x)$ выполнено $\mu$-почти всюду, если оно выполнено на множестве полной меры.
\end{df}
\newcommand{\lao}{\ensuremath \lambda_{1}}
\begin{exmpl*}
 Почти любая точка отрезка $[0,1]$ иррациональна. Имеется в виду, что свойство $P(x) \colon\; x \in [0,1] \setminus \Q$ выполнено $\lambda_1$-почти всюду.
\end{exmpl*}
\begin{exmpl*}
 Если $\mu = \delta_{\left\{ \frac{\sqrt{2}}{10} \right\}}$, то $\mu$-почти любая точка $\R$ --- это $\frac{\sqrt{2}}{10}$.
\end{exmpl*}

Мы будем рассматривать последовательности измеримых функций (любого знака, не обязательно возрастающих). Определим два различных вида сходимости таких последовательностей.

\begin{df}
 Пусть $\{f_{n}\}_{n=1}^{\infty} $ --- последовательность измеримых функций $f_n \colon\, X \to [-\infty, +\infty]  $, и $f \colon\, X \to [-\infty, +\infty] $ --- измеримая функция.

 Будем говорить, что функции \textit{$f_n$ сходятся к $f$ ($f_n \to f$) $\mu$-почти всюду}, если существует множество  $E \in \A$ такое, что $f_n(x) \to f(x)$ для любого $x \in E$ и $\mu(X \setminus E) = 0$.
\end{df}
\begin{remrk*}
 Если $\mu$ --- полная мера, то $f$ измерима автоматически, вне зависимости от значений на $X \setminus E$. В общем случае мы будем требовать измеримость $f$, например, доопределяя $f = 0$ на $X \setminus E$.
\end{remrk*}
\begin{df}
 Пусть $\{f_{n}\}_{n=1}^{\infty} $ --- последовательность измеримых функций $f_n \colon\, X \to [-\infty, +\infty]  $, и $f \colon\, X \to [-\infty, +\infty] $ --- измеримая функция.

 Будем говорить, что функции \textit{$f_n$ сходятся к $f$ по мере $\mu$} ($f_n \xrightarrow{\mu} f$), если для любого $\eps > 0$
 \begin{align*}
  \lim_{n \to \infty} \mu \left\{ x \mid \left| f_n(x) - f(x) \right| \geqslant \eps \right\}  = 0
 .\end{align*} 

 Иными словами, вероятность того, что значение в точке $x$ отличается на $\eps$ или более, стремится к нулю.
\end{df}

\begin{remrk}
 Если $f_n \to f$ и $f_n \to g$ $\mu$-почти всюду, то $f = g$ $\mu$-почти всюду.
\end{remrk}
\begin{proof}
 Имеем $f_n \to f$ на $E_1$ и $f_n \to g$ на $E_2$, где $E_1, E_2$ имеют полную меру. По единственности предела числовой последовательности имеем $f = g$ на $E_1 \cap E_2$, а это тоже множество полной меры: $\mu(X \setminus (E_1 \cup E_2)) \leqslant \mu(X \setminus E_1) + \mu(X \setminus E_2) = 0$.
\end{proof}
\begin{remrk}
 Если $f_n \xrightarrow{\mu} f$ и $f_n \xrightarrow{\mu} g$, то $f = g$  $\mu$-почти всюду.
\end{remrk}
\begin{proof}
 Для любого $n$ верно
 \begin{align*}
  \mu \left\{ \left| f - g \right| \geqslant 2\eps \right\} \leqslant \mu \left( \left\{ \left| f - f_n \right| \geqslant \eps \right\} \cup \left\{ \left| f_n - g \right| \geqslant \eps \right\} \right)
  ,\end{align*} так как по неравенству треугольника из условий $\left| f - f_n \right| < \eps$ и $\left| f_n - g \right| < \eps$ следует $\left| f-g \right| < 2\eps$. Тогда \begin{align*}
  \mu \left\{ \left| f-g \right| \geqslant 2\eps \right\} \leqslant \mu \left\{ \left| f - f_n \right| \geqslant \eps \right\} + \mu \left\{ \left| f_n - g \right| \geqslant \eps \right\}
 .\end{align*} Правая часть стремится к $0$ при $n \to \infty$, поэтому
 \begin{align*}
  \mu \left\{ \left| f-g \right| \geqslant 2\eps \right\} = 0
  .\end{align*} Так как это верно для любого $\eps > 0$, то \begin{align*}
  \mu \left( \left\{ \left| f - g \right| > 0 \right\} \right) = \mu \left( \bigcup_{n=1}^{\infty} \left\{ \left| f - g \right| \geqslant \frac{1}{n}\right\} \right) \leqslant \sum_{n=1}^{\infty} 0 = 0
 .\end{align*} 
\end{proof}

\begin{exmpl}[различия сходимостей по мере и $\mu$-почти всюду]\
 \begin{enumerate}
  \item Рассмотрим $f_n = \chi_{[n, n + 1]}$. Тогда $f_n(x) \to 0$ при $n \to \infty$ для любого $x \in \R$. Но неверно $f_n \xrightarrow{\lao} 0$, так как $\lao  \left\{ f_n \geqslant 1 \right\}  = 1$ для любого $n$.
  \item Рассмотрим последовательность функций $f_n$ как на рисунке \ref{fig:function_sequence_converges_by_measure_but_not_in_any_point}: $f_1 = \chi_{[0,1]}$, $f_2 = \chi_{\left[0, \frac{1}{2}\right]}$, $f_3 = \chi_{\left[ \frac{1}{2}, 1 \right]}$, $f_4 = \chi_{[0, \frac{1}{4}]}$, $f_5 = \chi_{\left[\frac{1}{4}, \frac{2}{4}\right]}$, $f_6 = \chi_{\left[ \frac{2}{4}, \frac{3}{4} \right]}$, $f_7 = \chi_{\left[ \frac{3}{4}, 1 \right]}$ и так далее. Тогда верно $f_n \xrightarrow{\lambda_1} 0$, но ни на одной точке $x \in [0, 1]$ не верно $f_n(x) \rightarrow 0$.

   \begin{figure}[ht]
    \centering
    \incfig{function_sequence_converges_by_measure_but_not_in_any_point}
    \caption{Последовательность функций $f_n$, сходящаяся к $0$ по мере Лебега, но не сходящаяся к $0$ ни в одной точке отрезка $[0,1]$.}
    \label{fig:function_sequence_converges_by_measure_but_not_in_any_point}
   \end{figure}
 \end{enumerate}
\end{exmpl}

Несмотря на приведенные контрпримеры, все же есть некоторая связь между этими двумя видами сходимости.

\begin{thm}
 \label{theorem:almost_everywhere_convergence_implies_measure_convergence_on_finite_measure}
 Пусть $\mu(X) < \infty$. Тогда из того, что $f_n \to f$ $\mu$-почти всюду на $X$, следует $f_n \xrightarrow{\mu} f$ на $X$.
\end{thm}
\begin{proof}\
 \begin{enumerate}
  \item Сначала пусть $f = 0$ и $f_n \geqslant f_{n+1} \geqslant 0$ для любого $n$ и любого $x \in X$. Для $\eps > 0$ рассмотрим множества \begin{align*}
    X_n = \left\{ x \in X \Mid f_n(x) \geqslant \eps \right\}
   .\end{align*} Возьмём их пересечение:
   \begin{align*}
    E = \bigcap_{n=1}^{\infty} X_n = \left\{ x \in X \mid \forall n \geqslant 1 \colon\; f_n(x) \geqslant \eps \right\}
   .\end{align*} На $E$ не может быть поточечной сходимости, а так как сходимость есть $\mu$-почти всюду, то $\mu(E) = 0$ ($E$ измеримо, ведь $X_n$ измеримы). Так как $\mu(X) < \infty$, то по непрерывности меры снизу (утверждение \ref{claim:downward_continuity_of_measure}):
   \begin{align*}
    \lim_{n \to \infty} \mu(X_n) = 0
   .\end{align*}  Получили $f_n \xrightarrow{\mu} 0 = f$ по определению.

  \item Сведём общий случай к шагу 1. Рассмотрим функции
   \begin{align*}
    \tilde f_n(x) = \sup_{k \geqslant n} \left| f(x) - f_k(x) \right|
   .\end{align*} Тогда $\tilde f_n \geqslant \tilde f_{n+1} \geqslant 0$ всюду на $X$. Кроме того, $\tilde f_n \to 0$ $\mu$-почти всюду (на самом деле в тех же точках, в которых выполнялось $f_n \to f$). По шагу 1 мы получаем оценку
   \begin{align*}
    \mu \left\{ \left| f_n - f \right| \geqslant \eps \right\} \leqslant \mu \left\{ \tilde f_n \geqslant \eps \right\} \to 0
   .\end{align*} 
 \end{enumerate}
\end{proof}

\begin{lm}[Бореля-Кантелли]
 \label{lemma:borel-cantelli}
 Пусть $\left\{ E_k \right\}_{k=1}^{\infty} \subset \A$ --- измеримые множества такие, что
 \begin{align*}
  \sum_{k=1}^{\infty} \mu(E_k) < \infty
 .\end{align*} Пусть $E$ --- это множество всех точек $x \in X$, которые принадлежат бесконечно многим $E_k$:
 \begin{align*}
  E = \bigcap_{n=1}^{\infty} \bigcup_{k=n}^{\infty} E_k
 .\end{align*} Тогда $\mu(E) = 0$.
\end{lm}

Смысл леммы в теории вероятностей: если есть сходимость ряда вероятностей событий $E_k$, то вероятность того, что произойдёт бесконечно много событий $E_k$, равна нулю.

Множество $E$ также называют \textit{верхним пределом} множеств $E_k$.

\begin{proof}
 Заметим, что для любого $n$ \begin{align*}
  \mu(E) \leqslant \mu \left( \bigcup_{k=n}^{\infty} E_k \right) \leqslant \underbrace{\sum_{k=n}^{\infty} \mu(E_k)}_{\text{хвост сходящегося ряда}} \to 0
 .\end{align*} 
\end{proof}

\begin{thm}
 \label{theorem:measure-convergence-subsequence}
 Пусть $f_n \xrightarrow{\mu} f$ --- измеримые функции. Тогда существует подпоследовательность $\{f_{n_k}\}_{k=1}^{\infty} $ такая, что $f_{n_k}(x) \to f(x)$ $\mu$-почти всюду на $X$.
\end{thm}
\begin{proof}
 Так как $f_n \xrightarrow{\mu} f$, то для любого $k \geqslant 1$ можно найти номер $n_k$ такой, что 
 \begin{align*}
  \mu  \left\{ x \in X \mid \left| f - f_{n} \right| \geqslant 1 / k \right\}  < \frac{1}{2^{k}}
 \end{align*} для любого $n \geqslant n_k$.

 Обозначим за $F$ множество точек $x \in X$, в которых $f_{n_k}(x)$ не сходится к $f(x)$. Пусть $x \in F$. Тогда существует $k_0 \in \N$: \begin{align*}
  \left| f_{n_k}(x) - f(x) \right| \geqslant \frac{1}{k_0}
  \end{align*} для бесконечно многих $k$. Из этого следует  \begin{align*}
  \left| f_{n_k}(x) - f(x) \right| \geqslant \frac{1}{k}
 \end{align*} для бесконечно многих $k$. Обозначив
 \begin{align*}
  E_k = \left\{ x \in X \Mid \left| f_{n_k}(x) - f(x) \right| \geqslant 1 / k \right\}
 ,  \end{align*} получаем вложение $F \subset E$, где $E$  --- множество точек $x \in X$, лежащих в бесконечно многих $E_k$. Так как
 \begin{align*}
  \sum_{k=1}^{\infty} \mu(E_k) < \sum_{k=1}^{\infty} \frac{1}{2^{k}} < \infty
 ,\end{align*} то по лемме \ref{lemma:borel-cantelli} Бореля-Кантелли $\mu(E) = 0 \implies \mu(F) = 0$. Значит, сходимость $f_{n_k} \to f$ есть $\mu$-почти всюду.
\end{proof}
\begin{crly}
 Если $f_n \to f$ $\mu$-почти всюду и $f_n \xrightarrow{\mu} g$, то $f=g$ $\mu$-почти всюду.
\end{crly}
\begin{crly}
 \label{corollary:measure_convergence_with_majoranta}
 Если $f_n \xrightarrow{\mu} f$ и $\left| f_n \right| \leqslant g$ $\mu$-почти всюду, то $\left| f \right| \leqslant g$ $\mu$-почти всюду.
\end{crly}
\begin{proof}
 $f_{n_k} \to f$, $\left| f_{n_k}(x) \right| \leqslant g(x) \implies \left| f(x) \right| \leqslant g(x)$  при $\mu$-почти всех $x \in X$.
\end{proof}

Наконец, мы подошли к двум новым предельным теоремам: теореме \ref{theorem:lebesgue-majoring-convergence} Лебега о мажорируемой сходимости, и лемме \ref{theorem:lemma_fatoo} Фату.

\begin{thm}[Лебега о мажорируемой сходимости]
 \label{theorem:lebesgue-majoring-convergence}
 Пусть $(X, \A, \mu)$ --- пространство с мерой, функции $\{f_{n}\}_{n=1}^{\infty} $, $f_n \colon\, X \to [-\infty, +\infty]$ измеримы. Пусть $f_n \to f$ $\mu$-почти всюду или $f_n \xrightarrow{\mu} f$.

 Пусть существует функция $g \colon\, X \to [0,\infty]$ --- неотрицательная суммируемая (то есть $\int_{X}  g \, d\mu < \infty$) такая, что $\left| f_n \right| \leqslant g$ $\mu$-почти всюду для любого $n$.

 Тогда \begin{align*}
  \lim_{n \to \infty} \int\limits_{X} f_n \, d\mu  = \int\limits_{X} f \, d\mu  
 .\end{align*} 
\end{thm}
Такая функция $g$  называется \textit{мажорантой}.
\begin{proof} По следствию \ref{corollary:measure_convergence_with_majoranta} (в случае сходимости $\mu$-почти всюду это и так очевидно) функция $g$ также мажорирует предел: неравенство $\left| f \right| \leqslant g$ выполнено $\mu$-почти всюду. По основной оценке интеграла (утверждение \ref{claim:basic_estimation_of_integral})  все интегралы конечны:
 \begin{align*}
  \left| \int\limits_{X} f_n \, d\mu \right| \leqslant \int\limits_{X} \left| f_n \right| \, d\mu \leqslant \int\limits_{X} g \, d\mu < \infty  \\
  \left| \int\limits_{X} f \, d\mu \right| \leqslant \int\limits_{X} \left| f \right| \, d\mu \leqslant \int\limits_{X} g \, d\mu < \infty 
 .\end{align*} Далее рассмотрим сначала случай конечной меры.

 \begin{enumerate}
  \item Пусть сначала $\mu(X) < \infty$. Тогда по теореме \ref{theorem:almost_everywhere_convergence_implies_measure_convergence_on_finite_measure} можно рассматривать только случай $f_n \xrightarrow{\mu} f$. Возьмём любой $\eps > 0$. Оценим разность:
   \begin{align*}
    \left| \int\limits_{X} f_n \, d\mu - \int\limits_{X} f \, d\mu \right|   &\leqslant \int\limits_{X} \left| f_n - f \right| \, d\mu  = \\
    &= \int\limits_{\left\{ x \mid \left| f_n - f \right| \geqslant \eps \right\}} \left| f_n - f \right| \, d\mu  + \int\limits_{\left\{ x \mid \left| f_n - f \right| < \eps \right\}} \left| f_n - f \right| \, d\mu   \leqslant \\
    &\leqslant \int\limits_{E_{n,\eps}} (\left| f_n \right| + \left| f \right|) \, d\mu + \eps \int\limits_{X} 1 \, d\mu    = \\
    &= 2 \int\limits_{E_{n,\eps}} g \, d\mu  + \eps \mu(X)
    ,\end{align*} где $E_{n,\eps} = \left\{ x \in X \mid \left| f_n - f \right| \geqslant \eps \right\}$. По абсолютной непрерывности интеграла (утверждение \ref{claim:absolute_continuity_of_integral}) \begin{align*}
    \int\limits_{F} g \, d\mu < \eps 
   \end{align*} для любого $F$ такого, что $\mu(F) < \delta(\eps)$. Так как $f_n \xrightarrow{\mu} f$, то при больших $n$
   \begin{align*}
    \mu \left( E_{n,\eps} \right) < \delta(\eps)
   .\end{align*} Тогда при больших $n$
   \begin{align*}
    \left| \int\limits_{X} f_n \, d\mu - \int\limits_{X} f \, d\mu   \right| \leqslant 2\eps + \eps \mu(X)
    .\end{align*} Так как это верно для любого $\eps > 0$, то \begin{align*}
    \lim_{n \to \infty} \left| \int\limits_{X} f_n \, d\mu - \int\limits_{X} f \, d\mu    \right|  = 0
   .\end{align*} 

  \item Теперь общий случай. По лемме \ref{lemma:about_exhaustion_of_integral} об исчерпывании интеграла найдём множество $A \in \A$ такое, что $\int_{X \setminus A} g \, d\mu   < \eps$ и $\mu(A) < \infty$. Тогда \begin{align*}
    \left| \int\limits_{X} f_n \, d\mu - \int\limits_{X} f \, d\mu    \right| &\leqslant \int\limits_{X \setminus A} \left| f_n - f \right| \, d\mu  + \int\limits_{A} \left| f_n - f \right| \, d\mu  \leqslant  \\
    &\leqslant 2 \int\limits_{X \setminus A}  g \, d\mu + \int\limits_{A} \left| f_n - f \right| \, d\mu < \\
    &< 2 \eps + \int\limits_{A} \left| f_n - f \right| \, d\mu
   .\end{align*} Так как по шагу 1 слагаемое $\int_{A} \left| f_n - f \right| \, d\mu$ стремится к нулю, то и
   \begin{align*}
    \lim_{n \to \infty} \left| \int\limits_{X} f_n \, d\mu - \int\limits_{X} f \, d\mu   \right| = 0 
   .\end{align*} 
 \end{enumerate}
\end{proof}

\begin{exmpl}[применение теоремы \ref{theorem:lebesgue-majoring-convergence}]
 Пусть $\mu$ --- мера на $\B(\R)$ такая, что
 \begin{align}
  \label{example:theorem_lebesgue_about_majoring_convergence:condition}
  \int\limits_{[0, 1)} \arctg(x) \, d\mu < \infty
 .\end{align} Вычислим предел
 \begin{align*}
  L = \lim_{n \to \infty} \int\limits_{[0, +\infty)} \arctg(x^{n}) \, d\mu
 .\end{align*} Разобьём пределы интегрирования на три промежутка:
 \begin{align*}
  L &= \lim_{n \to \infty} \int\limits_{[0, 1)} \arctg(x^{n}) \, d\mu + \lim_{n \to \infty} \int\limits_{\left\{ 1 \right\}}  \arctg(x^{n}) \, d\mu + \lim_{n \to \infty} \int\limits_{(1, +\infty)}    \arctg(x^{n}) \, d\mu
 .\end{align*} 
 Второе слагаемое --- интеграл по одноточечному множеству:
 \begin{align*}
  \lim_{n \to \infty} \int\limits_{\left\{ 1 \right\}}   \arctg(x^{n}) \, d\mu = \lim_{n \to \infty} \arctan(1^{n}) \cdot \mu(\left\{ 1 \right\}) = \frac{\pi}{4} \cdot \mu(\left\{ 1 \right\})
 .\end{align*} Для третьего слагаемого воспользуемся теоремой \ref{theorem:levi} Леви: функции $\arctg(x^{n})$  неотрицательны и возрастают к $\pi / 2$ на $(1,+\infty)$:
 \begin{align*}
  \lim_{n \to \infty} \int\limits_{(1,+\infty)}    \arctg(x^{n}) \, d\mu &= \int\limits_{(1,+\infty)}  \lim_{n \to \infty} \arctan(x^{n}) \, d\mu = \int\limits_{(1,+\infty)}  \frac{\pi}{2} \,d\mu = \frac{\pi}{2} \cdot \mu((1,+\infty))
 .\end{align*} Наконец, для первого слагаемого воспользуемся теоремой \ref{theorem:lebesgue-majoring-convergence} Лебега о мажорируемой сходимости. Функции $\arctg(x^{n})$ сходятся к $0$ всюду на $[0, 1)$ с мажорантой $\arctg(x)$ (суммируемость дана в формуле \eqref{example:theorem_lebesgue_about_majoring_convergence:condition}):
 \begin{align*}
  \lim_{n \to \infty} \int\limits_{[0, 1)}   \arctg(x^{n}) \, d\mu = \int\limits_{[0,1)} \lim_{n \to \infty} \arctg(x^{n}) \, d\mu = \int\limits_{[0, 1)} 0 \, d\mu = 0
 .\end{align*}
 Таким образом, ответ равен
 \begin{align*}
  L = \frac{\pi}{4} \cdot \mu(\left\{ 1 \right\}) + \frac{\pi}{2} \cdot \mu((1,+\infty))
 .\end{align*} 
\end{exmpl}

\begin{remrk}
 \label{remrk:finite_measure:theorem_lebesgue_majoring_convergence}
 Пусть мера пространства $X$ конечна ($\mu(X) < \infty$), $f_n \to f$ $\mu$-почти всюду, а также $\left| f_n \right| \leqslant C$ для любого $n$ ($f_n$ равномерно ограничены). Тогда \begin{align*}
  \lim_{n \to \infty} \int\limits_{X} f_n \, d\mu = \int\limits_{X} f \, d\mu    
 .\end{align*} 
\end{remrk}
\begin{proof}
 Теорема \ref{theorem:lebesgue-majoring-convergence} с мажорантой $g = C$.
\end{proof}
