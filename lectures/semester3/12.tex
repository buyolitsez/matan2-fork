% 2022.11.18 Lecture 12

Итак, мы получили измеримость образа под действием гладкого отображения. Как же вычислить меру образа? Для этого нужно воспользоваться основной идеей дифференциального исчисления --- приблизить гладкое линейным. Об этом будет следующая теорема. 

\begin{thm}[%
]
\label{theorem:measure_of_smooth_image}
 Пусть $\Omega_1, \Omega_2$ --- области в $\R^{n}$, $\Phi \colon\; \Omega_1 \to \Omega_2$ --- диффеоморфизм. Тогда для любого измеримого множества $E \subset \Omega_1$, $E \in \alan$ верна формула:
 \begin{align*}
  \lan(\Phi(E)) = \int\limits_{E} \left| \det J_{\Phi}(x) \right| \, d\lan(x)  
 ,\end{align*} где $J_{\Phi}(x)$ --- матрица Якоби (дифференциал) $\Phi$ в точке $x$.
\end{thm}
\begin{lm}
 \label{lemma:measure_of_smooth_image_leq}
 Если $\Psi \in C^{1}(\Omega_1, \Omega_2)$ --- диффеоморфизм и $E \in \alan$, $E \subset \Omega_1$, то \begin{align*}
  \lan(\Psi(E)) \leqslant \int\limits_{E} \left| \det J_{\Psi}(x) \right| \, d\lan
 .\end{align*} 
\end{lm}
\begin{proof}
 Образ $\Psi(E)$ измерим по теореме \ref{theorem:smooth_image_of_measurable_is_measurable}. Рассмотрим меру
 \begin{align*}
  \nu(E) = \lan(\Psi(E \cap \Omega_1))
 \end{align*} на $\alan$. Это мера, так как $\Psi$ --- биекция:
 \begin{align*}
  \nu \left( \bigsqcup_{k=1}^{\infty} E_k \right) &= \lan \left( \Psi \left( \bigsqcup_{k=1}^{\infty} (E_k \cap \Omega_1) \right) \right) = \lan \left( \bigsqcup_{k=1}^{\infty} \Psi (E_k \cap \Omega_1) \right) = \\
  &= \sum_{k=1}^{\infty} \lan(\Psi(E_k \cap \Omega_1)) = \sum_{k=1}^{\infty} \nu(E_k)
 .\end{align*} 

 Кроме того, мера $\nu$ непрерывна относительно $\lan$ ($\nu \prec \prec \lan$): если $\lan(e) = 0$, то $\nu(e) = 0$ --- это было показано в доказательстве теоремы \ref{theorem:smooth_image_of_measurable_is_measurable}. {\color{red} TODO: выделить лемму о том, что $\lan(e) = 0 \implies \lan(\Phi(e)) = 0$ и сослаться на неё}. По теореме \ref{theorem:radon_nikodim} Радона-Никодима существует неотрицательная суммируемая функция $f \in L^{1}(\R^{n},\lan)$, $f \geqslant 0$  такая, что \begin{align*}
  \nu(E) = \int\limits_{E} f(x) \, d\lan(x)  
 .\end{align*} Однако в условии теоремы Радона-Никодима мы требовали конечность мер. Но в силу счётной аддитивности мы можем разбить на кубы, и удовлетворить условие теоремы. {\color{red} Как именно разбить на кубы/приблизить снизу пространство $\R^{n}$? Условие суммируемости удовлетворить нужно, так как мы потом применяем \ref{theorem:almost_all_points_are_lebesgue_points}}.

 По теореме \ref{theorem:almost_all_points_are_lebesgue_points} при почти всех $x \in \Omega_1$ \begin{align}
  \label{eq1:theorem:measure_of_smooth_image}
  f(x) = \lim_{r \to 0} \frac{1}{\lan(B(x, r))} \int\limits_{B(X, r)} f(y) \, d\lan(y)
 ,\end{align} ведь пространство $\R^{n}$ сепарабельное, мера Лебега $\lan$ регулярная и удовлетворяет условию удвоения, а функция $f$ суммируема.

 Нам достаточно доказать, что $f(x) \leqslant \left| \det J_{\Psi}(x) \right|$ при почти всех $x \in \Omega_1$. Продолжим \eqref{eq1:theorem:measure_of_smooth_image}:
 \begin{align}
  \label{eq2:theorem:measure_of_smooth_image}
  f(x) = \lim_{r \to 0}  \frac{\nu(B(x,r ))}{\lan(B(x,r))} = \lim_{r \to 0} \frac{\lan(\Psi(B(x,r)))}{\lan(B(x,r))}
 .\end{align} Воспользуемся тем, что отображение $\Psi$ всюду дифференцируемо:
 \begin{align*}
  \Psi(y) = \Psi(x) + J_{\Psi}(x) (y - x) + o(y - x)
 \end{align*} при $y \to x$. Это можно записать так:
 \begin{align*}
  \Psi(B(x,r)) \subset \Psi(x) + J_{\Psi}(x) (B(0, r)) + F_r 
 \end{align*} при $r \to 0$, где $F_r$ --- некоторое множество такое, что $\diam F_r = o(r)$. При этом здесь сумма понимается в смысле суммы Минковского: каждая точка с каждой точкой. Тогда для любого фиксированного $\delta > 0$ и для достаточно малого $r$ верно
 \begin{align*}
  &\Psi(B(x,r)) \subset \Psi(x) + J_{\Psi}(x) (B(0, r) + B(0, \delta r)) \\
  \implies &\Psi(B(x,r)) \subset \Psi(x) + J_{\Psi}(x)(B(0,(1+\delta)r))
 .\end{align*} Следовательно, есть неравенство
 \begin{align*}
  \lan(\Psi(B(x,r))) &\leqslant \lan(\Psi(x) + J_{\Psi}(x) (B(0, (1+\delta)r))) = \\ &= \lan(J_{\Psi}(x) (B(0, (1 + \delta)r)))  = \\
  &= \left| \det J_{\Psi}(x) \right| \lan(B(0, (1 + \delta)r)) = \\
  &= \left| \det J_{\Psi}(x) \right|(1+\delta)^{n} \lan(B(0,r))
 .\end{align*} Продолжая \eqref{eq2:theorem:measure_of_smooth_image} получаем, что при почти всех $x \in \Omega_1$:
 \begin{align*}
 f(x) &\leqslant \lim_{r \to 0}  \frac{\left| \det J_{\Psi}(x) \right|(1+\delta)^{n} \lan(B(0, r))}{\lan(B(0,r))} = \left| \det J_{\Psi}(x) \right| (1 + \delta)^{n}
.\end{align*} Так как это верно для любого $\delta > 0$, то $f(x) \leqslant \left| \det J_{\Psi}(x) \right|$ для почти всех $x \in \Omega_1$. Лемма доказана.
\end{proof}

Следующая лемма в некотором смысле обобщит результат предыдущей.

\begin{lm}
 \begin{align}
  \label{eq:lemma:measure_of_smooth_image_weird}
  \int\limits_{\Psi(E)} h (\Psi^{-1}(y)) \, d\lan(y)   \leqslant \int\limits_{E} h(x) \left| \det J_{\Psi}(x) \right| \, d\lan(x)  
 \end{align} для любой измеримой относительно $\alan$ функции  $h \geqslant 0$ и любого множества $E \in \alan$.
\end{lm}
\begin{proof}\
 \begin{enumerate}
  \item Докажем лемму для характеристической функции $\chi_F$, $F \in \alan$ и при условии $E = \Omega_1$. По предыдущей лемме \ref{lemma:measure_of_smooth_image_leq}:
   \begin{align*}
    \lan(\Psi(F)) \leqslant \int\limits_{F}   \left| \det J_{\Psi}(x) \right|\, d\lan
   .\end{align*}  Левая часть равна
   \begin{align*}
    \lan(\Psi(F)) = \int\limits_{\Omega_2}  \chi_{\Psi(F)}(y) \, d\lan(y) = \int\limits_{\Omega_2} \chi_F(\Psi^{-1}(y))  \, d\lan(y)
   ,\end{align*} так как $\Psi$ --- биекция. Правая часть равна
   \begin{align*}
    \int\limits_{F} \left| \det J_{\Psi}(x) \right| \, d\lan(x) = \int\limits_{\Omega_1}  \chi_F(x) \left| \det J_{\Psi}(x) \right| \, d\lan(x)
   .\end{align*} Неравенство \eqref{eq:lemma:measure_of_smooth_image_weird} доказано для $h = \chi_F$ и $E = \Omega_1$.

  \item По линейности неравенство \eqref{eq:lemma:measure_of_smooth_image_weird} верно при $E = \Omega_1$ для любой простой неотрицательной функции $h \geqslant 0$.

  \item По теореме \ref{theorem:approximation} об аппроксимации и теореме \ref{theorem:levi} Леви неравенство \eqref{eq:lemma:measure_of_smooth_image_weird} верно для всех неотрицательных измеримых $h \geqslant 0$ при $E = \Omega_1$.

  \item Для произвольного множества $E \in \alan$, $E \subset \Omega_1$ рассмотрим функцию $\hat h = \chi_E \cdot h$:
   \begin{align*}
    &\int\limits_{\Omega_2}  \chi_E(\Psi^{-1}(y)) h(\Psi^{-1}(y)) \, d\lan(y) \leqslant \int\limits_{\Omega_1}  \chi_E(x) h(x) \left| \det J_{\Psi}(x) \right| \, d\lan(x) \\
    \implies & \int\limits_{\Psi(E)}  h(\Psi^{-1}(y))\,d\lan(y) \leqslant \int\limits_{E}  h(x) \left| \det J_{\Psi}(x) \right|\,d\lan(x)
   .\end{align*} 
 \end{enumerate}
\end{proof}

\begin{proof}[\normalfont\textsc{Доказательство теоремы \ref{theorem:measure_of_smooth_image} о мере гладкого образа}]
 По лемме \ref{lemma:measure_of_smooth_image_leq} сразу есть неравенство в одну сторону:
 \begin{align*}
  \lan(\Phi(E)) \leqslant \int\limits_{E} \left| \det J_{\Phi}(x) \right| d\lan(x)
 .\end{align*} Докажем в другую сторону. Воспользуемся неравенством \ref{eq:lemma:measure_of_smooth_image_weird} для отображения $\Psi = \Phi^{-1}$ и функции $h \colon\, \Omega_2 \to \Omega_1$, $h(y) = \left| \det J_{\Phi}(\Phi^{-1}(y)) \right|$:
 \begin{align*}
  \int\limits_{E} \left| \det J_{\Phi}(x) \right| \, d\lan(x) &= \int\limits_{\Psi(\Phi(E))} h(\Psi^{-1}(x)) \,d\lan(x) \leqslant \int\limits_{\Phi(E)}  h(y) \left| \det J_{\Psi}(y) \right| \,d\lan(y) = \\
  &= \int\limits_{\Phi(E)} \left| \det J_{\Phi}(\Phi^{-1}(y)) \right| \left| \det J_{\Phi^{-1}}(y) \right| \, d\lan(y) = \int\limits_{\Phi(E)}  d\lan(y) = \\
  &= \lan(\Phi(E))
 .\end{align*} Предпоследнее равенство верно, так как дифференциалы взаимно обратны.
\end{proof}

В качестве примера вычислим объём шара с радиусом $R$.

\begin{exmpl}
 $\lambda_3(B(0, R)) = \frac{4}{3}\pi R^{3}$.
\end{exmpl}
\begin{proof}[\normalfont\textsc{Доказательство}]
 Введём сферические координаты. Рассмотрим диффеоморфизм $\Phi \colon\, \Omega_1 \to \Omega_2$, заданный по формуле
 \begin{align*}
  \Phi \colon\; (r, \varphi, \psi) \mapsto \begin{pmatrix}
   r \cos \varphi \cos \psi \\
   r \sin \varphi \cos \psi \\
   r \sin \psi
  \end{pmatrix} = \begin{pmatrix}
   x(r,\varphi,\psi) \\
   y(r,\varphi,\psi) \\
   z(r,\varphi,\psi) \\
  \end{pmatrix}
 ,\end{align*} где $\Omega_1 = (0, R) \times (0, 2\pi) \times (-\pi / 2, \pi / 2)$. Отображение $\Phi$ параметризует весь шар, за исключением множества меры нуль:
 \begin{align*}
  \lambda_3(\Omega_2) = \lambda_3(B(0, R))
 .\end{align*} Вычислим якобиан $\Phi$:
 \begin{align*}
  \det J_{\Phi} &= \begin{vmatrix}
   \frac{\partial x}{\partial r} & \frac{\partial x}{\partial \varphi} & \frac{\partial x}{\partial \psi} \\
   \frac{\partial y}{\partial r} & \frac{\partial y}{\partial \varphi} & \frac{\partial y}{\partial \psi} \\
   \frac{\partial z}{\partial r} & \frac{\partial z}{\partial \varphi} & \frac{\partial z}{\partial \psi} \\
   \end{vmatrix} = \begin{vmatrix}
   \cos\varphi\cos\psi & -r \sin \varphi \cos\psi & -r \cos\varphi\sin\psi \\
   \sin\varphi\cos\psi & r \cos\varphi\cos\psi & -r\sin\varphi\sin\psi \\
   \sin\psi & 0 & r \cos\psi
  \end{vmatrix} = \\
  &= \sin\psi \begin{vmatrix}
   -r \sin\varphi\cos\psi & -r\cos\varphi\sin\psi \\
   r\cos\varphi\cos\psi & -r\sin\varphi\sin\psi
  \end{vmatrix} + r\cos\psi \begin{vmatrix}
  \cos\varphi\cos\psi & -r\sin\varphi\cos\psi \\
  \sin\varphi\cos\psi & r\cos\varphi\cos\psi
  \end{vmatrix} = \\
  &= r^{2}(\sin^{2}\varphi + \cos^{2}\varphi)\cos\psi\sin^{2}\psi + r^{2}(\cos^{2}\varphi + \sin^{2}\varphi)\cos^{3}\psi = \\
  &= r^{2}\cos\psi(\sin^{2}\psi + \cos^{2}\psi) = r^{2}\cos\psi
 .\end{align*} Так как $\cos \psi > 0$  на $(-\frac{\pi}{2}, \frac{\pi}{2})$, то $\left| \det J_{\Phi} \right| = r^{2}\cos\psi$. Вычислим, наконец, объём шара, пользуясь формулой из теоремы \ref{theorem:measure_of_smooth_image}, а также теоремой \ref{theorem:fubini} Фубини:
  \begin{align*}
   \lambda_3(B(0, R)) &= \lambda_3(\Omega_2) =  \int\limits_{\Omega_1} r^{2}\cos \psi \, d\lambda_3 = \int\limits_{0}^{R} \int\limits_{0}^{2\pi} \int\limits_{-\pi / 2}^{\pi / 2} r^{2} \cos\psi \, d\psi\,d\varphi\,dr = \\
   &= 2\pi \int\limits_{0}^{R} r^{2}\,dr \int\limits_{-\pi / 2}^{\pi / 2} \cos\psi = 2\pi \frac{R^{3}}{3} \sin\varphi \rvert_{-\pi / 2}^{\pi / 2} = \frac{4}{3}\pi R^{3}
  .\end{align*} 
\end{proof}
\begin{crly}[формула замены переменной в интеграле Лебега]
 Пусть $\Omega_1, \Omega_2 \subset \R^{n}$ --- области, $\Phi \colon\, \Omega_1 \to \Omega_2$ --- диффеоморфизм. Пусть $h \colon\, \Omega_1 \to [-\infty, +\infty]$ --- суммируемая функция. Тогда
 \begin{align*}
  \int\limits_{\Omega_2} h(\Phi^{-1}(y)) \, d\lan(y) = \int\limits_{\Omega_1} h(x) \left| \det J_{\Phi}(x) \right| \, d\lan(x)  
 .\end{align*} 
\end{crly}
\begin{proof}
 На характеристических функциях --- это предыдущая теорема  \ref{theorem:measure_of_smooth_image}. Переход к простым неотрицательным по линейности. Переход к измеримым неотрицательным по теореме \ref{theorem:approximation} об аппроксимации и теореме \ref{theorem:levi} Леви. Переход к функциях любого знака рассмотрением $f_{\pm} = \max(\pm f, 0)$.
\end{proof}
