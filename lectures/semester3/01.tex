В этом семестре в основном будет \textit{теория меры}. Программистам она нужна для того, чтобы понимать \textit{вероятность}. В этом семестре будет несколько \textit{глубоких} теорем (красивых фактов), каких не было в предыдущих семестрах. Глубокое осознание курса должно поднять слушателя на новый математический уровень.

\section{Суммирование неотрицательных семейств} \label{section:summation_of_nonnegative_families}

\begin{df}
 За $[0, \infty]$ обозначим множество неотрицательных вещественных чисел $[0, \infty)$ с новым элементом $\infty$. На этом множестве можно естественным образом доопределить отношение порядка, сложение и умножение: 
 \begin{itemize}
  \item $a < \infty$ для любого вещественного $a \geqslant 0$.
  \item $a + \infty = \infty + a = \infty$ для любого $a \in [0, \infty]$.
  \item $a \cdot \infty = \infty \cdot a = \infty$, если $a > 0$ (в том числе при $a = \infty$).
  \item $0 \cdot \infty = \infty \cdot 0 = 0$.
 \end{itemize}
\end{df}

Полезность множества $[0, \infty]$ в том, что в нём можно всегда взять сумму, причём по любому семейству индексов.

\newcommand{\I}{\ensuremath \mathcal{I}}
\begin{df}
 Пусть $ \I $ --- это множество (любое, возможно несчётное), и для каждого $ i \in \I $ сопоставлено неотрицательное число $ a_i \in [0, \infty] $. Определим
 \begin{align*}
  \sum_{i \in \I} a_i = \sup \left\{ \sum_{j \in F} a_j \Mid F \subset \I,\; F \text{ --- конечноe }\right\}
 .\end{align*}
\end{df}

\begin{thm}\
 \label{theorem:facts_about_infinite_nonnegative_sums}
 \begin{itemize}
  \item Если $ \sum_{i \in \I} a_i < \infty $, то существует не более чем счетное множество $ E \subset \I $ такое, что $ a_i = 0 $ для всех $ i \in \I \setminus E $.
  \item Eсли $ \I $ не более, чем счётно, и $ \I = \{i_k\}_{k \in \N} $, то
   \begin{align*}
    \sum_{i \in \I} a_i = \sum_{k=1}^\infty a_{i_k}
   ,\end{align*} где в правой части стоит сумма ряда в старом смысле --- предел последовательности частичных сумм.
 \end{itemize}
\end{thm}
\begin{proof}\
 \begin{itemize}
  \item Пусть $ \sum_{i\in \I} a_i < \infty $. Тогда для любого $ N \in \N $ существует лишь конечное число $ i \in \I $ таких, что $ a_i \geqslant 1/N $, ведь иначе из них можно составить множество с неограниченной суммой. Тогда
   \begin{align*}
    E = \left\{ i \in I \mid a_i > 0 \right\} = \bigcup_{N=1}^{\infty} \left\{ i \in I \mid a_i \geqslant 1/N \right\}
   .\end{align*} Получается, множество $ E $ не более, чем счётное (как счётное объединение конечных множеств).
  \item Докажем неравенство в обе стороны. Для любого $N$ верно  \begin{align*}
    \sum_{k=1}^{N} a_{i_k} \leqslant \sum_{i \in \I} a_i,
   \end{align*} потому что левая часть --- сумма по конечному подмножеству. Взяв предел при $N \to \infty$ получаем неравенство в одну сторону:
   \begin{align*}
    \sum_{k=1}^{\infty} a_{i_k} \leqslant \sum_{i \in \I} a_i
   .\end{align*}

   Теперь докажем в обратную сторону. Пусть $ F \subset \I $ конечноe. Тогда 
   \begin{align*}
    \sum_{i \in F} a_i \leqslant \sum_{k=1}^N a_{i_k} \leqslant \sum_{k = 1}^{\infty} a_{i_k}
   \end{align*} для некоторого $ N \in \N $ (например, максимальный номер индекса из $F$). Перейдя к супремуму в левой части, получаем нужное:
   \begin{align*}
    \sum_{i \in \I} a_i = \sup_{\substack{F \subset \I \\ F \text{ --- конечное }}} \sum_{i \in F} a_i \leqslant \sum_{k = 1}^{\infty} a_{i_k}
   .\end{align*}
 \end{itemize}
\end{proof}
\begin{crly}
 Пусть $ \I $ счётное множество, $ \{i_k\}_{k \in \N} = \{i_{\varphi(k)}\}_{k\in\N} = \I$ --- его нумерации двумя разными способами ($ \varphi \colon\; \N \to \N $ --- биекция). Тогда
 \begin{align*}
  \sum_{k = 1}^{\infty} a_{i_k} = \sum_{k = 1}^{\infty} a_{i_{\varphi(k)}}
 .\end{align*}
\end{crly}
\begin{proof}
 Применить второй пункт теоремы \ref{theorem:facts_about_infinite_nonnegative_sums}.
\end{proof}

Интересно, что данное следствие неверно, если разрешить отрицательные числа.

\section{Наивная длина}

В этом параграфе мы совершим первую, наивную попытку построить длину на прямой $\R$.

\begin{notatn*}\
 Введём некоторые обозначения, полезные в теории меры.
 \begin{itemize}
  \item Если $ X $ --- некоторое множество, то $ 2^X $ --- множество всех подмножеств множества $ X $.
  \item Будем писать $ A \sqcup B $ вместо $ A \cup B $, если дополнительно известно, что $ A \cap B = \varnothing $.
 \end{itemize}
\end{notatn*}

\begin{df}
 \textit{Наивная длина} --- это функция $ \mu \colon\, 2^\R \to [0,\infty] $ такая, что
 \begin{itemize}
  \item $ \mu ([0, 1)) = 1 $
  \item Если $\{A_{k}\}_{k=1}^{\infty} \subset \R $  --- последовательность не пересекающихся множеств, то
   \begin{align*}
    \mu \left( \bigsqcup_{k=1}^{\infty} A_k \right) = \sum_{k=1}^{\infty} \mu(A_k)
   .\end{align*} 
  \item $ \mu(A + x) = \mu(A) $ для любого множества $ A \subset \R $ и $ x \in \R $ (где $ A + x = \{ a + x \mid a \in A \} $ --- сумма по Минковскому).
 \end{itemize}
\end{df}
\begin{thm}
 Наивной длины не существует.
\end{thm}
\begin{proof}
 Введём на множестве $[0, 1)$ такое отношение эквивалентности: $ x \sim y $, если $ x-y \in \Q $. Это действительно отношение эквивалентности, поэтому единичный полуинтервал бьётся на классы эквивалентности:
 \begin{align*}
  [0, 1) = \bigsqcup_{J \text{ --- класс экв. }} J
 .\end{align*} По аксиоме выбора из каждого класса $ J $ можно выбрать представителя $ x_J $. Обозначим
 \begin{align*}
  E = \{ x_J \mid J \text { --- класс экв. } \}
 .\end{align*} Наблюдение:
 \begin{align*}
  [0, 1) \subset \bigcup_{q \in [-1, 1] \cap \Q} (q + E)
 .\end{align*} Действительно, если $ x \in [0, 1) $, то существует класс $ J $ такой, что $ x - x_J = q \in \Q $, то есть $ x \in q + E $. При этом $-1 < q < 1$.

 Второе наблюдение:
 \begin{align*}
  \bigcup_{q \in \Q \cap [-1, 1]} (q + E) = \bigsqcup_{q \in \Q \cap [-1, 1]} (q + E)
 .\end{align*} Действительно, пусть $ (q_1 + E) \cap (q_2 + E) \ni x $. Тогда $ q_1 + x_{J_1} = x = q_2 + x_{J_2} $. Тогда
 \begin{align*}
  x_{J_1} - x_{J_2} = q_2 - q_1 \in \Q
 .\end{align*} Следовательно, $ x_{J_1} = x_{J_2} \;\Longrightarrow q_1 = q_2$. Значит, если они пересеклись, то они совпадают.

 Мы доказали два полезных включения:
 \begin{align*}
  [0, 1) \subset \bigsqcup_{q \in [-1, 1] \cap \Q} (q + E) \subset [-1, 2)
 .\end{align*} Пользуясь аксиомами, получим неравенства на длины:
 \begin{align*}
  1 \leqslant &\sum_{q \in \Q \cap [-1, 1]} \mu(q + E) \leqslant 3 \\
  1 \leqslant &\sum_{q \in \Q \cap [-1, 1]} \mu(E) \leqslant 3 
 .\end{align*} Но то, что по середине --- это либо $ 0 $, если $ \mu(E)=0 $, либо $ +\infty $, иначе. В любом случае противоречие! 

 Для честности выведем монотонность по включению. Пусть $ A \subset B $. Тогда
 \begin{align*}
  \mu(B) = \mu(A \sqcup (B \setminus A)) = \mu(A) + \mu(B \setminus A) \geqslant \mu(A)
 .\end{align*} Показать $ \mu([-1, 2)) = 3 $ можно так:
 \begin{align*}
  \mu([-1, 2)) &= \mu([-1, 0) \sqcup [0, 1) \sqcup [1, 2)) = \\ &= \mu([-1, 0)) + \mu([0, 1)) + \mu([1, 2)) = 3 \cdot \mu([0, 1)) = 3
 .\end{align*}
\end{proof}

\section{Системы множеств и функции на них}

Итак, мы выяснили, что наивной длины нет. Классический способ обойти это (который мы будем изучать) --- определить длину не на всех множествах, а на \textit{хороших} множествах.

\begin{notatn*}
 Ещё одно полезное обозначение: если зафиксировано некоторое универсальное множество $X$, и есть подмножество $A \subset X$, то $A^{c} = X \setminus A$ --- дополнение подмножества $A$.
\end{notatn*}

\begin{prop*}
 Полезные формулы --- законы де Моргана:
 \begin{align*}
  \left( \bigcup_{i \in \I} A_i \right)^c &= \bigcap_{i \in \I} A_i^c \\
  \left( \bigcap_{i \in \I} A_i \right)^c &= \bigcup_{i \in \I} A_i^c
 \end{align*} для любого семейства $ \I $ и любых множеств $ A_i \subset X $.
\end{prop*}

\begin{df}
 Пусть $ X $ --- множество. Набор подмножеств $ \mathcal{P} \subset 2^X $ --- \textit{полукольцо}, если
 \begin{enumerate}
  \item $ \varnothing \in \mathcal{P} $
  \item $ P_1, P_2 \in \mathcal{P} \;\Longrightarrow\; P_1 \cap P_2 \in \mathcal{P} $
  \item $ P_1, P_2 \in \mathcal{P} \;\Longrightarrow\; P_1 \setminus P_2 = \bigsqcup_{j=1}^N Q_j$ для некоторых $ Q_1, \ldots, Q_N \in \mathcal{P} $.
   \label{enum3:definition:semiring}
 \end{enumerate}
\end{df}
\begin{df}
 Пусть $ X $ --- множество. Набор подмножеств $ \A \subset 2^X $ --- \textit{алгебра} множеств, если
 \begin{enumerate}
  \item $ \varnothing \in \A $
  \item $ A_1, A_2 \in \A \;\Longrightarrow\; A_1 \cap A_2 \in \A $
   \label{enum:set_algebra_axiom_2}
  \item $ A \in \A \;\Longrightarrow\; A^c \in \A $ (аксиома симметричности)
 \end{enumerate}
\end{df}
\begin{df}
 Набор подмножеств $ \A \subset 2^X $ --- \textit{$ \sigma $-алгебра}, если
 \begin{enumerate}
  \item $ \varnothing \in \A $
  \item Пусть $\{A_{k}\}_{k=1}^{\infty} $ --- счётный набор множеств $A_k \in \A$. Тогда
   \begin{align*}
    \bigcap_{k=1}^{\infty} A_k \in \A
   .\end{align*} 
   \label{enum:sigma_algebra_axiom_2}
  \item $ A \in \A \;\Longrightarrow\; A^c \in \A $ (аксиома симметричности)
 \end{enumerate}
\end{df}
\begin{exmpl}\
 \begin{itemize}
  \item $ \sigma $-алгебра $ \;\Longrightarrow\; $ алгебра $ \;\Longrightarrow\; $ полукольцо. Последняя импликация: $ A_1 \setminus A_2 = A_1 \cap (A_2^c) $.
  \item $ 2^X $ --- $ \sigma $-алгебра.
  \item $ \left\{ \varnothing, X \right\} $ --- $ \sigma $-алгебра.
  \item $ X = \R^2 $, $ \A = \left\{ E \subset \R^{2} \mid E \text{ --- ограничено или } E^c \text { --- ограничено } \right\}  $. Тогда $ \A $ --- алгебра:
   \begin{enumerate}
    \item $ \varnothing \in \A $
    \item Пусть $ A, B \in \A $ и одно из них ограничено. Тогда $ A \cap B $ ограничено и $ A \cap B \in \A $. А если $ A^c, B^c $ ограничены, тогда $ (A \cap B)^c = A^c \cup B^c $ ограничено. Тогда $ A \cap B \in \A $.
    \item $ A \in \A \;\Longleftrightarrow\; A^c \in \A $
   \end{enumerate}
   При этом $ \A $ не является $ \sigma $-алгеброй:
   \begin{align*}
    \A \not\ni \R^2 \setminus \left\{ (0, x) \mid x \geqslant 0 \right\}  = \bigcap_{k=1}^\infty A_k, \quad A_k \in \A
   .\end{align*} Нужно взять
   \begin{align*}
    A_k = \R^2 \setminus \left\{ (0, x) \mid 0 \leqslant x \leqslant k \right\} 
   .\end{align*}
 \end{itemize}
\end{exmpl}

\begin{prop}
 \label{proposition:axiom_2_in_set_algebras_equiv_to_union}
 Аксиома \ref{enum:set_algebra_axiom_2} алгебры множеств эквивалентна условию
 \begin{align*}
  \bigcup_{k=1}^{N} A_k \in \A
 \end{align*} для любых $A_1, \ldots, A_k \in \A$.

 Аналогично, аксиома \ref{enum:sigma_algebra_axiom_2} $\sigma$-алгебры эквивалентна условию
 \begin{align*}
  \bigcup_{k=1}^{\infty} A_k \in \A
 \end{align*} для любой последовательности $\{A_{k}\}_{k=1}^{\infty}$ множеств из $\A$.
\end{prop}
\begin{proof}[\normalfont\textsc{Доказательство}]
 Применить законы де Моргана.
\end{proof}

Далее рассмотрим самый важный пример полукольца.

\begin{df}
 \textit{Полукольцом ячеек} в $\R$ называется семейство 
 \begin{align*}
  \mathcal{P}_1 = \left\{ [a, b) \mid a \leqslant b,\; a, b \in \R \right\} 
 .\end{align*} Полуинтервал $[a, b) \in \p_1$ называется (одномерной) \textit{ячейкой}.
\end{df}
\begin{lm}
 $ \mathcal{P}_1 $ --- полукольцо.
\end{lm}
\begin{proof}\
 \begin{enumerate}
  \item $ \varnothing \in \mathcal{P}_1 $, так как $ \varnothing = [0, 0) $.
  \item $$
   [a_1, b_1) \cap [a_2, b_2) = 
   \begin{cases}
    \varnothing \text{, если } \max(a_1, a_2) \geqslant \min(b_1, b_2) \\
    [\max(a_1, a_2), \min(b_1, b_2) ) \text{, иначе}
   \end{cases}
   $$
   В обоих случаях имеем ячейку: $[a_1, b_1) \cap [a_2, b_2) \in \mathcal{P}_1 $.
  \item Здесь нужно разобрать несколько случаев: в каждом из них разность раскладывается на одну или две ячейки.
 \end{enumerate}
\end{proof}
\begin{remrk*}
 Полукольцо ячеек $ \mathcal{P}_1 $ не является алгеброй.
 Например, 
\begin{align*}
 [0, 1)^c = (-\infty, 0) \cup [1, +\infty) \not \in  \mathcal{P}_1.
\end{align*}
\end{remrk*}
\begin{lm}
 Пересечение любого числа $\sigma$-алгебр есть $\sigma$-алгебра: если $\left\{\A_i\right\}_{i \in \I}$  --- семейство $\sigma$-алгебр множеств из $X$, то
 \begin{align*}
  \A = \bigcap_{i \in \I} \A_i
 \end{align*} --- $\sigma$-алгебра множеств из $X$.
\end{lm}
\begin{proof}\
 \begin{enumerate}
  \item $ \varnothing \in \A $, так как $ \varnothing \in \A_i $ для всех $ i \in \I $.
  \item $ A_j \in \A \;\Longrightarrow\; A_j \in \A_i, \forall i, j  \;\Longrightarrow\; \bigcap_j A_j \in 
   \A_i \;\Longrightarrow\; \bigcap_j A_j \in \A$
  \item $ A \in \A \;\Longrightarrow\; A \in \A_i,  \forall i \in \I \;\Longrightarrow\; A^c \in \A_i, \forall i \in \I \;\Longrightarrow\; A^c \in \bigcap_{i \in \I} \A_i = \A$ 
 \end{enumerate}
\end{proof}
\begin{lm}
 Если $ E \subset 2^X $ --- произвольное семейство множеств, то существует единственная $\sigma$-алгебра $ \B(E) $ --- наименьшая по включению $ \sigma $-алгебра, содержащая $ E $.
\end{lm}
\begin{proof}
 Есть просто формула:
 \begin{align*}
  \B(E) := \bigcap_{E \subset \A \text{---} \sigma \text{-алгебра}} \A
 .\end{align*} Пересечение берётся не по пустому множеству, так как $ E \subset 2^X $ и  $ 2^X $ --- $ \sigma $-алгебра.
\end{proof}
\begin{df}
 Наименьшая $\sigma$-алгебра $\B(E)$, содержащая семейство множеств $E \subset 2^{X}$ называется \textit{борелевской оболочкой} семейства $E$.
\end{df}
Термин <<борелевская оболочка>> (или <<борелевская $\sigma$-алгебра>>) чаще используется в следующем контексте.
\begin{df}
 \textit{Борелевской оболочкой} (или \textit{боеревской $\sigma$-алгеброй}) топологического пространства $(X, \tau)$ называется наименьшая $\sigma$-алгебра, содержащая все открытые (и, следовательно, замкнутые) подмножества в $X$. Обозначение: $\B(X)$.

 Для борелевской оболочки $\R$ есть стандартное обозначение: $\B_1 = \B(\R)$.
\end{df}

Таким образом, наша текущая задача --- задать длину не на всех подмножествах $\R$, а на некоторой $\sigma$-алгебре подмножеств из $\R$, желательно хотя бы на борелевской оболочке $\B_1$.

Раз уж мы выяснили на чём мы хотим задавать длину, то теперь мы можем перейти к желаемым свойствам длины.

\begin{df}
 Пусть $\p$ --- полукольцо в $ X $ и $ \mu \colon\; \mathcal{P} \to [0, \infty] $ --- неотрицательная функция. Функция $ \mu $ называется \textit{конечно-аддитивной}, если для любых не пересекающихся множеств $ P_1, P_2, \ldots, P_N \in \p $ выполняется равенство
 \begin{align*}
  \mu\left( \bigsqcup_{k=1}^N P_k \right) = \sum_{k=1}^{N} \mu(P_k)
 .\end{align*}

 Функция $ \mu $ называется \textit{счётно-аддитивной}, если для любой последовательности $\{P_{k}\}_{k=1}^{\infty} \subset \p$ множеств из $\p$ выполняется
 \begin{align*}
  \mu \left( \bigsqcup_{k=1}^{\infty} P_k \right) = \sum_{k=1}^{\infty} \mu (P_k)
 .\end{align*} 
\end{df}

Эти свойства привели нас к определению \textit{меры}.

\begin{df}
 \label{definition:measure}
 Счётно-аддитивная неотрицательная функция $\mu \colon\, \A \to [0,\infty] $  на $\sigma$-алгебре $\A$ называется \textit{мерой}.

 Множества, лежащие в $\A$ называются \textit{измеримыми} по мере $\mu$.
\end{df}
\begin{exmpl}\
 \begin{itemize}
  \item Тождественный нуль $ \mu \equiv 0 $ --- мера на любой $\sigma$-алгебре $\A$.
  \item Тождественная бесконечность $ \mu \equiv \infty $ --- мера на любой $\sigma$-алгебре $\A$. В других источниках в определении меры требуют, чтобы $\mu(\varnothing) = 0$. Это не существенно, ведь если $\mu(\varnothing) > 0$, то $\mu \equiv \infty$.
  \item Пусть $ X $ --- множество, $ E \subset X $. Определим \textit{считающую меру}:
   \begin{align*}
    \mu_E(A) =
    \begin{cases}
     \left| A \cap E \right| \text{, если } A \cap E \text { конечно}, \\
     \infty \text {, иначе }
    \end{cases}
   .\end{align*} на $\sigma$-алгебре $ \A = 2^X $. Здесь $\left| F \right|$ --- количество элементов в конечном множестве $F$. Считающая мера сужает теорию меры на дискретный случай: из многих теорем в теории меры, можно получить частные дискретные случаи, подставив считающую меру.
  \item $ X = \R $, $\A = 2^{\R}$, $$ \delta_x(A) =
   \begin{cases}
    1, \text{ если } x \in A \\
    0, \text { иначе }
   \end{cases}$$ --- \textit{дельта мера Дирака}. Также называется \textit{точечной нагрузкой}.
  \item $ X = \R $, $  \left\{ a_k \right\}_{k=1}^\infty \subset [0, \infty] $, $ \delta_{x_k} $ --- дельта меры Дирака в точках $ x_k $. Тогда
   \begin{align*}
    \mu = \sum_{k=1}^\infty a_k \delta_{x_k}
   \end{align*} --- тоже мера.
  \item Более общо, линейная комбинация мер (конечно, с неотрицательными коэффициентами) --- тоже мера.
 \end{itemize}
\end{exmpl}

Приведём важный пример конечной-аддитивной функции на полукольце ячеек, из которой впоследствии мы и построим длину.

\begin{exmpl}
 Функция \begin{align*}
  \mu_0([a, b)) = b - a
 ,\end{align*} заданная на полукольце ячеек $\p_1$, является конечно-аддитивной. Эту функцию можно назвать \textit{длиной} ячеек.
\end{exmpl}
\begin{proof}[\normalfont\textsc{Доказательство}]
 Действительно, пусть $\left[a, b\right) = \bigsqcup_{k=1}^{N} \left[a_k, b_k\right)$. Отсортируем ячейки в порядке слева направо: $b_k \leqslant {a_{k+1}}$ для любого $1 \leqslant k < N$. Тогда $a_1 = a$, $b_N = b$ и $b_k = a_{k+1}$ для любого $1 \leqslant k < N$. Тогда
 \begin{align*}
  b - a = \sum_{k=1}^{N} (b_k - a_k)
 ,\end{align*} что и требовалось доказать.
\end{proof}

Техническая лемма:

\begin{lm}[о подчинённом разбиении]
 \label{lemma:about_subordinate_partition}
 Пусть $ X $ --- множество, $ \mathcal{P} $ --- полукольцо, $ P_1, \ldots, P_N \in \mathcal{P} $, $ P \in \mathcal{P} $. Тогда:
 \begin{enumerate}
  \item $ \displaystyle P \setminus \bigcup_{k=1}^N P_k = \bigsqcup_{j=1}^M Q_j $, где $ Q_j \in \mathcal{P} $
  \item $ \displaystyle \bigcup_{k=1}^N P_k = \bigsqcup_{k=1}^N \bigsqcup_{j=1}^{M_k} Q_{k,j} $, где $ Q_{k,j} \in \mathcal{P} $ и $ Q_{k,j} \subset P_k $
 \end{enumerate} 
\end{lm}
\begin{proof}\
 \begin{enumerate}
  \item По индукции. $ N = 1$ --- аксиома 3 полукольца. Индукционный переход:
   \begin{align*}
    P \setminus \bigcup_{k=1}^{N+1} P_k = \left( P \setminus \bigcup_{k=1}^N P_k \right) \setminus P_{N+1} =  \bigsqcup_{j=1}^{M_N} \underbrace{\left( Q_j \setminus P_{N+1} \right)}_{\text{уже знаем по базе индукции}}
   .\end{align*}
  \item Снова индукция:
   \begin{align*}
    &P_1 = P_1 = Q_{1,1} \\
    &P_1 \cup P_2 = P_1 \sqcup (P_2 \setminus P_1) = Q_{1,1} \sqcup (Q_{2,1} \sqcup \ldots \sqcup Q_{2, M_2}) \\
    &P_1 \cup P_2 \cup P_3 = \underbrace{(P_1 \cup P_2)}_{\text{ по инд.}} \sqcup \underbrace{(P_3 \setminus (P_1 \cup P_2))}_{\text{ знаем из пункта 1 }} \\
    &P_1 \cup P_2 \cup \dots \cup P_N = \underbrace{\left(\bigcup_{k = 1}^{N - 1} P_k\right)}_{\text{ по инд. }} \sqcup \underbrace{\left(P_N \setminus \left(\bigcup_{k = 1}^{N - 1} P_k\right)\right)}_{\text{ знаем из пункта 1  } }
   .\end{align*}
 \end{enumerate}
\end{proof}
\begin{remrk}
 Второй пункт леммы \ref{lemma:about_subordinate_partition} верен и для счётного покрытия: если $\{P_{k}\}_{k=1}^{\infty} \subset \p $, то существуют $Q_{kj} \in \p$ такие, что
 \begin{align*}
  \bigcup_{k=1}^{\infty} P_k = \bigsqcup_{k=1}^{\infty} \bigsqcup_{j=1}^{N_k} Q_{kj}, \quad Q_{kj} \subset P_k
 .\end{align*} Действительно, можно построить это разбиение по индукции для каждого конечного объединения $\bigcup_{k=1}^{N} P_k $. Тогда возьмём объедение всех разбиений, оно и будет являться разбиением счётного покрытия. 
\end{remrk}

Конечно- и счётно-аддитивные функции обладают монотонностью.

\begin{crly}
 \label{corollary:downward_monotonicity_of_finite_additive_set_function}
 Пусть $\p$ --- полукольцо, функция $ \mu \colon\, \mathcal{P} \rightarrow [0, \infty]$ конечно-аддитивна. Пусть $ \bigsqcup_{k=1}^N P_k \subset P $, где $ P_k, P \in \mathcal{P} $. Тогда
 \begin{align*}
  \sum_{k=1}^{N}  \mu(P_k) \leqslant \mu(P)
 .\end{align*}
\end{crly}
\begin{proof}
 По лемме \ref{lemma:about_subordinate_partition} 
 \begin{align*}
  P \setminus \bigsqcup_{k=1}^N P_k = \bigsqcup_{j=1}^M Q_j 
 .\end{align*} Тогда:
 \begin{align*}
  P &= \bigsqcup_{k=1}^N P_k \sqcup \bigsqcup_{j=1}^M Q_j \\
  \implies \mu(P) &= \sum_{k=1}^{N} \mu(P_k) + \sum_{j=1}^{M} \mu(Q_j) \geqslant \sum_{k=1}^{N} \mu(P_k)
 .\end{align*}
\end{proof}
\begin{crly}
 \label{corollary:upward_monotonicity_of_finite_additive_set_function}
 Пусть $ \mathcal{P} $ --- полукольцо, $ \mu \colon\, \p \to [0,\infty]  $ конечно-аддитивна, и $ P \subset \bigcup_{k=1}^N P_k $, где $ P, P_k \in \mathcal{P} $. Тогда
 \begin{align*}
  \mu(P) \leqslant \sum_{k=1}^N \mu(P_k)
 .\end{align*}
\end{crly}
\begin{proof} По лемме \ref{lemma:about_subordinate_partition} о подчинённом разбиении:
 \begin{align*}
  P &= \bigcup_{k=1}^N (P_k \cap P) = \bigsqcup_{k=1}^N \bigsqcup_{j=1}^{M_k} Q_{k,j} \\
  \implies \mu(P) &= \sum_{k=1}^{N} \underbrace{\sum_{j=1}^{M_k} \mu(Q_{k,j})}_{\leqslant \mu(P_k) \text{ по следствию \ref{corollary:downward_monotonicity_of_finite_additive_set_function} }} \leqslant \sum_{k=1}^{N} \mu(P_k)
 .\end{align*}

\end{proof}
\begin{thm}[счётная аддитивность длины на полукольце ячеек]
 \label{theorem:cell_length_has_sigma_additivity}
 Функция множества $ \mu_0 \colon\, [a, b) \mapsto b - a $ на полукольце ячеек $ \mathcal{P}_1 $ является счётно-аддитивной.
\end{thm}
\begin{proof}

 Пусть $[a, b) = \bigsqcup_{k=1}^{\infty} \left[a_k, b_k\right)$. Нужно доказать \begin{align*}
  \mu_0(\left[a, b\right)) = \sum_{k=1}^{\infty} \mu_0(\left[a_k, b_k\right))
 .\end{align*} Докажем неравенства в обе стороны.

 Для любого $N \in \N$ по следствию \ref{corollary:downward_monotonicity_of_finite_additive_set_function} верно
 \begin{align*}
  \sum_{k=1}^{N} \mu_0(\left[a_k, b_k\right)) \leqslant \mu_0(\left[a, b\right))
 .\end{align*}  Переходя к пределу при $N \to \infty$  в левой части, получаем неравенство в одну сторону:
 \begin{align*}
  \sum_{k=1}^{\infty} \mu_0(\left[a_k, b_k\right)) \leqslant \mu_0(\left[a, b\right))
 .\end{align*} 

 Докажем в обратную сторону. Возьмём любой $\eps > 0$. Положим $\tilde a_k = a_k - \eps / 2^{k}$  и $\tilde b_k = b_k$ для любого  $k \in \N$ . Тогда для любого $t \in (a, b)$  верно
 \begin{align*}
  [a,t] \subset \bigcup_{k=1}^{\infty} (\tilde a_k, \tilde b_k)
 .\end{align*} Заметим, что в левой части стоит \textbf{компактное} множество. Тогда существует конечное $N \in \N$  такое, что
 \begin{align*}
  \left[a, t\right) \subset \bigcup_{k=1}^{N} [\tilde a_k, \tilde b_k)
 \end{align*} (здесь мы ещё поменяли скобки, всё в нужную сторону). По следствию \ref{corollary:upward_monotonicity_of_finite_additive_set_function} имеем
 \begin{align*}
  t - a &= \mu_0([a, t)) \leqslant \sum_{k=1}^{N} \mu_0([\tilde a_k, \tilde b_k)) \leqslant \\
  &\leqslant \sum_{k=1}^{\infty} (\tilde b_k - \tilde a_k) = \sum_{k=1}^{\infty} \mu_0([a_k, b_k)) + \sum_{k=1}^{\infty} \frac{\eps}{2^{k}} \leqslant \\
  &\leqslant \sum_{k=1}^{\infty} \mu_0([a_k, b_k)) + \eps
 .\end{align*} 

 Устремляя $t \to b$  и $\eps \to 0$  получаем неравенство в обратную сторону:
 \begin{align*}
  b - a \leqslant \sum_{k=1}^{\infty}  \mu_0([a_k, b_k))
 .\end{align*}
\end{proof}
