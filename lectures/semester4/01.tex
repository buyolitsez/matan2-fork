В этом семестре будет \textit{комплексный анализ}. Эта наука в первую очередь изучает \textit{аналитические функции} --- то есть те функции, которые совпадают со своим рядом Тейлора в окрестности любой точки (где определена функция).

Рассмотрим две вещественные бесконечно гладкие функции $\frac{1}{1+x^{2}}$ и $e^{-x^{2}}$

\begin{figure}[ht]
    \centering
    \incfig{similar-plots}
    \caption{}
    \label{fig:similar-plots}
\end{figure}

Запишем ряды Тейлора этих функций в точке $0$:
\begin{align}
 \label{equation:introduction:taylor_series_of_1_over_1_plus_x_squared}
 \frac{1}{1+x^{2}} &= \sum_{k=0}^{\infty} (-x^{2})^{k}, \\
 \label{equation:introduction:taylor_series_of_e_to_minus_x_squared}
 e^{-x^{2}} &= \sum_{k=0}^{\infty}  \frac{(-x^{2})^{k}}{k!}.
\end{align} Ряд \eqref{equation:introduction:taylor_series_of_1_over_1_plus_x_squared} сходится при $\left| x \right| < 1$, а ряд \eqref{equation:introduction:taylor_series_of_e_to_minus_x_squared} --- при любых $x \in \R$.

Продолжим первую функцию в область $\CC$ комплексных чисел: получим функцию $\frac{1}{1 + z^{2}}$. Однако, есть две особые точки: $i$ и $-i$: в этих двух точках функция не определена. Тем не менее, ряд \begin{align*}
 \frac{1}{1+z^{2}} = \sum_{k=0}^{\infty} (-z^{2})^{k}
\end{align*}  сходится на комплексном единичном круге $\left| z \right| < 1$. {\color{red} ещё две картинки}

{\color{red} Не понятно, что показывает иллюстрация --- выпилить её или довести до ума.}

\section{Точные и замкнутые дифференциальные формы.}

На удивление, не смотря на то, что предмет комплексного анализа --- аналитические функции, чтобы эффективно и быстро построить теорию, нужно прибегнуть к дифференциальным формам порядка $1$ --- это будет существенное теоретическое упрощение.

Вспомним некоторые понятия из предыдущего семестра, связанные с дифференциальными формами. Важное отличие: теперь в качестве поля скаляров мы будем брать $\CC$. Так, значения форм теперь могут быть комплексными.
\begin{df*}
 Линейные формы $dx, dy \colon\, \R^{2} \to \CC$ определены формулами:
 \begin{align*}
  dx \colon\, \begin{pmatrix}
   h_1 \\ h_2
  \end{pmatrix} \mapsto h_1, & &dy \colon\, \begin{pmatrix}
   h_1 \\ h_2
  \end{pmatrix} \mapsto h_2,
 \end{align*} где $ \begin{psmallmatrix}
  h_1 \\ h_2
 \end{psmallmatrix} \in \R^{2}$. Формы $dx$ и $dy$ задают базис в пространстве $\LL(\R^{2}, \CC)$ линейных форм на плоскости $\R^{2}$: любая линейная форма имеет вид $a \, dx + b \, dy$ для некоторых $a,b \in \CC$.

 Плоскость $\R^{2}$ мы часто будем отождествлять с комплексной плоскостью $\CC$ и будем писать $\R^{2} \cong \CC$ (так, вектор $ \begin{psmallmatrix}
  h_1 \\ h_2
 \end{psmallmatrix}$ отождествляется с комплексным числом $h_1 + i h_2 \in \CC$). Таким образом, можно считать, что $dx, dy \colon\, \CC \to \CC$ ($dx$ соответствует функции $\Real$ --- вещественной части, а $dy$ --- функции $\Imaginary$ --- мнимой части).
 \end{df*}

 \begin{df*}
 \textit{Дифференциальной формой} порядка $1$ в $\R^{2}$ (\textit{$1$-формой}) называется отображение $\omega$ из области $\Omega \subset \R^{2}$ в пространство линейных форм $\LL(\R^{2}, \CC)$. Каждая $1$-форма в $\R^{2}$ имеет вид
 \begin{align}
  \label{equation:1-form}
  \omega = P(x,y)\,dx + Q(x,y)\,dy
 ,\end{align} где $P, Q \colon\, \Omega \to \CC$ --- функции. Если отождествлять $\R^{2}$ с $\CC$, то $1$-форма будет иметь вид
 \begin{align*}
  \omega = P(z)\,dx + Q(z)\,dy, \quad z \in \Omega \subset \CC
 .\end{align*} 
\end{df*}

\begin{exmpl*}
 $\omega = x \, dy$ --- $1$-форма. В точке $2 \in \CC$ на векторе $ \begin{psmallmatrix}
  1 \\ 3
 \end{psmallmatrix} \in \R^{2}$ эта форма равна
 \begin{align*}
  \left(\omega (2)\right) \begin{psmallmatrix}
   1 \\ 3
  \end{psmallmatrix}= 2 \, dy \begin{psmallmatrix}
   1 \\ 3
  \end{psmallmatrix} = 6
 .\end{align*} 
\end{exmpl*}
\begin{exmpl*}
 Важная линейная форма: $dz = dx + i\,dy$. По сути --- это тождественная линейная форма $\CC \to \CC$.
\end{exmpl*}
\begin{exmpl*}
 \begin{align*}
  z\,dz &= (x + iy)(dx + i\,dy) = (x + iy)\,dx + (-y +ix)\,dy
 .\end{align*} Так, $P(x,y) = x + iy$ и $Q(x,y) = -y + ix$.
\end{exmpl*}
\begin{df*}
 $1$-форма $\omega$, представленная в виде \eqref{equation:1-form}, называется \textit{непрерывной}, если функции $P,Q \colon\, \Omega \to \CC$ непрерывны. $\omega$ называется $C^{k}$-гладкой, если $P$ и $Q$ --- $C^{k}$-гладкие ($k \in \N \cup \left\{ \infty \right\}$).
\end{df*}
\begin{df*}
 \textit{Путём} в $\CC$ называется непрерывная функция $\gamma \colon [a,b] \to \CC$. Путь $\gamma$ называется \textit{кусочно-гладким}, если $\gamma$ --- кусочно-гладкая функция, то есть $\gamma = \gamma_1 + \ldots \gamma_n$, где $\gamma_i$ --- гладкие пути.
\end{df*}

\begin{df*}
 Если $\gamma_1 \colon [a,b] \to \CC$ и $\gamma_2 \colon [b,c] \to \CC$ --- пути, то суммой путей $\gamma_1$ и $\gamma_2$ называется путь $(\gamma_1 + \gamma_2)\colon\, [a,c] \to \CC$, определённый по формуле
 \begin{align*}
  (\gamma_1 + \gamma_2)(t) = \begin{cases}
   \gamma_1(t), \text{ если } t \in [a,b],  \\
   \gamma_2(t), \text{ если } t \in [b,c].
  \end{cases} 
 \end{align*} 
\end{df*}

\begin{notatn*}
 Если $\gamma \colon\, [a,b] \to \CC$ --- путь, то путь $-\gamma \colon [a,b] \to \CC$ определяется по формуле:
 \begin{align*}
  -\gamma(t) = \gamma(b - t + a)
 \end{align*} --- это тот же путь, но проходимый в обратном направлении.
\end{notatn*}
\begin{notatn*}
 Если $\gamma \colon\, [c, d] \to \CC$ --- путь, то $b(\gamma) = \gamma(c) \in \CC$ --- \textit{начало пути} (begin), а $e(\gamma) = \gamma(d) \in \CC$ --- конец пути (end).
\end{notatn*}
\begin{df*}[интеграл $1$-формы по пути]
 Пусть $\Omega \subset \CC$ --- область, $\gamma \colon [a,b] \to \Omega$ --- кусочно-гладкий путь, и $\omega \colon\, \Omega \to \LL(\CC, \CC)$  --- непрерывная $1$-форма, заданная на пути. Тогда \textit{интегралом} от $1$-формы $\omega$ по пути  $\gamma$ называется число
  \begin{align*}
  \int\limits_{\gamma} \omega = \int\limits_{a}^{b} \left( P(\gamma(t)) \gamma_1'(t) + Q(\gamma(t)) \gamma_2'(t) \right) dt,
 \end{align*} где $\gamma(t) = (\gamma_1(t),\gamma_2(t))$ (или $\gamma(t) = \gamma_1(t) + i \gamma_2(t)$).
\end{df*}
\begin{lm}[основная оценка интеграла]
 \begin{align*}
  \left| \int\limits_{\gamma} \omega  \right| \leqslant \max_{t \in [a,b]} \sqrt{\left| P(\gamma(t)) \right|^{2} + \left| Q(\gamma(t)) \right|^{2}} \cdot l(\gamma),
 \end{align*} где $l(\gamma)$ --- длина пути $\gamma$.
\end{lm}
\begin{proof}
 По неравенству \eqref{equation:lemma:cauchy_bunyakovsky_schwarz_inequality} КБШ для $\CC^{2}$:
 \begin{align*}
  \left| \int\limits_{\gamma} \omega  \right| &\leqslant \int\limits_{a}^{b} \left| P(\gamma(t))\gamma_1'(t) + Q(\gamma(t))\gamma_2'(t) \right|dt \leqslant \\
  &\leqslant \int\limits_{a}^{b} \sqrt{\left| P(\gamma(t)) \right|^{2} + \left| Q(\gamma(t)) \right|^{2}} \cdot \sqrt{\left| \gamma_1'(t) \right|^{2} + \left| \gamma_2'(t) \right|^{2}} dt \leqslant \\
  &\leqslant \max_{t \in [a,b]} \sqrt{\left| P(\gamma(t)) \right|^{2} + \left| Q(\gamma(t)) \right|^{2}} \cdot \int\limits_{a}^{b} \left| \gamma(t) \right|  dt
 .\end{align*} Интеграл $\int_{a}^{b} \left| \gamma(t) \right|dt$ и есть в точности длина пути $l(\gamma)$.
\end{proof}

Дадим новое важное топологическое определение.
\begin{df}[гомотопия]
 \textit{Гомотопией} в области $\Omega \subset \CC$ называется непрерывное отображение $H \colon [0,1] \times [0,1] \to \Omega$, $H = H(s,t)$. 
\end{df}

Гомотопии часто применяются, когда работают с путями.

\begin{df}
 Пусть $\Omega \subset \CC$ --- область. Путь $\gamma_0 \colon [0,1] \to \Omega$ \textit{гомотопен} пути $\gamma_1 \colon [0,1] \to \Omega$ в области $\Omega$, если $b(\gamma_0) = b(\gamma_1)$, $e(\gamma_0) = e(\gamma_1)$, а также существует гомотопия $H \colon [0,1] \times [0,1] \to \Omega$  такая, что для любого $s \in [0,1]$:
 \begin{align*}
  H(s, 0) &= \gamma_0(s) \\
  H(s,1) &= \gamma_1(s)
 \end{align*} 
\end{df}
\begin{exmpl}
	Пути и гомотопия между ними на рисунке \eqref{fig:path-homptopy}	
\end{exmpl}

\begin{figure}[ht]
    \centering
		\incfig[0.5]{path-homptopy}
    \caption{Пути и гомотопия между ними}
    \label{fig:path-homptopy}
\end{figure}

Фактически, пути гомотопны, если один можно переделать в другой непрерывным образом.

Область в определениях $\Omega$ важна.

\begin{exmpl}
 $\gamma_0$ --- верхняя полуокружность, $\gamma_1$ ---  нижняя полуокружность. Пути $\gamma_0$ и $\gamma_1$ гомотопны в $\CC$, но не гомотопны $\CC \setminus \left\{ 0 \right\}$.

\begin{figure}[ht]
    \centering
		\incfig[0.7]{paths-without-homptopy}
		\caption{Слева пути гомотопны в $\CC$. Справа пути не гомотопны в $\CC \setminus \{0\}$}
    \label{fig:paths-without-homptopy}
\end{figure}

 Интуитивное объяснение: мы не можем непрерывно перевести один путь в другой, миновав выколотую точку $0$. Формальное доказательство давать не будем (см. фундаментальную группу единичной окружности).
\end{exmpl}

\begin{df}
 $1$-форма $\omega \colon\, \Omega \to \LL(\CC,\CC)$ называется \textit{точной}, если существует гладкая функция $F \colon \Omega \to \CC $ ($0$-форма) такая, что $\omega = dF$.

 $\omega$ называется \textit{замкнутой} в области $\Omega$, если $\omega$ локально точна: для любой точки $p \in \Omega$ существует окрестность $U_p \ni p$, $U_p \subset \Omega$ такая, что $\omega$ точна в $U_p$.
\end{df}

\begin{thm}
 Пусть $\omega$ --- непрерывная $1$-форма в области $\Omega \subset \CC$. Следующие условия равносильны:
 \begin{enumerate}
  \item $\omega$ точна в $\Omega$.
  \item Для любых путей $\gamma_0, \gamma_1 \colon [0,1] \to \Omega$ с совпадающими началом и концом ($b(\gamma_0) = b(\gamma_1)$ и $e(\gamma_0) = e(\gamma_1)$) верно
   \begin{align*}
    \int\limits_{\gamma_0} \omega = \int\limits_{\gamma_1} \omega  
   .\end{align*} То есть, интеграл формы по пути зависит лишь от начала и от конца пути.
 \end{enumerate}
\end{thm}
\begin{proof}\

	$1 \implies 2$. Пусть
 \begin{align*}
  \omega = d F = F'_x \, dx + F'_y \, dy
 .\end{align*} Возьмём любой путь $\gamma$ и проинтегрируем, используя формулу Ньютона-Лейбница:
 \begin{align*}
   \int\limits_{\gamma} \omega &= \int\limits_{\gamma} dF = \int\limits_{0}^{1} \left( F'_x(\gamma(t)) \gamma_1'(t)  + F'_y(\gamma(t)) \gamma_2'(t) \right) dt = \\
   &= \int\limits_{0}^{1} \left( F(\gamma(t)) \right)'_t dt = F(\gamma(1)) - F(\gamma(0)) = F(e(\gamma)) - F(b(\gamma))
 .\end{align*} Полученное зависит только от начала и конца пути.

 $2 \implies 1$. Пусть верен второй пункт. Выберем произвольную точку $p_0 \in \Omega$. Для любой точки $p \in \Omega$ зададим
  \begin{align*}
  F(p) = \int\limits_{\gamma(p_0,p)}  \omega
 ,\end{align*} где $\gamma(p_0,p)$ --- произвольный путь в  $\Omega$, соединяющий $p_0$ и $p$.   Определение корректно, так как интеграл формы не зависит от выбора пути. Кроме того, путь между $p_0$ и $p$ существует, так как $\Omega$ --- связное множество.

 Проверим, что $F$  подходит. Запишем $\omega = P(z)\,dx + Q(z)\,dy$. Возьмём малое приращение $h \in \CC$, $h \to 0$:
 \begin{align*}
  &F(p + h) - F(p) = \int\limits_{\gamma(p_0,p + h)}  \omega - \int\limits_{\gamma(p_0,p)}  \omega = \int\limits_{\gamma(p_0,p) + [p,p+h]} \omega - \int\limits_{\gamma(p_0,p)}   \omega = \\
  = &\int\limits_{[p,p+h]}  \omega = \begin{bmatrix}
   [p,p+h] \text{ параметризуется как } \gamma(t) = p + th
  \end{bmatrix} = \\
  = &\int\limits_{0}^{1} \left( P(p+th) h_1 + Q(p+th) h_2 \right)dt = \\
  = &\int\limits_{0}^{1} \left( \left( P(p+th) - P(p) \right)h_1 + \left( Q(p+th)-Q(p) \right)h_2 \right)dt + P(p)h_1 + Q(p)h_2 = \\
  = & \int\limits_{0}^{1} \left( o(1)h_1 + o(1)h_2 \right)dt + P(p)h_1 + Q(p)h_2 = \\
  = &P(p)h_1 + Q(p)h_2 + o(h)
 .\end{align*} Получается,
 \begin{align*}
  d_p F \begin{pmatrix}
   h_1 \\ h_2
  \end{pmatrix} = P(p) h_1 + Q(p) h_2 = \left( \omega(p) \right) \begin{pmatrix}
   h_1 \\ h_2
  \end{pmatrix}
 ,\end{align*} что и требовалось доказать.
\end{proof}

\begin{thm}
 Пусть $\omega$ --- непрерывная $1$-форма в области $\Omega \subset \CC$. Тогда следующие условия эквивалентны:
 \begin{itemize}
  \item $\omega$ замкнута в $\Omega$.
  \item Если пути $\gamma_0, \gamma_1 \colon [0,1] \to \Omega$ гомотопны в $\Omega$, то $\int_{\gamma_0} \omega = \int_{\gamma_1} \omega $.
  \item $\int_{\partial \Pi} \omega = 0$, если $\Pi$ --- замкнутый прямоугольник в $\Omega$. 
	\begin{figure}[ht]
    \centering
		\incfig[0.5]{closed-rectangle-and-its-border}
    \label{fig:closed-rectangle-and-its-border}
	\end{figure}
 \end{itemize}
\end{thm}
\begin{proof}\

 $1 \implies 2$. Пусть пути $\gamma_0$, $\gamma_1$ гомотопны в $\Omega$, $H \colon [0,1] \times [0,1] \to \Omega$ --- их гомотопия. Заметим, что
 \begin{align*}
  H([0,1] \times [0,1]) = K
 \end{align*}  --- компакт (непрерывный образ компакта). Существует число $\eps_0 > 0$ такое, что для любой точки $p \in K$ шар $B(p,\eps_0) \subset \Omega$, и $\omega$ точна в $B(p,\eps_0)$. Действительно, так как $\omega$ замкнута, то для любой точки $p \in K$ существует $\eps(p)$ такое, что $B(p,2\eps(p))\subset\Omega$ и $\omega$ точна на $B(p,2\eps(p))$. Выделим конечное подпокрытие $\left\{ B(p_k, \eps(p_k)) \right\}_{k=1}^{N}$ компакта $K$, возьмём $\eps_0 = \min \left\{ \eps(p_1), \ldots, \eps(p_N) \right\}$. Тогда $\omega$ точна в $B(p,\eps_0)$ для любой точки $p \in K$:
 \begin{align*}
  p \in B(p_k,\eps(p_k)) \implies B(p,\eps_0) \subset B(p_k,2\eps(p_k))
 .\end{align*} 
	
 {\color{red} рисунок.}

 Гомотопия $H$ равномерно непрерывна на компакте $[0,1]\times[0,1]$ (по теореме Кантора с первого курса): для любого $\eps > 0$ существует число $\delta(\eps)$, такое, что для любых двух точек $w_1,w_2 \in [0,1] \times [0,1]$ условие $\left\| w_1 - w_2 \right\| < \delta(\eps)$ влечёт $\left| H(w_1)-H(w_2) \right| < \eps$.

 Возьмём натуральное число $N \geqslant 1$ такое, что $1 / N < \delta(\eps_0 / 4)$.

 Для всякого $t \in [0,1]$ обозначим путь $\gamma_t \colon\, s \mapsto H(s,t)$ ($s \in [0,1]$). Покажем, что
 \begin{align*}
  \int\limits_{\gamma_{\frac{k}{N}}}  \omega = \int\limits_{\gamma_{\frac{k+1}{N}}}  \omega,
 \end{align*} где $k = 0 \ldots N - 1$. Ясно, что из этого всё последует (будет цепочка из $N$ равенств, в конце установим равенство интегралов по $\gamma_0$ и $\gamma_1$).

 Зафиксируем $k$ и обозначим $\eta = \gamma_{\frac{k}{N}}$  и $\rho = \gamma_{\frac{k+1}{N}}$. Разобьём оба пути на маленькие кусочки:
 \begin{align*}
  \eta &= \eta_0 + \eta_1 + \ldots + \eta_{N-1}, \\
  \rho &= \rho_0 + \rho_1 + \ldots + \rho_{N-1},
 \end{align*} где
 \begin{align*}
  \eta_j = \eta \rvert_{\left[\frac{j}{N},\frac{j+1}{N}\right]  }, \\
  \rho_j = \rho \rvert_{\left[\frac{j}{N},\frac{j+1}{N}\right]  },
 \end{align*} $j = 0 \ldots N - 1$.

 {\color{red} рисунок} Соединим концы $\eta_j$ и $\rho_j$ отрезками. Тогда
 \begin{align*}
  \int\limits_{\eta_j + I_{1,j} + I_{2,j} + \rho_j} \omega = 0
 ,\end{align*} так как  контур лежит в $B(p,\eps_0)$, а здесь форма $\omega$ точна. Поэтому, просуммировав по всем прямоугольникам, тоже получим $0$. А сумма равна
 \begin{align*}
  \int\limits_{\eta - \rho}  \omega = 0
 .\end{align*} 
\end{proof}

