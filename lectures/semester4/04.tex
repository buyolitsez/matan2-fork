% 2023.03.16 lecture 4

\begin{crly}
 Если $ B(z_0, r(z_0)) \subset \Omega $, то
 \begin{align*}
  f = \sum_{n=0}^{\infty}c_n(z-z_0)^{n},
 \end{align*} где
 \begin{align*}
  c_n = \frac{1}{2\pi i} \int\limits_{C_\rho} \frac{f(z)\,dz}{(z-z_0)^{n+1}},
 \end{align*} где $ C_\rho = \left\{ \left| z-z_0 \right| = \rho \right\} $, и $ 0 < \rho < r(z_0) $.
\end{crly}
\begin{crly}
 Если $ f $ аналитическая, то $ f' $ тоже аналитическая. Следовательно, для любого $ k \in \N $ существует производная $ f^{(k)} $ в $ \Omega $. В том числе $ f \in C^{\infty}(\Omega \to \CC) $.
\end{crly}
\begin{crly}
 Если $ B(z_0, r(z_0)) \subset \Omega $ и $ w \in B(z_0,r(z_0)) $, то
 \begin{align*}
  f(w) = \frac{1}{2\pi i} \int\limits_{C_\rho} \frac{f(z)\,dz}{z - w},
 \end{align*} где $ C_\rho = \left\{ \left| z-z_0 \right| = \rho \right\} $, и $ \left| w - z_0 \right| < \rho < r(z_0) $.
\end{crly}

\begin{thm}[Лиувилля]
 \label{theorem:liuvill}
 Пусть $ f \colon\, \CC \to \CC $ --- целая функция и существует $ M \geq 0 $ такое, что $ \left| f(z) \right| \leqslant M $ для любого $ z \in \CC $. Тогда $ f \equiv \mathrm{const} $.
\end{thm}
\begin{proof}[\normalfont\textsc{Доказательство}]
 Ряд
 \begin{align*}
  f(z) = \sum_{k=0}^{\infty} c_k z^{k}.
 \end{align*} сходится всюду в $ \CC $. С другой стороны,
 \begin{align*}
  c_k = \frac{1}{2\pi i} \int\limits_{C_\rho} \frac{f(z)}{z^{k+1}} dz.
 \end{align*} При этом $ \rho > 0 $ можно взять любое. Тогда
 \begin{align*}
  \left| c_k \right| &\leqslant \frac{1}{2\pi} \left| C_\rho \right| \cdot \max_{z \in C_\rho} \left| \frac{f(z)}{z^{k+1}} \right| = \frac{2\pi \rho}{2\pi}  \cdot \max_{\left| z \right|=\rho} \frac{\left| f(z) \right|}{\left| z \right|^{k+1}} \leqslant \rho \frac{M}{\rho^{k+1}} = \frac{M}{\rho^{k}}.
 \end{align*} Следовательно, $ c_k = 0 $ для любого $ k \geqslant 1 $, потому что $ \frac{M}{\rho^{k}} \to 0 $ при $ \rho \to \infty $ и $ k \geqslant 1 $. Значит, $ f(z) = c_0 $ для любого $ z $.
\end{proof}

\begin{thm}[основная теорема алгебры]
 Если $ p \in \CC[z] $ --- многочлен, $ \mathrm{deg}\;p \geqslant 1 $, то существует точка $ z_0 \in \CC $ такая, что $ p(z_0) = 0 $.
\end{thm}
\begin{proof}[\normalfont\textsc{Доказательство}]
 Пусть нет корней. Рассмотрим функцию $ f = \frac{1}{p} $ , она аналитична в $ \CC $  по условию ($ f'(z) = -\frac{p'(z)}{p^{2}(z)} $). Но при $ \left| z \right| \leqslant R $ $ \left| f(z) \right| \leqslant M_R < \infty $ по непрерывности. С другой стороны, если $ \mathrm{deg}\;p \geqslant 1 $, то $ p = c_n z^{n} + \ldots + c_0 $, $ c_n \neq 0 $, и 
 \begin{align*}
  \lim_{z \to \infty} \frac{1}{\left| p(z) \right|} = \lim_{z \to \infty} \frac{1}{\left| c_n \right| \cdot \left| z \right|^{n}} = 0.
 \end{align*} Поэтому существует $ R $ такое, что $ \left| f(z) \right| \leqslant 1 $ для всех $ \left| z \right| \geqslant R $. Тогда $ \left| f \right| \leqslant M_R + 1 $ всюду в $ \CC $. По теореме \ref{theorem:liuvill} $ f \equiv \mathrm{const} $, а это противоречие!
\end{proof}

\begin{lm}[лемма о кратности нуля]
 Пусть $ f\colon\,\Omega \to \CC $ --- аналитическая, $ f \not\equiv 0 $ в шаре $ B(z_0,r(z_0)) \subset \Omega $. Пусть $ z_0 \in \Omega $ такая, что $ f(z_0) = 0 $. Тогда существует число $ n_0 \geqslant 1 $ такое, что $ f = (z - z_0)^{n_0}g $, где $ g $ --- аналитическая на $ \Omega $ такая, что $ g(z_0) \neq 0 $.
\end{lm}
\begin{proof}[\normalfont\textsc{Доказательство}]
 \begin{align*}
  f = \sum_{n=0}^{\infty}c_n(z-z_0)^{n}
 \end{align*} в $ B(z_0, r(z_0)) \subset \Omega $. Положим $ n_0 = \min_{n \geqslant 1} : c_n \neq 0 $. Тогда
 \begin{align*}
  f = (z - z_0)^{n_0} \cdot \sum_{n=n_0 + 1}^{\infty} c_n(z-z_0)^{n - n_0 - 1}.
 \end{align*} Тогда функция
 \begin{align*}
  g = \frac{f}{(z - z_0)^{n_0}}
 \end{align*}  аналитична в $ \Omega \setminus \left\{ z_0 \right\} $, и аналитична в $ B(z_0, r(z_0)) $, значит она аналитична в $ \Omega $.
\end{proof}
\begin{thm}[теорема единственности]
Пусть $ \Omega $ --- область, $ f,g $ --- аналитические в $ \Omega $. Пусть $ f |_E = g|_E $, где $ E \subset \Omega $ такое, что $ E $ имеет предельную точку в $ \Omega $. Тогда $ f = g $ в $ \Omega $.
\end{thm}
\begin{proof}[\normalfont\textsc{Доказательство}]
 По условию существует точка $ z_\ast \in \Omega $ такая, что для любого $ \rho > 0 $ $ B(z_\ast,\rho) \setminus \left\{ z_\ast \right\} \cap E \neq \varnothing $. Рассмотрим функцию $ h = f - g $ и докажем, что существует $ r > 0 $ такое, что $ B(z_\ast,r) \subset \Omega $ и $ h|_{B(z_\ast,r)} \equiv 0 $. По лемме о кратности нуля
 \begin{align}
  \label{equation:uniqueness_thm}
  h(z) = (z - z_\ast)^{n_0} \cdot \zeta(z),
 \end{align} где $ \zeta $ --- аналитическая функция, такая, что $ \zeta(z_\ast) \neq 0 $, либо  $ h \equiv 0 $ в $ B(z_\ast,r) $ (а тут доказывать нечего). В первом случае существует такое число  $ \eps > 0 $, что $ \left|\zeta(z) \right| > \frac{\left| \zeta(z_\ast) \right|}{2} $ в $ B(z_\ast,\eps) $. Тогда $ B(z_\ast,\eps) \setminus \left\{ z_\ast \right\} \cap E = \varnothing $, а это противоречит \eqref{equation:uniqueness_thm}. Мы доказали $ f \equiv g $ в $ B(z_\ast,r) $.

 Рассмотрим $ F = \mathrm{int} \left\{ z \mid h(z) = 0 \right\} $. $ F \neq \varnothing $ ведь $ F \supset B(z_\ast,r) $. $ F $ открыто. Докажем, что $ F = \Omega $. Пусть $ w \in \Omega \setminus F$. Соединим $ z_\ast $ и $ w $ некоторым путём $ \gamma\colon[0,1] \to \Omega $ (так можно, потому что $ \Omega $ линейно связно). Рассмотрим
 \begin{align*}
  t_\ast = \sup \left\{ t \in [0,1] \mid \gamma(t) \in F \right\}.
 \end{align*}  Положим $ \xi = \gamma(t_\ast) $. $ \xi $ --- это предельная точка $ F $ ($ \xi = \lim_{n \to \infty} \gamma(t_n) $, где $ t_n \to t_\ast $)  и $ \xi \in \Omega $. Воспользуемся уже доказанной частью рассуждения: тогда существует шар, содержащий $ \xi $, и лежащий в $ \Omega $, а это противоречие!

 Можно было сказать, что $ F $ открытое и замкнутое множество в связном пространстве. Здесь скрыт некоторый топологический аргумент.
\end{proof}
\begin{remrk}
 В лемме о кратности нуля можно стереть условие $ f \not\equiv 0 $ на шаре.
\end{remrk}
\begin{df}
 \begin{align*}
  \sin z = \frac{e^{iz} - e^{-iz}}{2i}, & & \cos z = \frac{e^{iz} + e^{-iz}}{2}.
 \end{align*} 
\end{df}
\begin{exmpl}
 Пусть $ f\colon\,\CC\to\CC $ --- аналитическая. Пусть $ f(1 / n) = \sin \frac{1}{n} , \forall n \in \N$. Тогда $ f(z) = \sin z $.
\end{exmpl}
\begin{proof}[\normalfont\textsc{Доказательство}]
 Рассмотрим $ E = \left\{ \frac{1}{n} \mid n \in \N \right\} $. Тогда $ f|_E = \sin|_E $ по условию, а $ 0 $ --- предельная точка, лежащая в $ \CC $. Далее теорема о единственности.
\end{proof}

\begin{df}
 $ E \subset \Omega $ --- \textit{дискретное множество}, если любая предельная точка $ E $ не лежит в $ \Omega $.
\end{df}
\begin{crly}
 Если $ f \not\equiv 0 $ --- аналитическая в $ \Omega $, то множество $ \left\{ z \colon f(z) = 0 \right\} $ --- дискретное подмножество $ \Omega $.
\end{crly}

\begin{thm}[условие Коши-Римана]
\label{theorem:cauchy_riman}
 Пусть $ \Omega \subset \CC $ --- область, $ f $--- функция в $ \Omega $. Функции $ u, v \colon \Omega \to \R $ такие, что $ f(x + iy) = u(x, y) + iv(x,y) $. Следующие условия равносильны.
 \begin{enumerate}
  \item $ f $ аналитична на $ \Omega $.
  \item $ u,v \in C^{1}(\Omega, \R) $. Выполнены равенства
   \begin{align*}
    \begin{cases}
     u'_x = v'_y \\
     u'_y = -v'_x
 \end{cases} \text{ в } \Omega.
   \end{align*}
 \end{enumerate}
\end{thm}
\begin{proof}[\normalfont\textsc{Доказательство}]
 $ f $ аналитична тогда и только тогда, когда $ f \, dz $ замкнута. Тогда и только тогда, когда 
\begin{align*}
d(f\,dz) = 0 \iff d((u+iv)dz) = 0 &\iff (u'_x\,dx+u'_y\,dy+iv'_x\,dx+iv'_y\,dy) \land dz = 0 \iff \\
&\iff ((u'_x + iv'_x)\,dx + (u'_y + iv'_y)dy) \land (dx + i\,dy) = \\
& = i(u'_x + iv'_x)\,dx \land dy +  (u'_y + iv'_y)\,dy\land dx = \\
& = (i u'_x - v'_x) dx \land dy - (u'_y + iv'_y) dx \land dy = \\
& = (iu'_x - v'_x - u'_y - iv'_y)\,dx \land dy \equiv 0 \iff \\
& \iff u'_x - v'_y = 0 \text{ и } -v'_x - u'_y \equiv 0
\end{align*} 
И $u, v \in C^1(\Omega, \R)$ т.к. это вещественная/мнимая часть от гладкой функции. 
\end{proof}

\begin{thm}[неравенство Лагранжа]
 \label{theorem:Lagrange_inequality}
 Пусть $ \Omega $ --- область, $ f \colon\, \Omega \to \CC $ аналитична, $ z_1, z_2 \in \Omega $. Тогда
 \begin{align*}
  \left| f(z_1) - f(z_2) \right| \leqslant \max_{z \in \gamma} \left| f'(z) \right| \cdot l(\gamma),
 \end{align*} где $ \gamma $ --- любой кусочно-гладкий путь, соединяющий $ z_1 $ и $ z_2 $.
\end{thm}
\begin{proof}[\normalfont\textsc{Доказательство}]
 Проверим $ df = f'(z) \, dz$:
 \begin{align*}
  d(f(x+iy)) = f'(x+iu) \frac{\partial}{\partial x}(x+iy)\,dx + f'(x +iy) \frac{\partial}{\partial y}(x+iy)\,dy = \\
  =f'(x+iy)\,(dx + i\,dy) = f'(z)\,dz.
 \end{align*} Следовательно, $ f'(z)\,dz $ --- точная форма с первообразной $ f $. Таким образом,
 \begin{align*}
  \int\limits_{\gamma}  f'(z)\,dz = f(z_2) - f(z_1).
 \end{align*} Но по основной оценке интеграла
 \begin{align*}
  \left| f(z_1) - f(z_2) \right| = \left| \int_{\gamma} f'(z)\,dz   \right| \leqslant \max_{z \in \gamma} \left| f'(z) \right| \cdot l(\gamma).
 \end{align*} 
\end{proof}
\begin{remrk}
  Если $ [z_1, z_2] \subset \Omega $, то теорема \eqref{theorem:Lagrange_inequality} влечёт\
 \begin{align*}
  \left| f(z_1) - f(z_2) \right| \leqslant \max_{z \in [z_1, z_2]} \left| f'(z) \right| \cdot \left| z_1 - z_2 \right|.
 \end{align*} 
\end{remrk}

