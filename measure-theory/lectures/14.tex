% 2022.12.02 Lecture 14
\documentclass[../measure-theory.tex]{subfiles}
\begin{document}

\section{Дифференциальные формы}

В последнем параграфе мы перейдём к более изощренной теории интегрирования, являющейся далёким и нетривиальным обобщением формулы Ньютона-Лейбница. Мы будем интегрировать сложные объекты, называемые \textit{дифференциальными формами}, которые ещё нужно определить. Интегрировать мы их будем по поверхностям в $\R^{n}$ (их мы уже определили в предыдущем параграфе). Точнее, интегрировать формы мы будем по <<ориентированным многообразиям с краем>>, однако точного определения этих объектов мы не дадим. В качестве интересного результата мы приблизимся к пониманию, что же \textit{на самом деле} значит запись $\int_{a}^{b} f(x) \, dx $.

Мы не будем углубляться в эту теорию интегрирования слишком сильно: основная цель параграфа --- скорее познакомиться с ней. В конспекте материалы с лекций были довольно обширно дополнены другим источником: С. М. Львовский <<Лекции по математическому анализу>>, параграф 29 <<Дифференциальные формы>>.

\subsection{Полилинейные и кососимметрические формы.}

Для начала нам нужно погрузиться в линейную алгебру. Точнее, мы будем изучать основы, так называемой, \textit{внешней алгебры}. Для начала вспомним определение полилинейных отображений с первого курса --- они были как на анализе, так и на алгебре.

\begin{notatn*}
 Для натурального числа $m \in \N$ обозначим $[m] = \left\{ 1,2,\ldots,m \right\}$. Для чисел $m,k \in \N$ обозначим
 \begin{align*}
  [m]^{k} = \left\{ (a_1, \ldots, a_k) \Mid a_1, \ldots, a_m \in [m] \right\}
 .\end{align*} 
\end{notatn*}

\begin{df*}
 Функция $L \colon\, \underbrace{\R^{n} \times \ldots \times \R^{n}}_{q} \to \R  $ называется \textit{полилинейной формой порядка} $q$ в $\R^{n}$, если она линейна по каждому аргументу. То есть, для любого $i \in [q]$ и для любых векторов $v_1, \ldots, v_q \in \R^{n}$, $\tilde v_i \in \R^{n}$ верно
 \begin{align*}
  L(v_1, \ldots, v_i + \tilde v_i, \ldots, v_q) = L(v_1, \ldots, v_i, \ldots, v_q) + L(v_1, \ldots, \tilde v_i, \ldots, v_q)
 ,\end{align*} а также для любого $\alpha \in \R$ верно
 \begin{align*}
  L(v_1, \ldots, \alpha v_i, \ldots, v_q) = \alpha L(v_1, \ldots, v_i, \ldots, v_q)
 .\end{align*} 
\end{df*}
\begin{prop*}
 Множество полилинейных форм порядка $q$ в $\R^{n}$ образует векторное пространство, обозначаемое $\mathcal{L}(\underbrace{\R^{n}, \ldots, \R^{n}}_{q}, \R)$.
\end{prop*}
Более того, пространство полилинейных форм в $\R^{n}$ нормированное и полное (то есть банахово) --- это было доказано на первом курсе.

\begin{exmpl*}
 Линейная форма порядка $0$ --- это просто число из $\R$. Линейная форма порядка $1$ --- это произвольное линейное отображение $\R^{n} \to \R$, также называемое \textit{линейной формой}.
\end{exmpl*}

\begin{exmpl}\
 \label{example:dx_polylinear_form}
 \begin{itemize}
  \item $dx_i \colon\, \R^{n} \to \R$ --- полилинейная форма порядка $1$ в $\R^{n}$, определяемая формулой
   \begin{align*}
    dx_i(h) = h_i, \quad h = (h_1, \ldots, h_n)
   .\end{align*} 
  \item $dx_i \otimes dx_j \colon\, \R^{n} \times \R^{n} \to \R^{n}$ --- полилинейная форма порядка $2$ в $\R^{n}$, определяемая формулой
   \begin{align*}
    (dx_i \otimes dx_j)(h,g) = dx_i(h) \cdot dx_j(g) = h_i \cdot g_j
   ,\end{align*} где $h = (h_1,\ldots,h_n)$, $g = (g_1,\ldots,g_n)$.
  \item Обобщая, если $L_1$, $L_2$ --- полилинейные формы порядка $k$, $l$ в $\R^{n}$, то
   \begin{align*}
    (L_1 \otimes L_2)(u_1, \ldots, u_k, v_1, \ldots, v_l) = L_1(u_1, \ldots, u_k) \cdot L_2(v_1,\ldots,v_l)
   \end{align*} --- полилинейная форма порядка $k+l$ в $\R^{n}$.
 \end{itemize}
\end{exmpl}

\begin{claim}
 \label{claim:polylinear_form_basis}
 Полилинейная форма порядка $q$ в $\R^{n}$ полностью определяется своими значениями на всевозможных наборах из $q$ стандартных базисных векторов.
\end{claim}
\begin{proof}[\normalfont\textsc{Доказательство}]
 Действительно, пусть есть полилинейная форма  $L$ порядка  $q$  в $\R^{n}$. Пусть $e_1, \ldots, e_n$  --- стандартный базис в $\R^{n}$. Пусть есть произвольный набор векторов $v_1, \ldots, v_q \in \R^{n}$. Разложим каждый из них по базису:
 \begin{align*}
  v_i = \sum_{j=1}^{n} dx_j(v_i)\cdot e_j, \quad i \in [q]
 .\end{align*} Запишем значение формы на этом наборе векторов и воспользуемся полилинейностью:
 \begin{align}
  \label{equation:formula_polylinear_form}
  L(v_1, \ldots, v_q) = \sum_{j \in [n]^{q}} dx_{j_1}(v_1) \cdot \ldots \cdot dx_{j_q}(v_q) \cdot L(e_{j_1}, \ldots, e_{j_q})
 .\end{align} Таким образом, $L$ полностью определяется числами $L(e_{j_1}, \ldots, e_{j_q})$ по всем наборам $j \in [n]^{q}$.  Наоборот, легко проверить, что если по произвольным числам $L(e_{j_1}, \ldots, e_{j_q})$ задать функцию $L$ по формуле \eqref{equation:formula_polylinear_form}, то $L$ окажется полилинейной формой порядка $q$ в $\R^{n}$.
\end{proof}
\begin{crly*}
 Пространство полилинейных форм порядка $q$ в $\R^{n}$ имеет размерность $n^{q}$. Кроме того, набор форм вида
 \begin{align*}
  L = dx_{j_1} \otimes dx_{j_2} \otimes \ldots \otimes dx_{j_q}
 \end{align*} по всем $j \in [n]^{q}$ задаёт базис в этом пространстве. 
\end{crly*}

Теперь вспомним определение \textit{кососимметрических} полилинейных форм, которое было дано на первом курсе алгебры.

\begin{df*}
 Полилинейная форма $L$ порядка $q$ в $\R^{n}$ называется \textit{кососимметрической}, если
 \begin{align*}
  L(v_1, \ldots, v_i, \ldots, v_j, \ldots, v_q) = -L(v_1, \ldots, v_j, \ldots, v_i, \ldots, v_q)
 \end{align*} для любых векторов $v_1, \ldots, v_q \in \R^{n}$ и любых $1 \leqslant i < j \leqslant q$.
\end{df*}
\begin{prop*}
 Для полилинейных кососимметрических форм верно:
 \begin{itemize}
  \item $ L(v_1, \ldots, v_q) = \sgn \sigma \cdot L(v_{\sigma(1)}, \ldots, v_{\sigma(q)}) $ для любой перестановки $\sigma \in S_{q}$.
  \item $L(v_1, \ldots, v_i, \ldots, v_i, \ldots, v_q) = 0$.
 \end{itemize}
\end{prop*}
\begin{prop*}
 Множество полилинейных кососимметрических форм порядка $q$ в $\R^{n}$ образует векторное пространство, обозначаемое $\bigwedge^{q} (\R^{n})^{\ast}$.
\end{prop*}

\begin{exmpl*}\
 \begin{itemize}
  \item Любая полилинейная форма порядка $0$ или $1$ является кососимметрической. В частности, форма $dx_i$ из примера \ref{example:dx_polylinear_form} является кососимметрической.
  \item Форма $dx_i \otimes dx_j$ из того же примера не является кососимметрической.
 \end{itemize}
\end{exmpl*}
\begin{exmpl*}
 Важный пример кососимметрической полилинейной формы --- это \textit{определитель} --- форма $dx_{i_1} \land \ldots \land dx_{i_q} \in \bigwedge^{q} (\R^{n})^{\ast}$ для набора $i \in [n]^{q}$, определяемая по формуле:
 \begin{align}
  \label{equation:det_as_alternating_polylinear_form}
  \left( dx_{i_1} \land \ldots \land dx_{i_q} \right)(v_1, \ldots, v_q) =\det \begin{pmatrix}
   dx_{i_1}(v_1) & \ldots & dx_{i_q}(v_1) \\
   \vdots & \ddots & \vdots \\
   dx_{i_1}(v_q) & \ldots & dx_{i_q}(v_q) \\
  \end{pmatrix}
 .\end{align} 

 Полилинейность понятна. Кососиммтричность обосновывается тем, что при транспозиции двух соседних векторов знак определителя (и, следовательно, знак формы) меняется.

 Для кососимметрических полилинейных форм вида \eqref{equation:det_as_alternating_polylinear_form} верны следующие полезные свойства:
 \begin{itemize}
  \item $dx_{i_1} \land \ldots \land dx_{r} \land \ldots \land dx_r \land \ldots \land dx_{i_q} = 0$.
  \item $dx_{i_1} \land \ldots \land dx_r \land \ldots \land dx_s \land \ldots \land dx_{i_q} = - dx_{i_1} \land \ldots \land dx_s \land \ldots \land dx_r \land \ldots \land dx_{i_q}$.
 \end{itemize}
\end{exmpl*}

Оказывается, кососимметрические полилинейные формы вида \eqref{equation:det_as_alternating_polylinear_form}, взятые по возрастающим наборам $i_1 < \ldots < i_q$ образуют базис в пространстве $\bigwedge^{q}(\R^{n})^{\ast}$.

\begin{thm}[%
 из алгебры]
 Любая полилинейная кососимметрическая форма порядка $q$ в $\R^{n}$ имеет вид
 \begin{align}
  \label{equation:general_alternating_form_formula}
  L = \sum_{1 \leqslant i_1 < \ldots < i_q \leqslant n} a_{i_1,\ldots,i_q} \cdot dx_{i_1} \land \ldots \land dx_{i_q}
 ,\end{align} где $a_{i_1,\ldots,i_q} \in \R$ --- числа.
\end{thm}
\begin{proof}
 Пусть $L \in \bigwedge^{q}(\R^{n})^{\ast}$. Так как $L$ полилинейна, то по утверждению \ref{claim:polylinear_form_basis} её можно записать в виде
 \begin{align*}
  L(v_1, \ldots, v_q) = \sum_{j \in [n]^{q}} L(e_{j_1}, \ldots, e_{j_q}) \cdot dx_{j_1}(v_1) \cdot \ldots \cdot dx_{j_q}(v_q)
 .\end{align*} Теперь воспользуемся кососимметричностью: в каждом множителе $L(e_{j_1}, \ldots, e_{j_q})$ можно упорядочить аргументы по возрастанию индексов $j$, возможно поменяв при этом знак. При этом, если среди индексов $j_1, \ldots, j_q$ есть совпадающие, то $L(e_{j_1}, \ldots, e_{j_q}) = 0$. Поэтому,
 \begin{align*}
  L(v_1,\ldots,v_q) = \sum_{1 \leqslant i_1 < \ldots < i_q \leqslant n} L(e_{i_1}, \ldots, e_{i_q}) \cdot \left( \sum_{\sigma \in S_{q}} \sgn \sigma \cdot dx_{i_{\sigma(1)}}(v_1) \cdot \ldots \cdot dx_{i_{\sigma(q)}}(v_q) \right)
 .\end{align*} Осталось заметить, что внутренняя сумма --- это в точности определитель $(dx_{i_1} \land \ldots \land dx_{i_q})(v_1,\ldots,v_q)$. Поэтому,
 \begin{align*}
  L = \sum_{1 \leqslant i_1 < \ldots < i_q \leqslant n} a_{i_1,\ldots,i_q} \cdot dx_{i_1} \land \ldots \land dx_{i_q}
 ,\end{align*} где $a_{i_1,\ldots,i_q} = L(e_{i_1},\ldots,e_{i_q})$.
\end{proof}
\begin{crly*}
 Пространство $\bigwedge^{q}(\R^{n})^{\ast}$ имеет размерность $\binom n q$.

 В случае $q > n$ пространство имеет размерность $0$, и поэтому $0$ --- единственная кососимметрическая полилинейная форма порядка $q > n$ в $\R^{n}$.

 В случае $q = n$ пространство имеет размерность $1$, поэтому единственная кососимметрическая полилинейная форма порядка $n$ в $\R^{n}$ --- это $C \cdot \det$ --- определитель, умноженный на константу.
\end{crly*}

\begin{df}[%
 внешнее произведение кососимметрических форм]
 Пусть есть две кососимметрические полилинейные формы $dx_{i_1} \land \ldots \land dx_{i_q} \in \bigwedge^{q}(\R^{n})^{\ast}$  и $dx_{j_1} \land \ldots \land dx_{j_r} \in \bigwedge^{r}(\R^{n})^{\ast}$. Тогда их \textit{внешним произведением} называется форма
 \begin{align*}
  (dx_{i_1} \land \ldots \land dx_{i_q}) \land (dx_{j_1} \land \ldots \land dx_{j_r}) = dx_{i_1} \land \ldots \land dx_{i_q} \land dx_{j_1} \land \ldots \land dx_{j_r}
 \end{align*} порядка $q + r$ в $\R^{n}$. Для общих форм определение продолжается по линейности.

 Например,
 \begin{align*}
  &(dx \land dy + 2 \, dy \land dz) \land dx = \underbrace{dx \land dy \land dx}_{=0} + 2 \, dy \land dz \land dx = \\
  =\; &2 \, dy \land dz \land dx = -2 \, dy \land dx \land dz = 2 \, dx \land dy \land dz
 .\end{align*} 
\end{df}

Внешнее произведение можно определить более аккуратно, независимо от представления формы в виде \eqref{equation:general_alternating_form_formula}, однако мы этого делать не будем. Также не будем доказывать корректность определения.

\subsection{Дифференциальные формы.}

Теперь мы готовы определить дифференциальные формы.

\begin{df}[дифференциальная форма]
 Пусть $\Omega \subset \R^{n}$ --- область. \textit{Дифференциальной формой} порядка $q$ на $\Omega$ называется отображение $\omega \colon\, \Omega \to \bigwedge^{q}(\R^{n})^{\ast}$, которое каждой точке $x \in \Omega$ сопоставляет кососимметрическую полилинейную форму $\omega(x)$ порядка $q$ в $\R^{n}$.

 Дифференциальные формы порядка $q$ также называют \textit{$q$-формами}. 
\end{df}

\begin{exmpl*}
 $x \, dx \land dy + (x + z)\, dy \land dz$ ---  $2$-форма на $\Omega = \R^{3}$. В точке $(1,2,3)$ эта дифференциальная форма принимает значение $ dx \land dy + 4 \, dy \land dz $.
\end{exmpl*}
\begin{exmpl*}
 Дифференциальными формами порядка $0$ являются всевозможные функции $f \colon\, \Omega \to \R$.
\end{exmpl*}

Как и обычные кососимметрические формы, дифференциальные формы можно перемножать внешним образом.

\begin{df}[внешнее произведение дифференциальных форм]
 Пусть $\omega$, $\eta$ --- дифференциальные формы порядков $q$, $r$ на $\Omega$. Тогда \textit{внешним произведением} форм $\omega$ и $\eta$ называется дифференциальная форма $\omega \land \eta$ порядка $q + r$ на $\Omega$, определённая по правилу
 \begin{align*}
  (\omega \land \eta)(x) = \omega(x) \land \eta(x)
 .\end{align*} Умножение функции на форму можно также рассматривать как внешнее произведение ($0$-формы с $q$-формой).
\end{df}

Крайне важный пример $1$-формы --- это дифференциал функции.

\begin{exmpl}[дифференциал функции как $1$-форма]
 \label{example:differential_of_function_as_1_form}
 Пусть $f \colon\, \Omega \to \R  $  --- гладкая функция, где $\Omega \subset \R^{n}$ --- область. Тогда дифференциал функции $f$
 \begin{align*}
  df = \frac{\partial f}{\partial x_1} dx_1 + \ldots + \frac{\partial f}{\partial x_n} dx_n
 \end{align*} является дифференциальной формой порядка $1$ на $\Omega$. Действительно, в каждой точке $x \in \Omega$ дифференциал  $f$  в точке $x$  --- это линейная форма, равная
 \begin{align*}
  d_x f =  \frac{\partial f}{\partial x_1}(x) dx_1 + \ldots + \frac{\partial f}{\partial x_n}(x) dx_n
 .\end{align*} 
\end{exmpl}

Имея в виду формулу \eqref{equation:general_alternating_form_formula} --- общий вид кососимметрической формы, можно записать следующее.

\begin{crly*}[общий вид дифференциальной формы]
 Всякая $q$-форма $\omega$ на $\Omega \subset \R^{n}$ имеет вид
 \begin{align}
  \label{equation:general_differential_form_formula}
  \omega(x) = \sum_{1 \leqslant i_1 < \ldots < i_q \leqslant n} a_{i_1,\ldots,i_q}(x) \, dx_{i_1} \land \ldots \land dx_{i_q}
 ,\end{align} где $a_{i_1,\ldots,i_q} \colon\, \Omega \to \R$ --- функции.
\end{crly*}

\begin{df*}
 $q$-форма $\omega$, представленная в виде \eqref{equation:general_differential_form_formula}, называется \textit{$C^{m}(\Omega)$-гладкой}, если все $a_{i_1, \ldots, i_k} \in C^{m}(\Omega, \R)$. Неформально, это означает, что кососимметрическая форма $\omega(x)$ $C^{m}$-гладким образом зависит от $x$.
\end{df*}

Теперь мы должны связать дифференциальные формы (объекты, которые мы будем интегрировать) с поверхностями (объекты, по которым мы будем интегрировать). Для дифференциальных форм, заданных на поверхности, существует понятие \textit{переноса} дифференциальной формы в карту. 

\begin{df}[перенос формы]
 Пусть $(\Omega,\Phi,S)$ --- поверхность размерности $k$ в $\R^{n}$. Пусть $\omega \colon\, S \to \bigwedge^{q}(\R^{n})^{\ast}$ --- $q$-форма в $\R^{n}$, определённая на поверхности $S$ (или на области, содержащей поверхность $S$). Тогда \textit{переносом} формы $\omega$ называется $q$-форма $\omega^{\ast} \colon\, \Omega \to \bigwedge^{q}(\R^{k})^{\ast}$ в $\R^{k}$, заданная формулой
 \begin{align*}
  \omega^{\ast}(x)(v_1,\ldots,v_q) = \omega(\Phi(x))(d_x \Phi v_1, \ldots, d_x \Phi v_q)
 .\end{align*} 

 Легко видеть, что $\omega^{\ast}$ действительно является дифференциальной формой порядка $q$ на $\Omega$. Кроме того, форму $\omega^{\ast}$ можно выразить формулой в терминах координатных функций $\Phi_i \colon\, \Omega \to \R$, $i \in [n]$ отображения  $\Phi$. Если форма $\omega$ представлена в виде \eqref{equation:general_differential_form_formula}, то
 \begin{align*}
  \omega^{\ast}(x) = \sum_{1 \leqslant i_1 < \ldots < i_q \leqslant n}  a_{i_1,\ldots,i_q}(\Phi(x)) \cdot d_x\Phi_{i_1} \land \ldots \land d_x\Phi_{i_q}
 ,\end{align*} где $d_x\Phi_j$ --- дифференциал координатной функции $\Phi_j$ в точке $x$ как линейная форма.
\end{df}

\begin{exmpl*}
 Найдём перенос $2$-формы $ \omega = x \, dy \land dz $ со сферы $S$ радиуса $R > 0$ в $\R^{3}$ в прямоугольник $(0, 2\pi) \times (-\pi / 2, \pi / 2) = \Omega \subset \R^{2}$, где карта задана отображением $\Phi \colon\, \Omega \to S$:
 \begin{align*}
  \Phi(\varphi, \psi) = \begin{pmatrix}
   x(\varphi,\psi)\\
   y(\varphi,\psi)\\
   z(\varphi,\psi)\\
   \end{pmatrix} = \begin{pmatrix}
   R \cos \varphi \cos \psi \\
   R \sin \varphi \cos \psi \\
   R \sin \psi
  \end{pmatrix}
 .\end{align*} По правде, образ $\Phi$ --- это не вся сфера, а сфера минус множество меры нуль, но мы на это закроем глаза.

 Перенесём:
 \begin{align*}
  \omega^{\ast}(\varphi,\psi) &= x(\varphi,\psi) \cdot d(R\sin \varphi \cos\psi) \land d(R \sin \psi) = \\
  &= R \cos \varphi \cos \psi \cdot (R \cos \varphi \cos \psi \, d\varphi - R \sin \varphi \sin \psi \, d\psi ) \land (R \cos \psi \, d\psi) = \\
  &= R^{3} \cos\varphi \cos\psi \cdot(\cos\varphi \cos \psi \,d\varphi) \land (\cos\psi \, d \psi) = \\
  &= R^{3} \cos^{2}\varphi\cos^{3}\psi \, d\varphi \land d\psi
 .\end{align*} 
\end{exmpl*}

В примере \ref{example:differential_of_function_as_1_form} был рассмотрен дифференциал функции как отображение $d$, которое каждой (гладкой) $0$-форме ставит в соответствие $1$-форму. Следующее определение обобщает это отображение на $q$-формы для произвольного $q$.

\begin{df}[внешнее дифференцирование форм]
 Пусть $\omega \colon\, \Omega \to \bigwedge^{q}(\R^{n})^{\ast}$ --- гладкая дифференциальная форма порядка $q$ в $\R^{n}$, представленная в виде \eqref{equation:general_differential_form_formula}:
 \begin{align*}
  \omega(x) = \sum_{1 \leqslant i_1 < \ldots < i_q \leqslant n}  a_{i_1,\ldots,i_q}(x) \, dx_{i_1} \land \ldots \land dx_{i_q}.
 \end{align*} \textit{Внешним дифференциалом} (или просто \textit{дифференциалом}) формы $\omega$ называется дифференциальная форма $d \omega \colon\, \Omega \to \bigwedge^{q + 1}(\R^{n})^{\ast}$ порядка $q + 1$, заданная по формуле
 \begin{align}
  \label{equation:differential_of_form}
  d \omega = \sum_{1 \leqslant i_1 < \ldots < i_q \leqslant n} d a_{i_1,\ldots,i_k} \land dx_{i_1} \land \ldots \land dx_{i_q}
 ,\end{align} где $d a_{i_1,\ldots,i_q}$ --- дифференциал функции $a_{i_1,\ldots,i_q}$ как $1$-форма.

 Отображение $d$ называется \textit{оператором внешнего дифференцирования}.
\end{df}
\begin{exmpl*}
 \begin{align*}
  d((x^{2} + y^{2})\,dx + z^{2}\,dz) &= (2x \,dx + 2y \,dy) \land dx + 2z \,dz \land dz = \\
  &= 2y\,dy\land dx = -2y \, dx \land dy
 .\end{align*} 
\end{exmpl*}

\begin{remrk*}
 Существует более замысловатое, аксиоматическое определение внешнего дифференциала форм. Оператор $d$ внешнего дифференцирования $q$-форм --- это линейное отображение, сопоставляющее каждой гладкой $q$-форме $\omega \colon\, \Omega \to \bigwedge^{q}(\R^{n})^{\ast}$ $(q+1)$-форму $d\omega \colon\, \Omega \to \bigwedge^{q+1}(\R^{n})^{\ast}$, обладающее следующими свойствами.
 \begin{itemize}
  \item \textit{Естественность:} для всякого гладкого отображения $\Phi \colon\, X \to Y$ (при условии $\Omega \subset Y$) верно
   \begin{align*}
    (d \omega)^{\ast} = d(\omega^{\ast})
   ,\end{align*} то есть дифференциал коммутирует с переносом формы в карту.
  \item \textit{Нормировка:} дифференциал $0$-форм совпадает с дифференциалом функции в смысле примера \ref{example:differential_of_function_as_1_form}.
  \item \textit{Тождество Лейбница:}
   \begin{align*}
    d(\omega \land \eta) = d\omega \land \eta + \omega \land (-1)^{q}\,d\eta
   ,\end{align*} где $q$ --- порядок формы $\omega$.
  \item \textit{Нулевой квадрат:} $d(d\omega) = 0$.
 \end{itemize}
 Такое линейное отображение существует и единственно: оно совпадает с определением по формуле \eqref{equation:differential_of_form}.

 Мы не будем доказывать эквивалентность этих определений ни в одну из сторон и не будем пользоваться ею, но тем не менее её полезно иметь в виду.
\end{remrk*}

Наконец, мы готовы определить интеграл от дифференциальной формы. Интегрировать мы будем $k$-формы по поверхностям размерности $k$: порядок формы должен совпадать с размерностью поверхности.

\begin{df}[интеграл дифференциальной формы по поверхности]
 Пусть $(\Omega, \Phi, S)$ --- $k$-мерная поверхность в $\R^{n}$: $\Omega \subset \R^{k}$ --- область, $\Phi \colon\, \Omega \to S$ --- гладкая биекция с всюду невырожденным дифференциалом, $S \subset \R^{n}$.

 Пусть $\omega \colon\, S \to \bigwedge^{k}(\R^{n})^{\ast}$ --- дифференциальная форма порядка $k$ в $\R^{n}$, заданная на поверхности $S$. Пусть форма $\omega^{\ast} \colon\, \Omega \to \bigwedge^{k}(\R^{k})^{\ast}$ --- перенос формы $\omega$ в карту $(\Omega, \Phi)$ --- выражается формулой
 \begin{align*}
  \omega^{\ast} = f \, dx_1 \land \ldots \land dx_k
 ,\end{align*} где $f \colon\, \Omega \to \R$ --- непрерывная функция (так как $\omega^{\ast}$ имеет порядок $k$, то форма обязательно так выражается).

 Тогда \textit{интегралом} от дифференциальной формы $\omega$ по поверхности $S$ называется число
 \begin{align*}
  \int\limits_{S} \omega = \int\limits_{\Omega} f \, dx_1 \, \ldots \, dx_k
 ,\end{align*} где в правой части стоит обычный интеграл Лебега.
\end{df}
\begin{remrk*}
 Интегрировать форму $\omega$ можно не по всей поверхности $S$, а по её измеримому подмножеству $\Phi(E) \subset S$, где $E \subset \Omega$ --- измеримое по Лебегу множество. В таком случае интеграл равен
 \begin{align*}
  \int\limits_{\Phi(E)} \omega = \int\limits_{E} f \, dx_1 \, \ldots \, dx_k 
 .\end{align*} 
\end{remrk*}
\begin{remrk*}
 Рассмотрим крайне важный частный случай: рассматриваемая поверхность $S$ просто является областью $\Omega \subset \R^{n}$ (то есть $n = k$ и отображение $\Phi$ тождественно). В таком случае дифференциальная форма $\omega \colon\, \Omega \to \bigwedge^{n}(\R^{n})^{\ast}$ обязательно имеет вид
 \begin{align*}
  \omega = f \, dx_1 \land \ldots \land dx_n
 ,\end{align*} где $f \colon\, \Omega \to \R$ --- непрерывная функция. Тогда перенос формы просто совпадает с самой формой, и интеграл равен
 \begin{align*}
  \int\limits_{\Omega}   \omega = \int\limits_{\Omega} f \, dx_1 \, \ldots \, dx_n
 .\end{align*}

 Таким образом, определение интеграла дифференциальной формы порядка $n$ по области в $\R^{n}$ полностью эквивалентно определению обычного интеграла Лебега от функции по области в $\R^{n}$: каждая $n$-форма соответствует функции $\Omega \to \R$, домноженной на $dx_1 \land \ldots \land dx_n$.
\end{remrk*}
\begin{remrk}
 \label{remark:surface_orientation_along_form_integreation}
 Определённый таким образом интеграл почти не зависит от выбора параметризации $\Phi$ поверхности $S$. Формально, если $\Psi \colon\, \Omega' \to \Omega$ --- диффеоморфизм со всюду положительным якобианом, то интегралы всякой $k$-формы $\omega$ по поверхности $S$ с картами $(\Omega, \Phi)$ и $(\Omega', \Phi \circ \Psi)$ совпадают. Это непосредственно следует из теоремы о замене переменной в интеграле Лебега, но доказывать мы это не будем.

 Условие <<со всюду положительным якобианом>> существенно, и, более того, скрывает за собой понятие \textit{ориентации} поверхности $S$. Если параметризовать поверхность $S$ по-другому, проведя замену переменной диффеоморфизмом со всюду отрицательным якобианом, то интеграл сменит свой знак на противоположный. Таким образом, интеграл от дифференциальной формы по поверхности определён \textbf{с точностью до знака}, который зависит от \textbf{ориентации} поверхности. Здесь мы намеренно опускаем формальное определение ориентации.
\end{remrk}
\begin{remrk}
 \label{remark:no_definition_for_objects_in_stox_theorem}
 На самом деле, интеграл от дифференциальной формы определяется не по $k$-мерным поверхностям, а по некоторым более общим объектам. Таких обобщений есть по крайней мере два: <<гладкие сингулярные цепи>> и <<ориентированные многообразия с краем>>. Соответственно, получаются две теории интегрирования дифференциальных форм.

 Необходимость этого обобщения вызвана тем, что в основном результате теории интегрирования дифференциальных форм --- теореме Стокса --- необходимо понятие \textit{края}, или \textit{границы} поверхности. По всей видимости, общее определение границы $k$-мерной поверхности в смысле определения \ref{definition:surface} дать невозможно.

 Тем не менее, мы не будем сильно углубляться в теорию: эти уточнения рассмотрены не будут.
\end{remrk}

\subsection{Интегрирование \texorpdfstring{$1$}{1}-форм по гладким кривым.}

Сейчас мы рассмотрим частный случай этой теории: интегрирование $1$-форм по гладким кривым в $\R^{n}$. Этот случай имеет много приложений (в том числе в физике).

\begin{claim}[интеграл $1$-формы по кривой]
 Пусть $\gamma \colon (a,b) \to \R^{n}$, $\gamma(t) = (\gamma_1(t), \ldots, \gamma_n(t))$ --- гладкая кривая, $\left| \gamma'(t) \right| \neq 0$ для всех $t \in (a,b)$. Пусть
 \begin{align*}
  \omega(x) = f_1(x) \, dx_1 + \ldots + f_n(x) \, dx_n
 \end{align*} --- дифференциальная форма порядка $1$ в $\R^{n}$, определённая на кривой $\gamma$. Тогда
 \begin{align*}
  \int\limits_{\gamma} \omega = \int\limits_{a}^{b} \left( f_1(\gamma(t)) \gamma_1'(t) + \ldots + f_n(\gamma(t))\gamma_n'(t) \right)  dt
 .\end{align*} 
\end{claim}
\begin{proof}
 Найдём перенос формы $\omega$ в интервал $(a, b)$:
 \begin{align*}
  \omega^{\ast} = \sum_{i=1}^{n} f_i(\gamma(t)) d \gamma_i(t) = \sum_{i=1}^{n} f_i(\gamma(t)) \gamma_i'(t) \, dt
 .\end{align*} Тогда по определению интеграла
 \begin{align*}
  \int\limits_{\gamma} \omega = \int\limits_{(a,b)} \omega^{\ast} = \int\limits_{a}^{b} \sum_{i=1}^{n} f_i(\gamma(t))\gamma_i'(t)  \, dt
 .\end{align*} Проверено. По правде, это рассуждение верно только если $\gamma$ является биекцией (не имеет самопересечений). В противном случае можно разбить кривую на маленькие кусочки так, чтобы на каждом из кусочков кривая была биекцией.
\end{proof}

У интеграла $1$-формы по кривой есть следующий \textbf{физический смысл}.

Рассмотрим постоянное силовое поле в $\R^{n}$: к каждой точке $p \in \R^{n}$ применяется сила, равная вектору $F \in \R^{n}$ (одна и та же для всех точек). Тогда работа постоянного поля $F$ вдоль вектора $v \in \R^{n}$ --- это скалярное произведение $(F,v) \in \R$.

Рассмотрим теперь переменное силовое поле $F \colon\, \R^{n} \to \R^{n}$:
\begin{align}
 \label{equation:changing_work_field}
 F(x) = \begin{pmatrix}
  f_1(x) \\
  \vdots \\
  f_n(x)
 \end{pmatrix}
.\end{align} Пусть $\gamma = \gamma(t) \colon (a,b) \to \R^{n}$  --- кривая, $\left| \gamma'(t) \right| \neq 0$. Тогда работа переменного поля $F$  вдоль кривой $\gamma$ --- это интеграл
\begin{align*}
 A_{F,\gamma} = \int\limits_{\gamma} (F,v) \, dS
,\end{align*} где $v(t)$ --- единичный касательный вектор к кривой $\gamma$  в точке $\gamma(t)$. При этом интеграл берётся по поверхностной мере Лебега на $\gamma$ (формула \eqref{equation:formula_integral_on_surface}), что обозначается символом $dS$.

Как мы сейчас выясним, интеграл $A_{F,\gamma}$ можно представить как интеграл некоторой $1$-формы $\omega$ по кривой $\gamma$. Эту форму называют \textit{формой работы} поля $F$.

\begin{figure}[ht]
 \centering
 \incfig[0.7]{work_under_a_curve}
 \caption{Работа переменного поля $F$ вдоль кривой $\gamma$.}
 \label{fig:work_under_a_curve}
\end{figure}

\begin{claim}
 Пусть $F$ --- переменное силовое поле в $\R^{n}$, заданное координатными функциями как в \eqref{equation:changing_work_field}. Пусть $\gamma \colon (a,b) \to \R^{n}$ --- гладкая кривая, $\left| \gamma'(t) \right| \neq 0$. Тогда работа поля $F$ вдоль кривой $\gamma$ равна
 \begin{align*}
  A_{F,\gamma} = \int\limits_{\gamma} \omega
 ,\end{align*} где
 \begin{align*}
  \omega = f_1 \, dx_1 + \ldots + f_n \, dx_n
 \end{align*} --- \textit{форма работы} переменного силового поля $F$.
\end{claim}
\begin{proof}
 Вычислим единичный касательный вектор  к кривой $\gamma$ в точке $\gamma(t)$:
 \begin{align*}
  v(t) = \frac{\gamma'(t)}{\left| \gamma'(t) \right|}
 ,\end{align*} где $\gamma'(t) = (\gamma_1'(t), \ldots, \gamma_n'(t)) \in \R^{n}$. Учитывая равенство $G_{\gamma}(t) = (\left| \gamma'(t) \right|^{2})$ (пример \ref{example:length_of_smooth_curve}) вычислим работу:
 \begin{align*}
  A_{F, \gamma} &= \int\limits_{\gamma} (F,v) \, dS = \int\limits_{a}^{b} (F(\gamma(t)), v(t)) \sqrt{\det G_{\gamma}(t)} \, dt = \\ 
  &= \int\limits_{a}^{b} \sum_{i=1}^{n} f_i(\gamma(t))\frac{\gamma_i'(t)}{\left| \gamma'(t) \right|} \left| \gamma'(t) \right| \, dt = \\
  &= \int\limits_{a}^{b} \sum_{i=1}^{n} f_i(\gamma(t))\gamma'_i(t) \, dt 
 .\end{align*} Полученное есть в точности интеграл формы работы по кривой:
 \begin{align*}
  \int\limits_{\gamma} \omega = \int\limits_{(a,b)} \omega^{\ast} = \int\limits_{a}^{b} \sum_{i=1}^{n} f_i(\gamma(t))\gamma'_i(t) \, dt
 .\end{align*} 
\end{proof}

Рассмотрим аналог неопределённого интеграла для дифференциальных форм.

\begin{df}
 \label{definition:}
 Пусть $\omega \colon\, \Omega \to \bigwedge^{q}(\R^{n})^{\ast}$ --- дифференциальная форма порядка $q$ с непрерывными коэффициентами. Дифференциальная форма $\omega$ называется \textit{точной}, если существует гладкая дифференциальная форма $\eta \colon\, \Omega \to \bigwedge^{q-1}(\R^{n})^{\ast}$ порядка $q - 1$ такая, что $d \eta = \omega$.

 Для $1$-форм это означает, что существует функция $F \colon\, \Omega \to \R$ ($0$-форма) такая, что
 \begin{align*}
  dF = f_1 \, dx_1 + \ldots f_n \, dx_n = \omega
 .\end{align*} 
\end{df}
\begin{claim}
 \label{claim:integral_of_potential_curve_is_difference_on_ends}
 Пусть $\omega$  --- точная $1$-форма в $\Omega$. Пусть $\gamma$  --- гладкая кривая в $\Omega$. Тогда
 \begin{align*}
  \int\limits_{\gamma} \omega = F(\gamma(b)) - F(\gamma(a))
 ,\end{align*} где $F \colon\, \Omega \to \R$ --- функция, такая, что $dF = \omega$.

 Иными словами, интеграл $\int_{\gamma} \omega  $  зависит лишь от начала и конца кривой $\gamma$.
\end{claim}

\textbf{Физический смысл}: для потенциального силового поля (форма работы которого точная) работа вдоль кривой зависит только от начала и конца кривой. Таким образом, поднявшись с первого этажа на второй мы совершим точно такую же работу, если бы мы поднялись с первого этажа на шестой, а затем спустились бы на второй.

\begin{proof}[\normalfont\textsc{Доказательство утверждения \ref{claim:integral_of_potential_curve_is_difference_on_ends}}]
 \begin{align*}
  \int\limits_{\gamma} \omega &= \int\limits_{a}^{b} \left( f_1(\gamma(t)) \gamma'_1(t) + \ldots + f_n(\gamma(t))\gamma_n'(t) \right) \, dt = \\
  &= \int\limits_{a}^{b} \left( F(\gamma(t)) \right)'_t \, dt
 \end{align*}  в силу того, что
 \begin{align*}
  F'_1 = f_1, \; \ldots, \; F'_n = f_n
 .\end{align*} Тогда по формуле Ньютона-Лейбница
 \begin{align*}
  \int\limits_{\gamma} \omega = F(\gamma(b)) - F(\gamma(a))
 .\end{align*} 
\end{proof}

Теперь обсудим момент ориентации поверхности в рассматриваемом случае интегрирования $1$-форм по кривым. \textit{Ориентация} кривой --- это попросту направление, в котором мы идем по кривой: либо от $\gamma(a)$ до $\gamma(b)$, либо наоборот. Проверим для этого случая утверждения ориентации, которые мы голословно объявили верными в замечании \ref{remark:surface_orientation_along_form_integreation}.

\begin{claim*}
 Пусть $\gamma \colon (a,b) \to \R^{n}$ --- гладкая кривая, $\omega$ --- $1$-форма в $\R^{n}$, определённая на носителе кривой $\gamma$. Пусть $\tilde \gamma \colon (a,b) \to \R^{n}$ --- кривая $\gamma$, проходимая в обратном направлении. Тогда
 \begin{align*}
  \int\limits_{\gamma} \omega = - \int\limits_{\tilde \gamma} \omega
 .\end{align*} Таким образом, смена ориентации кривой влечёт смену знака интеграла.
\end{claim*}
\begin{proof}
 Кривая $\tilde \gamma$ задаётся формулой
 \begin{align*}
  \tilde \gamma (t) = \gamma(b - t + a)
 .\end{align*} Тогда
 \begin{align*}
  \int\limits_{\tilde \gamma} \omega &= \int\limits_{a}^{b} \sum_{i=1}^{n} f_i(\tilde \gamma (t)) \tilde \gamma'_i(t)\, dt = \int\limits_{a}^{b} \sum_{i=1}^{n} f_i(\gamma(b-t+a)) \cdot (-\gamma'_i(s) \rvert_{s=b-t+a}) \, dt = \\
  &= \begin{bmatrix}
   u = b - t + a & du = -dt
  \end{bmatrix} = \int\limits_{b}^{a} \sum_{i=1}^{n} f_i(\gamma(u))\gamma'_i(u) \, du = -\int\limits_{\gamma} \omega
 .\end{align*} 
\end{proof}

\begin{df*}
 Гладкие кривые $\gamma_1 \colon (a, b) \to \R^{n}$, $\gamma_2 \colon (c,d) \to \R^{n}$ \textit{эквивалентные}, если существует гладкая биекция $\varphi \colon (c,d) \to (a,b) $  такая, что $\gamma_2(t) = \gamma_1(\varphi(t))$.
\end{df*}

\begin{remrk*}
 Геометрически $\gamma_1$ и $\gamma_2$ --- это одна и та же кривая, но по-разному параметризованная. При этом, эти параметризации \textit{согласованы:} кривая проходится в одном и том же направлении.
\end{remrk*}
\begin{claim*}
 Для $1$-форм с непрерывными коэффициентами интеграл по гладкой кривой не зависит от параметризации. То есть
 \begin{align*}
  \int\limits_{\gamma_1} \omega = \int\limits_{\gamma_2} \omega
 ,\end{align*} если $\omega$ --- $1$-форма, и гладкая кривая $\gamma_1$ эквивалентна гладкой кривой $\gamma_2$.
\end{claim*}
\begin{proof}
 Пусть $\varphi \colon (c,d)\to(a,b)$ --- гладкая биекция такая, что $\gamma_2(t) = \gamma_1(\varphi(t))$. Тогда
 \begin{align*}
  \int\limits_{\gamma_2} \omega &= \int\limits_{c}^{d} \sum_{i=1}^{n} f(\gamma_2(t))\gamma_{2,i}'(t) \, dt = \int\limits_{c}^{d} \sum_{i=1}^{n} f(\gamma_1(\varphi(t))) \gamma_{1,i}'(y) \rvert_{y = \varphi(t)} \cdot \varphi'(t) \, dt = \\
  &= \begin{bmatrix}
   u = \varphi(t) & du = \varphi'(t)\, dt
  \end{bmatrix} = \int\limits_{a}^{b} \sum_{i=1}^{n} f(\gamma(u)) \gamma_{1,i}'(u) \, du = \int\limits_{\gamma_1} \omega
 .\end{align*}
\end{proof}

Замечание \ref{remark:surface_orientation_along_form_integreation} обосновано для $1$-форм, интегрируемых по гладким кривым.

\begin{remrk*}
 Запись $\int_{a}^{b} f(x) \, dx $ можно понимать как интеграл дифференциальной формы $f(x) \, dx$ порядка $1$ в $\R$ по кривой --- интервалу $(a,b)$, проходимому по направлению от $a$ к $b$.
\end{remrk*}

\subsection{Теорема Стокса (формулировка).}

Сформулируем теперь основной результат теории интегрирования дифференциальных форм --- теорему Стокса --- то самое далёкое и нетривиальное обобщение формулы Ньютона-Лейбница. На данный момент мы не сможем в абсолютной строгости понять её формулировку, так как у нас не хватает некоторых определений (см. замечание \ref{remark:no_definition_for_objects_in_stox_theorem}). Тем более, мы не сможем сейчас её доказать. Мы всё же сформулируем теорему и дадим несколько комментариев к ней, дабы установить интуитивное понимание этого красивого утверждения.

\begin{thm}[%
 Стокса]
 \label{theorem:stox}
 Пусть $M$  --- гладкое $n$-мерное ориентированное многообразие с краем $\partial M$ ($n \geqslant 1$). Пусть ориентации $M$ и $\partial M$ согласованы. Пусть $\omega$  --- $C^{1}$-гладкая дифференциальная форма порядка $n-1$ на $M$ с компактным носителем. Тогда верна \textbf{формула Стокса:}
 \begin{align}
  \label{equation:formula_stox}
  \int\limits_{\partial M} \omega = \int\limits_{M} d \omega
 .\end{align} 
\end{thm}

Дадим несколько комментариев к теореме \ref{theorem:stox}.

\begin{remrk*}
 \textit{Край} многообразия $M$ --- это многообразие $\partial M$. Оно имеет размерность $n-1$. Соответственно, форма $d\omega$ имеет порядок $n$, поэтому размерности многообразий и порядки форм в формуле \eqref{equation:formula_stox} согласованы. Понятие \textit{края} многообразия похоже на понятие \textit{границы} (замыкание минус внутренность), но это разные вещи. Пример многообразия с краем --- полусфера --- показан на рисунке \ref{fig:halfsphere}.
 \begin{figure}[ht]
  \centering
  \incfig{halfsphere}
  \caption{Полусфера. Её край --- окружность.}
  \label{fig:halfsphere}
 \end{figure}
\end{remrk*}
\begin{remrk*}
 \textit{Ориентированное} многообразие --- это многообразие на котором задана ориентация. \textit{Ориентируемое} многообразие --- это многообразие, на котором возможно задать ориентацию. Общее определение этих понятий мы не сможем дать, однако мы обсудим частные случаи.

 Как мы обсуждали ранее, все многообразия размерности $1$ (то есть кривые) ориентируемы. Ориентация кривой соответствует выбору начала и конца кривой.

 Многообразие размерности $2$ в $\R^{3}$ называется \textit{оринтеируемым}, если на нём есть непрерывное поле единичных нормалей. Ориентации соответствует выбор такого поля. Например, сфера $S^{2}$ ориентируема (можно выбрать поле единичных нормалей, которые <<торчат наружу>>), а лента Мёбиуса неориентируема (при попытке задать непрерывное поле единичных нормалей мы можем обойти ленту по кругу, и прийти в ту же точку с нормалью, торчащей в другую сторону, чем было изначально).
\end{remrk*}

\begin{exmpl*}
 Рассмотрим частный случай формулы Стокса: пусть $M = [a,b]$ --- отрезок с направлением от $a$ до $b$, $\partial M = \left\{ a,b \right\}$ --- край отрезка. Пусть $\omega = F \colon\, [a,b] \to \R$ --- гладкая $0$-форма (функция). Тогда
 \begin{align*}
  \int\limits_{\partial M} F = \int\limits_{M} dF
 .\end{align*} Правая часть равна
 \begin{align*}
  \int\limits_{a}^{b} \frac{\partial F}{\partial x} \, dx = \int\limits_{a}^{b} F' \, dx
 ,\end{align*} а левая часть равна
 \begin{align}
  \label{equation:example:zero_deg_form_integral}
  \int\limits_{\partial M} F = F(b) - F(a)
 .\end{align} Мы получили формулу Ньютона-Лейбница. Равенство \eqref{equation:example:zero_deg_form_integral} выглядит странно, но оно верно по той причине, что ориентации многообразий $[a,b]$  и $\left\{ a,b \right\}$ согласованы.
\end{exmpl*}

\end{document}
