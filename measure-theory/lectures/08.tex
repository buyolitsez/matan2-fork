\documentclass[../measure-theory.tex]{subfiles}
\begin{document}

\section{Точки Лебега множеств и функций}

В этом параграфе мы исследуем частую ситуацию, когда пространство с мерой $X$ также является метрическим пространством.

Рассмотрим мотивирующую задачу.

\begin{problem}
 \label{problem:motivatinal_problem_good_points_in_closed_subset}
 Пусть $E \subset \R$ --- замкнутое подмножество. Будем говорить, что $x \in E$ --- \textit{хорошая точка}, если \begin{align*}
  \lim_{\eps \to 0} \frac{\left| E \cap (x - \eps, x + \eps) \right| }{\left| (x - \eps, x + \eps) \right|} = 1
 ,\end{align*} где $\left| S \right| = \lambda_1(S)$. Например, если $E$ --- отрезок, то все его точки хорошие, за исключением двух концов.

 Как понять, есть ли в $E$ хорошие точки, и много ли их?

 Для этой задачи мы сможем дать относительно удовлетворительный \textit{ответ:} $\lao$-почти все точки  $E$ --- хорошие.
\end{problem}

\begin{df}
 \label{definition:regular_measure}
 Пусть $(X, \tau)$ --- топологическое пространство. $\B(X)$ --- борелевская  $\sigma$-алгебра (наименьшая $\sigma$-алгебра, содержащая $\tau$). Мера $\mu$ на $\B(X)$ называется \textit{регулярной}, если для любого $E \in \B(X)$ выполняется \begin{align*}
  \mu(E) &= \inf \left\{ \mu(U) \mid U \text{ открытое и } E \subset U \right\}, \\
  \mu(E) &= \sup \left\{ \mu(K) \mid K \text{ компактно и } K \subset E \right\}
 .\end{align*} 
\end{df}
\begin{lm}
 \label{lemma:finite_measure_on_compact_space_is_regular}
 Пусть $X$ --- метрический компакт и  $\mu$ --- конечная борелевская мера на $X$. Тогда $\mu$ регулярна.
\end{lm}
\begin{proof}
 Доказательство будет структурным, в стиле теории меры.

 Построим $\sigma$-алгебру $\A$ следующим образом: борелевское множество $A \in \B(x)$ принадлежит $\A$, если для любого $\eps > 0$ существуют замкнутое $C$ и открытое $U$ такие, что $C \subset A \subset U$, $\mu(U \setminus A) < \eps$ и $\mu(A \setminus C) < \eps$.

 Ясно, что $\A \subset \B(X)$, и если борелевское множество $A \in \B(X)$ принадлежит $\A$, то для него выполняются соотношения из определения \ref{definition:regular_measure}: его можно сколь угодно хорошо приблизить сверху открытым множеством, а снизу --- компактом (вспомним, что все замкнутые подмножества $X$ компактны, так как $X$ компактно). Поэтому, достаточно показать, что $\A = \B(X)$.

 Сначала поймём, что все замкнутые множества принадлежат $\A$. Действительно, пусть $F$ замкнутое. В качестве $C$ можно взять $F$. Научимся приближать $F$ сверху открытыми множествами. Так как $F$ замкнуто, то точка $x \in F$ тогда и только тогда, когда $\mathrm{dist}(x, F) = 0$, где  \begin{align*}
  \mathrm{dist}(x, F) = \inf_{y \in F} \mathrm{dist}(x,y)
 \end{align*} (к точке на расстоянии ноль от $F$ можно построить сходящуюся последовательность точек из $F$). Поэтому, верно
 \begin{align*}
  F = \bigcap_{n=1}^{\infty} V_n
 ,\end{align*}
 где $V_n = \left\{ x \mid \mathrm{dist}(x, E) < \frac{1}{n} \right\}$. Эти множества $V_n$ открытые как объединения открытых шаров: \begin{align*}
  V_n = \bigcup_{x \in F} B \left( x, 1 / n \right)
 .\end{align*}  В силу того, что множества $V_n$ вложены ($V_n \supset V_{n+1}$ при $n \geqslant 1$), по непрерывности конечной меры снизу (утверждение \ref{claim:downward_continuity_of_measure}) имеем
 \begin{align*}
  \mu(F) = \lim_{n \to \infty} \mu(V_n) 
 .\end{align*} Поэтому существует $n$ такое, что $\mu(V_n) \leqslant \mu(F) + \eps$. Так как $V_n \supset F$, то $\mu(V_n \setminus F) < \eps$. Поэтому, в качестве $U$ можно взять $V_n$ для достаточно большого $n$.


 Итак, $\A$ содержит все замкнутые подмножества. Осталось проверить, что $\A$ --- $\sigma$-алгебра.

 \begin{enumerate}
  \item $\varnothing \in \A$, так как $\varnothing$ и замкнутое, и открытое (можно взять $C=U=\varnothing$).
  \item Пусть $A \in \A$. Покажем $A^{c} \in \A$. Действительно, если $C \subset A \subset U$ и $\mu(U \setminus A) < \eps$, $\mu(A \setminus C) < \eps$, то $U^{c} \subset A^{c} \subset C^{c}$, при этом $U^{c}$ замкнутое, а $C^{c}$ открытое. При этом 
   \begin{align*}
    \mu(C^{c} \setminus A^{c}) &= \mu(C^{c} \cap A) = \mu((X \setminus C) \cap A) = \mu(A \setminus C) < \eps, \\
    \mu(A^{c} \setminus U^{c}) &= \mu(A^{c} \cap U) = \mu((X \setminus A) \cap U) = \mu(U \setminus A) < \eps
   .\end{align*}
  \item Пусть $A_1, A_2, \ldots \in \A$. Покажем, что счётное объединение $A = A_1 \cup A_2 \cup \ldots$ тоже принадлежит $\A$. Возьмём любое наперёд заданное $\eps > 0$ и найдём замкнутые $C_1,C_2,\ldots$ и открытые $U_1,U_2,\ldots$ такие, что
   \begin{align*}
    C_k \subset A_k \subset U_k, \quad \mu(U_k \setminus A_k) < \eps / 2^{k}, \quad \mu(A_k \setminus C_k) < \eps / 2^{k}
    .\end{align*} В качестве открытого приближения возьмём $U = U_1 \cup U_2 \cup \ldots$ Тогда $U \supset A$, и \begin{align*}
    \mu(U \setminus A) &= \mu \left( \bigcup_{k=1}^{\infty} U_k \setminus A \right) \leqslant \sum_{k=1}^{\infty} \mu(U_k \setminus A) \leqslant \sum_{k=1}^{\infty} \mu(U_k \setminus A_k) < \eps 
   .\end{align*}
   В качестве замкнутого приближения можно взять конечное объединение замкнутых множеств $C_1 \cup \ldots \cup C_N$ для достаточно большого $N$. В самом деле, по непрерывности конечной меры снизу можно написать:
   \begin{align*}
    \lim_{N \to \infty} \mu \left( A \setminus \bigcup_{k=1}^{N} C_k \right) &= \mu \left( A \setminus \bigcup_{k=1}^{\infty} C_k \right) = \mu \left( \bigcup_{n=1}^{\infty} A_n \setminus \left( \bigcup_{k=1}^{\infty} C_k \right) \right) \leqslant \\ &\leqslant \sum_{n=1}^{\infty} \mu \left( A_n \setminus \left( \bigcup_{k=1}^{\infty} C_k \right) \right) \leqslant \sum_{k=1}^{\infty} \mu(A_k \setminus C_k) < \eps
    .\end{align*} Значит, при достаточно большом $N$ выполнится \begin{align*}
    \mu \left( A \setminus \bigcup_{k=1}^{N} C_k \right) \leqslant 2\eps
   .\end{align*}
 \end{enumerate} Таким образом, мы показали, что $\A$ --- $\sigma$-алгебра, что завершает доказательство леммы.
\end{proof}
\begin{lm}[Урысона]
 \label{lemma:urison_metric_space}
 Пусть $(X, d)$ --- метрическое пространство и  $F_0, F_1$ --- два замкнутых не пересекающихся подмножества в $X$ $(F_0 \cap F_1 = \varnothing)$. Тогда существует непрерывная функция $f \colon\, X \to [0,1] $ такая, что \begin{align*}
  f \rvert_{F_0} = 0, \\
  f \rvert_{F_1} = 1
 .\end{align*}
\end{lm}

Функцию $f$ из леммы называют \textit{непрерывным спуском с единицы}. Эта лемма очень полезная: она позволяет <<локализовывать>> многие задачи.

\begin{proof}
 Мы просто предъявим такую функцию: \begin{align*}
  f(x) = \frac{d(x, F_0)}{d(x, F_0) + d(x, F_1)}
  ,\end{align*}  где $d(x, F) = \inf_{y \in F} d(x, y)$ для замкнутого $F \subset X$. Так как для замкнутого $F$ верно \begin{align*}
  x \in F \iff d(x, F) = 0
 ,\end{align*} то функция задана корректно на всём пространстве $X$ (если $d(x, F_0) = d(x, F_1) = 0$, то $x \in F_0 \cap F_1$, чего не может быть). Кроме того, видно, что всюду выполнено $0 \leqslant f(x) \leqslant 1$. Далее, если $x \in F_0$, то 
 \begin{align*}
  f(x) = \frac{0}{0 + d(x,F_1)} = 0
 .\end{align*} Если  $x \in F_1$, то 
 \begin{align*}
  f(x) = \frac{d(x, F_0)}{0 + d(x, F_0)} = 1
 .\end{align*}

 Осталось проверить непрерывность $f$. Она верна, потому что функция $x \mapsto d(x, F)$ непрерывна при замкнутом $F$. Проверим это: возьмём любую последовательность точек $\{x_{n}\}_{n=1}^{\infty} \subset X$, сходящуюся к точке $y \in X$. По неравенству треугольника можно получить оценки с обеих сторон: \begin{align*}
  d(y, F) - \underbrace{d(x_n, y)}_{\to 0} \leqslant d(x_n, F) \leqslant \underbrace{d(x_n, y)}_{\to 0} + d(y, F) 
 .\end{align*} По теореме о двух милиционерах $d(x_n, F) \to d(y, F)$. Значит, $f$ непрерывна.
\end{proof}
\begin{df*}
 Подмножество $S$ топологического пространства $X$ называется \textit{всюду плотным}, если выполнено одно из эквивалентных условий:
 \begin{itemize}
  \item Замыкание этого множества совпадает со всем пространством: $\overline S = X$.
  \item Любую точку $x \in X$ можно приблизить последовательностью точек из $S$.
  \item Любое открытое множество пересекается с $S$.
 \end{itemize}
\end{df*}
Например, $\Q$ всюду плотно в $\R$.

Прежде, чем перейти к следующей теореме, нам понадобится некоторое улучшение теоремы \ref{theorem:approximation} об аппроксимации (на лекции об этом забыли упомянуть).

\begin{thm}[%
 о равномерной аппроксимации]
 \label{theorem:uniform_approximation}

 Пусть измеримая функция $f \colon\, X \to [-C, +C] $ ограничена $(0 < C < \infty)$. Тогда существует последовательность $\{f_{n}\}_{n=1}^{\infty} $ простых функций $f_n \colon\, X \to \R$, равномерно сходящихся к $f$. То есть для любого $\eps > 0$ и любой точки $x \in X$ существует $N \in \N$ такое, что \begin{align*}
  \left| f_n(x) - f(x) \right| < \eps
 \end{align*} при всех $n > N$.
\end{thm}
\begin{proof}
 Точно так же разобьём $f = f_+ - f_-$, где $f_+ = \max(f,0)$, $f_- = \max(-f, 0)$ и сведём к случаю, когда $0 \leqslant f \leqslant C$.

 Внимательно посмотрим на конструкцию из доказательства теоремы \ref{theorem:approximation} об аппроксимации. Оказывается, что при условии ограниченности $f \leqslant C$ построенная последовательность функций будет сходится равномерно. Действительно, для любой точки $x \in X$ и для любого $n > C$ будет верно \begin{align*}
  f(x) - f_n(x) \leqslant \frac{1}{n}
 ,\end{align*} так как при $n > C$ точка $x$ попадает одно из множеств $A_{n,k}$, $0 \leqslant k < n^{2}$. Таким образом, для любого $\eps > 0$ можно подобрать $N = \max(C, \eps^{-1})$, работающее для всех $x \in X$ одновременно.

 Дополнительно отметим, что для неограниченных функций такое доказательство не будет работать, потому что для сколь угодно большого $n$ всё равно может оказаться точка, не принадлежащая $A_{n,k}$, $k < n^{2}$.
\end{proof}

Теперь мы готовы к следующей теореме. Важно: изначальная формулировка теоремы на лекциях отличается от этой (см. замечание \ref{remark:compact:continous_functions_dense_set_in_lebesgue_space}).

\begin{thm}[%
 ]
 \label{theorem:continous_functions_dense_set_in_lebesgue_space}
 Пусть $X$ --- метрическое пространство, $\mu$ --- конечная регулярная борелевская мера на $X$. Тогда множество $C(X)$ непрерывных функций на $X$ является всюду плотным в пространстве Лебега $L^{p}(X, \mu)$ для любого $1 \leqslant p < \infty$.
\end{thm}
Эта теорема говорит нам о том, что в метрическом пространстве при условии конечной регулярной меры, любую суммируемую функцию можно приблизить по норме непрерывными.
\begin{proof}
 Возьмём любую функцию $f \in L^{p}(X,\mu)$. Наша цель: найти последовательность функций $f_n \in C(X)$ такую, что \begin{align*}
  \left\| f - f_n \right\|_{L^{p}(X,\mu)} \to 0
 .\end{align*} Мы последовательно будем сводить задачу к более простому случаю относительно функции $f$.

 \begin{itemize}
  \item Покажем, что всё можно свести к случаю, когда $f$ ограничена. Рассмотрим множества
   \begin{align*}
    E_N &= \left\{ x \in X \mid \left| f(x) \right| \geqslant N \right\}, \quad N \in \N
   .\end{align*} Тогда функция $f^{(N)} = f \cdot \chi_{X \setminus E_N} \in L^{p}(X,\mu)$ ограничена числом $N$. Так как $E_1 \supset E_2 \supset \ldots $ , то по непрерывности (конечной!) меры $E \mapsto \int_{E} \left| f \right|^{p} \, d\mu$  снизу имеем
   \begin{align*}
    \lim_{N \to \infty} \int\limits_{E_N} \left| f \right|^{p} \, d\mu = \int\limits_{\bigcap_{N=1}^{\infty} E_N} \left| f \right|^{p} \, d\mu = \int\limits_{E_{\infty}} \left| f \right|^{p} \, d\mu   = 0
   ,\end{align*}  где $E_{\infty} = \left\{ x \in X \Mid \left| f(x) \right| = \infty \right\}$. Последний интеграл равен нулю, так как $f$ суммируема. Следовательно,
   \begin{align*}
    \lim_{N \to \infty} \left\| f - f^{(N)} \right\|_{L^{p}(X,\mu)} = \lim_{N \to \infty} \left( \int\limits_{E_N} \left| f \right|^{p} \, d\mu \right)^{\frac{1}{p}} = 0
   .\end{align*} Таким образом, если мы приблизим каждую функцию $f^{(N)}$ сколь угодно хорошо непрерывными функциями:
   \begin{align*}
    \lim_{n \to \infty} \left\| f^{(N)} - f_n^{(N)} \right\|_{L^{p}(X,\mu)} = 0, & & f_n^{(N)} \in C(X)
   ,\end{align*} то устремив $N \to \infty$ мы приблизим и функцию $f$, ведь
   \begin{align*}
    \left\| f - f_n^{(N)} \right\|_{L^{p}(X,\mu)} \leqslant \underbrace{\left\| f - f^{(N)} \right\|_{L^{p(X,\mu)}}}_{\to 0\text{ при }N\to \infty} + \underbrace{\left\| f^{(N)} - f_n^{(N)} \right\|_{L^{p}(X,\mu)}}_{\to 0\text{ при }n \to \infty}
   .\end{align*} 

  \item Итак, можно считать, что $f$ ограничена. По теореме \ref{theorem:uniform_approximation} о равномерной аппроксимации существуют последовательность $\{g_{n}\}_{n=1}^{\infty} $ простых функций $g_n \colon\, X \to \R$, равномерно сходящаяся к $f$. Тогда для любого $\eps > 0$ существует $N$ такое, что для всех $n > N$ верно
   \begin{align*}
    \forall x \in X \colon\; \left| g_n - f \right| < \eps &\implies \sup \left| g_n - f \right| \leqslant \eps \\ 
    &\implies \esssup \left| g_n - f \right| \leqslant \eps \\
    &\implies \left\| g_n - f \right\|_{L^{\infty}(X, \mu)} \leqslant \eps \\
    &\implies \left\| g_n - f \right\|_{L^{p}(X, \mu)} \leqslant C\eps, \quad C > 0
    .\end{align*}  Последний переход верен (при условии $\mu(X) < \infty$), поскольку \begin{align*}
    \left\| h \right\|_{L^{p}(X,\mu)} &= \left( \int\limits_{X} \left| h \right|^{p} \, d\mu   \right)^{\frac{1}{p}} \leqslant \left( \int\limits_{X} \esssup \left| h \right|^{p} \, d\mu   \right)^{\frac{1}{p}} = \left( \mu(X) \cdot \esssup \left| h \right|^{p} \right)^{\frac{1}{p}} = \\
    &=\mu(X)^{\frac{1}{p}} \cdot  \esssup \left| h \right| = \mu(X)^{\frac{1}{p}} \cdot \left\| h \right\|_{L^{\infty}(X,\mu)} < \mu(X)^{\frac{1}{p}} \cdot \eps
   .\end{align*} Тогда, если мы сможем всякую простую функцию приблизить непрерывными, то и всякую ограниченную тоже.

  \item Следовательно, можно считать, что $f$ простая:
   \begin{align*}
    f = \sum_{k=1}^{M}  c_k \chi_{A_k}, \quad A_k \in \B(X)
    .\end{align*} Покажем, что можно свести к случаю характеристической функции. Если для любой $\chi_{A_k}$ и для любого $\eps > 0$ существует непрерывная функция $h_{k,\eps} \in C(X)$ такая, что  \begin{align*}
    \left\| \chi_{A_k} - h_{k,\eps} \right\|_{L^{p}(X,\mu)} < \frac{\eps}{2^{k}\left| c_k \right|}
   ,\end{align*} то можно взять функцию
   \begin{align*}
    g_{\eps} = \sum_{k=1}^{M} c_k h_{k,\eps}
   \end{align*}
   и получить оценку
   \begin{align*}
    \left\| f - g_{\eps} \right\|_{L^{p}(X,\mu)} &= \left\| \sum_{k=1}^{M} c_k \chi_{A_k} - \sum_{k=1}^{M} c_k h_{k,\eps} \right\|_{L^{p}(X,\mu)} \leqslant \\
    &\leqslant \sum_{k=1}^{M} \left| c_k \right| \left\| \chi_{A_k} - h_{k,\eps} \right\|_{L^{p}(X,\mu)} \leqslant \\
    & \leqslant \sum_{k=1}^{M}  \frac{\eps \left| c_k \right|}{\left| c_k \right| 2^{k}} \\
    &< \eps
   .\end{align*}
 \end{itemize}

 Таким образом, можно считать, что $f$ --- характеристическая функция: $f = \chi_A$, где $A \in \B(X)$. Так как мера $\mu$ регулярна, то для любого $\eps > 0$ существует компактное множество $C_{\eps}$ и открытое множество $U_{\eps}$ такие, что $C_{\eps} \subset A \subset U_{\eps}$ и
 \begin{align*}
  \mu(U_{\eps} \setminus C_{\eps}) = \mu(U_{\eps} \setminus A) + \mu(A \setminus C_{\eps}) < 2\eps
 .\end{align*}

 Приблизим функцию $f$ непрерывным спуском с единицы, построенным в лемме \ref{lemma:urison_metric_space} Урысона: возьмём непрерывную функцию $h_{\eps} \in C(X)$ такую, что $0 \leqslant h_{\eps} \leqslant 1$ всюду на $X$, $h_{\eps} \rvert_{C_{\eps}} = 1$ и $h_{\eps} \rvert_{U^{c}_{\eps}} = 0$. См рисунок \ref{fig:continuous-approximation-of-characteristic-function}.

 \begin{figure}[ht]
  \centering
  \incfig{continuous-approximation-of-characteristic-function}
  \caption{Непрерывное приближение характеристической функции.}
  \label{fig:continuous-approximation-of-characteristic-function}
 \end{figure}

 Наконец, оценим расстояние между $f = \chi_A$ и $h_{\eps}$:
 \begin{align*}
  \left\| f - h_{\eps} \right\|_{L^{p}(X,\mu)}^{p} &= \int\limits_{X} \left| \chi_A - h_\eps \right|^{p} \, d\mu  = \\
  &= \int\limits_{U_{\eps}^{c}} \left| \chi_A - h_{\eps} \right|^{p} \, d\mu   + \int\limits_{U_{\eps} \setminus C_{\eps}} \left| \chi_A - h_{\eps} \right|^{p} \, d\mu  + \int\limits_{C_{\eps}} \left| \chi_A - h_{\eps} \right|^{p} \, d\mu  = \\
  &= \int\limits_{U_{\eps}^{c}} \left| 0 - 0 \right|^{p} \, d\mu + \int\limits_{U_{\eps} \setminus C_{\eps}} \left| \chi_A - h_{\eps} \right|^{p} \, d\mu + \int\limits_{C_{\eps}}  \left| 1-1 \right|^{p} \, d\mu = \\
  &= \int\limits_{U_{\eps} \setminus C_{\eps}} \left| \chi_A - h_{\eps} \right|^{p} \, d\mu  \leqslant \int\limits_{U_{\eps} \setminus C_{\eps}} 1\,d\mu = \mu(U_{\eps} \setminus C_{\eps}) < 2\eps \\
  \implies \left\| f - h_{\eps} \right\|_{L^{p}(X,\mu)} &< (2\eps)^{\frac{1}{p}}
 .\end{align*}
 Итак, мы смогли сколь угодно хорошо приблизить характеристическую функцию непрерывной. Значит, и любую функцию из $L^{p}(X,\mu)$ можно сколь угодно хорошо приблизить непрерывной.
\end{proof}
\begin{remrk}
 \label{remark:compact:continous_functions_dense_set_in_lebesgue_space}
 По лемме \ref{lemma:finite_measure_on_compact_space_is_regular} в теореме \ref{theorem:continous_functions_dense_set_in_lebesgue_space} можно опустить требование регулярности меры, если добавить условие компактности пространства $X$. Таким образом, если $X$ --- метрический компакт с конечной борелевской мерой $\mu$, то $C(X)$ всюду плотно в $L^{p}(X,\mu)$, $1 \leqslant p < \infty$.
\end{remrk}
\begin{remrk*}
 Теорема \ref{theorem:continous_functions_dense_set_in_lebesgue_space} не верна при $p = \infty$: $C(X)$ может оказаться не плотным в $L^{\infty}(X, \mu)$, даже при условии компактности $X$ и конечности меры $\mu$.
\end{remrk*}
\begin{proof}[\normalfont\textsc{Пример (не был на лекциях)}]\

 Рассмотрим следующий пример: пусть $X = [0,1]$ --- отрезок, $\mu = \lao$ --- мера Лебега, и $f = \chi_{\mathbf{C}}$, где $\mathbf{C} \subset [0,1]$ --- толстое канторово множество --- множество, имеющее положительную длину, но не содержащее интервалов.

 Докажем, что не существует непрерывной функции $h \colon\, [0,1] \to \R$ такой, что $\esssup \left| f - h \right| < 1 / 3$. Предположим такая функция $h$ есть. Так как $\lao(\mathbf{C}) > 0$, то существует точка $x \in \mathbf{C}$ такая, что $h(x) \geqslant 2 / 3$. Докажем, что $h$ не может быть непрерывной в точке $x$. Возьмём $\eps = 1 / 3$ и любое $\delta > 0$. Так как множество $\mathbf{C}$ не содержит интервалов, то в $\delta$-окрестности точки $x$ есть точки не из $\mathbf{C}$. Более того, эти точки образуют множество положительной длины (это видно из построения канторова множества). Раз так, то существует точка $y \in [0,1] \setminus \mathbf{C}$ такая, что $\left| x-y \right| < \delta$, но $h(y) \leqslant 1 / 3$. Это означает, что $h$ разрывна в точке $x$: существует сколь угодно близкая к $x$ точка, значение которой отличается хотя бы на $1 / 3$.
\end{proof}

\begin{df*}
 Метрическое пространство $X$ называется \textit{cепарабельным}, если в нём существует счётное всюду плотное подмножество.
\end{df*}

Например, пространство $\R^{n}$ сеперабельное, так как $\Q^{n}$ --- счётное всюду плотное подмножество $\R^{n}$.

\begin{lm}[%
 Витали]
 \label{lemma:vitali}
 Пусть $X$ --- сепарабельное метрическое пространство. Пусть $G$ --- семейство открытых шаров в $X$ с ограниченным радиусом: $0 < \rho(B) \leqslant R$ для любого $B \in G$, где $\rho(B)$ --- радиус шара $B$.

 Тогда существует не более, чем счётное подсемейство $\hat G \subset G$, в котором шары не пересекаются ($B_1 \cap B_2 = \varnothing$ для любых $B_1, B_2 \in \hat G$, $B_1 \neq B_2$), такое, что
 \begin{align*}
  \bigcup_{B \in \hat G}  5B \supset \bigcup_{B \in G} B
 ,\end{align*} где $5B$ --- это шар с тем же центром, что и у  $B$, но увеличенным в $5$ раз радиусом: $\rho(5B) = 5 \cdot \rho(B)$.
\end{lm}
\begin{remrk*}
 Сепарабельность в условии леммы несущественна: лемму можно доказать для любого метрического пространства, но с использованием леммы Цорна --- эквиваленту аксиомы выбора. Однако, для несепарабельных пространств подсемейство $\hat G$ может оказаться несчётным.
\end{remrk*}
\begin{remrk*}
 В $\R^{n}$ вообще можно можно не увеличивать шары (теорема Безиковича).
\end{remrk*}
\begin{proof}[\normalfont\textsc{Доказательство леммы \ref{lemma:vitali} Витали}]
 Разобьём шары на счётное число семей-ств по радиусу:
 \begin{align*}
  G_j = \left\{ B \in G \Mid \frac{R}{2^{j+1}} < \rho(B) \leqslant \frac{R}{2^{j}} \right\}, \quad j \geqslant 0
 .\end{align*}

 Возьмём $G_0'$ --- любое максимальное по включению подмножество непересекающихся шаров из $G_0$. Максимальность по включению означает, что если $B \in G_0$, но $B \notin G_0'$, то существует шар $\tilde B \in G_0'$ такой, что $B \cap \tilde B \neq \varnothing$. В произвольном пространстве такое семейство можно построить с помощью леммы Цорна. В сепарабельном пространстве можно без неё: возьмём точки $\{x_{k}\}_{k=1}^{\infty} $, образующие счётное всюду плотное подмножество $\bigcup_{B \in G_0} B$. Сначала возьмём любой шар $B_1 \in G_0$, содержащий точку $x_1$. Далее, если рассмотрены точки $x_1, \ldots, x_n$ и построены шары $B_1, \ldots, B_{m_n}$, то при рассмотрении точки $x_{n+1}$ поступим следующим образом:
 \begin{itemize}
  \item Если любой шар $B \in G_0$, содержащий точку $x_{n+1}$, пересекает $B_1 \cup \ldots \cup B_{m_n}$, то переходим к $x_{n+2}$, и полагаем $m_{n+1} = m_n$.
  \item Если существует шар $B \in G_0$, содержащий точку $x_{n+1}$ и не пересекающийся с $B_1 \cup \ldots \cup B_{m_n}$, то возьмём его в семейство: положим $m_{n+1} = m_n + 1$ и  $B_{m_{n+1}} = B$.
 \end{itemize} Легко видеть, что эта конструкция подходит. По построению это семейство является дизъюнктным подсемейством $G_0$. Кроме того, в него нельзя добавить никакой шар $B \in G_0$ (не задевая остальные шары): иначе при рассмотрении некоторой точки $x_N \in B$  (которая есть в $B$, так как множество $\{x_{k}\}_{k=1}^{\infty} $ всюду плотно), мы бы его добавили тогда.

 Теперь пусть $G_j'$ --- максимальное по включению дизъюнктное семейство шаров из $G_j$, не пересекающих $G_0' \cup \ldots \cup G_{j-1}'$ ($j \geqslant 1$). Такое семейство строится совершенно аналогично $G_0'$ --- нужно только исключить шары, пересекающие $G_0' \cup \ldots \cup G_{j-1}'$. Наконец, возьмём
 \begin{align*}
  \hat G = \bigcup_{j=0}^{\infty} G_j'
 .\end{align*}

 Проверим, что $\hat G$ подходит. По построению  $\hat G \subset G$ и $\hat G$ дизъюнктное и, следовательно, не более, чем счётное (по сепарабельности пространства $X$ каждый шар содержит хотя бы одну точку из счётного всюду плотного множества). Рассмотрим любой шар $B \in G$  и покажем, что $B \subset 5\tilde B$ для некоторого $\tilde B \in \hat G$. Если $B \in \hat G$, то всё очевидно (можно взять $\tilde B = B$). Если $B \in G \setminus \hat G$, то существует $j \geqslant 0$ такое, что $B \in G_j \setminus G_j'$:
 \begin{align*}
  \frac{R}{2^{j+1}} < \rho(B) \leqslant \frac{R}{2^{j}}
 .\end{align*} Раз мы не взяли шар $B$, то существует $\tilde B \in G_k'$, где $0 \leqslant k \leqslant j$, такой, что $\tilde B \cap B \neq \varnothing$.

 Покажем, что $B \subset 5\tilde B$ (рисунок \ref{fig:vitali-lemma-intersecting-balls}).
 \begin{figure}[ht]
  \centering
  \incfig{vitali-lemma-intersecting-balls}
  \caption{Лемма Витали: шар $5 \tilde B$ содержит в себе шар $B$.}
  \label{fig:vitali-lemma-intersecting-balls}
 \end{figure}

 Так как $k \leqslant j$, то $\rho(\tilde B) > \rho(B)/2$. Действительно, худший случай --- это $k = j$, в котором радиусы отличаются менее, чем в два раза. В остальных случаях радиус $\rho(\tilde B)$ ещё больше. Возьмём любую точку $x \in B$. Пусть $y$ --- центр $B$, и $\tilde y$ --- центр $\tilde B$. Тогда \begin{align*}
  \rho(x, \tilde y) &\leqslant \rho(x, y) + \rho(y, \tilde y) \leqslant \\ & \leqslant \rho(B) + (\rho(B) + \rho(\tilde B)) = \\ &= 2\rho(B) + \rho(\tilde B) < \\ &< 5 \rho(\tilde B)
 ,\end{align*} где последнее неравенство эквивалентно $\rho(\tilde B) > \rho(B) / 2$. Таким образом, $x \in 5 \tilde B$.

\end{proof}

\begin{df}
 \label{definition:hardy_littlewood_maximal_function}
 Пусть $X$ --- метрическое пространство, $\mu$ --- борелевская мера на $X$, конечная на любом шаре $B \subset X$ ($\mu(B) < \infty$).

 Для функции $f \in L^{1}(X, \mu)$ \textit{максимальной функцией Харди-Литтлвуда} называется функция
 \begin{align*}
  (M^{\ast}f)(x) = \sup_{B_x \ni x} \frac{1}{\mu(B_x)} \int\limits_{B_x} \left| f \right| \, d\mu  
 ,\end{align*} где $B_x$ --- некоторый открытый шар, содержащий точку $x$.
\end{df}
\begin{remrk}
 Функция $M^{\ast}f$ измерима, так как множество
 \begin{align*}
  E_{\lambda} = \left\{ x \in X \Mid (M^{\ast} f)(x) > \lambda \right\} 
  \end{align*} открыто для любого $\lambda \in \R$. Действительно, если $(M^{\ast}f)(x) > \lambda$, то существует открытый шар $B$ такой, что \begin{align*}
  \frac{1}{\mu(B)} \int\limits_{B} \left| f \right| \, d\mu > \lambda
 .\end{align*} Значит, на всех точках $y \in B$ из шара будет выполнено $(M^{\ast}f)(x) > \lambda$.
\end{remrk}

\begin{df}
 Говорят, что борелевская мера $\mu$ на метрическом пространстве $X$ обладает \textit{свойством удвоения}, если для любого открытого шара $B(x,r)$, $r > 0$ выполняется
 \begin{align*}
  0 < \mu(B(x, 2r)) \leqslant c \mu(B(x,r)) < \infty
 \end{align*} для некоторого $c \geqslant 1$. В частности, мера любого шара конечна.
\end{df}

\begin{thm}[%
 Харди-Литтлвуда]
 \label{theorem:hardy_littlewood}
 Пусть $X$ --- сепарабельное ограниченное метрическое пространство (ограниченность означает $\rho(x, y) \leqslant R$ для любых $x,y \in X$ и некоторого $R > 0$). Пусть $\mu$ --- борелевская мера на $X$, обладающая свойством удвоения.

 Тогда существует константа $C > 0$, зависящая только от пространства с мерой, такая, что для любой функции $f \in L^{1}(X, \mu)$ и любого $\lambda > 0$ имеет место оценка \begin{align*}
  \mu \left\{ x \in X \mid (M^{\ast}f)(x) > \lambda \right\} \leqslant \frac{C}{\lambda} \left\| f \right\|_{L^{1}(X,\mu)}
 .\end{align*}
\end{thm}
\begin{proof}
 Обозначим $E_{\lambda} = \left\{ x \in X \mid (M^{\ast}f)(x) > \lambda \right\}$. По определению \ref{definition:hardy_littlewood_maximal_function}
 \begin{align*}
  E_{\lambda} \subset \bigcup_{B \in G}  B
  ,\end{align*} где $G$ --- семейство открытых шаров $B \subset X$ таких, что \begin{align}
  \label{equation:hardy_little_wood_theorem_1}
  \frac{1}{\mu(B)} \int\limits_{B} \left| f \right| \, d\mu   > \lambda
  .\end{align} Радиусы шаров из $G$ ограничены в связи с ограниченностью пространства $X$. Поэтому, воспользуемся леммой \ref{lemma:vitali} Витали и найдём не более, чем счётное подсемейство $\hat G \subset G$ такое, что \begin{align*}
  \bigcup_{B \in \hat G} 5B \supset \bigcup_{B \in G}  B \supset E_{\lambda}
  .\end{align*} Тогда меру $E_{\lambda}$ можно оценить сверху этими шарами: \begin{align*}
  \mu(E_{\lambda}) &\leqslant  \mu \left( \bigcup_{B \in \hat G} 5B  \right) \leqslant \sum_{B \in \hat G} \mu(5B)
 .\end{align*} Так как мера $\mu$ обладает свойством удвоения, то для некоторой константы $c > 0$ верно $\mu(5B) \leqslant c^{3} \mu(\frac{5}{8} B) \leqslant c^{3} \mu(B)$. Поэтому,
 \begin{align*}
  \mu(E_{\lambda}) \leqslant c^{3} \sum_{B \in \hat G}  \mu(B)
  .\end{align*} Для всякого $B \in \hat G$ по неравенству \eqref{equation:hardy_little_wood_theorem_1} имеем: \begin{align*}
  \mu(B) \leqslant \frac{1}{\lambda} \int\limits_{B} \left| f \right| \, d\mu
  .\end{align*} Поэтому, \begin{align*}
  \mu(E_{\lambda}) \leqslant c^{3} \sum_{B \in \hat G}  \mu(B) \leqslant \frac{c^{3}}{\lambda} \sum_{B \in \hat G}  \int\limits_{B} \left| f \right| \, d\mu   \leqslant \frac{c^{3}}{\lambda} \int\limits_{X} \left| f \right| \, d\mu   = \frac{c^{3}}{\lambda} \left\| f \right\|_{L^{1}(X,\mu)}
 ,\end{align*} так как шары из $\hat G$ дизъюнктны. Возьмём $C = c^{3}$. Теорема доказана.
\end{proof}
\begin{df}
 Пусть $X$ --- метрическое пространство с борелевской мерой $\mu$. Точка $x$ называется \textit{точкой Лебега} функции  $f \in L^{1}(X,\mu)$, если выполнено
 \begin{align}
  \label{equation:definition:lebesgue_point}
  \lim_{r \to 0} \frac{1}{\mu(B_{r})} \int\limits_{B_{r}} \left| f(y) - f(x) \right| \, d\mu (y) = 0
 ,\end{align} где $B_r = B_r(x)$ --- шар с центром в $x$ и радиусом $r > 0$.
\end{df}
\begin{df}
 Точка $x$ называется \textit{точкой Лебега} борелевского множества $E$, если $x \in E$ и $x$ --- точка Лебега функции $\chi_E$, то есть \begin{align*}
  \lim\limits_{r \to 0} \frac{\mu(E \cap B_{r}(x))}{\mu(B_{r}(x))} \to 1
 .\end{align*} 
\end{df}

\end{document}
