% 2022.09.30 lecture 4
\documentclass[../measure-theory.tex]{subfiles}
\begin{document}

\begin{thm}[об аппроксимации]
 \label{theorem:approximation}
 Пусть $(X, \A)$ --- измеримое пространство, неотрицательная функция $f \colon\, X \to [0, \infty]  $ измерима относительно $\A$.

 Тогда существует последовательность $\{f_{n}\}_{n=1}^{\infty} $ простых неотрицательных функций $f_n \colon\, X \to [0, +\infty)$, возрастающая к $f$ на $X$: для любого $n \in \N$ выполнено $0 \leqslant f_n \leqslant f_{n+1} \leqslant f$ на $X$ и \begin{align*}
  f(x) = \lim\limits_{n \to \infty} f_n(x)
 \end{align*} для любого $x \in X$.
\end{thm}
\begin{remrk*}
 В обратную сторону это тоже верно по лемме \ref{lemma:limit_of_measurable_functions_is_measurable}: если есть последовательность $\{f_{n}\}_{n=1}^{\infty} $ возрастающих простых неотрицательных функций, то функция \begin{align*}
  f(x) = \lim\limits_{n \to \infty} f_n(x)
 \end{align*} измерима.

 Так, теорема \ref{theorem:approximation} об аппроксимации даёт описание (критерий) всех неотрицательных измеримых функций.
\end{remrk*}
\begin{proof}[\normalfont\textsc{Доказательство теоремы \ref{theorem:approximation}}]
 Возьмём сколь угодно большое число $n \in \N$. Заведём ячейки \begin{align*}
  I_{n,k} = \left[\frac{k}{n},\; \frac{k+1}{n}\right)
  ,\end{align*} где $k = 0 \ldots n^{2} - 1$. Также, положим \begin{align*}
  I_{n,n^{2}} = [n, +\infty]
  .\end{align*} Так, для любого $n \in \N$ есть разбиение \begin{align*}
  [0, \infty] = \bigsqcup_{k=0}^{n^{2}} I_{n,k}
 .\end{align*}
 Рассмотрим прообразы этих ячеек:
 \begin{align*}
  A_{n,k} = f^{-1} \left( I_{n,k} \right), \quad k = 0, \ldots, n^{2}
 .\end{align*}
 Рассмотрим функцию \begin{align*}
  g_n = \sum_{k=0}^{n^{2}} \frac{k}{n} \chi_{A_{n,k}}
 \end{align*}
 Заметим, что $g_n$ --- простая функция, так как множества $A_{n,k}$ измеримы как прообразы ячеек. Для любого $x \in X$
 \begin{align*}
  0 \leqslant g_n(x) \leqslant f(x),
 \end{align*} так как $g_n(x) \rvert_{A_{n,k}} = \frac{k}{n}$, а $f(x) \rvert_{A_{n,k}} \geqslant \frac{k}{n}$, так как $f(A_{n,k}) \subset I_{n,k}$.

 Оценим разность $f$ и $g_n$. Если $x \in \bigsqcup_{k=0}^{n^{2} - 1} A_{n,k} \iff f(x) < n $, то
 \begin{align}
  \label{equation:theorem:approximation:g_n_and_f_difference}
  0 \leqslant f(x) - g_n(x) \leqslant \frac{1}{n}
 ,\end{align} так как на $A_{n,k}$ верно $\frac{k}{n} \leqslant f < \frac{k+1}{n}$ и  $g_n  = \frac{k}{n}$.

 Докажем теперь сходимость
 \begin{align*}
  \lim_{n\to \infty} g_n(x) = f(x)
 \end{align*} во всех точках $x \in X$.
 \begin{itemize}
  \item Если $f(x) < \infty$, то при больших $n$ верно $f(x) < n$, и, следовательно, выполняется \eqref{equation:theorem:approximation:g_n_and_f_difference}.
  \item Если $f(x) = \infty$, то $g_n(x) = n$ для любого $n$.
 \end{itemize}

 В любом случае поточечная сходимость имеет место. Осталось обеспечить возрастание: определим \begin{align*}
  f_n(x) = \max \{g_1(x), \ldots, g_n(x)\}
  .\end{align*} По следствию \ref{corollary:simple_operations_with_simple_functions_result_in_simple_function} функция $f_n$ простая. Монотонность выполнена: $0 \leqslant f_n \leqslant f_{n+1} \leqslant f$ для любого $n$. Осталось показать $f_n \to f$: \begin{align*}
  \underbrace{g_n(x)}_{\to f(x)} \leqslant f_n(x) \leqslant f(x) \implies \lim\limits_{n \to \infty} f_n(x) = f(x)
 .\end{align*} Теорема доказана.
\end{proof}
Есть вариант теоремы \ref{theorem:approximation} для функций с отрицательными значениями.
\begin{crly}
 \label{corollary:simple_function_approximation_maybe_negative}
 Любая измеримая функция $f \colon\, X \to [-\infty, +\infty] $ есть предел последовательности $\{f_{n}\}_{n=1}^{\infty} $ простых функций $f_n \colon\, X \to \R $:
 \begin{align*}
  f(x) = \lim_{n \to \infty} f_n(x)
 .\end{align*} 
\end{crly}
\begin{proof}
 Распилим $f$ на две части: $f_+ = \max(f, 0)$, $f_- = \max(-f, 0)$. Тогда $f_+ - f_- = f$ и  $f_+, f_- \geqslant 0$. Функции $f_{\pm}$ измеримы как максимумы измеримых. По теореме \ref{theorem:approximation} существуют простые функции $\{f_{\pm,n}\}_{n=1}^{\infty} $ такие, что $f_{\pm} = \lim f_{\pm,n}$. Тогда \begin{align*}
  f = \lim_{n \to \infty} \underbrace{(f_{+,n} - f_{-,n})}_{\text{ простые по следствию \ref{corollary:simple_operations_with_simple_functions_result_in_simple_function} }}
 .\end{align*}
\end{proof}
\begin{crly}
 \label{corollary:linear_combination_of_measurable_functions_is_measurable}
 Если функции $f_1,f_2 \colon\, X \to [-\infty, +\infty] $ измеримы, $\lambda_1, \lambda_2 \in \R$, то и функции $\lambda_1 f_1 + \lambda_2 f_2$ и $f_1 \cdot f_2$ измеримы, если определены.
\end{crly}
<<Если определены>> означает, что мы не складываем бесконечности разных знаков (см. определение \ref{definition:extended_reals}).
\begin{proof}
 По следствию \ref{corollary:simple_function_approximation_maybe_negative} $ f_{1,2} = \lim f_{1,2,n} $, где $f_{1,2,n}$ простые. Тогда \begin{align*}
  \lambda_1 f_1 + \lambda_2 f_2 = \lim_{n \to \infty} \left( \lambda_1 f_{1,n} + \lambda_2 f_{2,n} \right), \\
  f_1 \cdot f_2 = \lim_{n \to \infty} (f_{1,n} \cdot f_{2,n})
 .\end{align*} Под пределом стоят простые функции по следствию \ref{corollary:simple_operations_with_simple_functions_result_in_simple_function}.
\end{proof}
\begin{crly}
 Пусть функция $f \colon\, X \to \R  $ измерима, а функция $\varphi \colon\, \R \to \R  $ измерима по Борелю (то есть измерима в пространстве $(\R, \B_1)$). Тогда композиция $\varphi \circ f$ измерима.
\end{crly}
\begin{proof}
 \begin{align*}
  (\varphi \circ f)^{-1} ((a, +\infty]) = f^{-1} ( \underbrace{\varphi^{-1}((a, +\infty])}_{\in \B_1} ) \in \A
 ,\end{align*} так как $f^{-1}(E) \in \A$ при $E \in \B_1$.
\end{proof}
\begin{exercise}
 Если $f$ измерима, то $\left| f \right|^{p}$ измерима, $p \in \R$.
\end{exercise}

\section{Интеграл Лебега}

Интеграл Лебега придумал Лебег и Гротендик (см. <<Урожаи и посевы>>). Смысл такой: мы не хотим ограничиваться хорошими классами функций. Хочется интегрировать всё, что попало! Идея близка к тому, что мы сделали в параграфе \ref{section:summation_of_nonnegative_families} с рядами над произвольными множествами индексов.

\begin{df}
 $(X, \A, \mu)$ --- \textit{пространство с мерой}, если $X$ --- множество, $\A \subset 2^{X}$ --- $\sigma$-алгебра, а $\mu$ --- мера на $\A$.
\end{df}

Сначала мы определим интеграл Лебега от простой неотрицательной функции, затем продолжим определение на неотрицательные измеримые функции, и, наконец, на измеримые функции любого знака.

\begin{df}[интеграл Лебега простой неотрицательной функции]
 \label{definition:integral_simple_positive}
 Пусть $f \geqslant 0$ --- простая неотрицательная функция: \begin{align*}
  f = \sum_{k=1}^{N} c_k \chi_{A_k}
  .\end{align*} Пусть $E \in \A$ --- измеримое подмножество. Тогда \begin{align*}
  \int\limits_{E} f \, d\mu = \sum_{k=1}^{N} c_k \mu (E \cap A_k)
 .\end{align*} 
\end{df}
\begin{remrk}
 Определение \ref{definition:integral_simple_positive} корректно, то есть не зависит от выбора разбиения $X = \bigsqcup_{k=1}^{N} A_k$.
\end{remrk}
\begin{proof}
 Если есть другое разбиение $f = \sum_{k=1}^{N'} c_k' \chi_{A_k'}$, то рассмотрим общее мелкое разбиение (лемма \ref{lemma:common_small_partition_of_simple_functions}). Сведём вопрос к совпадению 
 \begin{align*}
  \sum_{k=1}^{N} c_k \mu (A_k \cap E) = \sum_{k=1}^{N'} c_k' \mu (A_k' \cap E)
 \end{align*} при условии, что $\bigsqcup_{k=1}^{N'} A_k' $ --- подразбиение $\bigsqcup_{k=1}^{N} A_k$.
 А тут по конечной аддитивности меры все понятно.
\end{proof}
\begin{remrk}
 Если $f,g$ --- простые неотрицательные и $f \leqslant g$, то \begin{align*}
  \int\limits_E f \, d\mu \leqslant \int\limits_E g\, d\mu
 \end{align*} 
\end{remrk}
\begin{proof}
 Рассмотреть общее мелкое разбиение для функций $f$ и $g$ (лемма \ref{lemma:common_small_partition_of_simple_functions}).
\end{proof}
\begin{df}[интеграл Лебега измеримой неотрицательной функции]
 \label{definition:integral_supremum_simple}
 Пусть $f \geqslant 0$ --- измерима, $E \in \A$. Тогда
 \begin{align}
  \label{equation:definition:integral_supremum_simple}
  \int\limits_E f \, d\mu = \sup \left\{ \int\limits_E g\, d\mu \Mid g \geqslant 0 \text{ --- простая, и } g \leqslant f \text{ на } E \right\}
 .\end{align} 
\end{df}
\begin{proof}[\normalfont\textsc{Корректность}]
 Нужно обосновать корректность: интегралы простой функции $f \geqslant 0$ в новом $\int_N$ и старом $\int_O$ смысле совпадают.

 Так как можно подставить $f = g$ в формулу \eqref{equation:definition:integral_supremum_simple}, то
 \begin{align*}
  \int\limits_{E,O} f \, d\mu \leqslant \int\limits_{E,N} f \, d\mu
  .\end{align*} С другой стороны, \begin{align*}
  \int\limits_{E,O} f \, d\mu \geqslant \int\limits_{E,N} f \, d\mu
 ,\end{align*} так как по монотонности интеграла в старом смысле число $\int_{E,O} f d\mu $ является верхней гранью множества из формулы \eqref{equation:definition:integral_supremum_simple}.
\end{proof}
\begin{remrk}
 Монотонность: если $0 \leqslant f \leqslant g$ измеримы, то $\int_E f \, d\mu \leqslant 
 \int_E g\,d\mu$.
\end{remrk}
\begin{proof}
 В правой части супремум берётся по большему множеству.
\end{proof}
\begin{df}[интеграл Лебега интегрируемой функции]
 Пусть функция $f \colon\, X \to [-\infty,+\infty] $ измерима. Обозначим $f_+ = \max(f, 0)$, $f_- = \max(-f, 0)$.

 Функция $f$ называется \textit{интегрируемой по Лебегу} на множестве $E \in \A$, если хотя бы один из интегралов $\int_E f_+ \, d\mu$  или $\int_E f_- \, d\mu$ конечен. Функция $f$ называется \textit{суммируемой}, если оба интеграла конечны.

 Для интегрируемой функции $f$ интеграл Лебега по множеству $E \in \A$ равен \begin{align*}
  \int\limits_E f \, d\mu = \int\limits_E f_+ \, d\mu - \int\limits_E f_- \, d\mu \in [-\infty,+\infty]
 .\end{align*} 
\end{df}
\begin{remrk}
 Корректность: пусть $f \geqslant 0$ --- измеримая. Тогда $f_+= f,\; f_- = 0$, и поэтому интегралы в новом и старом смысле совпадают.
\end{remrk}
\begin{remrk}
 Монотонность: $f \leqslant g \implies \int_E f \, d\mu \leqslant \int_E g \, d\mu$.
\end{remrk}
\begin{proof}
 Неравенство равносильно \begin{align*}
  \int\limits_E f_+ \, d\mu + \int\limits_E g_- \, d\mu \leqslant \int\limits_E f_- \, d\mu + \int\limits_E g_+ \, d\mu
 .\end{align*} Но $f_+ \leqslant g_+ \implies \int_E f_+ \, d\mu \leqslant \int_E g_+ \, d\mu$. И $g_- \leqslant f_- \implies \int_E g_- \, d\mu \leqslant \int_E f_- \, d\mu$.
\end{proof}

Следующая теорема имеет фундаментальное значение в теории меры и функциональном анализе: она позволяет менять местами знак интеграла и знак предела.

\begin{thm}[Леви]
 \label{theorem:levi}
 Пусть $(X, \A, \mu)$ --- пространство с мерой, $\{f_{n}\}_{n=1}^{\infty} $  --- последовательность неотрицательных возрастающих измеримых функций $f_n \colon\, X \to [0,\infty] $, то есть $0 \leqslant f_n(x) \leqslant f_{n+1}(x) \leqslant \infty$ для любого $n \in \N$ и любой точки $x \in X$.

 Пусть функция $f \colon\, X \to [0,\infty] $ --- поточечный предел этой последовательности:
 \begin{align*}
  f(x) = \lim_{n \to \infty} f_n(x) 
 .\end{align*}

 Тогда функция $f$ измерима, и
 \begin{align}
  \label{equation:theorem:levi}
  \lim_{n \to \infty} \int\limits_{E} f_n(x) \, d\mu  = \int\limits_{E}  f(x) \, d\mu
 \end{align} для любого измеримого $E \in \A$.
\end{thm}
\begin{proof}
 Измеримость функции $f$ уже проверена в лемме \ref{lemma:limit_of_measurable_functions_is_measurable}, нужно проверить только равенство \eqref{equation:theorem:levi}.

 По монотонности интеграла Лебега существует предел 
 \begin{align*}
  L = \lim_{n \to \infty} \int\limits_E f_n \, d\mu \in [0, \infty]
 ,\end{align*} при этом также по монотонности $L \leqslant \int_E f\,d\mu$. Нам нужно доказать, что это равенство.

 Возьмём любое число $\Theta \in (0, 1)$. Возьмём любую простую функцию $g$ такую, что $0 \leqslant g \leqslant f$ на $E$. Теперь такой трюк: рассмотрим множества \begin{align*}
  E_n = \left\{ x \in E \mid f_n(x) \geqslant \Theta g(x) \right\}
 .\end{align*}
 Множество $E_n$ можно выразить как
 \begin{align*}
  E_n = (f_n - \Theta g)([0, \infty])
  .\end{align*} Функция $f_n - \Theta g$ измерима (и определена, так как $g < \infty$), поэтому $E_n \in \A$. По возрастанию функций $f_n$ эти множества вложены: $E_n \subset E_{n+1}$ для любого $n \geqslant 1$. Кроме того, \begin{align*}
  \bigcup_{n=1}^{\infty} E_n = E
 .\end{align*} Действительно, если $x \in E$ и $f(x) > 0$, то при больших $n$ неравенство $f_n(x) \geqslant \Theta g(x)$ будет выполнятся, так как $f_n(x)$ будет сколь угодно близким к $f(x)$, и $\Theta < 1$. Если же  $x \in E$ и $f(x) = 0$, то  $g(x) = 0$, и $f_n(x) = 0$ для любого  $n$.

 Используя монотонность интеграла Лебега можно написать такую оценку снизу:
 \begin{align*}
  \int\limits_E f_n \, d\mu \geqslant \int\limits_{E} f_n \cdot \chi_{E_n} \, d\mu \geqslant \Theta \int\limits_E g \cdot \chi_{E_n} \, d\mu
  .\end{align*} Так как $g = \sum_{k=1}^{N} c_k \chi_{A_k}$, $A_k \in \A$, $c_k \geqslant 0$, то по определению \ref{definition:integral_simple_positive} интеграла Лебега простой функции \begin{align*}
  \int\limits_E f_n \, d\mu \geqslant \Theta \sum_{k=1}^{N} c_k \mu (A_k \cap E_n)
 .\end{align*} Перейдём к пределу при $n \to \infty$ в обеих частях:
 \begin{align*}
  L \geqslant \Theta \sum_{k=1}^{N} c_k \cdot \lim_{n \to \infty} \mu(A_k \cap E_n)
 .\end{align*} Но по непрерывности меры сверху (утверждение \ref{claim:upward_continuity_of_measure}) $\lim_{n \to \infty} \mu(A_k \cap E_n) = \mu(A_k \cap E) $, поэтому
 \begin{align*}
  L \geqslant \Theta \sum_{k=1}^{N} c_k \mu(A_k \cap E) = \Theta \int\limits_{E} g \, d\mu 
 .\end{align*} Так как это верно для любого $\Theta < 1$, то
 \begin{align*}
  L \geqslant \int\limits_{E} g \, d\mu 
 .\end{align*} А раз уж это верно для любой простой функции $g$, при условии $0 \leqslant g \leqslant f$, то
 \begin{align*}
  L \geqslant \int\limits_{E} f \, d\mu 
 .\end{align*} 
\end{proof}

В оставшейся части параграфа мы рассмотрим несколько полезных утверждений об интеграле Лебега, непосредственно следующих из теоремы \ref{theorem:levi}  Леви.

Очень много утверждений в теории меры будут доказываться по следующей схеме: сначала мы проверяем утверждение на простых функциях (обычно нужно рассмотреть общее мелкое разбиение по лемме \ref{lemma:common_small_partition_of_simple_functions}), затем по теореме \ref{theorem:approximation} об аппроксимации и теореме \ref{theorem:levi} Леви доказываем его для неотрицательных измеримых  функций, и наконец, проверяем утверждение на суммируемых функциях любого знака (обычно это делается рассмотрением функций $f_{\pm} = \max(\pm f, 0)$). Яркий пример --- доказательство линейности интеграла Лебега. 

\begin{crly}[линейность интеграла Лебега]
 Пусть $f,g$ --- суммируемые функции. Тогда для любого $E \in \A$ \begin{align}
  \label{equation:lebesgue_integral_linearity_1}
  \int\limits_E (f + g) \, d\mu &= \int\limits_E f\, d\mu + \int\limits_E g \, d\mu, \\
  \label{equation:lebesgue_integral_linearity_2}
  \int\limits_{E} (\alpha f) \, d\mu &= \alpha \int\limits_{E}   f \,d\mu, \quad \alpha \in \R
 .\end{align}
\end{crly}
\begin{proof}\
 \begin{enumerate}
  \item Проверим сначала на простых неотрицательных функциях $f,g \geqslant 0$. Рассмотрим общее мелкое разбиение (лемма \ref{lemma:common_small_partition_of_simple_functions}) $X = \bigsqcup_{k=1}^{N} A_k$, $A_k \in \A$,
   \begin{align*}
    f = \sum_{k=1}^{N} c_k \chi_{A_k}, \quad g = \sum_{k=1}^{N} d_k \chi_{A_k}, \quad c_k, d_k \geqslant 0
   .\end{align*} Тогда
   \begin{align*}
    f + g = \sum_{k=1}^{N} (c_k + d_k) \chi_{A_k}, \quad \alpha f = \sum_{k=1}^{N} (\alpha c_k) \chi_{A_k}, \quad \alpha \geqslant 0
   .\end{align*} Поэтому,
   \begin{align*}
    \int\limits_{E} (f + g) \, d\mu &= \sum_{k=1}^{N} (c_k + d_k) \mu(A_k \cap E)  = \sum_{k=1}^{N} c_k \mu(A_k \cap E) + \sum_{k=1}^{N}  d_k \mu(A_k \cap E) = \\
    &= \int\limits_{E} f \, d\mu + \int\limits_{E} g \, d\mu, \\
    \int\limits_{E} (\alpha f) \, d\mu &= \sum_{k=1}^{N} (\alpha c_k) \mu(A_k \cap E) = \alpha \sum_{k=1}^{N} c_k \mu(A_k \cap E) = \alpha \int\limits_{E} f \, d\mu 
   .\end{align*} 
  \item Пусть теперь $f, g \geqslant 0$  --- неотрицательные измеримые. По теореме \ref{theorem:approximation} об аппроксимации существуют последовательности $\{f_{n}\}_{n=1}^{\infty} $ и $\{g_{n}\}_{n=1}^{\infty} $ простых неотрицательных функций, возрастающих к $f$ и $g$ соответственно. По первому пункту для любого $n$ выполнено
   \begin{align*}
    \int\limits_{E} (f_n + g_n) \, d\mu &= \int\limits_{E} f_n \, d\mu + \int\limits_{E} g_n \, d\mu, \\
    \int\limits_{E} (\alpha f_n) \, d\mu  &= \alpha \int\limits_{E} f_n \, d\mu, \quad \alpha \geqslant 0 
   .\end{align*} Так как $\left\{ f_n + g_n \right\}$ и $\left\{ \alpha f_n \right\}$ --- последовательности простых неотрицательных функций, возрастающих к $f + g$ и $\alpha f$ соответственно (при $\alpha \geqslant 0$), то по теореме \ref{theorem:levi} Леви можно совершить предельный переход:
   \begin{align*}
    \int\limits_{E} (f + g) \, d\mu &= \int\limits_{E} f \, d\mu + \int\limits_{E} g \, d\mu, \\ 
    \int\limits_{E} (\alpha f)  \, d\mu &= \alpha \int\limits_{E}  f \, d\mu
   .\end{align*}
  \item Теперь пусть $f, g$ --- произвольные суммируемые функции. Рассмотрим $f_{\pm} = \max(\pm f, 0)$, $g_{\pm} = \max(\pm g, 0)$ и $(f+g)_{\pm} = \max(\pm(f+g),0)$. Тогда равенство \eqref{equation:lebesgue_integral_linearity_1} эквивалентно
   \begin{align*}
    \int\limits_{E} (f+g)_+  \, d\mu + \int\limits_{E} f_- \, d\mu + \int\limits_{E} g_ \, d\mu = \int\limits_{E} (f+g)_- \, d\mu + \int\limits_{E} f_+ \, d\mu + \int\limits_{E} g_+ \, d\mu    
   .\end{align*} Так как все подынтегральные функции неотрицательные, то по предыдущему пункту можно воспользоваться линейностью в обеих частях и свести равенство к
   \begin{align*}
    \int\limits_{E} ((f+g)_+ + f_- + g_-)\, d\mu = \int\limits_{E} ((f+g)_- + f_+ + g_+) \, d\mu
   .\end{align*} Но это равенство интегралов следует просто из равенства подынтегральных функций:
   \begin{align*}
    &(f+g)_+ + f_- + g_- = (f+g)_- + f_+ + g_+ \iff \\
    \iff &(f+g)_+ - (f+g)_- = (f_+ - f_-) + (g_+ - g_-) \iff \\
    \iff &f + g = f + g
   .\end{align*} 

   Осталось проверить \eqref{equation:lebesgue_integral_linearity_2}. Если $\alpha \geqslant 0$, то
   \begin{align*}
    \int\limits_{E} (\alpha f) \, d\mu &= \int\limits_{E} (\alpha f)_+ \, d\mu - \int\limits_{E} (\alpha f)_- \, d\mu = \int\limits_{E} \alpha f_+ \, d\mu - \int\limits_{E} \alpha f_- \, d\mu  = \\
    &= \alpha \left( \int\limits_{E} f_+ \, d\mu - \int\limits_{E} f_- \, d\mu   \right) = \alpha \int\limits_{E} f d\mu
   .\end{align*} Если $\alpha < 0$, то
   \begin{align*}
    \int\limits_{E} (\alpha f)  \, d\mu &= \int\limits_{E} (\alpha f)_+ \, d\mu - \int\limits_{E} (\alpha f)_- \, d\mu = \int\limits_{E} (-\alpha)f_- \, d\mu - \int\limits_{E} (-\alpha) f_+ \, d\mu = \\
    &= -\alpha \int\limits_{E} f_- \, d\mu + \alpha \int\limits_{E} f_+ \, d\mu = \alpha \int\limits_{E} f \, d\mu
   .\end{align*} 
 \end{enumerate}
\end{proof}

Продолжим определение интеграла Лебега на комплексно-значные функции. 

\begin{df}
 \label{definition:complex_lebesgue_integral}
 Комплексно-значная функция $f \colon\, X \to \mathbb{C} $ называется \textit{измеримой}, если функции $\Real f$, $\Imaginary f$ измеримы. $f$ называется \textit{суммируемой}, если $\Real f$ и $\Imaginary f$ суммируемые; в этом случае \textit{интеграл Лебега} равен
 \begin{align*}
  \int\limits_E f \, d\mu = \int\limits_E \mathrm{Re} f \, d\mu + i \int\limits_E \mathrm{Im} f \,d\mu
 \end{align*} для любого $E \in \A$.
\end{df}
\begin{proof}[\normalfont\textsc{Корректность}]
 Корректность понятна: для $f \colon\, X \to \R  $ $\Imaginary f = 0$, $\Real f = f$.
\end{proof}
\begin{claim}[линейность комплексного интеграла Лебега]
 Для суммируемых $f, g \colon\, X \to \CC$ верно
 \begin{align*}
  \int\limits_{E} (f + g) \, d\mu  &= \int\limits_{E} f \, d\mu + \int\limits_{E} g \, d\mu, \\ 
  \int\limits_{E} (\alpha f) \, d\mu &= \alpha \int\limits_{E} f \, d\mu, \quad \alpha \in \CC  
 .\end{align*} 
\end{claim}
\begin{proof}
 Выделим вещественную и мнимую части и сведём к вещественному случаю:
 \begin{align*}
  \int\limits_{E} (f + g)  \, d\mu &= \int\limits_{E} \Real (f + g) \, d\mu + i \int\limits_{E} \Imaginary(f+g)\,d\mu =  \\
  &= \int\limits_{E} (\Real f + \Real g) \, d\mu + i \int\limits_{E} (\Imaginary f + \Imaginary g) \, d\mu = \\
  &= \int\limits_{E} \Real f \, d\mu + \int\limits_{E} \Real g \, d\mu + i \int\limits_{E} \Imaginary f \, d\mu + i \int\limits_{E} \Imaginary g \, d\mu = \\
  &= \int\limits_{E} f \, d\mu + \int\limits_{E}  g \, d\mu , \\
  \int\limits_{E} (a + bi) f \, d\mu &= \int\limits_{E} (a + bi)(\Real f +i \Imaginary f) \, d\mu = \\
  &= \int\limits_{E} (a \Real f - b \Imaginary f + i (a \Imaginary f + b \Real f)) \, d\mu = \\
  &= \int\limits_{E} (a \Real f - b \Imaginary f) \, d\mu + i \int\limits_{E} (a \Imaginary f + b \Real f) \, d\mu = \\
  &= a \int\limits_{E} \Real f \, d\mu -b \int\limits_{E} \Imaginary f \, d\mu + ai \int\limits_{E} \Imaginary f \, d\mu + bi \int\limits_{E} \Real f \, d\mu = \\
  &= (a + bi) \int\limits_{E} \Real f \, d\mu  + (ai - b) \int\limits_{E} \Imaginary f \, d\mu = \\
  &= (a + bi) \left( \int\limits_{E} \Real f \, d\mu + i \int\limits_{E} \Imaginary f \, d\mu   \right) = \\
  &= (a + bi) \int\limits_{E} f \, d\mu 
 .\end{align*} 
\end{proof}

\begin{claim}[основная оценка интеграла]\
 \label{claim:basic_estimation_of_integral}
 Пусть $f \colon\, X \to \mathbb{C} $ измерима, $E \in \A$. Тогда \begin{align*}
  \left| \int\limits_E f \, d\mu \right| \leqslant \int\limits_E \left| f \right| \,d\mu
 .\end{align*} 
\end{claim}
\begin{proof}
 $\left| f \right| = \sqrt{(\mathrm{Re} f)^{2} + (\mathrm{Im} f)^{2}} \implies \left| f \right| $ измерима.

 Выберем $\alpha \in \mathbb{C}$ такое, что $\left| \alpha \right| = 1$, и \begin{align*}
  \left| \int\limits_E f\,d\mu \right| &= \alpha \int\limits_E f\,d\mu = \int\limits_E \alpha f \,d\mu = \int\limits_E \mathrm{Re}(\alpha f) \,d\mu + \underbrace{i \int\limits_E \mathrm{Im}(\alpha f) \, d\mu}_{= 0 \text{, слева вещественное число }} \leqslant \\
  &\leqslant \int\limits_E \left| \alpha f \right| \,d\mu = \int\limits_E \left| f \right| \,d\mu
 .\end{align*} 
\end{proof}
\begin{remrk}
 \begin{align}
  \label{equation:remark:integral_on_subset}
  \int\limits_E f\,d\mu = \int\limits_X \chi_E f \,d\mu
 \end{align} для любого $E \in \A$ и любой интегрируемой функции $f$.

 В других источниках интеграл Лебега сначала определяется только по всему пространству $X$, а по измеримому подмножеству $E \subset X$ интеграл определяется формулой \eqref{equation:remark:integral_on_subset}.
\end{remrk}
\begin{proof}
 На простых проверить легко. Потом, пользуясь теоремами \ref{theorem:approximation} и \ref{theorem:levi}, доказываем для измеримых неотрицательных, а далее проверяем на интегрируемых (и хоть на комплексно-значных).
\end{proof}

\begin{claim}[cвязь интегралов Лебега и Римана]
 Пусть $f \colon\, [0,1] \to \R  $ кусочно-непрерывна на $[0,1]$. Тогда \begin{align*}
  \int\limits_{0}^{1} f(x) \, dx = \int\limits_{[0,1]} f \, d\lambda_1
 .\end{align*}  Слева интеграл Римана, а справа --- интеграл Лебега.
\end{claim}
\begin{proof}
 Напишем, что \begin{align*}
  \sum_{k=0}^{n - 1} \inf_{I_k^{n}} f \cdot \left| I_k^{n} \right| \leqslant \sum_{k=0}^{n - 1} \int\limits_{I_k^{n}} f \, d\lambda_1 \leqslant \sum_{k=0}^{n - 1} \sup_{I_k^{n}} f \cdot \left| I_k^{n} \right|
 ,\end{align*}  где $I_k^{n} = \left[ \frac{k}{n}, \frac{k+1}{n} \right)$. Перейдём к пределу и всё получится.
\end{proof}
\begin{remrk*}
 Функция $\chi_{\R \setminus \Q}$ интегрируема по Лебегу на отрезке $[0,1]$:
 \begin{align*}
  \int\limits_{[0,1]} \chi_{\R \setminus \Q} \, d\lambda_1 = 1
 ,\end{align*} но не интегрируема по Риману.
\end{remrk*}

\end{document}
